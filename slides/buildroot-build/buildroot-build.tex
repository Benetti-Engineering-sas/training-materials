\setbeamerfont{block title}{size=\scriptsize}

\section{Managing the build and the configuration}

\begin{frame}{Default build organization}
  \begin{itemize}
  \item All the build output goes into a directory called
    \code{output/} within the top-level Buildroot source directory.
    \begin{itemize}
    \item \code{O = output}
    \end{itemize}
  \item The configuration file is stored as \code{.config} in the
    top-level Buildroot source directory.
    \begin{itemize}
    \item \code{CONFIG_DIR = $(TOPDIR)}
    \item \code{TOPDIR = $(shell pwd)}
    \end{itemize}
  \item \code{buildroot/}
    \begin{itemize}
    \item {\bf \code{.config}}
    \item \code{arch/}
    \item \code{package/}
    \item {\bf \code{output/}}
    \item \code{fs/}
    \item ...
    \end{itemize}
  \end{itemize}
\end{frame}

\begin{frame}{Out of tree build: introduction}
  \begin{itemize}
  \item Out of tree build allows to use an output directory different
    than \code{output/}
  \item Useful to build different Buildroot configurations from the
    same source tree.
  \item Customization of the output directory done by passing
    \code{O=/path/to/directory} on the command line.
  \item Configuration file stored inside the \code{$(O)} directory, as
    opposed to inside the Buildroot sources for the in-tree build
    case.
  \item \code{project/}
    \begin{itemize}
    \item \code{buildroot/}, Buildroot sources
    \item \code{foo-output/}, output of a first project
      \begin{itemize}
      \item \code{.config}
      \end{itemize}
    \item \code{bar-output/}, output of a second project
      \begin{itemize}
      \item \code{.config}
      \end{itemize}
    \end{itemize}
  \end{itemize}
\end{frame}

\begin{frame}[fragile]{Out of tree build: using}
  \begin{itemize}
  \item To start an out of tree build, two solutions:
    \begin{itemize}
    \item From the Buildroot source tree, simplify specify a \code{O=}
      variable:
      \begin{block}{}
\begin{verbatim}
make O=../foo-output/ menuconfig
\end{verbatim}
      \end{block}
    \item From an empty output directory, specify \code{O=} and the
      path to the Buildroot source tree:
      \begin{block}{}
\begin{verbatim}
make -C ../buildroot/ O=$(pwd) menuconfig
\end{verbatim}
      \end{block}
    \end{itemize}
  \item Once one out of tree operation has been done
    (\code{menuconfig}, loading a defconfig, etc.), Buildroot creates
    a small wrapper \code{Makefile} in the output directory.
  \item This wrapper \code{Makefile} then avoids the need to pass
    \code{O=} and the path to the Buildroot source tree.
  \end{itemize}
\end{frame}

\begin{frame}[fragile]{Out of tree build: example}
  \small
  \begin{enumerate}
  \item You are in your Buildroot source tree:\\
{\tiny
\begin{block}{}
\begin{verbatim}
$ ls
arch board boot ... Makefile ... package ...
\end{verbatim}
\end{block}
}
  \item Create a new output directory, and move to it:\\
{\tiny
\begin{block}{}
\begin{verbatim}
$ mkdir ../foobar-output
$ cd ../foobar-output
\end{verbatim}
\end{block}
}
  \item Start a new Buildroot configuration:\\
{\tiny
\begin{block}{}
\begin{verbatim}
$ make -C ../buildroot O=$(pwd) menuconfig
\end{verbatim}
\end{block}
}
  \item Start the build (passing \code{O=} and \code{-C} no longer needed thanks to the wrapper):\\
{\tiny
\begin{block}{}
\begin{verbatim}
$ make
\end{verbatim}
\end{block}
}
  \item Adjust the configuration again, restart the build, clean the build:\\
{\tiny
\begin{block}{}
\begin{verbatim}
$ make menuconfig
$ make
$ make clean
\end{verbatim}
\end{block}
}
  \end{enumerate}
\end{frame}

\begin{frame}{Full config file vs. {\em defconfig}}
  \begin{itemize}
  \item The \code{.config} file is a {\em full} config file: it
    contains the value for all options (except those having unmet
    dependencies)
  \item The default \code{.config}, without any customization, has
    4467 lines (as of Buildroot 2021.02)
    \begin{itemize}
    \item Not very practical for reading and modifying by humans.
    \end{itemize}
  \item A {\em defconfig} stores only the values for options for which
    the non-default value is chosen.
    \begin{itemize}
    \item Much easier to read
    \item Can be modified by humans
    \item Can be used for automated construction of configurations
    \end{itemize}
  \end{itemize}
\end{frame}

\begin{frame}[fragile]{{\em defconfig}: example}
  \begin{itemize}
  \item For the default Buildroot configuration, the {\em defconfig}
    is empty: everything is the default.
  \item If you change the architecture to be ARM, the {\em defconfig}
    is just one line:
{\small
\begin{block}{}
\begin{verbatim}
BR2_arm=y
\end{verbatim}
\end{block}
}
  \item If then you also enable the \code{stress} package, the {\em
      defconfig} will be just two lines:
{\small
\begin{block}{}
\begin{verbatim}
BR2_arm=y
BR2_PACKAGE_STRESS=y
\end{verbatim}
\end{block}
}
  \end{itemize}
\end{frame}

\begin{frame}{Using and creating a {\em defconfig}}
  \begin{itemize}
  \item To use a {\em defconfig}, copying it to \code{.config} is not
    sufficient as all the missing (default) options need to be
    expanded.
  \item Buildroot allows to load {\em defconfig} stored in the
    \code{configs/} directory, by doing:
    \code{make <foo>_defconfig}
    \begin{itemize}
    \item It overwrites the current \code{.config}, if any
    \end{itemize}
  \item To create a {\em defconfig}, run:\\
    \code{make savedefconfig}
    \begin{itemize}
    \item Saved in the file pointed by the \code{BR2_DEFCONFIG}
      configuration option
    \item By default, points to \code{defconfig} in the current
      directory if the configuration was started from scratch, or
      points to the original {\em defconfig} if the configuration was
      loaded from a defconfig.
    \item Move it to \code{configs/} to make it easily loadable with
      \code{make <foo>_defconfig}.
    \end{itemize}
  \end{itemize}
\end{frame}

\begin{frame}[fragile]{Existing {\em defconfigs}}

  \begin{itemize}

  \item Buildroot comes with a number of existing {\em defconfigs} for
    various publicly available hardware platforms:
    \begin{itemize}
    \item RaspberryPi, BeagleBone Black, CubieBoard, Microchip evaluation
      boards, Minnowboard, various i.MX6 boards
    \item QEMU emulated platforms
    \end{itemize}
  \item List them using \code{make list-defconfigs}
  \item Most built-in {\em defconfigs} are minimal: only build a
    toolchain, bootloader, kernel and minimal root filesystem.
    \begin{block}{}
\begin{verbatim}
$ make qemu_arm_vexpress_defconfig
$ make
\end{verbatim}
\end{block}

\item Additional instructions often available in
  \code{board/<boardname>}, e.g.: \code{board/qemu/arm-vexpess/readme.txt}.
\item Your own {\em defconfigs} can obviously be more featureful
  \end{itemize}
\end{frame}

\begin{frame}[fragile]{Assembling a {\em defconfig} (1/2)}

  \begin{itemize}
  \item {\em defconfigs} are trivial text files, one can use simple
    concatenation to assemble them from fragments.
  \end{itemize}

{\small
   \begin{block}{platform1.frag}
\begin{verbatim}
BR2_arm=y
BR2_TOOLCHAIN_BUILDROOT_WCHAR=y
BR2_GCC_VERSION_7_X=y
\end{verbatim}
   \end{block}
}

{\small
    \begin{block}{platform2.frag}
\begin{verbatim}
BR2_mipsel=y
BR2_TOOLCHAIN_EXTERNAL=y
BR2_TOOLCHAIN_EXTERNAL_CODESOURCERY_MIPS=y
\end{verbatim}
    \end{block}
}

{\small
   \begin{block}{packages.frag}
\begin{verbatim}
BR2_PACKAGE_STRESS=y
BR2_PACKAGE_MTD=y
BR2_PACKAGE_LIBCONFIG=y
\end{verbatim}
   \end{block}
}

\end{frame}

\begin{frame}[fragile]{Assembling a {\em defconfig} (2/2)}

  \begin{block}{debug.frag}
    {\scriptsize
\begin{verbatim}
BR2_ENABLE_DEBUG=y
BR2_PACKAGE_STRACE=y
\end{verbatim}
    }
  \end{block}

  \begin{block}{Build a release system for {\em platform1}}
    {\scriptsize
\begin{verbatim}
$ ./support/kconfig/merge_config.sh platform1.frag packages.frag
$ make
\end{verbatim}
    }
  \end{block}

  \begin{block}{Build a debug system for {\em platform2}}
    {\scriptsize
\begin{verbatim}
$ ./support/kconfig/merge_config.sh platform2.frag packages.frag \
        debug.frag
$ make
\end{verbatim}
    }
  \end{block}

  \begin{itemize}
  \item \code{make olddefconfig} expands a minimal {\em defconfig} to a
    full \code{.config}
  \item Saving fragments is not possible; it must be done manually
    from an existing {\em defconfig}
  \end{itemize}

\end{frame}

\begin{frame}[fragile]{Other building tips}
  \begin{itemize}
  \item Cleaning targets
    \begin{itemize}
    \item Cleaning all the build output, but keeping the configuration
      file:
      \begin{block}{}
\begin{verbatim}
$ make clean
\end{verbatim}
      \end{block}
    \item Cleaning everything, including the configuration file, and
      downloaded file if at the default location:
      \begin{block}{}
\begin{verbatim}
$ make distclean
\end{verbatim}
      \end{block}
    \end{itemize}
  \item Verbose build
    \begin{itemize}
    \item By default, Buildroot hides a number of commands it runs
      during the build, only showing the most important ones.
    \item To get a fully verbose build, pass \code{V=1}:
      \begin{block}{}
\begin{verbatim}
$ make V=1
\end{verbatim}
      \end{block}
    \item Passing \code{V=1} also applies to packages, like the Linux
      kernel, busybox...
    \end{itemize}
  \end{itemize}
\end{frame}
