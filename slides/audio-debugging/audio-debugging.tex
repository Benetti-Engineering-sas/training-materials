\section{Troubleshooting}

\begin{frame}{Troubleshooting: no sound}
  Audio seems to play for the correct duration but there is no sound:
  \begin{itemize}
  \item Unmute \code{Master} and the relevant controls
  \item Turn up the volume
  \item Check the codec analog muxing and mixing (use alsamixer)
  \item Check the amplifier configuration
  \item Check the routing
  \end{itemize}
\end{frame}

\begin{frame}{Troubleshooting: no sound}
  When trying to play sound but it seems stuck:
  \begin{itemize}
  \item Check pinmuxing
  \item Check the configured clock directions
  \item Check the producer/consumer configuration
  \item Check the clocks using an oscilloscope
  \item Check pinmuxing
  \item Some SoCs also have more muxing (NXP i.Mx AUDMUX, TI McASP)
  \end{itemize}
\end{frame}

\begin{frame}[fragile]{Troubleshooting: write error}
  \begin{block}{}
    \fontsize{10}{10}\selectfont
    \begin{minted}{console}
# aplay test.wav
Playing WAVE 'test.wav' : Signed 16 bit Little Endian, Rate 44100 Hz, Stereo
aplay: pcm_write:1737: write error: Input/output error
    \end{minted}
  \end{block}
  \begin{itemize}
  \item Usually caused by an issue in the routing
  \item Check that the codec driver exposes a stream named "Playback"
  \item Use \code{vizdapm}:
    \url{https://github.com/mihais/asoc-tools}
  \end{itemize}
\end{frame}

\begin{frame}[fragile]{Troubleshooting: over/underruns}
  \begin{block}{}
    \fontsize{10}{10}\selectfont
    \begin{minted}{console}
# aplay test.wav 
Playing WAVE 'test.wav' : Signed 16 bit Little Endian, Rate 44100 Hz, Stereo
underrun!!! (at least 1.899 ms long)
underrun!!! (at least 0.818 ms long)
underrun!!! (at least 2.912 ms long)
underrun!!! (at least 8.558 ms long)
    \end{minted}
  \end{block}
  \begin{itemize}
  \item Usually caused by an imprecise BCLK
  \item Try to find a better PLL and dividers combination
  \end{itemize}
\end{frame}

\begin{frame}{Troubleshooting: going further}
  \begin{itemize}
  \item Use \code{speaker-test} to generate audio and play tones.
  \item Be careful with the 440Hz tone, it may not expose all the
    errors. Rather play something that is not commonly divisible (e.g.
    441Hz)
  \item Generate tone with fade in and fade out as this allows to
    catch DMA transfer issues more easily.
  \end{itemize}
\end{frame}

\begin{frame}{Troubleshooting: going further}
  \begin{itemize}
  \item Have a look at the CPU DAI driver and its callback. In
    particular: \code{.set_clkdiv} and \code{.set_sysclk} to
    understand how the various clock dividers are setup.
    \code{.hw_params} or \code{.set_dai_fmt} may do some muxing
  \item Have a look at the codec driver callbacks, \code{.set_sysclk}
    as the \code{clk_id} parameter is codec specific.
  \item Remember using a codec as a clock consumer is an uncommon
    configuration and is probably untested.
  \item When in doubt, use \code{devmem} or \code{i2cget}
  \end{itemize}
\end{frame}

\setupdemoframe
{Troubleshooting}
{
  \begin{itemize}
  \item Using debugfs to find issues
  \item Using vizdapm
  \item Using ftrace to trace register writes and DAPM states
  \end{itemize}
}
