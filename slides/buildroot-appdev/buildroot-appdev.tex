\setbeamerfont{block title}{size=\scriptsize}

\section{Application development}

\begin{frame}{Building code for Buildroot}
  \begin{itemize}
  \item The Buildroot cross-compiler is installed in
    \code{$(HOST_DIR)/bin}
  \item It is already set up to:
    \begin{itemize}
    \item generate code for the configured architecture
    \item look for libraries and headers in \code{$(STAGING_DIR)}
    \end{itemize}
  \item Other useful tools that may be built by Buildroot are
    installed in \code{$(HOST_DIR)/bin}:
    \begin{itemize}
    \item \code{pkg-config}, to find libraries. Beware that it is
      configured to return results for {\em target} libraries: it
      should only be used when cross-compiling.
    \item \code{qmake}, when building Qt applications with this build
      system.
    \item \code{autoconf}, \code{automake}, \code{libtool}, to use
      versions independent from the host system.
    \end{itemize}
  \item Adding \code{$(HOST_DIR)/bin} to your \code{PATH} when
    cross-compiling is the easiest solution.
  \end{itemize}
\end{frame}

\begin{frame}[fragile]{Building code for Buildroot: C program}

\begin{block}{Building a C program for the host}
{\small
  \begin{verbatim}
$ gcc -o foobar foobar.c
$ file foobar
foobar: ELF 64-bit LSB executable, x86-64, version 1...
\end{verbatim}}
\end{block}

\begin{block}{Building a C program for the target}
{\small
  \begin{verbatim}
$ export PATH=$(pwd)/output/host/bin:$PATH
$ arm-linux-gcc -o foobar foobar.c
$ file foobar
foobar: ELF 32-bit LSB executable, ARM, EABI5 version 1...
\end{verbatim}}
\end{block}

\end{frame}

\begin{frame}[fragile]{Building code for Buildroot: pkg-config}

  \begin{block}{Using the system \code{pkg-config}}
{\small
\begin{verbatim}
$ pkg-config --cflags libpng
-I/usr/include/libpng12

$ pkg-config --libs libpng
-lpng12
\end{verbatim}}
  \end{block}

  \begin{block}{Using the Buildroot \code{pkg-config}}
{\small
\begin{verbatim}
$ export PATH=$(pwd)/output/host/bin:$PATH

$ pkg-config --cflags libpng
-I.../output/host/arm-buildroot-linux-uclibcgnueabi/
    sysroot/usr/include/libpng16

$ pkg-config --libs libpng
-L.../output/host/arm-buildroot-linux-uclibcgnueabi/
    sysroot/usr/lib -lpng16
\end{verbatim}}
  \end{block}

{\tiny Note: too long lines have been split.}

\end{frame}

\begin{frame}[fragile]{Building code for Buildroot: autotools}
  \begin{itemize}
  \item Building simple {\em autotools} components outside of
    Buildroot is easy:
    \begin{block}{}
{\small
\begin{verbatim}
$ export PATH=.../buildroot/output/host/bin/:$PATH
$ ./configure --host=arm-linux
\end{verbatim}}
\end{block}
\item Passing \code{--host=arm-linux} tells the configure script to
  use the cross-compilation tools prefixed by \code{arm-linux-}.
\item In more complex cases, some additional \code{CFLAGS} or
  \code{LDFLAGS} might be needed in the environment.
  \end{itemize}
\end{frame}

\begin{frame}[fragile]{Building code for Buildroot: CMake}
  \begin{itemize}
  \item Buildroot generates a {\em CMake toolchain file}, installed in
    \code{output/host/share/buildroot/toolchainfile.cmake}
  \item Tells {\em CMake} which cross-compilation tools to use
  \item Passed using the \code{CMAKE_TOOLCHAIN_FILE} {\em CMake} option
  \item \url{https://cmake.org/cmake/help/latest/manual/cmake-toolchains.7.html}
  \item With this file, building {\em CMake} projects outside of
    Buildroot is easy:
    \begin{block}{}
{\scriptsize
\begin{verbatim}
$ cmake -DCMAKE_TOOLCHAIN_FILE=.../buildroot/output/host/share/buildroot/toolchainfile.cmake .
$ make
$ file app
app: ELF 32-bit LSB executable, ARM, EABI5 version 1 (SYSV), dynamically linked...
\end{verbatim}}
\end{block}
\end{itemize}
\end{frame}

\begin{frame}{Building during development}
  \begin{itemize}
  \item Buildroot is mainly a {\em final integration} tool: it is
    aimed at downloading and building {\bf fixed} versions of software
    components, in a reproducible way.
  \item When doing active development of a software component, you
    need to be able to quickly change the code, build it, and deploy
    it on the target.
  \item The package build directory is temporary, and removed on
    \code{make clean}, so making changes here is not practical
  \item Buildroot does not automatically ``update'' your source code
    when the package is fetched from a version control system.
  \item Three solutions:
    \begin{itemize}
    \item Build your software component outside of Buildroot during
      development. Doable for software components that are easy to
      build.
    \item Use the \code{local} \code{SITE_METHOD} for your package
    \item Use the \code{<pkg>_OVERRIDE_SRCDIR} mechanism
    \end{itemize}
  \end{itemize}
\end{frame}

\begin{frame}[fragile]{{\tt local} site method}
  \begin{itemize}
  \item Allows to tell Buildroot that the source code for a package is
    already available locally
  \item Allows to keep your source code under version control,
    separately, and have Buildroot always build your latest changes.
  \item Typical project organization:
    \begin{itemize}
    \item \code{buildroot/}, the Buildroot source code
    \item \code{external/}, your \code{BR2_EXTERNAL} tree
    \item \code{custom-app/}, your custom application code
    \item \code{custom-lib/}, your custom library
    \end{itemize}
  \item In your package \code{.mk} file, use:
    \begin{block}{}
\begin{verbatim}
<pkg>_SITE = $(TOPDIR)/../custom-app
<pkg>_SITE_METHOD = local
\end{verbatim}
    \end{block}
  \end{itemize}
\end{frame}

\begin{frame}{Effect of {\tt local} site method}
  \begin{itemize}
  \item For the first build, the source code of your package is {\em
      rsync'ed} from \code{<pkg>_SITE} to the build directory, and
    built there.
  \item After making changes to the source code, you can run:
    \begin{itemize}
    \item \code{make <pkg>-reconfigure}
    \item \code{make <pkg>-rebuild}
    \item \code{make <pkg>-reinstall}
    \end{itemize}
  \item Buildroot will first {\em rsync} again the package source code
    (copying only the modified files) and restart the build from the
    requested step.
  \end{itemize}
\end{frame}

\begin{frame}{{\tt local} site method workflow}
  \begin{center}
    \includegraphics[height=0.8\textheight]{slides/buildroot-appdev/local-site-method.pdf}
  \end{center}
\end{frame}

\begin{frame}{{\tt <pkg>\_OVERRIDE\_SRCDIR}}
  \begin{itemize}
  \item The \code{local} site method solution is appropriate when the
    package uses this method for all developers
    \begin{itemize}
    \item Requires that all developers fetch locally the source code
      for all custom applications and libraries
    \end{itemize}
  \item An alternate solution is that packages for custom applications
    and libraries fetch their source code from version control systems
    \begin{itemize}
    \item Using the \code{git}, \code{svn}, \code{cvs}, etc. fetching
      methods
    \end{itemize}
  \item Then, locally, a user can {\bf override} how the package is
    fetched using \code{<pkg>_OVERRIDE_SRCDIR}
    \begin{itemize}
    \item It tells Buildroot to not {\em download} the package source
      code, but to copy it from a local directory.
    \end{itemize}
  \item The package then behaves as if it was using the \code{local}
    site method.
  \end{itemize}
\end{frame}

\begin{frame}[fragile]{Passing {\tt <pkg>\_OVERRIDE\_SRCDIR}}
  \begin{itemize}
  \item \code{<pkg>_OVERRIDE_SRCDIR} values are specified in a {\em
      package override file}, configured in
    \code{BR2_PACKAGE_OVERRIDE_FILE}, by default
    \code{$(CONFIG_DIR)/local.mk.}
  \end{itemize}

  \begin{block}{Example \code{local.mk}}
\begin{minted}{make}
LIBPNG_OVERRIDE_SRCDIR = $(HOME)/projects/libpng
LINUX_OVERRIDE_SRCDIR = $(HOME)/projects/linux
\end{minted}
  \end{block}
\end{frame}

\begin{frame}{Debugging: debugging symbols and stripping}
  \begin{itemize}
  \item To use debuggers, you need the programs and libraries to be
    built with debugging symbols.
  \item The \code{BR2_ENABLE_DEBUG} option controls whether programs
    and libraries are built with debugging symbols
    \begin{itemize}
    \item Disabled by default.
    \item Sub-options allow to control the amount of debugging symbols
      (i.e. gcc options \code{-g1}, \code{-g2} and \code{-g3}).
    \end{itemize}
  \item The \code{BR2_STRIP_none} and \code{BR2_STRIP_strip} options
    allow to disable or enable stripping of binaries on the target.
  \end{itemize}
\end{frame}

\begin{frame}{Debugging: debugging symbols and stripping}
  \begin{itemize}
  \item With \code{BR2_ENABLE_DEBUG=y} and \code{BR2_STRIP_strip=y}
    \begin{itemize}
    \item get debugging symbols in \code{$(STAGING_DIR)} for
      libraries, and in the build directories for everything.
    \item stripped binaries in \code{$(TARGET_DIR)}
    \item Appropriate for {\bf remote debugging}
    \end{itemize}
  \item With \code{BR2_ENABLE_DEBUG=y} and \code{BR2_STRIP_none=y}
    \begin{itemize}
    \item debugging symbols in both \code{$(STAGING_DIR)} and
      \code{$(TARGET_DIR)}
    \item appropriate for {\bf on-target debugging}
    \end{itemize}
  \end{itemize}
\end{frame}

\begin{frame}{Debugging: remote debugging requirements}
  \begin{itemize}
  \item To do remote debugging, you need:
    \begin{itemize}
    \item A {\bf cross-debugger}
      \begin{itemize}
      \item With the {\em internal toolchain backend}, can be built
        using \code{BR2_PACKAGE_HOST_GDB=y}.
      \item With the {\em external toolchain backend}, is either
        provided pre-built by the toolchain, or can be built using
        \code{BR2_PACKAGE_HOST_GDB=y}.
      \end{itemize}
    \item {\bf gdbserver}
      \begin{itemize}
      \item With the {\em internal toolchain backend}, can be built
        using \code{BR2_PACKAGE_GDB=y} + \code{BR2_PACKAGE_GDB_SERVER=y}
      \item With the {\em external toolchain backend}, if
        \code{gdbserver} is provided by the toolchain it can be copied
        to the target using
        \code{BR2_TOOLCHAIN_EXTERNAL_GDB_SERVER_COPY=y} or otherwise
        built from source like with the internal toolchain backend.
      \end{itemize}
    \end{itemize}
  \end{itemize}
\end{frame}

\begin{frame}{Debugging: remote debugging setup}
  \begin{itemize}
  \item On the target, start {\em gdbserver}
    \begin{itemize}
    \item Use a TCP socket, network connectivity needed
    \item The {\em multi} mode is quite convenient
    \item \code{$ gdbserver --multi localhost:2345}
    \end{itemize}
  \item On the host, start \code{<tuple>-gdb}
    \begin{itemize}
    \item \code{$ ./output/host/bin/<tuple>-gdb <program>}
    \item \code{<program>} is the path to the program to debug, with
      debugging symbols
    \end{itemize}
  \item Inside {\em gdb}, you need to:
    \begin{itemize}
    \item Connect to the target:\\
      \code{(gdb) target extended-remote <ip>:2345}
    \item Set the path to the {\em sysroot} so that {\em gdb} can find debugging symbols for libraries:\\
      \code{(gdb) set sysroot ./output/staging/}
    \item Start the program:\\
      \code{(gdb) run}
    \end{itemize}
  \end{itemize}
\end{frame}

\begin{frame}{Debugging tools available in Buildroot}
  \begin{itemize}
  \item Buildroot also includes a huge amount of other debugging or
    profiling related tools.
  \item To list just a few:
    \begin{itemize}
    \item strace
    \item ltrace
    \item LTTng
    \item perf
    \item sysdig
    \item sysprof
    \item OProfile
    \item valgrind
    \end{itemize}
  \item Look in \code{Target packages} $\rightarrow$ \code{Debugging,
      profiling and benchmark} for more.
  \end{itemize}
\end{frame}

\begin{frame}{Generating a SDK for application developers}
  \begin{itemize}
  \item If you would like application developers to build applications
    for a Buildroot generated system, without building Buildroot, you
    can generate a SDK.
  \item To achieve this:
    \begin{itemize}
    \item Run \code{make sdk}, which prepares the SDK to be
      relocatable
    \item Tarball the contents of the {\em host} directory, i.e
      \code{output/host}
    \item Share the tarball with your application developers
    \item They must uncompress it, and run the \code{relocate-sdk.sh}
      script
    \end{itemize}
  \item {\bf Warning}: the SDK must remain in sync with the root
    filesystem running on the target, otherwise applications built
    with the SDK may not run properly.
  \end{itemize}
\end{frame}

\setuplabframe
{Application development}
{
  \begin{itemize}
  \item Build and run your own application
  \item Remote debug your application
  \item Use \code{<pkg>_OVERRIDE_SRCDIR}
  \end{itemize}
}
