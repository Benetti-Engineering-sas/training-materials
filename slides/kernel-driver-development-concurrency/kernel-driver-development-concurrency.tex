\section{Concurrent Access to Resources: Locking}

\begin{frame}
  \frametitle{Sources of concurrency issues}
  \begin{itemize}
  \item In terms of concurrency, the kernel has the same constraint
    as a multi-threaded program: its state is global and visible
    in all executions contexts
  \item Concurrency arises because of
    \begin{itemize}
    \item \emph{Interrupts}, which interrupts the current thread to
      execute an interrupt handler. They may be using shared
      resources (memory addresses, hardware registers...)
    \item \emph{Kernel preemption}, if enabled, causes the kernel to
      switch from the execution of one system call to another. They
      may be using shared resources.
    \item \emph{Multiprocessing}, in which case code is really
      executed in parallel on different processors, and they may be
      using shared resources as well.
    \end{itemize}
  \item The solution is to keep as much local state as possible and
    for the shared resources that can't be made local (such as
    hardware ones), use locking.
  \end{itemize}
\end{frame}

\begin{frame}
  \frametitle{Concurrency protection with locks}
  \begin{center}
    \includegraphics[height=0.8\textheight]{slides/kernel-driver-development-concurrency/concurrency-protection.pdf}
  \end{center}
\end{frame}

\begin{frame}[fragile]
  \frametitle{Linux mutexes}
  {\em mutex = {\bf mut}ual {\bf ex}clusion}
  \begin{itemize}
  \item The kernel's main locking primitive. It's a {\em binary lock}.
    Note that {\em counting locks} ({\em semaphores}) are also available,
    but used 30x less frequently.
  \item The process requesting the lock blocks when the lock is
    already held.  Mutexes can therefore only be used in contexts
    where sleeping is allowed.
  \item Mutex definition:
    \begin{itemize}
    \item \mint{c}+#include <linux/mutex.h>+
    \end{itemize}
  \item Initializing a mutex statically (unusual case):
    \begin{itemize}
    \item \mint{c}+DEFINE_MUTEX(name);+
    \end{itemize}
  \item Or initializing a mutex dynamically (the usual case, on a per-device basis):
    \begin{itemize}
    \item \mint{c}+void mutex_init(struct mutex *lock);+
    \end{itemize}
  \end{itemize}
\end{frame}

\begin{frame}[fragile]
  \frametitle{Locking and unlocking mutexes 1/2}
  \begin{itemize}
  \item \mint{c}+void mutex_lock(struct mutex *lock);+
    \begin{itemize}
    \item Tries to lock the mutex, sleeps otherwise.
    \item Caution: can't be interrupted, resulting in processes you
      cannot kill!
    \end{itemize}
  \item \mint{c}+int mutex_lock_killable(struct mutex *lock);+
    \begin{itemize}
    \item Same, but can be interrupted by a fatal (\ksym{SIGKILL}) signal. If
      interrupted, returns a non zero value and doesn't hold the
      lock. Test the return value!!!
    \end{itemize}
  \item \mint{c}+int mutex_lock_interruptible(struct mutex *lock);+
    \begin{itemize}
    \item Same, but can be interrupted by any signal.
    \end{itemize}
  \end{itemize}
\end{frame}

\begin{frame}[fragile]
  \frametitle{Locking and unlocking mutexes 2/2}
  \begin{itemize}
  \item \mint{c}+int mutex_trylock(struct mutex *lock);+
    \begin{itemize}
    \item Never waits. Returns a non zero value if the mutex is not
      available.
    \end{itemize}
  \item \mint{c}+int mutex_is_locked(struct mutex *lock);+
    \begin{itemize}
    \item Just tells whether the mutex is locked or not.
    \end{itemize}
  \item \mint{c}+void mutex_unlock(struct mutex *lock);+
    \begin{itemize}
    \item Releases the lock. Do it as soon as you leave the critical
      section.
    \end{itemize}
  \end{itemize}
\end{frame}

\begin{frame}
  \frametitle{Spinlocks}
  \begin{itemize}
  \item Locks to be used for code that is not allowed to sleep
    (interrupt handlers), or that doesn't want to sleep (critical
    sections). Be very careful not to call functions which can sleep!
  \item Originally intended for multiprocessor systems
  \item Spinlocks never sleep and keep spinning in a loop until the
    lock is available.
  \item The critical section protected by a spinlock is not allowed to
    sleep.
  \end{itemize}
  \begin{center}
    \includegraphics[width=0.4\textwidth]{slides/kernel-driver-development-concurrency/spinlock.pdf}
  \end{center}
\end{frame}

\begin{frame}[fragile]
  \frametitle{Initializing spinlocks}
  \begin{itemize}
  \item Statically (unusual)
    \begin{itemize}
    \item \mint{c}+DEFINE_SPINLOCK(my_lock);+
    \end{itemize}
  \item Dynamically (the usual case, on a per-device basis)
    \begin{itemize}
    \item \mint{c}+void spin_lock_init(spinlock_t *lock);+
    \end{itemize}
  \end{itemize}
\end{frame}

\begin{frame}[fragile]
  \frametitle{Using spinlocks 1/3}
  Several variants, depending on where the spinlock is called:
  \begin{itemize}
  \item \mint{c}+void spin_lock(spinlock_t *lock);+
  \item \mint{c}+void spin_unlock(spinlock_t *lock);+
    \begin{itemize}
    \item Used for locking in process context (critical sections
          in which you do not want to sleep).
    \item Kernel preemption on the local CPU is disabled. We need
          to avoid deadlocks because of preemption from processes
	  that want to get the same lock:
    \end{itemize}
  \end{itemize}
  \includegraphics[width=0.8\textwidth]{common/spinlock-deadlock-with-preemption.pdf}
\end{frame}

\begin{frame}[fragile]
  \frametitle{Using spinlocks 2/3}
  \begin{itemize}
  \item \mint{c}+void spin_lock_irqsave(spinlock_t *lock, unsigned long flags);+
  \item \mint{c}+void spin_unlock_irqrestore(spinlock_t *lock, unsigned long flags);+
    \begin{itemize}
    \item Disables / restores IRQs on the local CPU.
    \item Typically used when the lock can be accessed in both process
      and interrupt context.
    \item We need to avoid deadlocks because of interrupts that want
          to get the same lock.
    \end{itemize}
  \end{itemize}
  \includegraphics[width=0.8\textwidth]{common/spinlock-deadlock-with-interrupt.pdf}
\end{frame}

\begin{frame}[fragile]
  \frametitle{Using spinlocks 3/3}
  \begin{itemize}
  \item \mint{c}+void spin_lock_bh(spinlock_t *lock);+
  \item \mint{c}+void spin_unlock_bh(spinlock_t *lock);+
    \begin{itemize}
    \item Disables software interrupts, but not hardware ones.
    \item Useful to protect shared data accessed in process context
      and in a soft interrupt (\emph{bottom half}).
    \item No need to disable hardware interrupts in this case.
    \end{itemize}
  \item Note that reader / writer spinlocks also exist, allowing
	for multiple simultaneous readers.
  \end{itemize}
\end{frame}

\begin{frame}[fragile]
  \frametitle{Spinlock example}
  \begin{itemize}
  \item From \kfile{drivers/tty/serial/uartlite.c}
  \item Spinlock structure embedded into \kstruct{uart_port}
    \begin{block}{}
    \begin{minted}[fontsize=\small]{c}
struct uart_port {
        spinlock_t lock;
        /* Other fields */
};
    \end{minted}
    \end{block}
  \item Spinlock taken/released with protection against interrupts
    \begin{block}{}
    \begin{minted}[fontsize=\small]{c}
static unsigned int ulite_tx_empty
    (struct uart_port *port) {
        unsigned long flags;

        spin_lock_irqsave(&port->lock, flags);
        /* Do something */
        spin_unlock_irqrestore(&port->lock, flags);
}
    \end{minted}
    \end{block}
  \end{itemize}
\end{frame}

\begin{frame}
  \frametitle{More deadlock situations} They can lock up your system. Make sure they never happen!
  \vspace{0.5cm}
  \begin{columns}
  \column[t]{0.5\textwidth}
      Rule 1: don't call a function that can try to get access to the same lock
      \includegraphics[width=0.9\textwidth]{slides/kernel-driver-development-concurrency/deadlock-same-lock.pdf}
  \column[t]{0.5\textwidth}
      Rule 2: if you need multiple locks, always acquire them in the same order!
      \includegraphics[width=0.9\textwidth]{slides/kernel-driver-development-concurrency/deadlock-two-locks.pdf}
  \end{columns}
\end{frame}

\begin{frame}
  \frametitle{Debugging locking and concurrency issues}
  \begin{itemize}
  \item Lock debugging: prove locking correctness
    \begin{itemize}
    \item \kconfig{CONFIG_PROVE_LOCKING}
    \item Adds instrumentation to kernel locking code
    \item Detect violations of locking rules during system life, such as:
      \begin{itemize}
      \item Locks acquired in different order (keeps track of locking
        sequences and compares them).
      \item Spinlocks acquired in interrupt handlers and also in
        process context when interrupts are enabled.
      \end{itemize}
    \item Not suitable for production systems but acceptable overhead
      in development.
    \item See \kdochtml{locking/lockdep-design} for details
    \end{itemize}
  \item Kernel Concurrency SANitizer framework
     \begin{itemize}
     \item \kconfig{CONFIG_KCSAN}, introduced in Linux 5.8.
     \item Can find concurrency issues in your system.
     \item See \url{https://lwn.net/Articles/816850/} for details.
     \end{itemize}
  \end{itemize}
\end{frame}

\begin{frame}
  \frametitle{Alternatives to locking}
  As we have just seen, locking can have a strong negative
  impact on system performance. In some situations, you could do
  without it.
  \begin{itemize}
  \item By using lock-free algorithms like \emph{Read Copy Update}
    (RCU).
  \item RCU API available in the kernel (See
    \url{https://en.wikipedia.org/wiki/Read-copy-update}).
  \item When available, use atomic operations.
  \end{itemize}
\end{frame}

\begin{frame}[fragile]
  \frametitle{Atomic variables 1/2}
  \begin{itemize}
  \item Useful when the shared resource is an integer value
  \item Even an instruction like \code{n++} is not guaranteed to be
    atomic on all processors!
  \item Atomic operations definitions
    \begin{itemize}
    \item \mint{c}+#include <asm/atomic.h>+
    \end{itemize}
  \item \ksym{atomic_t}
    \begin{itemize}
    \item Contains a signed integer (at least 24 bits)
    \end{itemize}
  \item Atomic operations (main ones)
    \begin{itemize}
    \item Set or read the counter:
      \begin{itemize}
      \item \mint{c}+void atomic_set(atomic_t *v, int i);+
      \item \mint{c}+int atomic_read(atomic_t *v);+
      \end{itemize}
    \item Operations without return value:
      \begin{itemize}
      \item \mint{c}+void atomic_inc(atomic_t *v);+
      \item \mint{c}+void atomic_dec(atomic_t *v);+
      \item \mint{c}+void atomic_add(int i, atomic_t *v);+
      \item \mint{c}+void atomic_sub(int i, atomic_t *v);+
      \end{itemize}
    \end{itemize}
  \end{itemize}
\end{frame}

\begin{frame}[fragile]
  \frametitle{Atomic variables 2/2}
  \begin{itemize}
  \item Similar functions testing the result:
    \begin{itemize}
    \item \mint{c}+int atomic_inc_and_test(...);+
    \item \mint{c}+int atomic_dec_and_test(...);+
    \item \mint{c}+int atomic_sub_and_test(...);+
    \end{itemize}
  \item Functions returning the new value:
    \begin{itemize}
    \item \mint{c}+int atomic_inc_return(...);+
    \item \mint{c}+int atomic_dec_return(...);+
    \item \mint{c}+int atomic_add_return(...);+
    \item \mint{c}+int atomic_sub_return(...);+
    \end{itemize}
  \end{itemize}
\end{frame}

\begin{frame}[fragile]
  \frametitle{Atomic bit operations}
  \begin{itemize}
  \item Supply very fast, atomic operations
  \item On most platforms, apply to an \code{unsigned long *} type.
  \item Apply to a \code{void *} type on a few others.
  \item Set, clear, toggle a given bit:
    \begin{itemize}
    \item \mint{c}+void set_bit(int nr, unsigned long *addr);+
    \item \mint{c}+void clear_bit(int nr, unsigned long *addr);+
    \item \mint{c}+void change_bit(int nr, unsigned long *addr);+
    \end{itemize}
  \item Test bit value:
    \begin{itemize}
    \item \mint{c}+int test_bit(int nr, unsigned long *addr);+
    \end{itemize}
  \item Test and modify (return the previous value):
    \begin{itemize}
    \item \mint{c}+int test_and_set_bit(...);+
    \item \mint{c}+int test_and_clear_bit(...);+
    \item \mint{c}+int test_and_change_bit(...);+
    \end{itemize}
  \end{itemize}
\end{frame}

\begin{frame}
  \frametitle{Kernel locking: summary and references}
  \begin{columns}
  \column[t]{0.65\textwidth}
    \begin{itemize}
    \item Use mutexes in code that is allowed to sleep
    \item Use spinlocks in code that is not allowed to sleep (interrupts)
      or for which sleeping would be too costly (critical sections)
    \item Use atomic operations to protect integers or addresses
    \end{itemize}
    See \kdochtml{kernel-hacking/locking} in kernel documentation
    for many details about kernel locking mechanisms.
  \column[t]{0.35\textwidth}
    \small
    Further reading: see the classical
    {\em \href{https://en.wikipedia.org/wiki/Dining_philosophers_problem}
    {dining philosophers problem}} for a nice illustration of synchronization
    and concurrency issues.
    \includegraphics[width=\textwidth]{slides/kernel-driver-development-concurrency/An_illustration_of_the_dining_philosophers_problem.jpg}
    \tiny Image source: Wikipedia (\url{https://frama.link/xg3Wnd0F})
  \end{columns}
\end{frame}
