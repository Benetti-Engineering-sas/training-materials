\section{Linux Common Analysis \& Observability Tools}

\begin{frame}
  \frametitle{Pseudo Filesystems}
  \begin{itemize}
    \item Some virtual filesystems are exposed by the kernel and provide a lot
          of information on the system.
    \item {\em procfs} contains information about processes and system
          information.
    \begin{itemize}
      \item Mounted on \code{/proc}
      \item Often parsed by tools to display raw data in a more user-friendly
            way.
    \end{itemize}
    \item {\em sysfs} provides informations about hardware/logical devices,
          association between devices and drivers.
    \begin{itemize}
      \item Mounted on \code{/sys}
    \end{itemize}
    \item {\em debugfs} exposes information related to debug.
    \begin{itemize}
      \item Typically mounted on \code{/sys/kernel/debug/}
      \item \code{mount -t debugfs none /sys/kernel/debug}
    \end{itemize}
  \end{itemize}
\end{frame}

\begin{frame}
  \frametitle{procfs}
  \begin{itemize}
    \item {\em procfs} exposes information about processes and system
          (\manpage{proc}{5}).
    \begin{itemize}
      \item \code{/proc/cpuinfo} CPU information.
      \item \code{/proc/meminfo} memory information (used, free, total, etc).
      \item \code{/proc/<pid>/} process related information
      \begin{itemize}
        \item\code{/proc/<pid>/status} process basic information
        \item \code{/proc/<pid>/maps} process memory mappings
        \item \code{/proc/<pid>/fd} file descriptors of the process
        \item \code{/proc/<pid>/task} descriptors of threads belonging
          to the process
      \end{itemize}
      \item \code{/proc/self/} will refer to the process used to access the file
      \item \code{/proc/sys/} contains system parameters that can be tuned. The
            list of parameters that can be modified is available at
            \kdochtml{admin-guide/sysctl/index}
      \item \code{/proc/irq/<irq>} interrupt information.
    \end{itemize}
    \item A list of all available {\em procfs} file and their content is
          described at \kdochtml{filesystems/proc} and \manpage{proc}{5}
  \end{itemize}
\end{frame}

\begin{frame}
  \frametitle{sysfs}
  \begin{itemize}
    \item {\em sysfs} filesystem exposes information about various kernel
          subsystems, hardware devices and association with drivers
          (\manpage{sysfs}{5}).
    \item This allows to find the link between drivers and devices through a
          file hierarchy representing the kernel internal tree of devices.
    \item \code{/sys/kernel} contains interesting files for kernel debugging:
    \begin{itemize}
      \item \code{irq} with information about interrupts (mapping, count, etc).
      \item \code{tracing} for tracing control.
    \end{itemize}
    \item \kdochtml{admin-guide/abi-stable}
  \end{itemize}
\end{frame}

\begin{frame}
  \frametitle{debugfs}
  \begin{itemize}
    \item {\em debugfs} is a simple RAM-based filesystem which exposes debugging
          information.
    \item Used by some subsystems ({\em clk}, {\em block}, {\em dma}, {\em gpio},
          etc) to expose debugging information related to the internals.
    \item Usually mounted on \code{/sys/kernel/debug}
    \begin{itemize}
      \item Dynamic debug features exposed through \code{/sys/kernel/debug/dynamic_debug}.
      \item Clock tree exposed through \code{/sys/kernel/debug/clk/clk_summary}.
    \end{itemize}
  \end{itemize}
\end{frame}

\begin{frame}[fragile]
  \frametitle{binutils for ELF analysis}
  \begin{itemize}
    \item The binutils are used to deal with binary files, either object files
          or executables.
    \begin{itemize}
      \item Includes \code{ld}, \code{as} and other useful tools.
    \end{itemize}
    \item {\em readelf} displays information about ELF files (header, section,
          segments, etc).
    \item {\em objdump} allows to display information and disassemble ELF
          files.
    \item {\em objcopy} can convert ELF files or extract/translate some
          parts of it.
    \item {\em nm} displays the list of symbols embedded in ELF files.
    \item {\em addr2line} finds the source code line/file pair from an address using
          an ELF file with debug information
  \end{itemize}
\end{frame}

\begin{frame}[fragile]
  \frametitle{binutils example (1/2)}
  \begin{itemize}
    \item Finding the address of \code{ksys_read()} kernel function using {\em nm}:
    \begin{block}{}
      \begin{minted}[fontsize=\footnotesize]{console}
$ nm vmlinux | grep ksys_read
c02c7040 T ksys_read
      \end{minted}
    \end{block}

    \item Using {\em addr2line} to match a kernel OOPS address or a symbol name
      with source code:
    \begin{block}{}
      \begin{minted}[fontsize=\footnotesize]{console}
$ addr2line -s -f -e vmlinux ffffffff8145a8b0
queue_wc_show
blk-sysfs.c:516
      \end{minted}
% The following is supported only by recent versions of addr2line, not yet on Ubuntu 22.04
% TODO add it back later, when switching to Ubuntu 24.04?
%$ addr2line -e vmlinux printk+0x10
%/home/training/debugging-labs/buildroot/output/build/linux-5.13/kernel/printk/printk.c:2211
    \end{block}
  \end{itemize}
\end{frame}

\begin{frame}[fragile]
  \frametitle{binutils example (2/2)}
  \begin{itemize}
    \item Display an elf header with {\em readelf}:
    \begin{block}{}
      \begin{minted}[fontsize=\footnotesize]{console}
$ readelf -h binary
ELF Header:
Magic:   7f 45 4c 46 02 01 01 00 00 00 00 00 00 00 00 00
Class:                             ELF64
Data:                              2's complement, little endian
Version:                           1 (current)
OS/ABI:                            UNIX - System V
ABI Version:                       0
Type:                              DYN (Position-Independent Executable file)
Machine:                           Advanced Micro Devices X86-64
...
      \end{minted}
    \end{block}

    \item Convert an elf file to a flat binary file using {\em objcopy}:
    \begin{block}{}
      \begin{minted}[fontsize=\footnotesize]{console}
$ objcopy -O binary file.elf file.bin
      \end{minted}
    \end{block}
  \end{itemize}
\end{frame}


\begin{frame}[fragile]
  \frametitle{{\em ldd}}
  \begin{itemize}
    \item In order to display the shared libraries used by an ELF binary, one
          can use {\em ldd} (Generally packaged with C library. See \manpage{ldd}{1}).
    \item {\em ldd} will list all the libraries that were used at link time.
    \begin{itemize}
      \item Libraries that are loaded at runtime using \code{dlopen()} are not
            displayed.
    \end{itemize}
  \end{itemize}
  \begin{block}{}
    \begin{minted}[fontsize=\footnotesize]{console}
$ ldd /usr/bin/bash
linux-vdso.so.1 (0x00007ffdf3fc6000)
libreadline.so.8 => /usr/lib/libreadline.so.8 (0x00007fa2d2aef000)
libc.so.6 => /usr/lib/libc.so.6 (0x00007fa2d2905000)
libncursesw.so.6 => /usr/lib/libncursesw.so.6 (0x00007fa2d288e000)
/lib64/ld-linux-x86-64.so.2 => /usr/lib64/ld-linux-x86-64.so.2 (0x00007fa2d2c88000)
    \end{minted}
  \end{block}
\end{frame}

\begin{frame}
  \frametitle{Monitoring Tools}
  \begin{itemize}
    \item Lots of monitoring tools on Linux to allow monitoring various part of
          the system.
    \item Most of the time, these are CLI interactive programs.
    \begin{itemize}
        \item Processes with {\em ps}, {\em top}, {\em htop}, etc
        \item Memory with {\em free}, {\em vmstat}
        \item Networking
    \end{itemize}
    \item Almost all these tools relies on the {\em sysfs} or {\em procfs}
          filesystem to obtain the processes, memory and system information but
          will display them in a more human readable way.
    \begin{itemize}
      \item Networking tools uses a netlink interface with the networking
            subsystem of the kernel.
    \end{itemize}
  \end{itemize}
\end{frame}

\begin{frame}[fragile]
  \frametitle{Processes with {\em ps}}
  \begin{itemize}
    \item The {\em ps} command allows to display a snapshot of active processes and
          their associated information (\manpage{ps}{1})
    \begin{itemize}
      \item Lists both user processes and kernel threads.
      \item Displays PID, CPU usage, memory usage, uptime, etc.
    \end{itemize}
    \begin{itemize}
      \item Uses {\em /proc/<pid>/} directory to obtain process information.
      \item Always present on almost all embedded platforms (provided by
            {\em Busybox}).
    \end{itemize}
    \item By default, displays only the current user/current tty processes.
    \item Useful for scripting and parsing since its output is static.
  \end{itemize}
\end{frame}

\begin{frame}[fragile]
  \frametitle{{\em ps} example}
  \begin{itemize}
  \frametitle{Processes with {\em ps}}
    \item Display all processes in a friendly way:
  \end{itemize}
  \begin{block}{}
    \begin{minted}[fontsize=\footnotesize]{console}
$ ps aux
USER         PID %CPU %MEM    VSZ   RSS TTY      STAT START   TIME COMMAND
root           1  0.0  0.0 168864 12800 ?        Ss   09:08   0:00 /sbin/init
root           2  0.0  0.0      0     0 ?        S    09:08   0:00 [kthreadd]
root           3  0.0  0.0      0     0 ?        I<   09:08   0:00 [rcu_gp]
root           4  0.0  0.0      0     0 ?        I<   09:08   0:00 [rcu_par_gp]
root           5  0.0  0.0      0     0 ?        I<   09:08   0:00 [netns]
...
root         914  0.0  0.0 396216 16220 ?        Ssl  09:08   0:04 /usr/libexec/udisks2/udisksd
avahi        929  0.0  0.0   8728   412 ?        S    09:08   0:00 avahi-daemon: chroot helper
root         956  0.0  0.1 260304 19024 ?        Ssl  09:08   0:02 /usr/sbin/NetworkManager --no-daemon
root         960  0.0  0.0  17040  5704 ?        Ss   09:08   0:00 /sbin/wpa_supplicant -u -s -O /run/wpa_suppli
root         962  0.0  0.0 317644 11896 ?        Ssl  09:08   0:00 /usr/sbin/ModemManager
vnstat       987  0.0  0.0   5516  3696 ?        Ss   09:08   0:00 /usr/sbin/vnstatd -n
    \end{minted}
  \end{block}
\end{frame}

\begin{frame}[fragile]
  \frametitle{Processes with {\em top}}
  \begin{itemize}
    \item {\em top} command output information similar to {\em ps} but dynamic
          and interactive (\manpage {top}{1}).
    \begin{itemize}
      \item Also almost always present on embedded platforms (provided by
            {\em Busybox})
    \end{itemize}
  \end{itemize}
  \begin{block}{}
    \begin{minted}[fontsize=\footnotesize]{console}
$ top
top - 18:38:11 up  9:29,  1 user,  load average: 2.84, 2.74, 2.02
Tasks: 371 total,   1 running, 370 sleeping,   0 stopped,   0 zombie
%Cpu(s):  5.8 us,  2.1 sy,  0.0 ni, 77.4 id, 14.7 wa,  0.0 hi,  0.0 si,  0.0 st
MiB Mem :  15947.6 total,   1476.9 free,   7685.7 used,   6784.9 buff/cache
MiB Swap:  15259.0 total,  15238.7 free,     20.2 used.   7742.3 avail Mem

    PID USER      PR  NI    VIRT    RES    SHR S  %CPU  %MEM     TIME+ COMMAND
   2988 cleger    20   0 5184816   1.2g 430244 S  26.7   7.9  60:24.27 firefox-esr
   4326 cleger    20   0   16.4g 208104  81504 S  26.7   1.3   9:27.33 code
    909 root     -51   0       0      0      0 S  13.3   0.0  15:12.15 irq/104-nvidia
  41704 cleger    20   0   38.4g 373744 116984 S  13.3   2.3  13:25.76 code
  91926 cleger    20   0 2514784 145360  95144 S  13.3   0.9   1:29.85 Web Content
    \end{minted}
  \end{block}
\end{frame}

\begin{frame}[fragile]
  \frametitle{{\em free}}
  \begin{itemize}
    \item {\em free} is a simple program that displays the amount of free and
          used memory in the system (\manpage{free}{1}).
    \begin{itemize}
      \item Useful to check if the system suffers from memory exhaustion
      \item Uses \code{/proc/meminfo} to obtain memory information.
    \end{itemize}
  \end{itemize}
  \begin{block}{}
    \begin{minted}[fontsize=\footnotesize]{console}
$ free -h
               total        used        free      shared  buff/cache   available
Mem:            15Gi       7.5Gi       1.4Gi       192Mi       6.6Gi       7.5Gi
Swap:           14Gi        20Mi        14Gi
    \end{minted}
  \end{block}
\end{frame}

\begin{frame}[fragile]
  \frametitle{vmstat}
  \begin{itemize}
    \item {\em vmstat} displays information about system virtual memory usage
    \item Can also display stats from processes, memory, paging, block IO,
          traps, disks and cpu activity (\manpage{vmstat}{8}).
    \item Can be used to gather data at periodic interval using \code{vmstat <interval> <number>}
  \end{itemize}
  \begin{block}{}
    \begin{minted}[fontsize=\footnotesize]{console}
$ vmstat 1 6
procs -----------memory----------   ---swap--  -----io---- -system-- ------cpu-----
r  b   swpd   free   buff  cache     si   so    bi    bo    in   cs  us sy id wa st
3  0 253440 1237236 194936 9286980    3    6   186   540    134  157  3  5 82 10  0
    \end{minted}
  \end{block}
  \begin{itemize}
    \item {\em Note: vmstat consider a kernel block to be 1024 bytes}
  \end{itemize}
\end{frame}

\begin{frame}[fragile]
  \frametitle{mpstat}
  \begin{itemize}
    \item {\em mpstat} displays Multiprocessor statistics (\manpage{mpstat}{1}).
    \item Useful to detect unbalance CPU workloads, bad IRQ affinity, etc.
  \end{itemize}
  \begin{block}{}
    \begin{minted}[fontsize=\scriptsize]{console}
$ mpstat -P ALL 
Linux 6.0.0-1-amd64 (fixe)      19/10/2022      _x86_64_        (4 CPU)

17:02:50     CPU    %usr   %nice    %sys %iowait    %irq   %soft  %steal  %guest  %gnice   %idle
17:02:50     all    6,77    0,00    2,09   11,67    0,00    0,06    0,00    0,00    0,00   79,40
17:02:50       0    6,88    0,00    1,93    8,22    0,00    0,13    0,00    0,00    0,00   82,84
17:02:50       1    4,91    0,00    1,50    8,91    0,00    0,03    0,00    0,00    0,00   84,64
17:02:50       2    6,96    0,00    1,74    7,23    0,00    0,01    0,00    0,00    0,00   84,06
17:02:50       3    9,32    0,00    2,80   54,67    0,00    0,00    0,00    0,00    0,00   33,20
17:02:50       4    5,40    0,00    1,29    4,92    0,00    0,00    0,00    0,00    0,00   88,40
    \end{minted}
  \end{block}
\end{frame}


\begin{frame}[fragile]
  \frametitle{iostat}
  \begin{itemize}
    \item {\em iostat} displays information about IOs per device on the system.
    \item Useful to see if a device is overloaded by IOs.
  \end{itemize}
  \begin{block}{}
    \begin{minted}[fontsize=\footnotesize]{console}
$ iostat 
Linux 5.19.0-2-amd64 (fixe)     11/10/2022      _x86_64_        (12 CPU)

avg-cpu:  %user   %nice %system %iowait  %steal   %idle
           8,43    0,00    1,52    8,77    0,00   81,28

Device      tps  kB_read/s  kB_wrtn/s  kB_dscd/s  kB_read  kB_wrtn  kB_dscd
nvme0n1   55,89    1096,88     149,33       0,00  5117334   696668        0
sda        0,03       0,92       0,00       0,00     4308        0        0
sdb      104,42     274,55    2126,64       0,00  1280853  9921488        0
    \end{minted}
  \end{block}
\end{frame}

\begin{frame}[fragile]
  \frametitle{iotop}
  \begin{itemize}
    \item {\em iotop} displays information about IOs much like {\em top} for each
          process.
    \item Useful to find applications generating too much I/O traffic.
    \begin{itemize}
    \item Needs \kconfigval{CONFIG_TASKSTATS}{y},
          \kconfigval{CONFIG_TASK_DELAY_ACCT}{y} and
          \kconfigval{CONFIG_TASK_IO_ACCOUNTING}{y} to be enabled in the kernel.
    \end{itemize}
  \end{itemize}
  \begin{block}{}
    \begin{minted}[fontsize=\small]{console}
# iotop
Total DISK READ:        20.61 K/s | Total DISK WRITE:        51.52 K/s
Current DISK READ:      20.61 K/s | Current DISK WRITE:      24.04 K/s
    TID  PRIO  USER     DISK READ DISK WRITE>    COMMAND                                                                                                                                                                                                                        
    2629 be/4 cleger     20.61 K/s   44.65 K/s firefox-esr [Cache2 I/O]
    322 be/3 root        0.00 B/s    3.43 K/s [jbd2/nvme0n1p1-8]
  39055 be/4 cleger      0.00 B/s    3.43 K/s firefox-esr [DOMCacheThread]
      1 be/4 root        0.00 B/s    0.00 B/s init
      2 be/4 root        0.00 B/s    0.00 B/s [kthreadd]
      3 be/0 root        0.00 B/s    0.00 B/s [rcu_gp]
      4 be/0 root        0.00 B/s    0.00 B/s [rcu_par_gp]
      ...
    \end{minted}
  \end{block}
\end{frame}

\begin{frame}[fragile]
  \frametitle{pmap}
  \begin{itemize}
    \item \code{pmap} displays process mappings more easily than
          accessing \code{/proc/<pid>/maps} (\manpage{pmap}{1}).
  \end{itemize}
  \begin{block}{}
    \begin{minted}[fontsize=\tiny]{console}
# pmap 2002
2002:   /usr/bin/dbus-daemon --session --address=systemd: --nofork --nopidfile --systemd-activation --syslog-only
...
00007f3f958bb000     56K r---- libdbus-1.so.3.32.1
00007f3f958c9000    192K r-x-- libdbus-1.so.3.32.1
00007f3f958f9000     84K r---- libdbus-1.so.3.32.1
00007f3f9590e000      8K r---- libdbus-1.so.3.32.1
00007f3f95910000      4K rw--- libdbus-1.so.3.32.1
00007f3f95937000      8K rw---   [ anon ]
00007f3f95939000      8K r---- ld-linux-x86-64.so.2
00007f3f9593b000    152K r-x-- ld-linux-x86-64.so.2
00007f3f95961000     44K r---- ld-linux-x86-64.so.2
00007f3f9596c000      8K r---- ld-linux-x86-64.so.2
00007f3f9596e000      8K rw--- ld-linux-x86-64.so.2
00007ffe13857000    132K rw---   [ stack ]
00007ffe13934000     16K r----   [ anon ]
00007ffe13938000      8K r-x--   [ anon ]
 total            11088K
    \end{minted}
  \end{block}
\end{frame}

\setuplabframe
{System Status}
{
  Check what is running on a system and its load
  \begin{itemize}
    \item Observe processes and IOs
    \item Display memory mappings
    \item Monitor resources
  \end{itemize}
}
