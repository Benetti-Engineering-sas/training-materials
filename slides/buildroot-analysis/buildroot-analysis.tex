\setbeamerfont{block title}{size=\scriptsize}

\section{Analyzing the build}

\begin{frame}{Analyzing the build: available tools}
  \begin{itemize}
  \item Buildroot provides several useful tools to analyze the build:
    \begin{itemize}
    \item The {\bf licensing report}, covered in a previous section,
      which allows to analyze the list of packages and their licenses.
    \item The {\bf dependency graphing} tools
    \item The {\bf build time graphing} tools
    \item The {\bf filesystem size} tools
    \end{itemize}
  \end{itemize}
\end{frame}

\begin{frame}{Dependency graphing}
  \begin{itemize}
  \item Exploring the dependencies between packages is useful to
    understand
    \begin{itemize}
    \item why a particular package is being brought into the
      build
    \item if the build size and duration can be reduced
    \end{itemize}
  \item \code{make graph-depends} to generate a full dependency graph,
    which can be huge!
  \item \code{make <pkg>-graph-depends} to generate the dependency
    graph of a given package
  \item The graph is done according to the current Buildroot
    configuration.
  \item Resulting graphs in \code{$(O)/graphs/}
  \end{itemize}
\end{frame}

\begin{frame}{Dependency graph example}
  \begin{center}
    \includegraphics[height=0.8\textheight]{slides/buildroot-analysis/graph-depends.pdf}
  \end{center}
\end{frame}

\begin{frame}{Build time graphing}
  \begin{itemize}
  \item When the generated embedded Linux system grows bigger and
    bigger, the build time also increases.
  \item It is sometimes useful to analyze this build time, and see if
    certain packages are particularly problematic.
  \item Buildroot collects build duration data in the file
    \code{$(O)/build/build-time.log}
  \item \code{make graph-build} generates several graphs in
    \code{$(O)/graphs/}:
    \begin{itemize}
    \item \code{build.hist-build.pdf}, build time in build order
    \item \code{build.hist-duration.pdf}, build time by duration
    \item \code{build.hist-name.pdf}, build time by package name
    \item \code{build.pie-packages.pdf}, pie chart of the per-package
      build time
    \item \code{build.pie-steps.pdf}, pie chart of the per-step build
      time
    \end{itemize}
  \item Note: only works properly after a complete clean rebuild.
  \end{itemize}
\end{frame}

\begin{frame}{Build time graphing: example}
  \begin{center}
    \includegraphics[width=\textwidth]{slides/buildroot-analysis/build-hist-build.pdf}
  \end{center}
\end{frame}

\begin{frame}{Filesystem size graphing}
  \begin{itemize}
  \item In many embedded systems, storage resources are limited.
  \item For this reason, it is useful to be able to analyze the size
    of your root filesystem, and see which packages are consuming the
    biggest amount of space.
  \item Allows to focus the size optimizations on the relevant
    packages.
  \item Buildroot collects data about the size installed by each
    package.
  \item \code{make graph-size} produces:
    \begin{itemize}
    \item \code{file-size-stats.csv}, CSV with the raw data of the
      per-file size
    \item \code{package-size-stats.csv}, CSV with the raw data of the
      per-package size
    \item \code{graph-size.pdf}, pie chart of the per-package size
      consumption
    \end{itemize}
  \end{itemize}
\end{frame}

\begin{frame}{Filesystem size graphing: example}
  \begin{center}
    \includegraphics[height=0.8\textheight]{slides/buildroot-analysis/graph-size.pdf}
  \end{center}
\end{frame}

\end{frame}

