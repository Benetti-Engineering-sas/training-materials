\subsection{CPU DAI driver}

\begin{frame}{CPU DAI driver}
  \begin{itemize}
  \item The CPU DAI driver is now a component driver, like the codec
    ones.
  \item However, it is usually more complex as it need to handle IRQs
    and take care of pinmuxing, clocks and DMA.
  \item Also, the list of supported format and rates is usually very
    large.
  \end{itemize}
\end{frame}

\begin{frame}{DMA handling}
  \begin{itemize}
  \item When a DMA controller is available, handling DMA in ALSA is
    done almost completely in the core, through \code{dmaengine_pcm}.
  \item The DMA is simply registered using
    \kfunc{devm_snd_dmaengine_pcm_register}. This handles parsing the
    device tree if necessary.
  \item In the DAI driver probe callback, the DMA engine is simply
    configured using \kfunc{snd_soc_dai_init_dma_data} which takes the
    DMA configuration for playback and capture.
  \end{itemize}
\end{frame}

\begin{frame}[fragile]{DMA handling example}
  \begin{block}{\code{sound/soc/atmel/atmel-i2s.c}}
    \fontsize{8}{7}\selectfont
    \begin{minted}{c}
struct atmel_i2s_dev {
        struct device                           *dev;
        struct regmap                           *regmap;
        struct clk                              *pclk;
        struct clk                              *gclk;
        struct snd_dmaengine_dai_dma_data       playback;
        struct snd_dmaengine_dai_dma_data       capture;
        unsigned int                            fmt;
        const struct atmel_i2s_gck_param        *gck_param;
        const struct atmel_i2s_caps             *caps;
        int                                     clk_use_no;
};
[...]
static int atmel_i2s_dai_probe(struct snd_soc_dai *dai)
{
        struct atmel_i2s_dev *dev = snd_soc_dai_get_drvdata(dai);

        snd_soc_dai_init_dma_data(dai, &dev->playback, &dev->capture);
        return 0;
}
    \end{minted}
  \end{block}
\end{frame}

\begin{frame}[fragile]{DMA handling example}
  \begin{block}{\code{sound/soc/atmel/atmel-i2s.c}}
    \fontsize{8}{8}\selectfont
    \begin{minted}{c}
static int atmel_i2s_probe(struct platform_device *pdev)
{
[...]
        /* Prepare DMA config. */
        dev->playback.addr        = (dma_addr_t)mem->start + ATMEL_I2SC_THR;
        dev->playback.maxburst        = 1;
        dev->capture.addr        = (dma_addr_t)mem->start + ATMEL_I2SC_RHR;
        dev->capture.maxburst        = 1;

        if (of_property_match_string(np, "dma-names", "rx-tx") == 0)
                pcm_flags |= SND_DMAENGINE_PCM_FLAG_HALF_DUPLEX;
        err = devm_snd_dmaengine_pcm_register(&pdev->dev, NULL, pcm_flags);
        if (err) {
                dev_err(&pdev->dev, "failed to register PCM: %d\n", err);
                clk_disable_unprepare(dev->pclk);
                return err;
        }
[...]
}
    \end{minted}
  \end{block}
\end{frame}

\begin{frame}{DMA handling}
  \begin{itemize}
  \item When a peripheral DMA controller is used, this is more
    complex.
  \item The driver will have to handle all the aspects of the PCM
    stream life cycle.
  \item Understandable example in \kfile{sound/soc/atmel/atmel-pcm-pdc.c}
  \end{itemize}
\end{frame}


