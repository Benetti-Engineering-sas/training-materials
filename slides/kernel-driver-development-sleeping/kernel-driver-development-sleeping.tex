\subsection{Sleeping}

\begin{frame}
  \frametitle{Sleeping}
  \begin{center}
    \includegraphics[width=\textwidth]{slides/kernel-driver-development-sleeping/sleeping.pdf}\\
    Sleeping is needed when a process (user space or kernel space) is
    waiting for data.
  \end{center}
\end{frame}

\begin{frame}[fragile]
  \frametitle{How to sleep with a wait queue 1/3}
  \begin{itemize}
  \item Must declare a wait queue, which will be used to store the list of threads
        waiting for an event
  \item Dynamic queue declaration:
      \begin{itemize}
      \item Typically one queue per device managed by the driver
      \item It's convenient to embed the wait queue inside a per-device data
        structure.
      \item Example from \kfile{drivers/net/ethernet/marvell/mvmdio.c}:
\begin{minted}{c}
struct orion_mdio_dev {
        ...
        wait_queue_head_t smi_busy_wait;
};
struct orion_mdio_dev *dev;
...
init_waitqueue_head(&dev->smi_busy_wait);
\end{minted}
    \end{itemize}
    \item Static queue declaration:
      \begin{itemize}
      \item Using a global variable when a global resource is sufficient
      \item \mint{c}+DECLARE_WAIT_QUEUE_HEAD(module_queue);+
      \end{itemize}
  \end{itemize}
\end{frame}

\begin{frame}[fragile]
  \frametitle{How to sleep with a waitqueue 2/3}
  Several ways to make a kernel process sleep
  \begin{itemize}
  \item \mint{c}+void wait_event(queue, condition);+
    \begin{itemize}
    \item Sleeps until the task is woken up {\bf and} the given C
      expression is true. Caution: can't be interrupted (can't kill
      the user space process!)
    \end{itemize}
  \item \mint{c}+int wait_event_killable(queue, condition);+
    \begin{itemize}
    \item Can be interrupted, but only by a \emph{fatal} signal
      (\ksym{SIGKILL}). Returns \code{-}\ksym{ERESTARTSYS} if interrupted.
    \end{itemize}
  \item \mint{c}+int wait_event_interruptible(queue, condition);+
    \begin{itemize}
    \item The most common variant
    \item Can be interrupted by any signal. Returns
      \code{-}\ksym{ERESTARTSYS} if interrupted.
    \end{itemize}
  \end{itemize}
\end{frame}

\begin{frame}[fragile]
  \frametitle{How to sleep with a waitqueue 3/3}
  \begin{itemize}
  \item \mint{c}+int wait_event_timeout(queue, condition, timeout);+
    \begin{itemize}
    \item Also stops sleeping when the task is woken up {\bf or} the
      timeout expired (a timer is used).
    \item Returns \code{0} if the timeout elapsed, non-zero if
      the condition was met.
    \end{itemize}
  \item \begin{minted}{c}
int wait_event_interruptible_timeout(queue, condition, timeout);
  \end{minted}
    \begin{itemize}
    \item Same as above, interruptible.
    \item Returns \code{0} if the timeout
      elapsed, \code{-}\ksym{ERESTARTSYS} if interrupted, positive value if
      the condition was met.
    \end{itemize}
  \end{itemize}
\end{frame}

\begin{frame}[fragile]
\frametitle{How to sleep with a waitqueue - Example}
\begin{minted}{c}
sig = wait_event_interruptible(ibmvtpm->wq,
                               !ibmvtpm->tpm_processing_cmd);
if (sig)
        return -EINTR;
\end{minted}
From \kfile{drivers/char/tpm/tpm_ibmvtpm.c}
\end{frame}

\begin{frame}
  \frametitle{Waking up!}
  Typically done by interrupt handlers when data sleeping
  processes are waiting for become available.
  \begin{itemize}
  \item \code{wake_up(&queue);}
    \begin{itemize}
    \item Wakes up all processes in the wait queue
    \end{itemize}
  \item \code{wake_up_interruptible(&queue);}
    \begin{itemize}
    \item Wakes up all processes waiting in an interruptible sleep
      on the given queue
    \end{itemize}
  \end{itemize}
\end{frame}

\begin{frame}
  \frametitle{Exclusive vs. non-exclusive}
  \begin{itemize}
  \item \kfunc{wait_event_interruptible} puts a task in a
    non-exclusive wait.
    \begin{itemize}
    \item All non-exclusive tasks are woken up by \kfunc{wake_up} /
      \kfunc{wake_up_interruptible}
    \end{itemize}
  \item \kfunc{wait_event_interruptible_exclusive} puts a task in an
    exclusive wait.
    \begin{itemize}
    \item \kfunc{wake_up} / \kfunc{wake_up_interruptible} wakes up
      all non-exclusive tasks and only one exclusive task
    \item \kfunc{wake_up_all} / \kfunc{wake_up_interruptible_all}
      wakes up all non-exclusive and all exclusive tasks
    \end{itemize}
  \item Exclusive sleeps are useful to avoid waking up multiple tasks
    when only one will be able to ``consume'' the event.
  \item Non-exclusive sleeps are useful when the event can ``benefit''
    to multiple tasks.
  \end{itemize}
\end{frame}

\begin{frame}[fragile]
  \frametitle{Sleeping and waking up - Implementation}
  \begin{columns}
    \column{0.5\textwidth}
      \includegraphics[height=0.8\textheight]{slides/kernel-driver-development-sleeping/wait-event.pdf}\\
    \column{0.5\textwidth}
       The scheduler doesn't keep evaluating the sleeping condition!
       \begin{itemize}
       \item \mint{c}+wait_event(queue, cond);+
         The process is put in the \ksym{TASK_UNINTERRUPTIBLE} state.
       \item \mint{c}+wake_up(&queue);+
         All processes waiting in \code{queue} are woken up, so they get
         scheduled later and have the opportunity to evaluate the
         condition again and go back to sleep if it is not met.
       \end{itemize}
       See \kfile{include/linux/wait.h} for implementation details.
  \end{columns}
\end{frame}

\begin{frame}[fragile]
  \frametitle{How to sleep with completions 1/2}
  \begin{itemize}
  \item Use \kfunc{wait_for_completion} when no particular condition
    must be enforced at the time of the wake-up
    \begin{itemize}
    \item Leverages the power of wait queues
    \item Simplifies its use
    \item Highly efficient using low level scheduler facilities
    \end{itemize}
  \item Preparation of the completion structure:
    \begin{itemize}
    \item Static declaration and initialization:
      \mint{c}+DECLARE_COMPLETION(setup_done);+
    \item Dynamic declaration:
      \mint{c}+init_completion(&object->setup_done);+
    \item The completion object should get a meaningful name (eg. not
      just ``done'').
    \end{itemize}
  \item Ready to be used by signal consumers and providers as soon as
    the completion object is initialized
  \item See \kfile{include/linux/completion.h} for the full API
  \item Internal documentation at \kdochtml{scheduler/completion.rst}
  \end{itemize}
\end{frame}

\begin{frame}[fragile]
  \frametitle{How to sleep with completions 2/2}
  \begin{itemize}
  \item Enter a wait state with
    \mint{c}+void wait_for_completion(struct completion *done)+
    \begin{itemize}
    \item All \kfunc{wait_event} flavors are also supported, such as:
      \kfunc{wait_for_completion_timeout},
      \kfunc{wait_for_completion_interruptible\{,_timeout\}},
      \kfunc{wait_for_completion_killable\{,_timeout\}}, etc
    \end {itemize}
  \item Wake up consumers with
    \mint{c}+void complete(struct completion *done)+
    \begin{itemize}
    \item Several calls to \kfunc{complete} are valid, they will wake up
      the same number of threads waiting on this object (acts as a FIFO).
    \item A single \kfunc{complete_all} call would wake up all present and
      future threads waiting on this completion object
    \end {itemize}
  \item Reset the counter with
    \mint{c}+void reinit_completion(struct completion *done)+
    \begin{itemize}
    \item Resets the number of ``done'' completions still pending
    \item Mind not to call \kfunc{init_completion} twice, which could
      confuse the enqueued tasks
    \end{itemize}
  \end{itemize}
\end{frame}

\begin{frame}[fragile]
  \frametitle{Waiting when there is no interrupt}
  \begin{itemize}
  \item When there is no interrupt mechanism tied to a particular
        hardware state, it is tempting to implement a custom busy-wait loop.
    \begin{itemize}
    \item Spoiler alert: this is highly discouraged!
    \end{itemize}
  \item For long lasting pauses, rely on helpers which leverage the
    system clock
    \begin{itemize}
    \item \kfunc{wait_event} helpers are (also) very useful outside of
      interruption situations
    \item Release the CPU with \kfunc{schedule}
    \end{itemize}
  \item For shorter pauses, use helpers which implement software loops
    \begin{itemize}
    \item \kfunc{msleep}/\kfunc{msleep_interruptible} put the process
      in sleep for a given amount of milliseconds
    \item \kfunc{udelay}/\kfunc{udelay_range} waste CPU cycles in order
      to save a couple of context switches for a sub-millisecond period
    \item \kfunc{cpu_relax} does nothing, but may be used as a way to
      not being optimized out by the compiler when busy looping for very
      short periods
    \end{itemize}
  \end{itemize}
\end{frame}

\begin{frame}[fragile]
  \frametitle{Waiting when hardware is involved}
  \begin{itemize}
  \item When hardware is involved in the waiting process
    \begin{itemize}
    \item but there is no interrupt available
    \item or because a context switch would be too expensive
    \end{itemize}
  \item Specific polling I/O accessors may be used:
    \begin{itemize}
    \item Exhaustive list in \kfile{include/iopoll.h}
      \begin{minted}{c}
int read[bwlq]_poll[_timeout[_atomic]](addr, val, cond,
                                       delay_us, timeout_us)
      \end{minted}
        \begin{itemize}
        \item \code{addr}: I/O memory location
        \item \code{val}: Content of the register pointed with
        \item \code{cond}: Boolean condition based on \code{val}
        \item \code{delay_us}: Polling delay between reads
        \item \code{timeout_us}: Timeout delay after which the operation
          fails and returns -ETIMEDOUT
        \end{itemize}
        \item \code{_atomic} variant uses \kfunc{udelay} instead of \kfunc{usleep}.
    \end{itemize}
  \end{itemize}
\end{frame}
