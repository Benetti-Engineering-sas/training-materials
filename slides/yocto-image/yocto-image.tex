\section{Images}
\subsection{Introduction to images}

\begin{frame}
  \frametitle{Overview 1/3}
  \begin{itemize}
    \item An \code{image} is the top level recipe and is used
      alongside the \code{machine} definition.
    \item Whereas the \code{machine} describes the hardware used and
      its capabilities, the \code{image} is architecture agnostic and
      defines how the root filesystem is built, with what packages.
    \item By default, several images are provided in Poky:
      \begin{itemize}
        \item \code{meta*/recipes*/images/*.bb}
      \end{itemize}
  \end{itemize}
\end{frame}

\begin{frame}
  \frametitle{Overview 2/3}
  \begin{itemize}
    \item Common images are:
      \begin{description}
        \item[core-image-base] Console-only image, with full support
          of the hardware.
        \item[core-image-minimal] Small image, capable of booting a
          device.
        \item[core-image-minimal-dev] Small image with extra tools,
          suitable for development.
        \item[core-image-x11] Image with basic X11 support.
        \item[core-image-rt] \code{core-image-minimal} with real time
          tools and test suite.
      \end{description}
  \end{itemize}
\end{frame}

\begin{frame}
  \frametitle{Overview 3/3}
  \begin{itemize}
    \item An \code{image} is no more than a recipe.
    \item It has a description, a license and inherits the
      \code{core-image} class.
  \end{itemize}
\end{frame}

\begin{frame}
  \frametitle{Organization of an image recipe}
  \begin{itemize}
    \item Some special configuration variables are used to describe an
      image:
      \begin{description}
        \item[IMAGE\_BASENAME] The name of the output image files.
          Defaults to \code{${PN}}.
        \item[IMAGE\_INSTALL] List of packages and package groups to
          install in the generated image.
        \item[IMAGE\_ROOTFS\_SIZE] The final root filesystem size.
        \item[IMAGE\_FEATURES] List of features to enable in the
          image.
        \item[IMAGE\_FSTYPES] List of formats the OpenEmbedded build
          system will use to create images.
        \item[IMAGE\_LINGUAS] List of the locales to be supported in
          the image.
        \item[IMAGE\_PKGTYPE] Package type used by the build system.
          One of \code{deb}, \code{rpm}, \code{ipk} and \code{tar}.
        \item[IMAGE\_POSTPROCESS\_COMMAND] Shell commands to run at
          post process.
      \end{description}
  \end{itemize}
\end{frame}

\begin{frame}[fragile]
  \frametitle{Example of an image}
  \begin{block}{}
    \begin{minted}{sh}
require recipes-core/images/core-image-minimal.bb

DESCRIPTION = "Example image"

IMAGE_INSTALL += "ninvaders"

LICENSE = "MIT"
    \end{minted}
  \end{block}
\end{frame}

\subsection{Image types}

\begin{frame}
  \frametitle{\code{IMAGE_FSTYPES}}
  \begin{itemize}
    \item Configures the resulting root filesystem image format.
    \item If more than one format is specified, one image per format
      will be generated.
    \item Image formats instructions are delivered in Poky, thanks to
      \code{meta/classes/image_types.bbclass}
    \item Common image formats are: ext2, ext3, ext4, squashfs,
      squashfs-xz, cpio, jffs2, ubifs, tar.bz2, tar.gz\dots
  \end{itemize}
\end{frame}

\begin{frame}
  \frametitle{Creating an image type}
  \begin{itemize}
    \item If you have a particular layout on your storage (for example
      bootloader location on an SD card), you may want to create your
      own image type.
    \item This is done through a class that inherits from
      \code{image_types}.
    \item It has to define a function named \code{IMAGE_CMD_<type>}.
    \item Append it to \code{IMAGE_TYPES}
  \end{itemize}
\end{frame}

\begin{frame}
  \frametitle{Creating an image conversion type}
  \begin{itemize}
    \item Common conversion types are: gz, bz2, sha256sum, bmap\dots
    \item This is done through a class that inherits from
      \code{image_types}.
    \item It has to define a function named \code{CONVERSION_CMD_<type>}.
    \item Append it to \code{CONVERSIONTYPES}
    \item Append valid combinations to \code{IMAGE_TYPES}
  \end{itemize}
\end{frame}

\begin{frame}[fragile]
  \frametitle{wic}
  \begin{itemize}
    \item \code{wic} is a tool that can create a flashable image from
      the compiled packages and artifacts.
    \item It can create partitions.
    \item It can select which files are located in
      which partition through the use of plugins.
    \item The final image layout is described in a \code{.wks} or
      \code{.wks.in} file.
    \item It can be extended in any layer.
    \item Usage example:
      \begin{block}{}
        \begin{minted}{sh}
WKS_FILE = "imx-uboot-custom.wks.in"
IMAGE_FSTYPES = "wic.bmap wic"
        \end{minted}
      \end{block}
  \end{itemize}
\end{frame}

\begin{frame}[fragile]
  \frametitle{imx-uboot-custom.wks.in}
      \begin{block}{}
        \fontsize{7}{7}\selectfont
        \begin{minted}{sh}
part u-boot --source rawcopy --sourceparams="file=imx-boot" --ondisk sda --no-table --align ${IMX_BOOT_SEEK}
part /boot --source bootimg-partition --ondisk sda --fstype=vfat --label boot --active --align 8192 --size 64
part / --source rootfs --ondisk sda --fstype=ext4 --label root --exclude-path=home/ --exclude-path=opt/ --align 8192
part /home --source rootfs --rootfs-dir=${IMAGE_ROOTFS}/home --ondisk sda --fstype=ext4 --label home --align 8192
part /opt --source rootfs --rootfs-dir=${IMAGE_ROOTFS}/opt --ondisk sda --fstype=ext4 --label opt --align 8192

bootloader --ptable msdos
        \end{minted}
      \end{block}
  \begin{itemize}
  \item Copies \code{imx-boot} from \code{$DEPLOY_DIR} in the image,
    aligned on (and so at that offset) \code{${IMX_BOOT_SEEK}}.
  \item Creates a first partition, formatted in FAT32, with the files
    listed in the \code{IMAGE_BOOT_FILES} variable.
  \item Creates an \code{ext4} partition with the contents on the root
    filesystem, excluding the content of \code{/home} and \code{/opt}
  \item Creates two \code{ext4} partitions, one with the content of
    \code{/home}, the other one with the content of \code{/opt}, from
    the image root filesystem.
  \end{itemize}
\end{frame}

\subsection{Package groups}

\begin{frame}
  \frametitle{Overview}
  \begin{itemize}
    \item \code{Package groups} are a way to group packages by
      functionality or common purpose.
    \item Package groups are used in image recipes to help building
      the list of packages to install.
    \item They can be found under
      \code{meta*/recipes-core/packagegroups/}
    \item A package group is yet another recipe.
    \item The prefix \code{packagegroup-} is always used.
    \item Be careful about the \code{PACKAGE_ARCH} value:
      \begin{itemize}
      \item Set to the value \code{all} by default,
      \item Must be explicitly set to \code{${MACHINE_ARCH}} when there is a machine
        dependency.
      \end{itemize}
  \end{itemize}
\end{frame}

\begin{frame}
  \frametitle{Common package groups}
  \begin{itemize}
    \item packagegroup-core-boot
    \item packagegroup-core-buildessential
    \item packagegroup-core-nfs-client
    \item packagegroup-core-nfs-server
    \item packagegroup-core-tools-debug
    \item packagegroup-core-tools-profile
  \end{itemize}
\end{frame}

\begin{frame}[fragile]
  \frametitle{Example}
  \code{./meta/recipes-core/packagegroups/packagegroup-core-tools-debug.bb}:
  \begin{block}{}
    \begin{minted}{sh}
SUMMARY = "Debugging tools"
LICENSE = "MIT"

inherit packagegroup

RDEPENDS:${PN} = "\
    gdb \
    gdbserver \
    strace"
    \end{minted}
  \end{block}
\end{frame}
