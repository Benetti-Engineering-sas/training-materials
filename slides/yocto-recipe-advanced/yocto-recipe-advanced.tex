\section{Writing recipes - advanced}

\subsection{Extending a recipe}

\begin{frame}
  \frametitle{Introduction to recipe extensions}
  \begin{itemize}
    \item It is a good practice {\bf not} to modify recipes available
          in Poky.
    \item But it is sometimes useful to modify an existing recipe, to
          apply a custom patch for example.
    \item The BitBake \emph{build engine} allows to modify a recipe by
          extending it.
    \item Multiple extensions can be applied to a recipe.
  \end{itemize}
\end{frame}

\begin{frame}
  \frametitle{Introduction to recipe extensions}
  \begin{itemize}
    \item Metadata can be changed, added or appended.
    \item Tasks can be added or appended.
    \item Operators are used extensively, to add, append, prepend or
          assign values.
  \end{itemize}
\end{frame}

\begin{frame}{Extend a recipe}
  \begin{center}
    \includegraphics[width=\textwidth]{slides/yocto-recipe-advanced/yocto-recipe-advanced-extend.pdf}
  \end{center}
\end{frame}

\begin{frame}
  \frametitle{Extend a recipe}
  \begin{itemize}
    \item The recipe extensions end in \code{.bbappend}
    \item Append files must have the same root name as the recipe they
          extend.
    \begin{itemize}
      \item \code{example_0.1.bbappend} applies to
            \code{example_0.1.bb}
    \end{itemize}
    \item Append files are {\bf version specific}. If the recipe is
          updated to a newer version, the append files must also be
          updated.
    \item If adding new files, the path to their directory must
          be prepended to the \code{FILESEXTRAPATHS} variable.
    \begin{itemize}
      \item Files are looked up in paths referenced in
            \code{FILESEXTRAPATHS}, from left to right.
      \item Prepending a path makes sure it has priority over the recipe's
            one. This allows to override recipes' files.
    \end{itemize}
  \end{itemize}
\end{frame}

\subsection{Append file example}

\begin{frame}[fragile]
  \frametitle{Hello world append file}
  \begin{block}{}
    \begin{minted}{sh}
FILESEXTRAPATHS:prepend := "${THISDIR}/files:"

SRC_URI += "file://custom-modification-0.patch \
            file://custom-modification-1.patch \
            "
    \end{minted}
  \end{block}
\end{frame}

\subsection{Advanced recipe configuration}

\begin{frame}
  \frametitle{Advanced configuration}
  \begin{itemize}
    \item In the real word, more complex configurations are often needed
          because recipes may:
    \begin{itemize}
      \item Provide virtual packages
      \item Inherit generic functions from classes
    \end{itemize}
  \end{itemize}
\end{frame}

\begin{frame}
  \frametitle{Providing virtual packages}
  \begin{itemize}
    \item BitBake allows to use virtual names instead of the actual
          package name. We saw a use case with \emph{package
          variants}.
    \item The virtual name is specified through the \code{PROVIDES}
          variable.
    \item Several recipes can provide the same virtual name. Only one
          will be built and installed into the generated image.
    \item \code{PROVIDES = "virtual/kernel"}
  \end{itemize}
\end{frame}

\subsection{Classes}

\begin{frame}
  \frametitle{Introduction to classes}
  \begin{itemize}
    \item Classes provide an abstraction to common code, which can be
          re-used in multiple recipes.
    \item Common tasks do not have to be re-developed!
    \item Any metadata and task which can be put in a recipe can be
          used in a class.
    \item Classes extension is \code{.bbclass}
    \item Classes are located in the \code{classes} folder of a layer.
    \item Recipes can use this common code by inheriting a class:
    \begin{itemize}
      \item \code{inherit <class>}
    \end{itemize}
  \end{itemize}
\end{frame}

\begin{frame}
  \frametitle{Common classes}
  \begin{itemize}
    \item Common classes can be found in \code{meta/classes/}
    \begin{itemize}
      \item \code{base.bbclass}
      \item \code{kernel.bbclass}
      \item \code{autotools.bbclass}
      \item \code{autotools-brokensep.bbclass}
      \item \code{cmake.bbclass}
      \item \code{native.bbclass}
      \item \code{systemd.bbclass}
      \item \code{update-rc.d.bbclass}
      \item \code{useradd.bbclass}
      \item \dots
    \end{itemize}
  \end{itemize}
\end{frame}

\begin{frame}
  \frametitle{The base class}
  \begin{itemize}
    \item Every recipe inherits the base class automatically.
    \item Contains a set of basic common tasks to fetch, unpack or
          compile applications.
    \item Inherits other common classes, providing:
    \begin{itemize}
      \item Mirrors definitions: \code{DEBIAN_MIRROR},
            \code{GNU_MIRROR}, \code{KERNELORG_MIRROR}\dots
      \item The ability to filter patches by \code{SRC_URI}
      \item Some tasks: \code{clean}, \code{listtasks} or
            \code{fetch}.
    \end{itemize}
    \item Defines \code{oe_runmake}, using \code{EXTRA_OEMAKE} to use
      custom arguments.
  \end{itemize}
\end{frame}

\begin{frame}
  \frametitle{The kernel class}
  \begin{itemize}
    \item Used to build Linux kernels.
    \item Defines tasks to configure, compile and install a kernel and
          its modules.
    \item The kernel is divided into several packages: \code{kernel},
          \code{kernel-base}, \code{kernel-dev},
          \code{kernel-modules}\dots
    \item Automatically provides the virtual package
          \code{virtual/kernel}.
    \item Configuration variables are available:
    \begin{itemize}
      \item \code{KERNEL_IMAGETYPE}, defaults to \code{zImage}
      \item \code{KERNEL_EXTRA_ARGS}
      \item \code{INITRAMFS_IMAGE}
    \end{itemize}
  \end{itemize}
\end{frame}

\begin{frame}
  \frametitle{The autotools class}
  \begin{itemize}
    \item Defines tasks and metadata to handle applications using the
          autotools build system (autoconf, automake and libtool):
    \begin{itemize}
      \item \code{do_configure}: generates the configure script using
            \code{autoreconf} and loads it with standard arguments or
            cross-compilation.
      \item \code{do_compile}: runs \code{make}
      \item \code{do_install}: runs \code{make install}
    \end{itemize}
    \item Extra configuration parameters can be passed with
          \code{EXTRA_OECONF}.
    \item Compilation flags can be added thanks to the
          \code{EXTRA_OEMAKE} variable.
  \end{itemize}
\end{frame}

\begin{frame}[fragile]
  \frametitle{Example: use the autotools class}
  \begin{block}{}
    \begin{minted}[fontsize=\scriptsize]{sh}
DESCRIPTION = "Print a friendly, customizable greeting"
HOMEPAGE = "https://www.gnu.org/software/hello/"
PRIORITY = "optional"
SECTION = "examples"
LICENSE = "GPL-3.0-or-later"

SRC_URI = "${GNU_MIRROR}/hello/hello-${PV}.tar.gz"
SRC_URI[md5sum] = "67607d2616a0faaf5bc94c59dca7c3cb"
SRC_URI[sha256sum] = "ecbb7a2214196c57ff9340aa71458e1559abd38f6d8d169666846935df191ea7"
LIC_FILES_CHKSUM = "file://COPYING;md5=d32239bcb673463ab874e80d47fae504"

inherit autotools
    \end{minted}
  \end{block}
\end{frame}

\begin{frame}
  \frametitle{The useradd class}
  \begin{itemize}
    \item This class helps to add users to the resulting image.
    \item Adding custom users is required by many services to avoid
          running them as root.
    \item \code{USERADD_PACKAGES} must be defined when the
          \code{useradd} class is inherited. Defines the list of
          packages which needs the user.
    \item Users and groups will be created before the packages using
          it perform their \code{do_install}.
    \item At least one of the two following variables must be set:
    \begin{itemize}
      \item \code{USERADD_PARAM}: parameters to pass to
            \code{useradd}.
      \item \code{GROUPADD_PARAM}: parameters to pass to
            \code{groupadd}.
    \end{itemize}
  \end{itemize}
\end{frame}

\begin{frame}[fragile]
  \frametitle{Example: use the useradd class}
  \begin{block}{}
    \begin{minted}[fontsize=\scriptsize]{sh}
DESCRIPTION = "useradd class usage example"
PRIORITY = "optional"
SECTION = "examples"
LICENSE = "MIT"

SRC_URI = "file://file0"
LIC_FILES_CHKSUM = "file://${COREBASE}/meta/files/common-licenses/MIT;md5=0835ade698e0bc..."

inherit useradd

USERADD_PACKAGES = "${PN}"
USERADD_PARAM = "-u 1000 -d /home/user0 -s /bin/bash user0"

do_install() {
    install -m 644 file0 ${D}/home/user0/
    chown user0:user0 ${D}/home/user0/file0
}
    \end{minted}
  \end{block}
\end{frame}

\subsection{Binary packages}

\begin{frame}
  \frametitle{Specifics for binary packages}
  \begin{itemize}
    \item It is possible to install binaries into the generated root
      filesystem.
    \item Set the \code{LICENSE} to \code{CLOSED}.
    \item Use the \code{do_install} task to copy the binaries into the
      root file system.
  \end{itemize}
\end{frame}

\subsection{BitBake file inclusions}

\begin{frame}
  \frametitle{Locate files in the build system}
  \begin{itemize}
    \item Metadata can be shared using included files.
    \item \code{BitBake} uses the \code{BBPATH} to find the files to
      be included. It also looks into the current directory.
    \item Three keywords can be used to include files from recipes,
      classes or other configuration files:
      \begin{itemize}
        \item \code{inherit}
        \item \code{include}
        \item \code{require}
      \end{itemize}
  \end{itemize}
\end{frame}

\begin{frame}
  \frametitle{The \code{inherit} keyword}
  \begin{itemize}
    \item \code{inherit} can be used in recipes or classes, to inherit
      the functionalities of a class.
    \item To inherit the functionalities of the \code{kernel} class,
      use: \code{inherit kernel}
    \item \code{inherit} looks for files ending in \code{.bbclass}, in
      \code{classes} directories found in \code{BBPATH}.
    \item It is possible to include a class conditionally using a
      variable: \code{inherit ${FOO}}
  \end{itemize}
\end{frame}

\begin{frame}
  \frametitle{The \code{include} and \code{require} keywords}
  \begin{itemize}
    \item \code{include} and \code{require} can be used in all files,
      to insert the content of another file at that location.
    \item If the path specified on the \code{include} (or
      \code{require}) path is relative, BitBake will insert the first
      file found in \code{BBPATH}.
    \item \code{include} does not produce an error when a file cannot
      be found, whereas \code{require} raises a parsing error.
    \item To include a local file: \code{include ninvaders.inc}
    \item To include a file from another location (which could be
      in another layer): \code{include path/to/file.inc}
  \end{itemize}
\end{frame}

\subsection{Debugging recipes}

\begin{frame}[fragile]
  \frametitle{Debugging recipes}
  \begin{itemize}
    \item For each task, logs are available in the \code{temp}
      directory in the work folder of a recipe. This includes both
      the actual tasks code that ran and the output of the task.
    \item bitbake can dump the whole environment, including the
      variable values and how they were set:
      \begin{block}{}
        \begin{minted}[fontsize=\tiny]{console}
$ bitbake -e ninvaders
# $DEPENDS [4 operations]
#   set /yocto-labs/poky/meta/conf/bitbake.conf:268
#     ""
#   set /yocto-labs/poky/meta/conf/documentation.conf:130
#     [doc] "Lists a recipe's build-time dependencies (i.e. other recipe files)."
#   :prepend /yocto-training/yocto-labs/poky/meta/classes/base.bbclass:74
#     "${BASEDEPENDS} "
#   set /yocto-labs/meta-bootlinlabs/recipes-games/ninvaders/ninvaders.inc:11
#     "ncurses"
# pre-expansion value:
#   "${BASEDEPENDS} ncurses"
DEPENDS="virtual/arm-poky-linux-gnueabi-gcc virtual/arm-poky-linux-gnueabi-compilerlibs virtual/libc ncurses"
        \end{minted}
      \end{block}

  \end{itemize}
\end{frame}

\begin{frame}[fragile]
  \frametitle{Debugging recipes}
  \begin{itemize}
    \item A development shell, exporting the full environment can be
      used to debug build failures:
      \begin{block}{}
        \begin{minted}{console}
$ bitbake -c devshell <recipe>
        \end{minted}
      \end{block}
    \item To understand what a change in a recipe implies, you can
      activate build history in \code{local.conf}:
      \begin{block}{}
        \begin{minted}{sh}
INHERIT += "buildhistory"
BUILDHISTORY_COMMIT = "1"
        \end{minted}
      \end{block}
      Then use the \code{buildhistory-diff} tool to examine
      differences between two builds.
      \begin{itemize}
        \item \code{buildhistory-diff}
      \end{itemize}
  \end{itemize}
\end{frame}

\subsection{Network usage}

\begin{frame}
  \frametitle{Source fetching}
  \begin{itemize}
    \item BitBake will look for files to retrieve at the following
      locations, in order:
      \begin{enumerate}
        \item \code{DL_DIR} (the local download directory).
        \item The \code{PREMIRRORS} locations.
        \item The upstream source, as defined in \code{SRC_URI}.
        \item The \code{MIRRORS} locations.
      \end{enumerate}
    \item If all the mirrors fail, the build will fail.
  \end{itemize}
\end{frame}

\begin{frame}[fragile]
  \frametitle{Mirror configuration in Poky}
  \begin{block}{}
    \begin{minted}[fontsize=\scriptsize]{sh}
PREMIRRORS ??= "\
bzr://.*/.*   http://downloads.yoctoproject.org/mirror/sources/ \n \
cvs://.*/.*   http://downloads.yoctoproject.org/mirror/sources/ \n \
git://.*/.*   http://downloads.yoctoproject.org/mirror/sources/ \n \
hg://.*/.*    http://downloads.yoctoproject.org/mirror/sources/ \n \
osc://.*/.*   http://downloads.yoctoproject.org/mirror/sources/ \n \
p4://.*/.*    http://downloads.yoctoproject.org/mirror/sources/ \n \
svk://.*/.*   http://downloads.yoctoproject.org/mirror/sources/ \n \
svn://.*/.*   http://downloads.yoctoproject.org/mirror/sources/ \n"

MIRRORS =+ "\
ftp://.*/.*   http://downloads.yoctoproject.org/mirror/sources/ \n \
http://.*/.*  http://downloads.yoctoproject.org/mirror/sources/ \n \
https://.*/.* http://downloads.yoctoproject.org/mirror/sources/ \n"
    \end{minted}
  \end{block}
\end{frame}

\begin{frame}[fragile]
  \frametitle{Configuring the mirrors}
  \begin{itemize}
    \item It's possible to prepend custom mirrors, using the
      \code{PREMIRRORS} variable:
  \end{itemize}
  \begin{block}{}
    \begin{minted}{sh}
PREMIRRORS:prepend = "\
 git://.*/.* http://www.yoctoproject.org/sources/   \n \
 ftp://.*/.* http://www.yoctoproject.org/sources/   \n \
 http://.*/.* http://www.yoctoproject.org/sources/  \n \
 https://.*/.* http://www.yoctoproject.org/sources/ \n"
    \end{minted}
  \end{block}
  \begin{itemize}
    \item Another solution is to use the \code{own-mirrors} class:
  \end{itemize}
  \begin{block}{}
    \begin{minted}{sh}
INHERIT += "own-mirrors"
SOURCE_MIRROR_URL = "http://example.com/my-source-mirror"
    \end{minted}
  \end{block}
\end{frame}

\begin{frame}[fragile]
  \frametitle{Forbidding network access}
  \begin{itemize}
    \item You can use \code{BB_GENERATE_MIRROR_TARBALLS = "1"} to
      generate tarballs of the git repositories in \code{DL_DIR}
    \item You can also completely disable network access using
      \code{BB_NO_NETWORK = "1"}
    \item Or restrict BitBake to only download files from the
      \code{PREMIRRORS}, using \code{BB_FETCH_PREMIRRORONLY = "1"}
  \end{itemize}
\end{frame}
