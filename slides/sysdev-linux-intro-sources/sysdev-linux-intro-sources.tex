\subsection{Linux kernel sources}

\begin{frame}
  \frametitle{Location of official kernel sources}
  \begin{itemize}
  \item The mainline versions of the Linux kernel, as released by Linus
    Torvalds, are available at \url{https://kernel.org}
    \begin{itemize}
    \item These versions follow the development model of the kernel
    \item They may not contain the latest developments from a specific
      area yet
    \item A good pick for products development phase
    \end{itemize}
    \item The stable versions of the Linux kernel, as maintained by a
      maintainers group, are also available at \url{https://kernel.org}
    \begin{itemize}
    \item These versions do not bring new features compared to Linus'
      tree
    \item Only bug fixes and security fixes are pulled there
    \item Each version is stabilized during the development period of
      the next mainline kernel
    \item Certain versions can be maintained for much longer, 2$+$ years
    \item A good pick for products commercialization phase
    \end{itemize}
  \end{itemize}
\end{frame}

\begin{frame}
  \frametitle{Location of non-official kernel sources}
  \begin{itemize}
  \item Many chip vendors supply their own kernel sources
    \begin{itemize}
    \item Focusing on hardware support first
    \item Can have a very important delta with mainline Linux
    \item Sometimes they break support for other platforms/devices
      without caring
    \item Useful in early phases only when mainline hasn't caught up yet
      (many vendors invest in the mainline kernel at the same time)
    \item Suitable for PoC, not suitable for products on the long term
      as usually no updates are provided to these kernels
    \item Getting stuck with a deprecated system with broken software
      that cannot be udpated has a real cost in the end
    \end{itemize}
  \item Many kernel sub-communities maintain their own kernel, with
    usually newer but fewer stable features, only for cutting-edge
    development
    \begin{itemize}
    \item Architecture communities (ARM, MIPS, PowerPC, etc)
    \item Device drivers communities (I2C, SPI, USB, PCI, network, etc)
    \item Other communities (real-time, etc)
    \item Not suitable to be used in products
    \end{itemize}
  \end{itemize}
\end{frame}

\begin{frame}
  \frametitle{Getting Linux sources}
  \begin{itemize}
  \item The kernel sources are available from
    \url{https://kernel.org/pub/linux/kernel} as {\bf full tarballs}
    (complete kernel sources) and {\bf patches} (differences between
    two kernel versions).
  \item However, more and more people use the \code{git} version
    control system. Absolutely needed for kernel development!
    \begin{itemize}
    \item Fetch the entire kernel sources and history\\
      {\footnotesize
      \code{git clone
https://git.kernel.org/pub/scm/linux/kernel/git/torvalds/linux}}
    \item Create a branch that starts at a specific mainline version\\
      \code{git checkout -b <name-of-branch> v5.6}
    \item Web interface available at
      \url{https://git.kernel.org/cgit/linux/kernel/git/torvalds/linux.git/tree/}
    \item Read more about Git at \url{https://git-scm.com/}
    \end{itemize}
  \end{itemize}
\end{frame}

\begin{frame}
  \frametitle{Linux kernel size (1)}
  \begin{itemize}
  \item Linux 5.10.11 sources:\\
    \begin{itemize}
	\item 70,639 files (\code{git ls-files | wc -l})
	\item 29,746,102 lines (\code{git ls-files | xargs cat | wc -l})
	\item 962,810,769 bytes (\code{git ls-files | xargs cat | wc -c})
    \end{itemize}
  \item But a compressed Linux kernel just sizes a few megabytes.
  \item So, why are these sources so big?\\
    Because they include thousands of device drivers, many network
    protocols, support many architectures and filesystems...
  \item The Linux core (scheduler, memory management...) is pretty
    small!
  \end{itemize}
\end{frame}

\begin{frame}
  \frametitle{Linux kernel size (2)}
  As of kernel version 5.7 (in percentage of total number of lines).
  % Update the data by running utils/source-code-line-statistics
  % in the Linux kernel source directory
  \begin{columns}
    \column[t]{0.5\textwidth}
    \begin{itemize}
       \item \kdir{drivers}: 60.1\%
       \item \kdir{arch}: 12.9\%
       \item \kdir{fs}: 4.7\%
       \item \kdir{sound}: 4.2\%
       \item \kdir{net}: 4.0\%
       \item \kdir{include}: 3.6\%
       \item \kdir{tools}: 3.2\%
       \item \kdir{Documentation}: 3.2\%
       \item \kdir{kernel}: 1.3\%
    \end{itemize}
    \column[t]{0.5\textwidth}
    \begin{itemize}
       \item \kdir{lib}: 0.6\%
       \item \kdir{mm}: 0.5\%
       \item \kdir{scripts}: 0.4\%
       \item \kdir{crypto}: 0.4\%
       \item \kdir{security}: 0.3\%
       \item \kdir{block}: 0.2\%
       \item \kdir{samples}: 0.1\%
       \item \kdir{virt}: 0.1\%
       \item ...
    \end{itemize}
  \end{columns}
\end{frame}
