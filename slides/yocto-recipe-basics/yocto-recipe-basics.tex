\section{Writing recipes - basics}

\subsection{Recipes: overview}

\begin{frame}{Recipes}
  \begin{center}
    \includegraphics[width=\textwidth]{slides/yocto-recipe-basics/yocto-recipe-basics-overview.pdf}
  \end{center}
\end{frame}

\begin{frame}
  \frametitle{Basics}
  \begin{itemize}
    \item A recipe describes how to handle a given software component
      (application, library, \dots).
    \item It is a set of instructions to describe how to retrieve, patch,
      compile, install and generate binary packages.
    \item It also defines what build or runtime dependencies are
      required.
    \item Recipes are parsed by the BitBake build engine.
    \item The format of a recipe file name is
      \code{<application-name>_<version>.bb}
  \end{itemize}
\end{frame}

\begin{frame}
  \frametitle{Content of a recipe}
  \begin{itemize}
    \item A recipe contains configuration variables: name, license,
      dependencies, path to retrieve the source code\dots
    \item It also contains functions that can be run (fetch,
      configure, compile\dots) which are called {\bf tasks}.
    \item Tasks provide a set of actions to perform.
    \item Remember the \code{bitbake -c <task> <target>} command?
  \end{itemize}
\end{frame}

\begin{frame}
  \frametitle{Common variables}
  \begin{itemize}
    \item To make it easier to write a recipe, some variables are
      automatically available:
      \begin{description}
        \item[PN] package name, as specified in the recipe file name
        \item[BPN] \code{PN} with prefixes and suffixes removed such
          as \code{nativesdk-}, or \code{-native}
        \item[PV] package version, as specified in the recipe file
          name
        \item[BP] defined as \code{${BPN}-${PV}}
      \end{description}
    \item The recipe name and version usually match the upstream ones.
    \item When using the recipe \code{bash_5.1.bb}:
      \begin{itemize}
        \item \code{${PN} = "bash"}
        \item \code{${PV} = "5.1"}
      \end{itemize}
  \end{itemize}
\end{frame}

\subsection{Organization of a recipe}

\begin{frame}{Organization of a recipe}
  \begin{center}
    \includegraphics[width=\textwidth]{slides/yocto-recipe-basics/yocto-recipe-basics-organisation.pdf}
  \end{center}
\end{frame}

\begin{frame}
  \frametitle{Organization of a recipe}
  \begin{itemize}
    \item Many applications have more than one recipe, to support
      different versions. In that case the common metadata is
      included in each version specific recipe and is in a \code{.inc}
      file:
      \begin{itemize}
        \item \code{<application>.inc}
          \begin{itemize}
            \item version agnostic metadata
          \end{itemize}
        \item \code{<application>_<version>.bb}
          \begin{itemize}
            \item \code{require <application>.inc}
            \item any version specific metadata
          \end{itemize}
      \end{itemize}
    \item We can divide a recipe into three main parts:
      \begin{itemize}
        \item The header: what/who
        \item The sources: where
        \item The tasks: how
      \end{itemize}
  \end{itemize}
\end{frame}

\begin{frame}
  \frametitle{The header}
  \begin{itemize}
    \item Configuration variables to describe the application:
      \begin{description}
        \item[DESCRIPTION] describes what the software is about
        \item[HOMEPAGE] URL to the project's homepage
        \item[PRIORITY] defaults to \code{optional}
        \item[SECTION] package category (e.g. \code{console/utils})
        \item[LICENSE] the application's license, using SPDX identifiers
          (\url{https://spdx.org/licenses/})
      \end{description}
  \end{itemize}
\end{frame}

\begin{frame}
  \frametitle{The source locations: overview}
  \begin{itemize}
    \item We need to retrieve both the raw sources from an official
      location and the resources needed to configure, patch or install
      the application.
    \item \code{SRC_URI} defines where and how to retrieve the
      needed elements. It is a set of URI schemes pointing to the
      resource locations (local or remote).
    \item URI scheme syntax: \code{scheme://url;param1;param2}
    \item \code{scheme} can describe a local file using \code{file://}
      or remote locations with \code{https://}, \code{git://},
      \code{svn://}, \code{hg://}, \code{ftp://}\dots
    \item By default, sources are fetched in
      \code{$BUILDDIR/downloads}. Change it with the \code{DL_DIR}
      variable in \code{conf/local.conf}
  \end{itemize}
\end{frame}

\begin{frame}
  \frametitle{The source locations: remote files 1/2}
  \begin{itemize}
    \item The \code{git} scheme:
      \begin{itemize}
        \item \code{git://<url>;protocol=<protocol>;branch=<branch>}
        \item When using git, it is necessary to also define
          \code{SRCREV}. It has to be a commit hash and not a tag to
          be able to do offline builds. The \code{branch} parameter is
          mandatory unless \code{nobranch=1} is used.
      \end{itemize}
    \item The \code{http}, \code{https} and \code{ftp} schemes:
      \begin{itemize}
        \item \code{https://example.com/application-1.0.tar.bz2}
        \item A few variables are available to help pointing to remote
          locations: \code{${SOURCEFORGE_MIRROR}},
          \code{${GNU_MIRROR}}, \code{${KERNELORG_MIRROR}}\dots
        \item Example:
          \code{${SOURCEFORGE_MIRROR}/<project-name>/${BPN}-${PV}.tar.gz}
        \item See \code{meta/conf/bitbake.conf}
      \end{itemize}
  \end{itemize}
\end{frame}

\begin{frame}[fragile]
  \frametitle{The source locations: remote files 2/2}
  \begin{itemize}
    \item An md5 or an sha256 sum must be provided when the protocol
      used to retrieve the file(s) does not guarantee their integrity.
      This is the case for \code{https}, \code{http} or \code{ftp}.
  \end{itemize}
  \begin{block}{}
    \begin{minted}{sh}
SRC_URI[md5sum] = "97b2c3fb082241ab5c56ab728522622b"
SRC_URI[sha256sum] = "..."
    \end{minted}
  \end{block}
  \begin{itemize}
    \item It's possible to use checksums for more than one file, using
      the \code{name} parameter:
  \end{itemize}
  \begin{block}{}
    \begin{minted}{sh}
SRC_URI = "http://example.com/src.tar.bz2;name=tarball \
           http://example.com/fixes.patch;name=patch"

SRC_URI[tarball.md5sum] = "97b2c3fb082241ab5c56..."
SRC_URI[patch.md5sum]   = "b184acf9eb39df794ffd..."
    \end{minted}
  \end{block}
\end{frame}

\begin{frame}[fragile]
  \frametitle{The source locations: local files}
  \begin{itemize}
    \item All local files found in \code{SRC_URI} are copied into the
      recipe's working directory, in \code{$BUILDDIR/tmp/work/}.
    \item The searched paths are defined in the \code{FILESPATH}
      variable.
  \end{itemize}
  \begin{block}{}
  \begin{minted}{sh}
FILESPATH = "${@base_set_filespath(["${FILE_DIRNAME}/${BP}",
                "${FILE_DIRNAME}/${BPN}","${FILE_DIRNAME}/files"], d)}

FILESOVERRIDES = "${TRANSLATED_TARGET_ARCH}:
                  ${MACHINEOVERRIDES}:${DISTROOVERRIDES}"
  \end{minted}
  \end{block}
  \begin{itemize}
    \item The \code{base_set_filespath(path)} function uses its
      \code{path} parameter,  \code{FILESEXTRAPATHS} and
      \code{FILESOVERRIDES} to fill the \code{FILESPATH} variable.
    \item Custom paths and files can be added using
      \code{FILESEXTRAPATHS} and \code{FILESOVERRIDES}.
    \item Prepend the paths, as the order matters.
  \end{itemize}
\end{frame}

\begin{frame}
  \frametitle{The source locations: tarballs}
  \begin{itemize}
    \item When extracting a tarball, BitBake expects to find the
      extracted files in a directory named
      \code{<application>-<version>}. This is controlled by the
      \code{S} variable. If the directory has another name, you must
      explicitly define \code{S}.
    \item If the scheme is \code{git}, \code{S} must be set to
      \code{${WORKDIR}/git}
  \end{itemize}
\end{frame}

\begin{frame}[fragile]
  \frametitle{The source locations: license files}
  \begin{itemize}
    \item License files must have their own checksum.
    \item \code{LIC_FILES_CHKSUM} defines the URI pointing to the
      license file in the source code as well as its checksum.
  \end{itemize}
  \begin{block}{}
    \begin{minted}{sh}
LIC_FILES_CHKSUM = "file://gpl.txt;md5=393a5ca..."
LIC_FILES_CHKSUM =  \
    "file://main.c;beginline=3;endline=21;md5=58e..."
LIC_FILES_CHKSUM =  \
    "file://${COMMON_LICENSE_DIR}/MIT;md5=083..."
    \end{minted}
  \end{block}
  \begin{itemize}
    \item This allows to track any license update: if the license
      changes, the build will trigger a failure as the checksum won't
      be valid anymore.
  \end{itemize}
\end{frame}

\begin{frame}
  \frametitle{Dependencies 1/2}
  \begin{itemize}
    \item A recipe can have dependencies during the build or at
          runtime. To reflect these requirements in the recipe, two
          variables are used:
    \begin{description}
      \item[DEPENDS] List of the recipe build-time dependencies.
      \item[RDEPENDS] List of the package runtime
        dependencies. Must be package specific (e.g. with
        \code{:${PN}}).
    \end{description}
    \item \code{DEPENDS = "recipe-b"}: the local \code{do_prepare_recipe_sysroot}
      task depends on the \code{do_populate_sysroot} task of recipe-b.
    \item \code{RDEPENDS:${PN} = "package-b"}: the local
      \code{do_build} task depends on the
      \code{do_package_write_<archive-format>} task of recipe b.
  \end{itemize}
\end{frame}

\begin{frame}
  \frametitle{Dependencies 2/2}
  \begin{itemize}
    \item Sometimes a recipe have dependencies on specific versions
      of another recipe.
    \item BitBake allows to reflect this by using:
    \begin{itemize}
      \item \code{DEPENDS = "recipe-b (>= 1.2)"}
      \item \code{RDEPENDS:${PN} = "recipe-b (>= 1.2)"}
    \end{itemize}
    \item The following operators are supported: \code{=}, \code{>},
      \code{<}, \code{>=} and \code{<=}.
    \item A graphical tool can be used to explore dependencies or
      reverse dependencies:
    \begin{itemize}
      \item \code{bitbake -g -u taskexp core-image-minimal}
    \end{itemize}
  \end{itemize}
\end{frame}

\begin{frame}[fragile]
  \frametitle{Tasks}
  Default tasks already exists, they are defined in classes:
  \begin{itemize}
    \item do\_fetch
    \item do\_unpack
    \item do\_patch
    \item do\_configure
    \item do\_compile
    \item do\_install
    \item do\_package
    \item do\_rootfs
  \end{itemize}
  You can get a list of existing tasks for a recipe with:
  \code{bitbake <recipe> -c listtasks}
\end{frame}

\begin{frame}[fragile]
  \frametitle{Writing tasks 1/2}
  \begin{itemize}
    \item Functions use the sh shell syntax, with available
      OpenEmbedded variables and internal functions available.
      \begin{description}
        \item[WORKDIR] the recipe's working directory
        \item[S] The directory where the source code is extracted
        \item[B] The directory where bitbake places the objects
          generated during the build
        \item[D] The destination directory (root directory of where
          the files are installed, before creating the image).
      \end{description}
    \item Syntax of a task:
  \end{itemize}
  \begin{block}{}
    \begin{minted}{sh}
do_task() {
    action0
    action1
    ...
}
    \end{minted}
  \end{block}
\end{frame}

\begin{frame}[fragile]
  \frametitle{Writing tasks 2/2}
  \begin{itemize}
    \item Example:
  \end{itemize}
  \begin{block}{}
    \begin{minted}[fontsize=\small]{sh}
do_compile() {
    oe_runmake
}

do_install() {
    install -d ${D}${bindir}
    install -m 0755 hello ${D}${bindir}
}
    \end{minted}
  \end{block}
\end{frame}

\begin{frame}{The main tasks}
  \begin{center}
    \includegraphics[width=\textwidth]{slides/yocto-recipe-basics/tasks-basics.pdf}
  \end{center}
\end{frame}

\begin{frame}[fragile]
  \frametitle{Modifying existing tasks}
  Tasks can be extended with \code{:prepend} or \code{:append}
  \begin{block}{}
    \begin{minted}{sh}
do_install:append() {
    install -d ${D}${sysconfdir}
    install -m 0644 hello.conf ${D}${sysconfdir}
}
    \end{minted}
  \end{block}
\end{frame}

\begin{frame}[fragile]
  \frametitle{Adding new tasks}
  Tasks can be added with \code{addtask}
  \begin{block}{}
    \begin{minted}{sh}
do_mkimage () {
    uboot-mkimage ...
}

addtask do_mkimage after do_compile before do_install
    \end{minted}
  \end{block}
\end{frame}

\subsection{Applying patches}

\begin{frame}
  \frametitle{Patches use cases}
  Patches can be applied to resolve build-system problematics:
  \begin{itemize}
    \item To support old versions of a software: bug and security
      fixes.
    \item To fix cross-compilation issues.
      \begin{itemize}
        \item In certain simple cases the \code{-e} option of
          \code{make} can be used.
        \item The \code{-e} option gives variables taken from the
          environment precedence over variables from \code{Makefiles}.
        \item Helps when an upstream \code{Makefile} uses hardcoded
          \code{CC} and/or \code{CFLAGS}.
      \end{itemize}
    \item To apply patches before they get their way into the upstream
      version.
  \end{itemize}
\end{frame}

\begin{frame}[fragile]
  \frametitle{The source locations: patches}
  \begin{itemize}
    \item Files ending in \code{.patch}, \code{.diff} or having the
      \code{apply=yes} parameter will be applied after the sources are
      retrieved and extracted, during the \code{do_patch} task.
  \end{itemize}
  \begin{block}{}
    \begin{minted}{sh}
SRC_URI += "file://joystick-support.patch \
            file://smp-fixes.diff \
           "
    \end{minted}
  \end{block}
  \begin{itemize}
    \item Patches are applied in the order they are listed in
      \code{SRC_URI}.
    \item It is possible to select which tool will be used to apply
      the patches listed in \code{SRC_URI} variable with
      \code{PATCHTOOL}.
    \item By default, \code{PATCHTOOL = 'quilt'} in Poky.
    \item Possible values: \code{git}, \code{patch} and \code{quilt}.
  \end{itemize}
\end{frame}

\begin{frame}
  \frametitle{Resolving conflicts}
  \begin{itemize}
    \item The \code{PATCHRESOLVE} variable defines how to handle
    conflicts when applying patches.
    \item It has two valid values:
      \begin{itemize}
        \item \code{noop}: the build fails if a patch cannot be
          successfully applied.
        \item \code{user}: a shell is launched to resolve manually the
          conflicts.
      \end{itemize}
    \item By default, \code{PATCHRESOLVE = "noop"} in
      \code{meta-poky}.
  \end{itemize}
\end{frame}

\subsection{Example of a recipe}

\begin{frame}[fragile]
  \frametitle{Hello world recipe}
  \begin{block}{}
    \begin{minted}[fontsize=\scriptsize]{sh}
DESCRIPTION = "Hello world program"
HOMEPAGE = "http://example.net/hello/"
PRIORITY = "optional"
SECTION = "examples"
LICENSE = "GPL-2.0-or-later"

SRC_URI = "git://git.example.com/hello;protocol=https;branch=master"
SRCREV = "2d47b4eb66e705458a17622c2e09367300a7b118"
S = "${WORKDIR}/git"
LIC_FILES_CHKSUM = "file://hello.c;beginline=3;endline=21;md5=58e..."

do_compile() {
    oe_runmake
}
do_install() {
    install -d ${D}${bindir}
    install -m 0755 hello ${D}${bindir}
}
    \end{minted}
  \end{block}
\end{frame}

\subsection{Example of a recipe with a version agnostic part}

\begin{frame}[fragile]
  \frametitle{tar.inc}
  \begin{block}{}
    \begin{minted}{sh}
SUMMARY = "GNU file archiving program"
HOMEPAGE = "https://www.gnu.org/software/tar/"
SECTION = "base"

SRC_URI = "${GNU_MIRROR}/tar/tar-${PV}.tar.bz2"

do_configure() { ... }

do_compile() { ... }

do_install() { ... }
    \end{minted}
  \end{block}
\end{frame}

\begin{frame}[fragile]
  \frametitle{tar\_1.17.bb}
  \begin{block}{}
    \begin{minted}[fontsize=\small]{sh}
require tar.inc

LICENSE = "GPL-2.0-only"
LIC_FILES_CHKSUM =  \
  "file://COPYING;md5=59530bdf33659b29e73d4adb9f9f6552"

SRC_URI += "file://avoid_heap_overflow.patch"

SRC_URI[md5sum] = "c6c4f1c075dbf0f75c29737faa58f290"
    \end{minted}
  \end{block}
\end{frame}

\begin{frame}[fragile]
  \frametitle{tar\_1.26.bb}
  \begin{block}{}
    \begin{minted}[fontsize=\small]{sh}
require tar.inc

LICENSE = "GPL-3.0-only"
LIC_FILES_CHKSUM =  \
  "file://COPYING;md5=d32239bcb673463ab874e80d47fae504"

SRC_URI[md5sum] = "2cee42a2ff4f1cd4f9298eeeb2264519"
    \end{minted}
  \end{block}
\end{frame}
