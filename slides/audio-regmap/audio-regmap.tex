\subsection{regmap}

\begin{frame}[fragile]{regmap}
  \begin{itemize}
    \item has its roots in ASoC (ALSA)
    \item can use I2C, SPI and MMIO (also SPMI)
    \item actually abstracts the underlying bus
    \item can handle locking when necessary
    \item can cache registers
    \item can handle endianness conversion
    \item can handle IRQ chips and IRQs
    \item can check register ranges
    \item handles read only, write only, volatile, precious registers
    \item handles register pages
    \item API is defined in \kfile{include/linux/regmap.h}
    \item implemented in \kfile{drivers/base/regmap/}
  \end{itemize}
\end{frame}

\begin{frame}[fragile]{regmap: creation}
  \begin{itemize}
    \item 
      \begin{block}{}
        \begin{minted}{c}
#define regmap_init(dev, bus, bus_context, config)                        \
        __regmap_lockdep_wrapper(__regmap_init, #config,                  \
                                dev, bus, bus_context, config)
        \end{minted}
      \end{block}
    \item 
      \begin{block}{}
        \begin{minted}{c}
#define regmap_init_i2c(i2c, config)                                        \
        __regmap_lockdep_wrapper(__regmap_init_i2c, #config,                \
                                i2c, config)
        \end{minted}
      \end{block}
    \item 
      \begin{block}{}
        \begin{minted}{c}
#define regmap_init_spi(dev, config)                                        \
        __regmap_lockdep_wrapper(__regmap_init_spi, #config,                \
                                dev, config)
        \end{minted}
      \end{block}
  \end{itemize}
  \begin{itemize}
    \item Also \code{devm_} versions
    \item and \code{_clk} versions, preparing, enabling and disabling
      clocks when necessary
  \end{itemize}
\end{frame}

\begin{frame}[fragile]{regmap: config}
  \begin{block}{\kfile{include/linux/regmap.h}}
    \fontsize{9}{9}\selectfont
    \begin{minted}{c}
struct regmap_config {
[...]
        int reg_bits;
        int reg_stride;
[...]
        bool (*writeable_reg)(struct device *dev, unsigned int reg);
        bool (*readable_reg)(struct device *dev, unsigned int reg);
        bool (*volatile_reg)(struct device *dev, unsigned int reg);
        bool (*precious_reg)(struct device *dev, unsigned int reg);
[...]
        int (*reg_read)(void *context, unsigned int reg, unsigned int *val);
        int (*reg_write)(void *context, unsigned int reg, unsigned int val);
        int (*reg_update_bits)(void *context, unsigned int reg,
                               unsigned int mask, unsigned int val);

[...]
        const struct reg_default *reg_defaults;
        unsigned int num_reg_defaults;
[...]
};
    \end{minted}
  \end{block}
\end{frame}

\begin{frame}[fragile]{regmap: config}
  \begin{itemize}
  \item \code{reg_bits} Number of bits in a register address, mandatory.
  \item \code{reg_stride} The register address stride. Valid register
    addresses are a multiple of this value. If set to 0, a value of 1
    will be used.
  \item \code{writeable_reg}, \code{readable_reg},
    \code{volatile_reg}, \code{precious_reg}: Optional callbacks
    returning true if the register is writeable, readable, volatile or
    precious. volatile registers won't be cached. precious registers
    will not be read unless the driver explicitly calls a read
    function. There are also tables in the \code{struct regmap_config}
    for the same purpose.
  \item \code{reg_read}, \code{reg_write}, \code{reg_update_bits}:
    Optional callbacks that if filled will be used to perform
    accesses. \code{reg_update_bits} should only be provided is
    specific locking is required.
  \item \code{reg_defaults}: Power on reset values for registers (for
    use with register cache support).
  \item \code{num_reg_defaults}: Number of elements in
    \code{reg_defaults}.
  \end{itemize}
\end{frame}

\begin{frame}[fragile]{regmap: access}
  \begin{itemize}
    \fontsize{9}{9}\selectfont
    \item 
      \begin{block}{}
        \begin{minted}{c}
int regmap_read(struct regmap *map, unsigned int reg, unsigned int *val);
        \end{minted}
      \end{block}
    \item 
      \begin{block}{}
        \begin{minted}{c}
int regmap_write(struct regmap *map, unsigned int reg, unsigned int val);
        \end{minted}
      \end{block}
    \item 
      \begin{block}{}
        \begin{minted}{c}
static inline int regmap_update_bits(struct regmap *map, unsigned int reg,
                                     unsigned int mask, unsigned int val)
        \end{minted}
      \end{block}
    \item 
      \begin{block}{}
        \begin{minted}{c}
#define regmap_read_poll_timeout(map, addr, val, cond, sleep_us, timeout_us)
        \end{minted}
      \end{block}
    \item 
      \begin{block}{}
        \begin{minted}{c}
int regmap_test_bits(struct regmap *map, unsigned int reg, unsigned int bits);
        \end{minted}
      \end{block}
    \item 
      \begin{block}{}
        \begin{minted}{c}
static inline int regmap_update_bits_check(struct regmap *map, unsigned int reg,
                                           unsigned int mask, unsigned int val,
                                           bool *change)
        \end{minted}
      \end{block}
  \end{itemize}
\end{frame}

\begin{frame}[fragile]{regmap: cache management}
  \begin{itemize}
    \item
      \begin{block}{}
        \begin{minted}{c}
int regcache_sync(struct regmap *map);
        \end{minted}
      \end{block}
    \item 
      \begin{block}{}
        \begin{minted}{c}
int regcache_sync_region(struct regmap *map, unsigned int min,
                         unsigned int max);
        \end{minted}
      \end{block}
    \item 
      \begin{block}{}
        \begin{minted}{c}
int regcache_drop_region(struct regmap *map, unsigned int min,
                         unsigned int max);
        \end{minted}
      \end{block}
    \item 
      \begin{block}{}
        \begin{minted}{c}
void regcache_cache_only(struct regmap *map, bool enable);
        \end{minted}
      \end{block}
    \item 
      \begin{block}{}
        \begin{minted}{c}
void regcache_cache_bypass(struct regmap *map, bool enable);
        \end{minted}
      \end{block}
    \item 
      \begin{block}{}
        \begin{minted}{c}
void regcache_mark_dirty(struct regmap *map);
        \end{minted}
      \end{block}
  \end{itemize}
\end{frame}

\begin{frame}[fragile]{regmap: example}
  \begin{block}{\kfile{sound/soc/codecs/max9877.c}}
  \fontsize{8}{8}\selectfont
    \begin{minted}{c}
static const struct regmap_config max9877_regmap = {
        .reg_bits = 8,
        .val_bits = 8,

        .reg_defaults = max9877_regs,
        .num_reg_defaults = ARRAY_SIZE(max9877_regs),
        .cache_type = REGCACHE_RBTREE,
};

static int max9877_i2c_probe(struct i2c_client *client)
{
        struct regmap *regmap;
        int i;

        regmap = devm_regmap_init_i2c(client, &max9877_regmap);
        if (IS_ERR(regmap))
                return PTR_ERR(regmap);

        /* Ensure the device is in reset state */
        for (i = 0; i < ARRAY_SIZE(max9877_regs); i++)
                regmap_write(regmap, max9877_regs[i].reg, max9877_regs[i].def);

        return devm_snd_soc_register_component(&client->dev,
                        &max9877_component_driver, NULL, 0);
}
    \end{minted}
  \end{block}
\end{frame}

\begin{frame}[fragile]{regmap: i2c and spi device example}
  \begin{block}{\kfile{sound/soc/codecs/adau1372.c}}
  \fontsize{8}{8}\selectfont
    \begin{minted}{c}
const struct regmap_config adau1372_regmap_config = {
        .val_bits = 8,
        .reg_bits = 16,
        .max_register = 0x4d,

        .reg_defaults = adau1372_reg_defaults,
        .num_reg_defaults = ARRAY_SIZE(adau1372_reg_defaults),
        .volatile_reg = adau1372_volatile_register,
        .cache_type = REGCACHE_RBTREE,
};
EXPORT_SYMBOL_GPL(adau1372_regmap_config);
    \end{minted}
  \end{block}
  \begin{block}{\kfile{sound/soc/codecs/adau1372-i2c.c}}
  \fontsize{8}{8}\selectfont
    \begin{minted}{c}
static int adau1372_i2c_probe(struct i2c_client *client)
{
        return adau1372_probe(&client->dev,
                devm_regmap_init_i2c(client, &adau1372_regmap_config), NULL);
}
    \end{minted}
  \end{block}
\end{frame}

\begin{frame}[fragile]{regmap: i2c and spi device example}
  \begin{block}{\kfile{sound/soc/codecs/adau1372-spi.c}}
  \fontsize{8}{8}\selectfont
    \begin{minted}{c}
static int adau1372_spi_probe(struct spi_device *spi)
{
        struct regmap_config config;

        config = adau1372_regmap_config;
        config.read_flag_mask = 0x1;

        return adau1372_probe(&spi->dev,
                devm_regmap_init_spi(spi, &config), adau1372_spi_switch_mode);
}
    \end{minted}
  \end{block}
\end{frame}

\begin{frame}[fragile]{regmap: ASoC components}
  \begin{itemize}
  \item \code{snd_soc_component} regmap accessors also exist, they are
    available either implicitly as the component core calls
    \code{dev_get_regmap(component->dev, NULL)} to retrieve or create
    a regmap for the device or explicitly by calling
    \code{snd_soc_component_init_regmap()}
  \end{itemize}
  \begin{block}{\kfile{include/sound/soc-component.h}}
  \fontsize{9}{9}\selectfont
    \begin{minted}{c}
/* component IO */
unsigned int snd_soc_component_read(struct snd_soc_component *component,
                                      unsigned int reg);
int snd_soc_component_write(struct snd_soc_component *component,
                            unsigned int reg, unsigned int val);
int snd_soc_component_update_bits(struct snd_soc_component *component,
                                  unsigned int reg, unsigned int mask,
                                  unsigned int val);
[...]
int snd_soc_component_test_bits(struct snd_soc_component *component,
                                unsigned int reg, unsigned int mask,
                                unsigned int value);
    \end{minted}
  \end{block}
\end{frame}
