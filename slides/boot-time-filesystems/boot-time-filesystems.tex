\section{Filesystem optimizations}

\begin{frame}
\frametitle{Filesystem impact on performance}
Tuning the filesystem is usually one of the first things
we work on in boot time projects.
\begin{itemize}
\item Different filesystems can have different initialization
      and mount times. In particular, the type of filesystem
      for the root filesystem directly impacts boot time.
\item Different filesystems can exhibit different read, write
      and access time performance, according to the type
      of filesystem activity and to the type of files in the
      system.
\end{itemize}
\end{frame}

\begin{frame}
\frametitle{Different filesystem for different storage types}
\begin{itemize}
\item Block storage (including memory cards, eMMC)
      \begin{itemize}
      \item ext2, ext4
      \item xfs, btrfs
      \item f2fs
      \item SquashFS, EROFS
      \end{itemize}
\item Raw flash storage
      \begin{itemize}
      \item JFFS2
      \item YAFFS2
      \item UBIFS
      \item ubiblock + (SquashFS or EROFS)
      \end{itemize}
\end{itemize}
See our embedded Linux training materials for full details:
{\small
\url{https://bootlin.com/doc/training/embedded-linux/}
}
\end{frame}

\begin{frame}
\frametitle{Block filesystems}
For block storage
\begin{itemize}
\item ext4: pretty good read and write performance.
\item xfs: can be good in some read or write scenarii
      as well.
\item btrfs, f2fs: can achieve best read and write performance,
      taking advantage of the characteristics of flash-based block
      devices.
\item SquashFS: very good mount time and read performance, for read-only
      partitions. Gives priority to compression rate vs performance.
\item EROFS: newer read-only file system for block storage. Gives
      priority to read performance vs compression rate.
\end{itemize}
\end{frame}

\begin{frame}
\frametitle{JFFS2}
For raw flash storage
\begin{itemize}
\item Mount time depending on filesystem size: the kernel has to
      scan the whole storage at mount time, to read which block
      belongs to each file.
\item Need to use the \kconfig{CONFIG_JFFS2_SUMMARY} kernel option
      to store such information in flash. This dramatically reduces
      mount time.
\item Benchmark on ARM:\\
      from 16 s to 0.8 s for a 128 MB partition.
\item Rather poor read and write performance,\\
      compared to YAFFS2 and UBIFS.
\item JFFS2 only makes sense on small storage space,
      where UBI would have too much overhead.
\end{itemize}
\end{frame}

\begin{frame}
\frametitle{YAFFS2}
For raw flash storage
\begin{itemize}
\item Good mount time
\item Good read and write performance
\item Drawbacks: no compression, not in the mainline Linux kernel
\end{itemize}
\end{frame}

\begin{frame}
\frametitle{UBIFS}
For raw flash storage, on top of the UBI layer
\begin{itemize}
\item Advantages:
      \begin{itemize}
      \item Good read and write performance (similar to YAFFS2)
      \item Other advantages: better for wear leveling (can operate on the
	    whole UBI space, not only within a single partition).
      \end{itemize}
\item Drawbacks:
      \begin{itemize}
      \item Not appropriate for small partitions (too much metadata
	    overhead). Use JFFS2 or YAFFS2 instead.
      \item Not so good mount time, because of the time needed
            to initialize UBI ({\em UBI Attach}: at boot time or running
            \code{ubi_attach} in user space).
      \item Addressed by {\em UBI Fastmap}, introduced in Linux 3.7. \\
            See next slides.
      \end{itemize}
\end{itemize}
\end{frame}

\begin{frame}
\frametitle{How UBI Fastmap works}
\begin{itemize}
\item {\em UBI Attach}: needs to read UBI metadata by
      scanning all erase blocks. Time proportional
      to the storage size.
\item {\em UBI Fastmap} stores such information in a few flash
      blocks (typically at UBI detach time during system
      shutdown) and finds it there at boot time.
\item This makes {\em UBI Attach} time constant.
\item If {\em Fastmap} information is invalid (unclean system
      shutdown, for example), it falls back to scanning
      (slower, but correct, and {\em Fastmap} will work again
      during the next boot).
\item Details: ELCE 2012 presentation from Thomas Gleixner:
      \url{https://elinux.org/images/a/ab/UBI_Fastmap.pdf}
\end{itemize}
\end{frame}

\begin{frame}
\frametitle{Using UBI Fastmap}
\begin{itemize}
\item Compile your kernel with \kconfig{CONFIG_MTD_UBI_FASTMAP}
\item Boot your system at least once with the
      \code{ubi.fm_autoconvert=1} kernel parameter.
\item Reboot your system in a clean way
\item You can now remove \code{ubi.fm_autoconvert=1}
\end{itemize}
\end{frame}

\begin{frame}
\frametitle{UBI Fastmap benchmark}
\begin{itemize}
\item Measured on the Microchip SAMA5D3 Xplained board (ARM), Linux 3.10
\item UBI space: 216 MB
\item Root filesystem: 80 MB used (Yocto)
\item Average results:
    \newline \newline
    \begin{tabular}{| l || c | c | c |}
    \hline
    & Attach time & Diff & Total time\\
    \hline
    Without {\em UBI Fastmap} & 968 ms & & \\
    With {\em UBI Fastmap} & 238 ms & -731 ms & -665 ms \\
    \hline
    \end{tabular}
    \newline
\item Expect to save more with bigger UBI spaces!
\end{itemize}
Note: total boot time reduction a bit lower probably
because of other kernel threads executing during the
attach process.
\end{frame}

\begin{frame}
\frametitle{ubiblock + (SquashFS or EROFS)}
For raw flash storage
\begin{itemize}
\item {\em ubiblock}: read-only block device on top of UBI
      (\kconfig{CONFIG_MTD_UBI_BLOCK}).
\item Allows to put SquashFS or EROFS on a UBI volume.
\item Expecting great boot time and read performance. Great
      for read-only root filesystems.
\item Benchmarks not available yet.
\end{itemize}
\end{frame}

\begin{frame}
\frametitle{Finding the best filesystem}
\begin{itemize}
\item Raw flash storage: UBIFS with \kconfig{CONFIG_UBI_FASTMAP} is
      probably the best solution.
\item Block storage: SquashFS best solution for root filesystems
      which can be read-only. Btrfs and f2fs probably the best solutions
      for read/write filesystems.
\item Fortunately, changing filesystem types is quite cheap,
      and completely transparent for applications. Just try
      several filesystem options, as see which one works best
      for you!
\item Do not focus only on boot time. \\
      For systems in which read and write performance matters, we
      recommend to use separate root filesystem (for quick
      boot time) and data partitions (for good runtime performance).
\end{itemize}
\end{frame}
