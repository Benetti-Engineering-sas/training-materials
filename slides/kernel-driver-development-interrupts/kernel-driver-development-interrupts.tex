\subsection{Interrupt Management}

\begin{frame}[fragile]
  \frametitle{Registering an interrupt handler 1/2}
  The {\em managed} API is recommended:
  \begin{minted}[fontsize=\small]{c}
int devm_request_irq(struct device *dev,
                     unsigned int irq,
                     irq_handler_t handler,
                     unsigned long irq_flags,
                     const char *devname,
                     void *dev_id);
  \end{minted}
  \begin{itemize}
  \item \code{device} for automatic freeing at device or module
        release time.
  \item \code{irq} is the requested IRQ channel. For platform
        devices, use \kfunc{platform_get_irq} to retrieve the
        interrupt number.
  \item \code{handler} is a pointer to the IRQ handler
  \item \code{irq_flags} are option masks (see next slide)
  \item \code{devname} is the registered name (for
	\code{/proc/interrupts})
  \item \code{dev_id} is an opaque pointer. It can typically
        be used to pass a pointer to a per-device data structure.
        It cannot be \code{NULL} as it is used as an identifier for
        freeing interrupts on a shared line.
  \end{itemize}
\end{frame}

\begin{frame}[fragile]
  \frametitle{Releasing an interrupt handler}
  \begin{minted}{c}
void devm_free_irq(struct device *dev,
                   unsigned int irq, void *dev_id);
  \end{minted}
  \begin{itemize}
  \item Explicitly release an interrupt handler. Done automatically
        in normal situations.
  \end{itemize}
  Defined in \kfile{include/linux/interrupt.h}
\end{frame}

\begin{frame}
  \frametitle{Registering an interrupt handler 2/2}
  Here are the most frequent \code{irq_flags} bit values
  in drivers (can be combined):
  \begin{itemize}
     \item \ksym{IRQF_SHARED}: interrupt channel can be shared by several devices.
     \begin{itemize}
	\item When an interrupt is received, all the interrupt
	handlers registered on the same interrupt line are called.
	\item This requires a hardware status register telling whether
        an IRQ was raised or not.
     \end{itemize}
     \item \ksym{IRQF_ONESHOT}: for use by threaded interrupts (see
     next slides). Keeping the interrupt line disabled until the thread
     function has run.
  \end{itemize}
\end{frame}

\begin{frame}
  \frametitle{Interrupt handler constraints}
  \begin{itemize}
  \item No guarantee in which address space the system will be in when
    the interrupt occurs: can't transfer data to and from user space.
  \item Interrupt handler execution is managed by the CPU, not by the
    scheduler.  Handlers can't run actions that may sleep, because
    there is nothing to resume their execution. In particular, need to
    allocate memory with \ksym{GFP_ATOMIC}.
  \item Interrupt handlers are run with all interrupts disabled on
    the local CPU (see \url{https://lwn.net/Articles/380931}).
    Therefore, they have to complete their job quickly
    enough, to avoiding blocking interrupts for too long.
  \end{itemize}
\end{frame}

\begin{frame}[fragile]
  \frametitle{/proc/interrupts on Raspberry Pi 2 (ARM, Linux 4.19)}
\begin{block}{}
  \tiny
\begin{verbatim}
           CPU0       CPU1       CPU2       CPU3
 17:    1005317          0          0          0  ARMCTRL-level   1 Edge      3f00b880.mailbox
 18:         36          0          0          0  ARMCTRL-level   2 Edge      VCHIQ doorbell
 40:          0          0          0          0  ARMCTRL-level  48 Edge      bcm2708_fb DMA
 42:     427715          0          0          0  ARMCTRL-level  50 Edge      DMA IRQ
 56:  478426356          0          0          0  ARMCTRL-level  64 Edge      dwc_otg, dwc_otg_pcd, dwc_otg_hcd:usb1
 80:     411468          0          0          0  ARMCTRL-level  88 Edge      mmc0
 81:        502          0          0          0  ARMCTRL-level  89 Edge      uart-pl011
161:          0          0          0          0  bcm2836-timer   0 Edge      arch_timer
162:   10963772    6378711   16583353    6406625  bcm2836-timer   1 Edge      arch_timer
165:          0          0          0          0  bcm2836-pmu   9 Edge      arm-pmu
FIQ:              usb_fiq
IPI0:          0          0          0          0  CPU wakeup interrupts
IPI1:          0          0          0          0  Timer broadcast interrupts
IPI2:    2625198    4404191    7634127    3993714  Rescheduling interrupts
IPI3:       3140      56405      49483      59648  Function call interrupts
IPI4:          0          0          0          0  CPU stop interrupts
IPI5:    2167923     477097    5350168     412699  IRQ work interrupts
IPI6:          0          0          0          0  completion interrupts
Err:          0
\end{verbatim}
\end{block}
  \footnotesize
  Note: interrupt numbers shown on the left-most column are virtual
  numbers when the Device Tree is used. The real physical interrupt
  numbers can be seen in \code{/sys/kernel/debug/irq_domain_mapping}.
\end{frame}

\begin{frame}[fragile]
  \frametitle{Interrupt handler prototype}
  \begin{itemize}
  \item \mint{c}+irqreturn_t foo_interrupt(int irq, void *dev_id)+
    \begin{itemize}
    \item \code{irq}, the IRQ number
    \item \code{dev_id}, the per-device pointer that was
      passed to \kfunc{devm_request_irq}
    \end{itemize}
  \item Return value
    \begin{itemize}
    \item \ksym{IRQ_HANDLED}: recognized and handled interrupt
    \item \ksym{IRQ_NONE}: used by the kernel to detect spurious
	interrupts, and disable the interrupt line if none of the
	interrupt handlers has handled the interrupt.
    \end{itemize}
  \end{itemize}
\end{frame}

\begin{frame}
  \frametitle{Typical interrupt handler's job}
  \begin{itemize}
  \item Acknowledge the interrupt to the device (otherwise no more
    interrupts will be generated, or the interrupt will keep firing
    over and over again)
  \item Read/write data from/to the device
  \item Wake up any process waiting for such data, typically on a
    per-device wait queue:\\
    \code{wake_up_interruptible(&device_queue);}
\end{itemize}
\end{frame}

\begin{frame}[fragile]
  \frametitle{Threaded interrupts}
  The kernel also supports threaded interrupts:
  \begin{itemize}
  \item The interrupt handler is executed inside a thread.
  \item Allows to block during the interrupt handler, which is often
        needed for I2C/SPI devices as the interrupt handler needs to
        communicate with them.
  \item Allows to set a priority for the interrupt handler
        execution, which is useful for real-time usage of Linux
  \end{itemize}
  \begin{minted}{c}
int devm_request_threaded_irq(
    struct device *dev,
    unsigned int irq,
    irq_handler_t handler, irq_handler_t thread_fn,
    unsigned long flags, const char *name, void *dev);
  \end{minted}
  \begin{itemize}
  \item \code{handler}, ``hard IRQ'' handler
  \item \code{thread_fn}, executed in a thread
  \end{itemize}
\end{frame}

\begin{frame}
  \frametitle{Top half and bottom half processing}
  Splitting the execution of interrupt handlers in 2 parts
  \begin{itemize}
  \item Top half
    \begin{itemize}
    \item This is the real interrupt handler, which should complete
      as quickly as possible since all interrupts are disabled.
      It takes the data out of the device and if substantial
      post-processing is needed, schedule a bottom half to handle it. 
    \end{itemize}
  \item Bottom half
    \begin{itemize}
    \item Is the general Linux name for various mechanisms which
      allow to postpone the handling of interrupt-related
      work. Implemented in Linux as softirqs, tasklets or
      workqueues.
    \end{itemize}
  \end{itemize}
\end{frame}

\begin{frame}
  \frametitle{Top half and bottom half diagram}
  \begin{center}
    \includegraphics[width=\textwidth]{slides/kernel-driver-development-interrupts/thread-halves.pdf}
  \end{center}
\end{frame}

\begin{frame}
  \frametitle{Softirqs}
  \begin{itemize}
  \item Softirqs are a form of bottom half processing
  \item The softirqs handlers are executed with all interrupts
    enabled, and a given softirq handler can run simultaneously on
    multiple CPUs
  \item They are executed once all interrupt handlers have completed,
    before the kernel resumes scheduling processes, so sleeping is not
    allowed.
  \item The number of softirqs is fixed in the system, so softirqs are
    not directly used by drivers, but by complete kernel subsystems
    (network, etc.)
  \item The list of softirqs is defined in
    \kfile{include/linux/interrupt.h}: \ksym{HI_SOFTIRQ}, \ksym{TIMER_SOFTIRQ},
    \ksym{NET_TX_SOFTIRQ}, \ksym{NET_RX_SOFTIRQ}, \ksym{BLOCK_SOFTIRQ},
    \ksym{IRQ_POLL_SOFTIRQ}, \ksym{TASKLET_SOFTIRQ},
    \ksym{SCHED_SOFTIRQ}, \ksym{HRTIMER_SOFTIRQ}, \ksym{RCU_SOFTIRQ}
  \item \ksym{HI_SOFTIRQ} and \ksym{TASKLET_SOFTIRQ} are used to execute
    tasklets
  \end{itemize}
\end{frame}

\begin{frame}
  \frametitle{Example usage of softirqs - NAPI}
  NAPI = {\em New API}
  \begin{itemize}
  \item Interface in the Linux kernel used for interrupt mitigation in
        network drivers
  \item Principe: when the network traffic exceeds a given threshhold
        ("budget"), disable network interrupts and consume incoming packets
        through a polling function, instead of processing each new
        packet with an interrupt.
  \item This reduces overhead due to interrupts and yields better
        network throughput.
  \item The polling function is run by \kfunc{napi_schedule}, which uses
        \ksym{NET_RX_SOFTIRQ}.
  \item See \url{https://en.wikipedia.org/wiki/New_API} for details
  \item See also our commented network driver on \url{https://frama.link/qCaWu1-U}
  \end{itemize}

\end{frame}

\begin{frame}
  \frametitle{Tasklets}
  \begin{itemize}
  \item Tasklets are executed within the \ksym{HI_SOFTIRQ} and
    \ksym{TASKLET_SOFTIRQ} softirqs. They are executed with all interrupts enabled, but a
    given tasklet is guaranteed to execute on a single CPU at a time.
  \item Tasklets are typically created with the \kfunc{tasklet_init}
    function, when your driver manages multiple devices, otherwise
    statically with \kfunc{DECLARE_TASKLET}. A tasklet is simply
    implemented as a function. Tasklets can easily
    be used by individual device drivers, as opposed to softirqs.
  \item The interrupt handler can schedule tasklet execution with:
    \begin{itemize}
    \item \kfunc{tasklet_schedule} to get it executed in
      \ksym{TASKLET_SOFTIRQ}
    \item \kfunc{tasklet_hi_schedule} to get it executed in
      \ksym{HI_SOFTIRQ} (highest priority)
    \end{itemize}
  \end{itemize}
\end{frame}

\begin{frame}[fragile]
  \frametitle{Tasklet Example: drivers/crypto/atmel-sha.c 1/2}
\begin{minted}[fontsize=\scriptsize]{c}
/* The tasklet function */
static void atmel_sha_done_task(unsigned long data)
{
        struct atmel_sha_dev *dd = (struct atmel_sha_dev *)data;
        [...]
}

/* Probe function: registering the tasklet */
static int atmel_sha_probe(struct platform_device *pdev)
{
        struct atmel_sha_dev *sha_dd; // Per device structure
        [...]
        platform_set_drvdata(pdev, sha_dd);
        [...]
        tasklet_init(&sha_dd->done_task, atmel_sha_done_task,
                                        (unsigned long)sha_dd);
        [...]
        err = devm_request_irq(&pdev->dev, sha_dd->irq, atmel_sha_irq,
                               IRQF_SHARED, "atmel-sha", sha_dd);
        [...]
}
\end{minted}
\end{frame}

\begin{frame}[fragile]
  \frametitle{Tasklet Example: drivers/crypto/atmel-sha.c 2/2}
\begin{minted}[fontsize=\scriptsize]{c}
/* Remove function: removing the tasklet */
static int atmel_sha_remove(struct platform_device *pdev)
{
        static struct atmel_sha_dev *sha_dd;
        sha_dd = platform_get_drvdata(pdev);
        [...]
        tasklet_kill(&sha_dd->done_task);
        [...]
}

/* Interrupt handler: triggering execution of the tasklet */
static irqreturn_t atmel_sha_irq(int irq, void *dev_id)
{
        struct atmel_sha_dev *sha_dd = dev_id;
        [...]
        tasklet_schedule(&sha_dd->done_task);
        [...]
}
\end{minted}
\end{frame}

\begin{frame}[fragile]
  \frametitle{Workqueues}
  \begin{itemize}
  \item Workqueues are a general mechanism for deferring work. It is
    not limited in usage to handling interrupts. It can typically
    be used for background work which can be scheduled.
  \item The function registered as workqueue is executed in a thread,
    which means:
    \begin{itemize}
    \item All interrupts are enabled
    \item Sleeping is allowed
    \end{itemize}
  \item A workqueue, usually allocated in a per-device structure,
    is registered with \kfunc{INIT_WORK} and typically
    triggered with \kfunc{queue_work}
  \item The complete API, in \kfile{include/linux/workqueue.h}, provides
    many other possibilities (creating its own workqueue threads, etc.)
  \item Example (\kfile{drivers/crypto/atmel-i2c}):
\begin{minted}{c}
INIT_WORK(&work_data->work, atmel_i2c_work_handler);
schedule_work(&work_data->work);
\end{minted}
  \end{itemize}
\end{frame}

\begin{frame}
  \frametitle{Interrupt management summary}
  \begin{itemize}
  \item Device driver
    \begin{itemize}
    \item In the \code{probe()} function, for each device,
      use \kfunc{devm_request_irq} to register an interrupt handler
      for the device's interrupt channel.
    \end{itemize}
  \item Interrupt handler
    \begin{itemize}
    \item Called when an interrupt is raised.
    \item Acknowledge the interrupt
    \item If needed, schedule a per-device tasklet taking care of handling
      data.
    \item Wake up processes waiting for the data on a per-device queue
    \end{itemize}
  \item Device driver
    \begin{itemize}
    \item In the \code{remove()} function, for each device, the
      interrupt handler is automatically unregistered.
    \end{itemize}
  \end{itemize}
\end{frame}
