\section{Automating layer management}

\begin{frame}[fragile]
  \frametitle{Release management}
  There are multiple tasks that OE/bitbake based projects let you do
  on your own to ensure build reproducibility:
  \begin{itemize}
  \item Code distribution and project setup.
  \item Release tagging
  \end{itemize}
  A separate tool is needed for that, usual solutions are:
  \begin{itemize}
  \item git submodules + setup script. Great example in YOE:
    \url{https://github.com/YoeDistro/yoe-distro}
  \item repo and \code{templateconf} or setup script
  \item kas
  \end{itemize}
\end{frame}

\begin{frame}
  \frametitle{Google repo}
  \begin{itemize}
    \item A good way to distribute a distribution (Poky, custom
      layers, BSP, \code{.templateconf}\dots) is to use Google's
      \code{repo}.
    \item \code{Repo} is used in Android to distribute its source
      code, which is split into many \code{git} repositories. It's a
      wrapper to handle several \code{git} repositories at once.
    \item The only requirement is to use \code{git}.
    \item The \code{repo} configuration is stored in a \code{manifest}
      file, usually available in its own \code{git} repository.
  \item It could also be in a specific branch of your custom layer.
  \item It only handles fetching code, handling \code{local.conf} and
    \code{bblayers.conf} is done separately
  \end{itemize}
\end{frame}

\begin{frame}[fragile]
  \frametitle{Manifest example}
  \begin{block}{}
  \fontsize{9}{9}\selectfont
  \begin{minted}{xml}
<?xml version="1.0" encoding="UTF-8"?>
<manifest>
  <remote name="yocto-project" fetch="git.yoctoproject.org" />
  <remote name="private" fetch="git.example.net" />

  <default revision="kirkstone" remote="private" />

  <project name="poky" remote="yocto-project" />
  <project name="meta-ti" remote="yocto-project" />
  <project name="meta-custom" />
  <project name="meta-custom-bsp" />
  <project path="meta-custom-distro" name="distro">
    <copyfile src="templateconf" dest="poky/.templateconf" />
  </project>
</manifest>
  \end{minted}
  \end{block}
\end{frame}

\begin{frame}[fragile]
  \frametitle{Retrieve the project using \code{repo}}
  \begin{minted}{sh}
$ mkdir my-project; cd my-project
$ repo init -u https://git.example.net/manifest.git
$ repo sync -j4
  \end{minted}
  \begin{itemize}
    \item \code{repo init} uses the \code{default.xml} manifest in the
    repository, unless specified otherwise.
    \item You can see the full \code{repo} documentation at
      \url{https://source.android.com/source/using-repo.html}.
  \end{itemize}
\end{frame}

\begin{frame}[fragile]
  \frametitle{repo: release}
  To tag a release, a few steps have to be taken:
  \begin{itemize}
  \item Optionally tag the custom layers
  \item For each project entry in the manifest, set the revision
    parameter to either a tag or a commit hash.
  \item Commit and tag this version of the manifest.
  \end{itemize}
\end{frame}

\begin{frame}[fragile]{kas}
  \begin{itemize}
  \item Specific tool developed by Siemens for OpenEmbedded:
    \url{https://github.com/siemens/kas}
  \item Will fetch layers and build the image in a single command
  \item Uses a single JSON or YAML configuration file part of the
    custom layer
  \item Can generate and run inside a Docker container
  \item Can setup \code{local.conf} and \code{bblayers.conf}
  \end{itemize}
\end{frame}

\begin{frame}[fragile]
  \frametitle{kas configuration}
  \begin{block}{}
  \fontsize{9}{9}\selectfont
  \begin{minted}{yaml}
header:
  version: 8
machine: mymachine
distro: mydistro
target:
  - myimage

repos:
  meta-custom:

  bitbake:
    url: "https://git.openembedded.org/bitbake"
    refspec: "2.0"
    layers:
      .: excluded

  openembedded-core:
    url: "https://git.openembedded.org/openembedded-core"
    refspec: kirkstone
    layers:
      meta:
  \end{minted}
  \end{block}
\end{frame}

\begin{frame}[fragile]
  \frametitle{kas configuration}
  \fontsize{9}{9}\selectfont
  \begin{block}{}
  \begin{minted}{yaml}
  meta-freescale:
    url: "https://github.com/Freescale/meta-freescale"
    refspec: kirkstone

  meta-openembedded:
   url: https://git.openembedded.org/meta-openembedded
   refspec: kirkstone
   layers:
     meta-oe:
     meta-python:
     meta-networking:
  \end{minted}
  \end{block}


  \begin{itemize}
  \item Then a single command will build all the listed targets for the
machine:

  \begin{minted}{sh}
$ kas build meta-custom/mymachine.yaml
  \end{minted}

  \item Or, alternatively, invoke \code{bitbake} commands:

  \begin{minted}{sh}
$ kas shell /path/to/kas-project.yml -c 'bitbake dosfsutils-native'
  \end{minted}
  \end{itemize}
\end{frame}
