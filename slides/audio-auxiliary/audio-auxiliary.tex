\subsection{Auxiliary devices}

\begin{frame}{Amplifier}
  What about the amplifier?
  \begin{itemize}
  \item Supported using {\em auxiliary devices}
  \item Register a \kstruct{snd_soc_aux_dev} array using the
    \code{.aux_dev} and \code{.num_aux_devs} fields of the registered
    \kstruct{snd_soc_card}
  \item This will expose the auxiliary devices control widgets as part
    of the sound card
  \item There is a driver for simple amplifiers driven by a single
    GPIO, \code{simple-amplifier}.
  \end{itemize}
\end{frame}

\begin{frame}[fragile]{Auxiliary devices}
  \begin{block}{\kfile{sound/soc/samsung/neo1973_wm8753.c}}
    \fontsize{9}{9}\selectfont
    \begin{minted}{c}
static struct snd_soc_aux_dev neo1973_aux_devs[] = {
        {
                .name = "dfbmcs320",
                .codec_name = "dfbmcs320.0",
        },
};

static struct snd_soc_card neo1973 = {
        .name = "neo1973",
        .owner = THIS_MODULE,
        .dai_link = neo1973_dai,
        .num_links = ARRAY_SIZE(neo1973_dai),
        .aux_dev = neo1973_aux_devs,
        .num_aux_devs = ARRAY_SIZE(neo1973_aux_devs),
    \end{minted}
  \end{block}
\end{frame}

\begin{frame}[fragile]{simple-amplifier - example 1}
\kfile{arch/arm64/boot/dts/allwinner/sun50i-a64-pinebook.dts}
  \begin{block}{}
    \fontsize{7}{6}\selectfont
    \begin{minted}{c}

        speaker_amp: audio-amplifier {
                compatible = "simple-audio-amplifier";
                VCC-supply = <&reg_ldo_io0>;
                enable-gpios = <&pio 7 7 GPIO_ACTIVE_HIGH>; /* PH7 */
                sound-name-prefix = "Speaker Amp";
        };

&sound {
        status = "okay";
        simple-audio-card,aux-devs = <&codec_analog>, <&speaker_amp>;
        simple-audio-card,widgets = "Microphone", "Internal Microphone Left",
                                    "Microphone", "Internal Microphone Right",
                                    "Headphone", "Headphone Jack",
                                    "Speaker", "Internal Speaker";
        simple-audio-card,routing =
                        "Left DAC", "AIF1 Slot 0 Left",
                        "Right DAC", "AIF1 Slot 0 Right",
                        "Speaker Amp INL", "LINEOUT",
                        "Speaker Amp INR", "LINEOUT",
                        "Internal Speaker", "Speaker Amp OUTL",
                        "Internal Speaker", "Speaker Amp OUTR",
                        "Headphone Jack", "HP",
    \end{minted}
  \end{block}
\end{frame}

\begin{frame}[fragile]{simple-amplifier - example 2}
  \begin{block}{}
    \fontsize{7}{6}\selectfont
    \begin{minted}{c}
        dio2133: analog-amplifier {
                compatible = "simple-audio-amplifier";
                sound-name-prefix = "AU2";
                VCC-supply = <&hdmi_5v>;
                enable-gpios = <&gpio GPIOH_5 GPIO_ACTIVE_HIGH>;
        };

        sound {
                compatible = "amlogic,gx-sound-card";
                model = "GXL-LIBRETECH-S905X-CC";
                audio-aux-devs = <&dio2133>;
                audio-widgets = "Line", "Lineout";
                audio-routing = "AU2 INL", "ACODEC LOLN",
                                "AU2 INR", "ACODEC LORN",
                                "Lineout", "AU2 OUTL",
                                "Lineout", "AU2 OUTR";
    \end{minted}
  \end{block}
  Audio is routed through\code{AU2}, the amplifier.
\end{frame}

\begin{frame}{Input Muxing}
  \begin{itemize}
  \item There may be a muxer on the analog input lines.
  \item If controlled using a gpio, the \code{simple-mux} driver is
    available.
  \item It exposes two inputs: "IN1" and "IN2" and one output, "OUT".
  \item The device tree binding allows to provide a prefix to make the
    routes specific.
  \end{itemize}
\end{frame}

\begin{frame}[fragile]{\code{simple-mux} example}
  \begin{block}{}
    \fontsize{8}{8}\selectfont
    \begin{minted}{c}
        mic_mux: mic-mux {
                compatible = "simple-audio-mux";
                pinctrl-names = "default";
                pinctrl-0 = <&pinctrl_micsel>;
                mux-gpios = <&gpio5 5 GPIO_ACTIVE_LOW>;
                sound-name-prefix = "Mic Mux";
        };
    \end{minted}
  \end{block}
  \begin{itemize}
  \item This exposes routes between \code{Mic Mux IN1} and \code{Mic
    Mux IN2} to \code{Mic Mux OUT}.
  \item This route is controlled by \code{gpio5 5}.
  \item A control named \code{Mic Mux Muxer} will be exposed to
    userspace.
  \end{itemize}
\end{frame}

\begin{frame}[fragile]{\code{simple-mux} example}
  \begin{block}{\kfile{arch/arm64/boot/dts/freescale/imx8mq-librem5-devkit.dts}}
    \fontsize{7}{6}\selectfont
    \begin{minted}{c}
        sound {
                compatible = "simple-audio-card";
                pinctrl-names = "default";
                pinctrl-0 = <&pinctrl_hpdet>;
                simple-audio-card,aux-devs = <&speaker_amp>, <&mic_mux>;
                simple-audio-card,name = "Librem 5 Devkit";
                simple-audio-card,format = "i2s";
                simple-audio-card,widgets =
                        "Microphone", "Builtin Microphone",
                        "Microphone", "Headset Microphone",
                        "Headphone", "Headphones",
                        "Speaker", "Builtin Speaker";
                simple-audio-card,routing =
                        "MIC_IN", "Mic Mux OUT",
                        "Mic Mux IN1", "Headset Microphone",
                        "Mic Mux IN2", "Builtin Microphone",
                        "Mic Mux OUT", "Mic Bias",
                        "Headphones", "HP_OUT",
                        "Builtin Speaker", "Speaker Amp OUTR",
                        "Speaker Amp INR", "LINE_OUT";
                simple-audio-card,hp-det-gpio = <&gpio3 20 GPIO_ACTIVE_HIGH>;

                simple-audio-card,cpu {
                        sound-dai = <&sai2>;
                };

                simple-audio-card,codec {
                        sound-dai = <&sgtl5000>;
                        clocks = <&clk IMX8MQ_CLK_SAI2_ROOT>;
                        frame-master;
                        bitclock-master;
                };
        };
    \end{minted}
  \end{block}
\end{frame}
