\section{Layers}

\subsection{Introduction to layers}

\begin{frame}
  \frametitle{Layers' principles}
  \begin{itemize}
    \item The OpenEmbedded \emph{build system} manipulates
          \emph{metadata}.
    \item Layers allow to isolate and organize the metadata.
    \begin{itemize}
      \item A layer is a collection of recipes.
    \end{itemize}
    \item It is a good practice to begin a layer name with the prefix
      \code{meta-}.
  \end{itemize}
\end{frame}

\begin{frame}{Layers in Poky}
  \begin{center}
    \includegraphics[height=0.8\textheight]{slides/yocto-overview/yocto-overview-poky.pdf}
  \end{center}
\end{frame}

\begin{frame}
  \frametitle{Layers in Poky}
  \begin{itemize}
    \item The Poky \emph{reference system} is a set of basic common
          layers:
    \begin{itemize}
      \item meta
      \item meta-skeleton
      \item meta-poky
      \item meta-yocto-bsp
    \end{itemize}
    \item Poky is not a final set of layers. It is the common base.
    \item Layers are added when needed.
    \item When making modifications to the existing recipes or when
      adding new ones, it is a good practice not to modify Poky.
      Instead you can create your own layers!
  \end{itemize}
\end{frame}

\begin{frame}{Third party layers}
  \begin{center}
    \includegraphics[height=0.8\textheight]{slides/yocto-layer/yocto-layer-intro.pdf}
  \end{center}
\end{frame}

\begin{frame}
  \frametitle{Integrate and use a layer 1/3}
  \begin{itemize}
    \item A list of existing and maintained layers can be found at
      \url{https://layers.openembedded.org}
    \item Instead of redeveloping layers, always check the work hasn't
      been done by others.
    \item It takes less time to download a layer providing a package
      you need and to add an append file if some modifications are
      needed than to do it from scratch.
  \end{itemize}
\end{frame}

\begin{frame}
  \frametitle{Integrate and use a layer 2/3}
  \begin{itemize}
    \item The location where a layer is saved on the disk doesn't
      matter.
    \begin{itemize}
      \item But a good practice is to save it where all others layers
        are stored.
    \end{itemize}
    \item The only requirement is to let BitBake know about the new
          layer:
    \begin{itemize}
      \item The list of layers BitBake uses is defined in
        \code{$BUILDDIR/conf/bblayers.conf}
      \item To include a new layer, add its absolute path to the
        \code{BBLAYERS} variable.
      \item BitBake parses each layer specified in \code{BBLAYERS} and
        adds the recipes, configurations files and classes it
        contains.
    \end{itemize}
  \end{itemize}
\end{frame}

\begin{frame}
  \frametitle{Integrate and use a layer 3/3}
  \begin{itemize}
    \item The \code{bitbake-layers} tool is provided alongside
      \code{bitbake}.
    \item It can be used to inspect the layers and to manage
      \code{$BUILDDIR/conf/bblayers.conf}:
    \begin{itemize}
      \item \code{bitbake-layers show-layers}
      \item \code{bitbake-layers add-layer meta-custom}
      \item \code{bitbake-layers remove-layer meta-qt5}
    \end{itemize}
  \end{itemize}
\end{frame}

\begin{frame}
  \frametitle{Some useful layers}
  \begin{itemize}
    \item Many SoC specific layers are available, providing support
      for the boards using these SoCs. Some examples: \code{meta-ti},
      \code{meta-freescale} and \code{meta-raspberrypi}.
    \item Other layers offer to support applications not available in
      the Poky reference system:
    \begin{itemize}
      \item \code{meta-browser}: web browsers (Chromium, Firefox).
      \item \code{meta-filesystems}: support for additional
        filesystems.
      \item \code{meta-gstreamer10}: support for GStreamer 1.0.
      \item \code{meta-java} and \code{meta-oracle-java}: Java support.
      \item \code{meta-linaro-toolchain}: Linaro toolchain recipes.
      \item \code{meta-qt5}: QT5 modules.
      \item \code{meta-realtime}: real time tools and test programs.
      \item \code{meta-telephony} and many more\dots
    \end{itemize}
    Notice that some of these layers do not come with all the Yocto
    branches. The \code{meta-browser} did not have a \code{krogoth} branch,
    for example.
  \end{itemize}
\end{frame}

\subsection{Creating a layer}

\begin{frame}{Custom layer}
  \begin{center}
    \includegraphics[height=0.8\textheight]{slides/yocto-layer/yocto-layer-create.pdf}
  \end{center}
\end{frame}

\begin{frame}
  \frametitle{Create a custom layer 1/2}
  \begin{itemize}
    \item A layer is a set of files and directories and can be created
      by hand.
    \item However, the \code{bitbake-layers create-layer} command helps us create new
      layers and ensures this is done right.
    \item \code{bitbake-layers create-layer -p <PRIORITY> <layer>}
    \item The \textbf{priority} is used to select which recipe to use when multiple layers contains the same recipe
    \item The recipe priority takes precedence over the version number ordering
  \end{itemize}
\end{frame}

\begin{frame}
  \frametitle{Create a custom layer 2/2}
  \begin{itemize}
    \item The layer created will be pre-filled with the following
          files:
    \begin{description}
      \item[conf/layer.conf] The layer's configuration. Holds its
        priority and generic information. No need to modify it in many
        cases.
      \item[COPYING.MIT] The license under which a layer is released.
        By default MIT.
      \item[README] A basic description of the layer. Contains a
        contact e-mail to update.
    \end{description}
    \item By default, all metadata matching \code{./recipes-*/*/*.bb}
          will be parsed by the BitBake \emph{build engine}.
  \end{itemize}
\end{frame}

\begin{frame}
  \frametitle{Use a layer: best practices}
  \begin{itemize}
    \item Do not copy and modify existing recipes from other layers.
      Instead use append files.
    \item Avoid duplicating files. Use append files or explicitly use
      a path relative to other layers.
    \item Save the layer alongside other layers, in \code{OEROOT}.
    \item Use \code{LAYERDEPENDS} to explicitly define layer
      dependencies.
    \item Use \code{LAYERSERIES_COMPAT} to define the Yocto version(s)
    with which the layer is compatible.
  \end{itemize}
\end{frame}
