\section{Application development workflow}

\begin{frame}
  \frametitle{Recommended workflows}
  \begin{itemize}
    \item Different development workflows are possible given the
      needs:
      \begin{itemize}
        \item Low-level application development (bootloader, kernel).
        \item Application development.
        \item Temporary modifications on an external project (bug
          fixes, security fixes).
      \end{itemize}
    \item Three workflows exists for theses needs: the \code{SDK},
      \code{devtool} and \code{quilt}.
  \end{itemize}
\end{frame}

\subsection{The Yocto Project SDK}

\begin{frame}
  \frametitle{Overview}
  \begin{itemize}
    \item An SDK (Software Development Kit) is a set of tools allowing
      the development of applications for a given target (operating
      system, platform, environment\dots).
    \item It generally provides a set of tools including:
      \begin{itemize}
        \item Compilers or cross-compilers.
        \item Linkers.
        \item Library headers.
        \item Debuggers.
        \item Custom utilities.
      \end{itemize}
  \end{itemize}
\end{frame}

\begin{frame}
  \frametitle{The Yocto Project SDK}
  \begin{itemize}
    \item The Poky reference system is used to generate images, by
      building many applications and doing a lot of configuration work.
      \begin{itemize}
        \item When developing an application, we only care about the
          application itself.
        \item We want to be able to develop, test and debug easily.
      \end{itemize}
    \item The Yocto Project SDK is an application development SDK,
      which can be generated to provide a full environment compatible
      with the target.
    \item It includes a toolchain, libraries headers and all the
      needed tools.
    \item This SDK can be installed on any computer and is
      self-contained. The presence of Poky is not required for the SDK
      to fully work.
  \end{itemize}
\end{frame}

\begin{frame}
  \frametitle{Available SDKs}
  \begin{itemize}
    \item Two different SDKs can be generated:
      \begin{itemize}
        \item A generic SDK, including:
          \begin{itemize}
            \item A toolchain.
            \item Common tools.
            \item A collection of basic libraries.
          \end{itemize}
        \item An image-based SDK, including:
          \begin{itemize}
            \item The generic SDK.
            \item The sysroot matching the target root filesystem.
          \end{itemize}
      \end{itemize}
    \item The toolchain in the SDKs is self-contained (linked to an SDK
      embedded libc).
    \item The SDKs generated with Poky are distributed in the form of a
      shell script.
    \item Executing this script extracts the tools and sets up the
      environment.
  \end{itemize}
\end{frame}

\begin{frame}
  \frametitle{The generic SDK}
  \begin{itemize}
    \item Mainly used for low-level development, where only the
      toolchain is needed:
      \begin{itemize}
        \item Bootloader development.
        \item Kernel development.
      \end{itemize}
    \item The recipe \code{meta-toolchain} generates this SDK:
      \begin{itemize}
        \item \code{bitbake meta-toolchain}
      \end{itemize}
    \item The generated script, containing all the tools for this SDK,
      is in:
      \begin{itemize}
        \item \code{$BUILDDIR/tmp/deploy/sdk}
        \item Example:
          \code{poky-glibc-x86_64-meta-toolchain-cortexa8hf-neon-toolchain-2.5.sh}
      \end{itemize}
    \item The SDK will be configured to be compatible with the
      specified \code{MACHINE}.
  \end{itemize}
\end{frame}

\begin{frame}
  \frametitle{The image-based SDK}
  \begin{itemize}
    \item Used to develop applications running on the target.
    \item One task is dedicated to the process. The task behavior
      can vary between the images.
      \begin{itemize}
        \item \code{populate_sdk}
      \end{itemize}
    \item To generate an SDK for \code{core-image-minimal}:
      \begin{itemize}
        \item \code{bitbake -c populate_sdk core-image-minimal}
      \end{itemize}
    \item The generated script, containing all the tools for this SDK,
      is in:
      \begin{itemize}
        \item \code{$BUILDDIR/tmp/deploy/sdk}
        \item Example:
          \code{poky-glibc-x86_64-core-image-minimal-cortexa8hf-neon-toolchain-2.5.sh}
      \end{itemize}
    \item The SDK will be configured to be compatible with the
      specified \code{MACHINE}.
  \end{itemize}
\end{frame}

\begin{frame}[fragile]
  \frametitle{Adding packages to the SDK}
  \begin{itemize}
    \item Two variables control what will be installed in the SDK
  \end{itemize}
  \begin{description}
    \item[TOOLCHAIN\_TARGET\_TASK]
      List of target packages to be included in the SDK
    \item[TOOLCHAIN\_HOST\_TASK]
      List of host packages to be included in the SDK
  \end{description}
  \begin{itemize}
    \item Both can be appended to install more tools or libraries useful
      for development.
    \item Example: to have native {\tt curl} on the SDK:
  \end{itemize}
  \begin{block}{}
    \fontsize{9}{9}\selectfont
    \begin{minted}{bash}
TOOLCHAIN_HOST_TASK:append = "nativesdk-curl"
    \end{minted}
  \end{block}
\end{frame}

\begin{frame}[fragile]
  \frametitle{SDK format}
  \begin{itemize}
    \item Both SDKs are distributed as bash scripts.
    \item These scripts self extract themselves to install the
      toolchains and the files they provide.
    \item To install an SDK, retrieve the generated script and execute
      it.
      \begin{itemize}
        \item The script asks where to install the SDK. Defaults to
          \code{/opt/poky/<version>}
        \item Example: \code{/opt/poky/2.5}
        \begin{block}{}
          \begin{minted}[fontsize=\scriptsize]{console}
$ ./poky-glibc-x86_64-meta-toolchain-cortexa8hf-neon-toolchain-2.5.sh
Poky (Yocto Project Reference Distro) SDK installer version 2.5
===============================================================
Enter target directory for SDK (default: /opt/poky/2.5):
You are about to install the SDK to "/opt/poky/2.5". Proceed[Y/n]?
Extracting SDK.................done
Setting it up...done
SDK has been successfully set up and is ready to be used.
Each time you wish to use the SDK in a new shell session, you need to source
the environment setup script e.g.
 $ . /opt/poky/2.5/environment-setup-cortexa8hf-neon-poky-linux-gnueabi
          \end{minted}
        \end{block}
      \end{itemize}
  \end{itemize}
\end{frame}

\begin{frame}[fragile]
  \frametitle{Use the SDK}
  \begin{itemize}
    \item To use the SDK, a script is available to set up the
      environment:
      \begin{block}{}
        \begin{minted}[fontsize=\scriptsize]{console}
$ cd /opt/poky/2.5
$ source ./environment-setup-cortexa8hf-neon-poky-linux-gnueabi
        \end{minted}
      \end{block}
    \item The \code{PATH} is updated to take into account the binaries
      installed alongside the SDK.
    \item Environment variables are exported to help using the tools.
  \end{itemize}
\end{frame}

\begin{frame}
  \frametitle{SDK installation}
  \begin{description}
    \item[environment-setup-cortexa8hf-neon-poky-linux-gnueabi] Exports environment
      variables.
    \item[site-config-cortexa8hf-neon-poky-linux-gnueabi] Variables used during the
      toolchain creation
    \item[sysroots] SDK binaries, headers and libraries. Contains
      one directory for the host and one for the target.
    \item[version-cortexa8hf-neon-poky-linux-gnueabi] Version information.
  \end{description}
\end{frame}

\begin{frame}
  \frametitle{SDK environment variables}
  \begin{description}
    \item[CC] Full path to the C compiler binary.
    \item[CFLAGS] C flags, used by the C compiler.
    \item[CXX] C++ compiler.
    \item[CXXFLAGS] C++ flags, used by \code{CPP}
    \item[LD] Linker.
    \item[LDFLAGS] Link flags, used by the linker.
    \item[ARCH] For kernel compilation.
    \item[CROSS\_COMPILE] For kernel compilation.
    \item[GDB] SDK GNU Debugger.
    \item[OBJDUMP] SDK objdump.
  \end{description}
  \begin{itemize}
    \item To see the full list, open the environment script.
  \end{itemize}
\end{frame}

\begin{frame}[fragile]
  \frametitle{Examples}
  \begin{itemize}
    \item To build an application for the target:
      \begin{block}{}
        \begin{minted}{console}
$ $CC -o example example.c
        \end{minted}
      \end{block}
    \item The \code{LDFLAGS} variables is set to be used with the C
      compiler (\code{gcc}).
      \begin{itemize}
        \item When building the Linux kernel, unset this variable.
          \begin{block}{}
            \begin{minted}{console}
$ unset LDFLAGS
$ make menuconfig
$ make
            \end{minted}
          \end{block}
      \end{itemize}
  \end{itemize}
\end{frame}

\subsection{Devtool}

\begin{frame}
  \frametitle{Overview}
  \begin{itemize}
    \item \code{Devtool} is a set of utilities to ease the integration
    and the development of OpenEmbedded recipes.
    \item It can be used to:
      \begin{itemize}
        \item Generate a recipe for a given upstream application.
        \item Modify an existing recipe and its associated sources.
        \item Upgrade an existing recipe to use a newer upstream
          application.
      \end{itemize}
    \item \code{Devtool} adds a new layer, automatically managed, in
      \code{$BUILDDIR/workspace/}.
    \item It then adds or appends recipes to this layer so that the
      recipes point to a local path for their sources. In
      \code{$BUILDDIR/workspace/sources/}.
      \begin{itemize}
        \item Local sources are managed by \code{git}.
        \item All modifications made locally should be commited.
      \end{itemize}
  \end{itemize}
\end{frame}

\begin{frame}
  \frametitle{\code{devtool} usage 1/3}
  There are three ways of creating a new \code{devtool} project:
  \begin{itemize}
    \item To create a new recipe:
      \code{devtool add <recipe> <fetchuri>}
      \begin{itemize}
        \item Where \code{recipe} is the recipe's name.
        \item \code{fetchuri} can be a local path or a remote {\em
          uri}.
      \end{itemize}
    \item To modify the source for an existing recipe: \code{devtool modify <recipe>}
    \item To upgrade a given recipe:
      \code{devtool upgrade -V <version> <recipe>}
      \begin{itemize}
        \item Where \code{version} is the new version of the upstream
          application.
      \end{itemize}
  \end{itemize}
\end{frame}

\begin{frame}
  \frametitle{\code{devtool} usage 2/3}
  Once a \code{devtool} project is started, commands can be issued:
  \begin{itemize}
    \item \code{devtool edit-recipe <recipe>}: edit \code{recipe} in a text
      editor (as defined by the \code{EDITOR} environment variable).
    \item \code{devtool build <recipe>}: build the given
      \code{recipe}.
    \item \code{devtool build-image <image>}: build \code{image} with
      the additional \code{devtool} recipes' packages.
  \end{itemize}
\end{frame}

\begin{frame}
  \frametitle{\code{devtool} usage 3/3}
  \begin{itemize}
    \item \code{devtool deploy-target <recipe> <target>}: upload the
      \code{recipe}'s packages on \code{target}, which is a live
      running target with an SSH server running (\code{user@address}).
    \item \code{devtool update-recipe <recipe>}: generate patches from
      git commits made locally.
    \item \code{devtool reset <recipe>}: remove \code{recipe} from the
      control of \code{devtool}. Standard layers and remote sources
      are used again as usual.
  \end{itemize}
\end{frame}

\subsection{Quilt}

\begin{frame}
  \frametitle{Overview}
  \begin{itemize}
    \item Quilt is a utility to manage patches which can be used
      without having a clean source tree.
    \item It can be used to create patches for recipes already
      available in the build system.
    \item Be careful when using this workflow: the modifications won't
      persist across builds!
  \end{itemize}
\end{frame}

\begin{frame}
  \frametitle{Using Quilt}
  \begin{enumerate}
    \item Find the recipe working directory in
      \code{$BUILDDIR/tmp/work/}.
    \item Create a new \code{Quilt} patch:
      \code{$ quilt new topic.patch}
    \item Add files to this patch: \code{$ quilt add file0.c file1.c}
    \item Make the modifications by editing the files.
    \item Test the modifications:
      \code{$ bitbake -c compile -f recipe}
    \item Generate the patch file: \code{$ quilt refresh}
    \item Move the generated patch into the recipe's directory.
  \end{enumerate}
\end{frame}
