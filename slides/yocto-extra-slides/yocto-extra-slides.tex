\section{Extra slides}

\subsection{Quilt}

\begin{frame}
  \frametitle{Overview}
  \begin{itemize}
    \item Quilt is a utility to manage patches which can be used
      without having a clean source tree.
    \item It can be used to create patches for recipes already
      available in the build system.
    \item Be careful when using this workflow: the modifications won't
      persist across builds!
  \end{itemize}
\end{frame}

\begin{frame}
  \frametitle{Using Quilt}
  \begin{enumerate}
    \item Find the recipe working directory in
      \code{$BUILDDIR/tmp/work/}.
    \item Create a new \code{Quilt} patch:
      \code{$ quilt new topic.patch}
    \item Add files to this patch: \code{$ quilt add file0.c file1.c}
    \item Make the modifications by editing the files.
    \item Test the modifications:
      \code{$ bitbake -c compile -f recipe}
    \item Generate the patch file: \code{$ quilt refresh}
    \item Move the generated patch into the recipe's directory.
  \end{enumerate}
\end{frame}
