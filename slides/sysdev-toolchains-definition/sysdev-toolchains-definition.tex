\subsection{Definition and Components}

\begin{frame}
  \frametitle{Toolchain definition (1)}
  \begin{itemize}
  \item The usual development tools available on a GNU/Linux
    workstation is a {\bf native toolchain}
  \item This toolchain runs on your workstation and generates code for
    your workstation, usually x86
  \item For embedded system development, it is usually impossible or not
    interesting to use a native toolchain
    \begin{itemize}
    \item The target is too restricted in terms of storage and/or memory
    \item The target is very slow compared to your workstation
    \item You may not want to install all development tools on your target.
    \end{itemize}
  \item Therefore, {\bf cross-compiling toolchains} are generally
    used. They run on your workstation but generate code for your
    target.
  \end{itemize}
\end{frame}

\begin{frame}
  \frametitle{Toolchain definition (2)}
  \begin{center}
    \includegraphics[width=0.8\textwidth]{slides/sysdev-toolchains-definition/cross-toolchain.pdf}
  \end{center}
\end{frame}

\begin{frame}
  \frametitle{Machines in build procedures}
  \begin{itemize}
  \item Three machines must be distinguished when discussing toolchain creation
    \begin{itemize}
    \item The {\bf build} machine, where the toolchain is built.
    \item The {\bf host} machine, where the toolchain will be executed.
    \item The {\bf target} machine, where the binaries created by the
      toolchain are executed.
    \end{itemize}
  \item Four common build types are possible for toolchains
  \end{itemize}
\end{frame}

\begin{frame}
  \frametitle{Different toolchain build procedures}
  \begin{center}
    \includegraphics[height=0.8\textheight]{slides/sysdev-toolchains-definition/toolchain-build-types.pdf}
  \end{center}
\end{frame}

\begin{frame}
  \frametitle{Components of gcc toolchains}
  \begin{center}
    \includegraphics[width=0.8\textwidth]{slides/sysdev-toolchains-definition/components.pdf}
  \end{center}
\end{frame}

\begin{frame}
  \frametitle{Binutils}
  \begin{itemize}
  \item {\bf Binutils} is a set of tools to generate and manipulate
    binaries for a given CPU architecture
    \begin{itemize}
    \item \code{as}, the assembler, that generates binary code from
      assembler source code
    \item \code{ld}, the linker
    \item \code{ar}, \code{ranlib}, to generate \code{.a} archives
     (static libraries)
    \item \code{objdump}, \code{readelf}, \code{size}, \code{nm},
      \code{strings}, to inspect binaries. Very useful analysis tools!
    \item \code{objcopy}, to modify binaries
    \item \code{strip}, to strip parts of binaries that are just needed
      for debugging (reducing their size).
    \end{itemize}
  \item GNU Binutils: \url{https://www.gnu.org/software/binutils/}, GPL license
  \item The LLVM project now provides alternatives to GNU Binutils:
        \code{llvm-strip}, \code{llvm-readelf}, \code{lld}...
        (\url{https://www.llvm.org/docs/CommandGuide/})
  \end{itemize}
\end{frame}

\begin{frame}
  \frametitle{Kernel headers (1)}
  \begin{columns}
    \column{0.7\textwidth}
    \begin{itemize}
    \item The C library and compiled programs needs to interact with the kernel
      \begin{itemize}
      \item Available system calls and their numbers
      \item Constant definitions
      \item Data structures, etc.
      \end{itemize}
    \item Therefore, compiling the C library requires kernel headers, and many
      applications also require them.
    \item Available in \code{<linux/...>} and \code{<asm/...>} and a few
      other directories corresponding to the ones visible in
      \kdir{include/uapi} and in \code{arch/<arch>/include/uapi} in the kernel sources
    \item The kernel headers are extracted from the kernel sources using
      the \code{headers_install} kernel Makefile target.
    \end{itemize}
    \column[c]{0.3\textwidth}
    \includegraphics[width=\textwidth]{slides/sysdev-toolchains-definition/kernel-headers.pdf}
  \end{columns}
\end{frame}

\begin{frame}[fragile]
  \frametitle{Kernel headers (2)}
  \begin{itemize}
  \item System call numbers, in \code{<asm/unistd.h>}
\begin{verbatim}
#define __NR_exit         1
#define __NR_fork         2
#define __NR_read         3
\end{verbatim}
  \item Constant definitions, here in \code{<asm-generic/fcntl.h>},
    included from \code{<asm/fcntl.h>}, included from
    \code{<linux/fcntl.h>}
\begin{verbatim}
#define O_RDWR 00000002
\end{verbatim}
\item Data structures, here in \code{<asm/stat.h>} (used by the
\code{stat} command)
\begin{verbatim}
struct stat {
    unsigned long st_dev;
    unsigned long st_ino;
    [...]
};
\end{verbatim}
\end{itemize}
\end{frame}

\begin{frame}[fragile]
  \frametitle{Kernel headers (3)}
  The kernel to user space ABI is {\bf backward compatible}
  \begin{itemize}
  \item ABI = {\em Application Binary Interface} - It's about binary compatibility
  \item Kernel developers are doing their best to {\bf never}
        break existing programs when the kernel is upgraded.
        Otherwise, users would stick to older kernels, which
        would be bad for everyone.
  \item Hence, binaries generated with a toolchain using kernel headers
        older than the running kernel will work without problem, but
        won't be able to use the new system calls, data structures, etc.
  \item Binaries generated with a toolchain using kernel headers
        newer than the running kernel might work only if they don't use
        the recent features, otherwise they will break.
  \end{itemize}
  What to remember: fine to keep an old toolchain as long as it
  works fine for your project, even if you update the kernel.
\end{frame}

\begin{frame}
  \frametitle{C/C++ compiler}
  \begin{columns}
    \column{0.8\textwidth}
    \begin{itemize}
    \item GCC: GNU Compiler Collection, the famous free software compiler
    \item \url{https://gcc.gnu.org/}
    \item Can compile C, C++, Ada, Fortran, Java, Objective-C,
      Objective-C++, Go, etc. Can generate code for a large number of CPU
      architectures, including x86, ARM, RISC-V, and many others.
    \item Available under the GPL license, libraries under the GPL with
      linking exception.
    \item Alternative: Clang / LLVM compiler (\url{https://clang.llvm.org/})
      getting increasingly popular and able to compile most programs
      (license: MIT/BSD type). It can offer better optimizations and
      make errors easier to interpret.
    \end{itemize}
    \column{0.2\textwidth}
    \includegraphics[width=0.7\textwidth]{slides/sysdev-toolchains-definition/gcc.png}
  \end{columns}
\end{frame}

\begin{frame}
  \frametitle{C library}
  \begin{columns}
    \column{0.6\textwidth}
    \begin{itemize}
    \item The C library is an essential component of a Linux system
      \begin{itemize}
      \item Interface between the applications and the kernel
      \item Provides the well-known standard C API to ease application
        development
      \end{itemize}
    \item Several C libraries are available:
      {\em glibc}, {\em uClibc}, {\em musl}, {\em klibc}, {\em
        newlib}...
    \item The choice of the C library must be made at
      cross-compiling toolchain generation time, as the GCC compiler is
      compiled against a specific C library.
    \end{itemize}
    \column{0.4\textwidth}
    \includegraphics[width=\textwidth]{slides/sysdev-toolchains-definition/Linux_kernel_System_Call_Interface_and_uClibc.pdf}\\
    {\tiny Source: Wikipedia (\url{https://bit.ly/2zrGve2})}
  \end{columns}
  Comparing libcs by feature: \url{https://www.etalabs.net/compare_libcs.html}
\end{frame}
