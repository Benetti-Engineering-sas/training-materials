\subsection{Toolchain Options}
\begin{frame}
  \frametitle{ABI}
  \begin{itemize}
  \item When building a toolchain, the ABI used to generate binaries
    needs to be defined
  \item ABI, for {\em Application Binary Interface}, defines the
    calling conventions (how function arguments are passed, how the
    return value is passed, how system calls are made) and the
    organization of structures (alignment, etc.)
  \item All binaries in a system are typically compiled with the same ABI,
    and the kernel must understand this ABI.
  \item On ARM 32-bit, two main ABIs: {\em EABI} and {\em EABIhf}
    \begin{itemize}
    \item {\em EABIhf} passes floating-point arguments in
      floating-point registers $\rightarrow$ needs an ARM processor
      with a FPU
    \end{itemize}
  \item On RISC-V, several ABIs: {\em ilp32}, {\em ilp32f}, {\em ilp32d},
        {\em lp64}, {\em lp64f}, and {\em lp64d}
  \item {\fontsize{10}{15} \url{https://en.wikipedia.org/wiki/Application_Binary_Interface}}
  \end{itemize}
\end{frame}

\begin{frame}{Floating point support}
  \begin{itemize}
  \item All ARMv7-A (32-bit) and ARMv8-A (64-bit) processors have a
    floating point unit
  \item RISC-V cores with the \code{F} extension have a floating point
    unit
  \item Some older ARM cores (ARMv4/ARMv5) or some RISC-V cores may
    not have a floating point unit
  \item For processors without a floating point unit, two solutions
    for floating point computation:
    \begin{itemize}
    \item Generate {\em hard float code} and rely on the kernel to
      emulate the floating point instructions. This is very slow.
    \item Generate {\em soft float code}, so that instead of
      generating floating point instructions, calls to a user space
      library are generated
    \end{itemize}
  \item Decision taken at toolchain configuration time
  \item For processors with a floating point unit, sometimes different
    FPU are possible. For example on ARM: VFPv3, VFPv3-D16, VFPv4,
    VFPv4-D16, etc.
  \end{itemize}
\end{frame}

\begin{frame}
  \frametitle{CPU optimization flags}
  \begin{itemize}
  \item GNU tools (gcc, binutils) can only be compiled for a specific
        target architecture at a time (ARM, x86, RISC-V...)
  \item gcc offers further options:
    \begin{itemize}
    \item \code{-march} allows to select a specific target instruction set
    \item \code{-mtune} allows to optimize code for a specific CPU
    \item For example: \code{-march=armv7 -mtune=cortex-a8}
    \item \code{-mcpu=cortex-a8} can be used instead to allow gcc to infer the target
      instruction set (\code{-march=armv7}) and cpu optimizations (\code{-mtune=cortex-a8})
    \item \url{https://gcc.gnu.org/onlinedocs/gcc/ARM-Options.html}
    \end{itemize}
  \item At the GNU toolchain compilation time, values can be chosen. They are used:
    \begin{itemize}
    \item As the default values for the cross-compiling tools, when no
      other \code{-march}, \code{-mtune}, \code{-mcpu} options are
      passed
    \item To compile the C library
    \end{itemize}
  \item Note: LLVM (Clang, LLD...) utilities support multiple target architectures at once.
  \end{itemize}
\end{frame}
