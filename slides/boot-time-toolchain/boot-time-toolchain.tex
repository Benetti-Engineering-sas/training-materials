\section{Toolchain optimizations}

\begin{frame}
\frametitle{Best toolchain for your project}
Optimizing the cross-compiling toolchain is typically one of the first
things to do:
\begin{itemize}
  \item The benefits of a toolchain change will be more significant and easier to
        measure if other optimizations haven't been done yet.
  \item Here's what you can change in a toolchain, with a potential
        impact on boot time, performance and size:
  \begin{itemize}
    \item Components: versions of {\em gcc} and {\em binutils}\\
	  More recent versions can feature better optimization capabilities.
    \item C library: {\em glibc}, {\em uClibc}, {\em musl}\\
	  {\em uClibc} and {\em musl} libraries make a smaller root
	  filesystem
    \item Instruction set variant: {\em ARM} or {\em Thumb2} (on 32 bit
          only), {\em Hard Float} support or not.\\
	  Can have an impact on code performance and code size.
          \begin{itemize}
            \item {\em Thumb2}, available only on ARM 32, encodes the same instructions
                  as {\em ARM} but in a more compact way, at least significantly reducing size.
	    \item {\em ARM EABIhf}, in addition to being more efficient,
                  also reduces code size compared to {\em ARM EABI}, but
                  only on binaries doing floating point computation. For
                  example, \code{libavcodec} size is only reduced by 4K
                  (-0.03\%). That's negligible.
	  \end{itemize}
  \end{itemize}
\end{itemize}
\end{frame}

\begin{frame}
\frametitle{Choosing the C library}
\begin{itemize}
  \item The C library is hardcoded at toolchain creation time
  \item Available C libraries:
  \begin{itemize}
    \item {\em glibc}: most standard and featureful
    \item {\em uClibc}: smaller and configurable. Has been around for
	about 20 years.
    \item {\em musl}: an alternative to {\em uClibc}, developed more
        recently but mature too.
  \end{itemize}
\end{itemize}
\end{frame}

