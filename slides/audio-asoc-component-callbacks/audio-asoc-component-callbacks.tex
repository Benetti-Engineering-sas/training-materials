\subsection{ASoC component callbacks}

\begin{frame}[fragile]{\code{snd_soc_dai_ops}}
  \begin{block}{\code{include/sound/soc-dai.h}}
    \fontsize{9}{9}\selectfont
    \begin{minted}{c}
struct snd_soc_dai_ops {
        /*
         * DAI clocking configuration, all optional.
         * Called by soc_card drivers, normally in their hw_params.
         */
        int (*set_sysclk)(struct snd_soc_dai *dai,
                int clk_id, unsigned int freq, int dir);
        int (*set_pll)(struct snd_soc_dai *dai, int pll_id, int source,
                unsigned int freq_in, unsigned int freq_out);
        int (*set_clkdiv)(struct snd_soc_dai *dai, int div_id, int div);
        int (*set_bclk_ratio)(struct snd_soc_dai *dai, unsigned int ratio);

        /*
         * DAI format configuration
         * Called by soc_card drivers, normally in their hw_params.
         */
        int (*set_fmt)(struct snd_soc_dai *dai, unsigned int fmt);
        int (*xlate_tdm_slot_mask)(unsigned int slots,
                unsigned int *tx_mask, unsigned int *rx_mask);
        int (*set_tdm_slot)(struct snd_soc_dai *dai,
                unsigned int tx_mask, unsigned int rx_mask,
                int slots, int slot_width);
    \end{minted}
  \end{block}
\end{frame}

\begin{frame}[fragile]{\code{snd_soc_dai_ops}}
  \begin{block}{\code{include/sound/soc-dai.h}}
    \fontsize{9}{9}\selectfont
    \begin{minted}{c}
        /*
         * DAI digital mute - optional.
         * Called by soc-core to minimise any pops.
         */
        int (*mute_stream)(struct snd_soc_dai *dai, int mute, int stream);

        /*
         * ALSA PCM audio operations - all optional.
         * Called by soc-core during audio PCM operations.
         */
        int (*startup)(struct snd_pcm_substream *,
                struct snd_soc_dai *);
        void (*shutdown)(struct snd_pcm_substream *,
                struct snd_soc_dai *);
        int (*hw_params)(struct snd_pcm_substream *,
                struct snd_pcm_hw_params *, struct snd_soc_dai *);
        int (*hw_free)(struct snd_pcm_substream *,
                struct snd_soc_dai *);
        int (*prepare)(struct snd_pcm_substream *,
                struct snd_soc_dai *);
[...]
};
    \end{minted}
  \end{block}
\end{frame}

\begin{frame}{\code{hw_params}}
  \begin{itemize}
  \item The most useful callback
  \item It is used to configure the component to match the parameters
    of the audio stream.
  \item Called when the stream is ready to be played, before any data
    is transferred.
  \item If the requested parameters cannot be supported by the
    hardware, the \code{hw_params} callback can return an error code
    to indicate that the stream cannot be opened. Otherwise, the
    callback returns successfully, and the audio stream can be
    started.
  \end{itemize}
\end{frame}

\begin{frame}{\code{snd_pcm_hw_params}}
  \begin{itemize}
  \item Holds the stream parameters
  \item Usually not accessed directly but through accessors:
    \begin{itemize}
    \item \code{params_channels}: the number of channels
    \item \code{params_rate}: the sample rate
    \item \code{params_width}: the number of bits per sample
    \end{itemize}
  \end{itemize}
\end{frame}

\begin{frame}[fragile]{\code{hw_params} example}
  \begin{block}{\code{sound/soc/codecs/tlv320aic31xx.c}}
    \fontsize{8}{7}\selectfont
    \begin{minted}{c}
static int aic31xx_hw_params(struct snd_pcm_substream *substream,
                             struct snd_pcm_hw_params *params,
                             struct snd_soc_dai *dai)
{
        struct snd_soc_component *component = dai->component;
        struct aic31xx_priv *aic31xx = snd_soc_component_get_drvdata(component);
        u8 data = 0;

        switch (params_width(params)) {
        case 16:
                break;
        case 20:
                data = (AIC31XX_WORD_LEN_20BITS <<
                        AIC31XX_IFACE1_DATALEN_SHIFT);
                break;
        case 24:
                data = (AIC31XX_WORD_LEN_24BITS <<
                        AIC31XX_IFACE1_DATALEN_SHIFT);
                break;
        case 32:
                data = (AIC31XX_WORD_LEN_32BITS <<
                        AIC31XX_IFACE1_DATALEN_SHIFT);
                break;
        default:
                dev_err(component->dev, "%s: Unsupported width %d\n",
                        __func__, params_width(params));
                return -EINVAL;
        }
    \end{minted}
  \end{block}
\end{frame}

\begin{frame}[fragile]{\code{hw_params} example}
  \begin{block}{\code{sound/soc/codecs/tlv320aic31xx.c}}
    \fontsize{8}{8}\selectfont
    \begin{minted}{c}
        snd_soc_component_update_bits(component, AIC31XX_IFACE1,
                            AIC31XX_IFACE1_DATALEN_MASK,
                            data);

        /*
         * If BCLK is used as PLL input, the sysclk is determined by the hw
         * params. So it must be updated here to match the input frequency.
         */
        if (aic31xx->sysclk_id == AIC31XX_PLL_CLKIN_BCLK) {
                aic31xx->sysclk = params_rate(params) * params_width(params) *
                                  params_channels(params);
                aic31xx->p_div = 1;
        }

        return aic31xx_setup_pll(component, params);
}
    \end{minted}
  \end{block}
  \kfunc{aic31xx_setup_pll} then uses the parameters to set the CODEC
  PLLs and clocks properly. The usual ways to achieve that are to
  either do the calculations or prepare an array matching parameters
  to register values.
\end{frame}

\begin{frame}{\code{set_sysclk}}
  \begin{itemize}
  \item This sets the system clock parameters of the component, in
    particular which one is selected, its frequency and the direction.
  \item This allows the component to set up PLLs and clocks.
  \item This is called from the machine driver, using
    \kfunc{snd_soc_dai_set_sysclk}
  \item It can return an error in case the clock is not available or
    the frequency is not in the supported range.
  \item A component wide version exists, called using
    \kfunc{snd_soc_component_set_sysclk}, very rarely used.
  \end{itemize}
\end{frame}


\begin{frame}[fragile]{\code{set_sysclk} example}
  \begin{block}{}
    \fontsize{7}{7}\selectfont
    \begin{minted}{c}
static int aic31xx_set_dai_sysclk(struct snd_soc_dai *codec_dai,
                                  int clk_id, unsigned int freq, int dir)
{
        struct snd_soc_component *component = codec_dai->component;
        struct aic31xx_priv *aic31xx = snd_soc_component_get_drvdata(component);
        int i;
[...]
        for (i = 1; i < 8; i++)
                if (freq / i <= 20000000)
                        break;
        if (freq/i > 20000000) {
                dev_err(aic31xx->dev, "%s: Too high mclk frequency %u\n",
                        __func__, freq);
                return -EINVAL;
        }
        aic31xx->p_div = i;

        for (i = 0; i < ARRAY_SIZE(aic31xx_divs); i++)
                if (aic31xx_divs[i].mclk_p == freq / aic31xx->p_div)
                        break;
        if (i == ARRAY_SIZE(aic31xx_divs)) {
                dev_err(aic31xx->dev, "%s: Unsupported frequency %d\n",
                        __func__, freq);
                return -EINVAL;
        }

        /* set clock on MCLK, BCLK, or GPIO1 as PLL input */
        snd_soc_component_update_bits(component, AIC31XX_CLKMUX, AIC31XX_PLL_CLKIN_MASK,
                            clk_id << AIC31XX_PLL_CLKIN_SHIFT);

        aic31xx->sysclk_id = clk_id;
        aic31xx->sysclk = freq;

        return 0;
}
    \end{minted}
  \end{block}
\end{frame}

\begin{frame}{\code{set_fmt}}
  \begin{itemize}
  \item This sets the format of the PCM bus
  \item This is called from the machine driver, using
    \kfunc{snd_soc_dai_set_fmt}
  \item Available formats are:
    \begin{itemize}
    \item \ksym{SND_SOC_DAIFMT_I2S}
    \item \ksym{SND_SOC_DAIFMT_RIGHT_J}
    \item \ksym{SND_SOC_DAIFMT_LEFT_J}
    \item \ksym{SND_SOC_DAIFMT_DSP_A}
    \item \ksym{SND_SOC_DAIFMT_DSP_B}
    \item \ksym{SND_SOC_DAIFMT_AC97}
    \item \ksym{SND_SOC_DAIFMT_PDM}
    \end{itemize}
  \item Also the polarity can be changed:
    \begin{itemize}
    \item \ksym{SND_SOC_DAIFMT_NB_NF}: normal bit clock + normal frame
    \item \ksym{SND_SOC_DAIFMT_NB_IF}: normal bit clock + invert frame
    \item \ksym{SND_SOC_DAIFMT_IB_NF}: invert bit clock + normal frame
    \item \ksym{SND_SOC_DAIFMT_IB_IF}: invert bit clock + invert frame
    \end{itemize}
  \end{itemize}
\end{frame}

\begin{frame}[fragile]{\code{set_fmt}}
  \begin{itemize}
  \item The clock directions can also be set:
    \begin{itemize}
    \item \ksym{SND_SOC_DAIFMT_CBP_CFP}: codec clk provider and frame provider
    \item \ksym{SND_SOC_DAIFMT_CBC_CFP}: codec clk consumer and frame provider
    \item \ksym{SND_SOC_DAIFMT_CBP_CFC}: codec clk provider and frame consumer
    \item \ksym{SND_SOC_DAIFMT_CBC_CFC}: codec clk consumer and frame consumer
    \end{itemize}
  \item These used to have another name:
  \begin{block}{include/sound/soc-dai.h}
    \fontsize{9}{9}\selectfont
    \begin{minted}{c}
/* previous definitions kept for backwards-compatibility, do not use in new contributions */
#define SND_SOC_DAIFMT_CBM_CFM                SND_SOC_DAIFMT_CBP_CFP
#define SND_SOC_DAIFMT_CBS_CFM                SND_SOC_DAIFMT_CBC_CFP
#define SND_SOC_DAIFMT_CBM_CFS                SND_SOC_DAIFMT_CBP_CFC
#define SND_SOC_DAIFMT_CBS_CFS                SND_SOC_DAIFMT_CBC_CFC
    \end{minted}
  \end{block}
  \end{itemize}
\end{frame}

\begin{frame}[fragile]{\code{set_fmt} example}
  \begin{block}{}
    \fontsize{7}{7}\selectfont
    \begin{minted}{c}
static int aic31xx_set_dai_fmt(struct snd_soc_dai *codec_dai,
                               unsigned int fmt)
{
        struct snd_soc_component *component = codec_dai->component;
        u8 iface_reg1 = 0;
        u8 iface_reg2 = 0;
        u8 dsp_a_val = 0;
[...]
        switch (fmt & SND_SOC_DAIFMT_CLOCK_PROVIDER_MASK) {
        case SND_SOC_DAIFMT_CBP_CFP:
                iface_reg1 |= AIC31XX_BCLK_MASTER | AIC31XX_WCLK_MASTER;
                break;
[...]
        }

        /* signal polarity */
        switch (fmt & SND_SOC_DAIFMT_INV_MASK) {
        case SND_SOC_DAIFMT_NB_NF:
[...]
        }

        /* interface format */
        switch (fmt & SND_SOC_DAIFMT_FORMAT_MASK) {
[...]
        }

        snd_soc_component_update_bits(component, AIC31XX_IFACE1,
                            AIC31XX_IFACE1_DATATYPE_MASK |
                            AIC31XX_IFACE1_MASTER_MASK,
                            iface_reg1);
    \end{minted}
  \end{block}
\end{frame}

\begin{frame}{\code{set_tdm_slot}}
  \begin{itemize}
  \item This callback configures the DAI for TDM operation.
  \item \code{slots} is the total number of slots of the TDM stream and
    \code{slot_width} the width of each slot in bit clock cycles.
  \item \code{tx_mask} and \code{rx_mask} are bitmasks specifying the
    active slots of the TDM stream for the specified DAI, i.e. which slots the
    DAI should write to or read from. A set bit means the channel is
    active.
  \item This is called from the machine driver, using
    \kfunc{snd_soc_dai_set_tdm_slot}
  \item This allows to explicitly configure mismatching stream and bus
    sample width.
  \item TDM mode must be disabled when \code{slots} is 0.
  \end{itemize}
\end{frame}

\begin{frame}{\code{trigger}}
  \begin{itemize}
  \item This callback is called when the stream status is updated.
  \item It allows to listen for events.
  \item This is called from the Alsa core, in \kfunc{soc_pcm_trigger}
    using \kfunc{ snd_soc_pcm_dai_trigger}
  \item A component version exists.
  \item Available states are:
    \begin{itemize}
    \item \ksym{SNDRV_PCM_TRIGGER_STOP}
    \item \ksym{SNDRV_PCM_TRIGGER_START}
    \item \ksym{SNDRV_PCM_TRIGGER_PAUSE_PUSH}
    \item \ksym{SNDRV_PCM_TRIGGER_PAUSE_RELEASE}
    \item \ksym{SNDRV_PCM_TRIGGER_SUSPEND}
    \item \ksym{SNDRV_PCM_TRIGGER_RESUME}
    \item \ksym{SNDRV_PCM_TRIGGER_DRAIN}
    \end{itemize}
  \end{itemize}
\end{frame}

\begin{frame}{\code{trigger} example}
  \begin{itemize}
  \item The PCM1789 needs the system clock, bit clock and frame clock
    to be synchronized as soon as it gets out of reset.
  \item With DAPM, those clocks are disabled until a stream is ready
    to be played.
  \item A solution is to reset the device when a stream is played.
  \end{itemize}
\end{frame}

\begin{frame}[fragile]{\code{trigger} example}
  \begin{block}{sound/soc/codecs/pcm1789.c}
    \fontsize{8}{8}\selectfont
    \begin{minted}{c}
static int pcm1789_trigger(struct snd_pcm_substream *substream, int cmd,
                           struct snd_soc_dai *dai)
{
        struct snd_soc_component *component = dai->component;
        struct pcm1789_private *priv = snd_soc_component_get_drvdata(component);
        int ret = 0;

        switch (cmd) {
        case SNDRV_PCM_TRIGGER_START:
        case SNDRV_PCM_TRIGGER_RESUME:
        case SNDRV_PCM_TRIGGER_PAUSE_RELEASE:
                schedule_work(&priv->work);
                break;
        case SNDRV_PCM_TRIGGER_STOP:
        case SNDRV_PCM_TRIGGER_SUSPEND:
        case SNDRV_PCM_TRIGGER_PAUSE_PUSH:
                break;
        default:
                ret = -EINVAL;
        }

        return ret;
}
    \end{minted}
  \end{block}
\end{frame}

\begin{frame}[fragile]{\code{trigger} example}
  \begin{block}{sound/soc/codecs/pcm1789.c}
    \fontsize{8}{8}\selectfont
    \begin{minted}{c}
static void pcm1789_work_queue(struct work_struct *work)
{
        struct pcm1789_private *priv = container_of(work,
                                                    struct pcm1789_private,
                                                    work);

        /* Perform a software reset to remove codec from desynchronized state */
        if (regmap_update_bits(priv->regmap, PCM1789_MUTE_CONTROL,
                               0x3 << PCM1789_MUTE_SRET, 0) < 0)
                dev_err(priv->dev, "Error while setting SRET");
}
    \end{minted}
  \end{block}
\end{frame}

\begin{frame}{\code{set_bias_level}}
  \begin{itemize}
  \item This callback is called by DAPM through
    \kfunc{snd_soc_dapm_set_bias_level} and
    \kfunc{snd_soc_component_set_bias_level} once the component gets
    activated.
  \item It allows to listen for power events.
  \item Available events are:
    \begin{itemize}
    \item \code{SND_SOC_BIAS_ON}: Bias is fully on for audio playback
      and capture operations.
    \item \code{SND_SOC_BIAS_PREPARE}: Prepare for audio operations.
      Called before DAPM switching for
      stream start and stop operations.
    \item \code{SND_SOC_BIAS_STANDBY}: Low power standby state when no
      playback/capture operations are
      in progress. NOTE: The transition time between STANDBY and ON
      should be as fast as possible and no longer than 10ms.
    \item \code{SND_SOC_BIAS_OFF}: Power Off. No restrictions on
      transition times.
    \end{itemize}
  \end{itemize}
\end{frame}

\begin{frame}{\code{set_bias_level} example}
  \begin{itemize}
  \item There are CODECs that won't even listen on the control
    bus until there are clocks on the PCM bus or that will stay
    powered off as much as possible.
  \item A solution is to use regcache.
  \end{itemize}
\end{frame}

\begin{frame}[fragile]{\code{set_bias_level} example}
  \begin{block}{sound/soc/codecs/ssm2518.c}
    \fontsize{8}{8}\selectfont
    \begin{minted}{c}
static int ssm2518_set_bias_level(struct snd_soc_component *component,
        enum snd_soc_bias_level level)
{
        struct ssm2518 *ssm2518 = snd_soc_component_get_drvdata(component);
        int ret = 0;

        switch (level) {
        case SND_SOC_BIAS_ON:
                break;
        case SND_SOC_BIAS_PREPARE:
                break;
        case SND_SOC_BIAS_STANDBY:
                if (snd_soc_component_get_bias_level(component) == SND_SOC_BIAS_OFF)
                        ret = ssm2518_set_power(ssm2518, true);
                break;
        case SND_SOC_BIAS_OFF:
                ret = ssm2518_set_power(ssm2518, false);
                break;
        }

        return ret;
}
    \end{minted}
  \end{block}
\end{frame}

\begin{frame}[fragile]{\code{set_bias_level} example}
  \begin{block}{sound/soc/codecs/ssm2518.c}
    \fontsize{8}{8}\selectfont
    \begin{minted}{c}
static int ssm2518_set_power(struct ssm2518 *ssm2518, bool enable)
{
        int ret = 0;

        if (!enable) {
                ret = regmap_update_bits(ssm2518->regmap, SSM2518_REG_POWER1,
                        SSM2518_POWER1_SPWDN, SSM2518_POWER1_SPWDN);
                regcache_mark_dirty(ssm2518->regmap);
        }

        if (ssm2518->enable_gpio)
                gpiod_set_value_cansleep(ssm2518->enable_gpio, enable);

        regcache_cache_only(ssm2518->regmap, !enable);

        if (enable) {
                ret = regmap_update_bits(ssm2518->regmap, SSM2518_REG_POWER1,
                        SSM2518_POWER1_SPWDN | SSM2518_POWER1_RESET, 0x00);
                regcache_sync(ssm2518->regmap);
        }

        return ret;
}
    \end{minted}
  \end{block}
\end{frame}
