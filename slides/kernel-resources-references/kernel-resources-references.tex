\section{Kernel Resources}

\begin{frame}
  Linux Weekly News
  \frametitle{Kernel Development News}
  \begin{columns}
    \column{0.7\textwidth}
    \begin{itemize}
    \item \url{https://lwn.net/}
    \item The weekly digest off all Linux and free software
      information sources
    \item In depth technical discussions about the kernel
    \item Subscribe to finance the editors (\$7 / month)
    \item Articles available for non subscribers after 1 week.
    \end{itemize}
    \column{0.3\textwidth}
    \includegraphics[width=\textwidth]{slides/kernel-resources-references/lwn.png}
  \end{columns}
\end{frame}

\begin{frame}
  \frametitle{Useful Online Resources}
  \begin{itemize}
  \item Kernel documentation
    \begin{itemize}
    \item \url{https://kernel.org/doc/}
    \end{itemize}
  \item Linux kernel mailing list FAQ
    \begin{itemize}
    \item \url{http://vger.kernel.org/lkml/}
    \item Complete Linux kernel FAQ
    \item Read this before asking a question to the mailing list
    \end{itemize}
  \item Linux kernel mailing lists
    \begin{itemize}
    \item \url{http://lore.kernel.org/}
    \item Easy browsing and referencing of all e-mail threads
    \item Easy access to an mbox in order to answer to e-mails you were
      not Cc'ed to
    \end{itemize}
  \item Kernel Newbies
    \begin{itemize}
    \item \url{https://kernelnewbies.org/}
    \item Articles, presentations, HOWTOs, recommended reading, useful
      tools for people getting familiar with Linux kernel or driver
      development.
    \item Glossary: \url{https://kernelnewbies.org/KernelGlossary}
    \end{itemize}
  \item The \url{https://elinux.org} wiki
\end{itemize}
\end{frame}

\begin{frame}
  \frametitle{International Conferences (1)}
  \begin{columns}
  \column{0.7\textwidth}
  \begin{itemize}
    \item Embedded Linux Conference:
  \begin{itemize}
  \item \url{https://embeddedlinuxconference.com/}
  \item Organized by the Linux Foundation
  \item Once per year, alternating North America/Europe
  \item Very interesting kernel and user space topics for embedded
        systems developers. Many kernel and embedded project maintainers are present.
  \item Presentation slides and videos freely available on
    \url{https://elinux.org/ELC_Presentations} 
  \end{itemize}

    \item Linux Plumbers
  \begin{itemize}
  \item \url{https://linuxplumbersconf.org}
  \item About the low-level plumbing of Linux: kernel, audio, power
    management, device management, multimedia, etc.
  \item Not really a conventional conference with formal
    presentations, but rather a place where contributors on each topic
    meet, share their progress and make plans for work ahead.
  \end{itemize}

  \end{itemize}
  \column{0.3\textwidth}
     \includegraphics[width=\textwidth]{common/elc-logo.png}\\
     \vspace{1cm}
     \includegraphics[width=\textwidth]{common/lpc-logo.jpg}\\
  \end{columns}
\end{frame}

\begin{frame}
  \frametitle{International Conferences (2)}
  \begin{columns}
  \column{0.7\textwidth}
    \begin{itemize}
      \item Kernel Recipes: \url{https://kernel-recipes.org/}
      \begin{itemize}
      \item Well attended conference in Europe (Paris), only one track
        at a time, with a format that really allows for discussions.
    \end{itemize}
    \item linux.conf.au: \url{https://linux.org.au/conf/}
      \begin{itemize}
      \item In Australia / New Zealand
      \item Features a few presentations by key kernel hackers.
      \end{itemize}
    \item Currently, most conferences are available on-line. They
	  are much more affordable and often free.
  \end{itemize}
  \column{0.3\textwidth}
     \includegraphics[width=\textwidth]{slides/kernel-resources-references/kernel-recipes-logo.png}\\
     \vspace{1cm}
     \includegraphics[width=\textwidth]{common/lca-logo.png}
  \end{columns}
\end{frame}

\begin{frame}
  \frametitle{After the course}
  \begin{columns}
  \column{0.5\textwidth}
  Continue to learn:
  \begin{itemize}
  \item Run your labs again on your own hardware. The Nunchuk lab should
        be rather straightforward, but the serial lab will be quite different
	if you use a different processor.
  \item Learn by reading the kernel code by yourself, ask questions and
	propose improvements.
  \item Implement and share drivers for your own hardware, of course!
  \end{itemize}
  \column{0.5\textwidth}
  Hobbyists can make their first contributions by:
  \begin{itemize}
  \item Helping with tasks keeping the kernel code clean and up-to-date:\\
	\url{https://kernelnewbies.org/KernelJanitors/Todo}
  \item Proposing fixes for issues reported by the {\em Coccinelle} tool:\\
	\code{make coccicheck}
  \item Participating to improving drivers in \kdir{drivers/staging}
  \item Investigating and do the triage of issues reported by Coverity Scan:
        \url{https://scan.coverity.com/projects/linux}
  \end{itemize}
  \end{columns}
\end{frame}
