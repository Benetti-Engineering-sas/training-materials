\section{BSP Layers}

\subsection{Introduction to BSP layers in the Yocto Project}

\begin{frame}{BSP layers}
  \begin{center}
    \includegraphics[height=0.8\textheight]{slides/yocto-bsp/yocto-bsp-overview.pdf}
  \end{center}
\end{frame}

\begin{frame}
  \frametitle{BSP layers overview}
  \begin{itemize}
  \item BSP layers are a subset of the layers.
  \item They hold metadata with the purpose of supporting a specific class
    of hardware devices.
    \item They usually provide:
      \begin{itemize}
        \item Hardware configuration files (\code{machines})
        \item Custom kernel and bootloader recipes and configurations
        \item Modules and drivers to enable specific hardware features
          (e.g. multimedia accelerators)
        \item Pre-built user binaries and firmwares
      \end{itemize}
    \item A good practice is to name it \code{meta-<bsp_name>}.
    \item Examples:
      \href{https://git.yoctoproject.org/meta-ti/tree/meta-ti-bsp}{\tt meta-ti-bsp},
      \href{https://github.com/STMicroelectronics/meta-st-stm32mp}{\tt meta-st-stm32mp}.
  \end{itemize}
\end{frame}

\subsection{Hardware configuration files}

\begin{frame}
  \frametitle{Overview 1/2}
  \begin{itemize}
    \item A layer provides one machine file (hardware configuration
      file) per machine it supports.
    \item These configuration files are stored under
      \code{meta-<bsp_name>/conf/machine/*.conf}
    \item The file names correspond to the values set in the
      \code{MACHINE} configuration variable.
      \begin{itemize}
        \item \code{meta-ti/meta-ti-bsp/conf/machine/beaglebone.conf}
        \item \code{MACHINE = "beaglebone"}
      \end{itemize}
    \item Each machine should be described in the \code{README} file
      of the BSP.
  \end{itemize}
\end{frame}

\begin{frame}
  \frametitle{Overview 2/2}
  \begin{itemize}
    \item The hardware configuration file contains configuration
      variables related to the architecture and to the machine
      features.
    \item Some other variables help customize the kernel image or the
      filesystems used.
  \end{itemize}
\end{frame}

\begin{frame}
  \frametitle{Machine configuration}
  \begin{description}
    \item[TARGET\_ARCH] The architecture of the device being built.
    \item[PREFERRED\_PROVIDER\_virtual/kernel] The default kernel.
    \item[MACHINE\_FEATURES] List of hardware features provided by the
      machine, e.g. \code{usbgadget usbhost screen wifi}
    \item[SERIAL\_CONSOLES] Speed and device for the serial consoles to
	    attach. Used to configure \code{getty},
      e.g. \code{115200;ttyS0}
    \item[KERNEL\_IMAGETYPE] The type of kernel image to build, e.g.
      \code{zImage}
  \end{description}
\end{frame}

\begin{frame}
  \frametitle{\code{MACHINE_FEATURES}}
  \begin{itemize}
    \item Lists the hardware features provided by the machine.
    \item These features are used by package recipes to enable or
      disable functionalities.
    \item Some packages are automatically added to the resulting root
    filesystem depending on the feature list.
      \begin{itemize}
        \item The machine feature \code{keyboard} adds the \code{keymaps}
          to the image.
      \end{itemize}
  \end{itemize}
\end{frame}

\begin{frame}[fragile]
  \frametitle{\code{conf/machine/include/cfa10036.inc}}
  \begin{block}{}
    \begin{minted}[fontsize=\scriptsize]{sh}
# Common definitions for cfa-10036 boards
include conf/machine/include/imx-base.inc
include conf/machine/include/tune-arm926ejs.inc

SOC_FAMILY = "mxs:mx28:cfa10036"

PREFERRED_PROVIDER_virtual/kernel ?= "linux-cfa"
PREFERRED_PROVIDER_virtual/bootloader ?= "barebox"
IMAGE_BOOTLOADER = "barebox"
BAREBOX_BINARY = "barebox"
IMAGE_FSTYPES:mxs = "tar.bz2 barebox.mxsboot-sdcard sdcard.gz"
IMXBOOTLETS_MACHINE = "cfa10036"

KERNEL_IMAGETYPE = "zImage"
KERNEL_DEVICETREE = "imx28-cfa10036.dtb"
# we need the kernel to be installed in the final image
IMAGE_INSTALL:append = " kernel-image kernel-devicetree"
SDCARD_ROOTFS ?= "${DEPLOY_DIR_IMAGE}/${IMAGE_NAME}.rootfs.ext3"
SERIAL_CONSOLES = "115200;ttyAMA0"
MACHINE_FEATURES = "usbgadget usbhost vfat"
    \end{minted}
  \end{block}
\end{frame}

\begin{frame}[fragile]{\code{conf/machine/cfa10057.conf}}
  \begin{block}{}
    \begin{minted}[fontsize=\small]{sh}
#@TYPE: Machine
#@NAME: Crystalfontz CFA-10057
#@SOC: i.MX28
#@DESCRIPTION: Machine configuration for CFA-10057, also called CFA-920
#@MAINTAINER: Alexandre Belloni <alexandre.belloni@bootlin.com>

require conf/machine/include/cfa10036.inc

KERNEL_DEVICETREE += "imx28-cfa10057.dtb"

MACHINE_FEATURES += "touchscreen"
    \end{minted}
  \end{block}
\end{frame}

\subsection{Bootloader}

\begin{frame}
  \frametitle{Default bootloader 1/2}
  \begin{itemize}
    \item By default the bootloader used is the mainline version of
      \code{U-Boot}, with a fixed version (per Poky release).
    \item All the magic is done in
      \code{meta/recipes-bsp/u-boot/u-boot.inc}
    \item Some configuration variables used by the U-Boot recipe can
      be customized, in the machine file.
  \end{itemize}
\end{frame}

\begin{frame}
  \frametitle{Default bootloader 2/2}
  \begin{description}
    \item[SPL\_BINARY] If an SPL is built, describes the name of the
      output binary. Defaults to an empty string.
    \item[UBOOT\_SUFFIX] \code{bin} (default) or \code{img}.
    \item[UBOOT\_MACHINE] The target used to build the configuration.
    \item[UBOOT\_ENTRYPOINT] The bootloader entry point.
    \item[UBOOT\_LOADADDRESS] The bootloader load address.
    \item[UBOOT\_MAKE\_TARGET] Make target when building the
      bootloader.  Defaults to \code{all}.
  \end{description}
\end{frame}

\begin{frame}
  \frametitle{Customize the bootloader}
  \begin{itemize}
    \item It is possible to support a custom U-Boot by creating an
      extended recipe and to append extra metadata to the original
      one.
    \item This works well when using a mainline version of U-Boot.
    \item Otherwise it is possible to create a custom recipe.
      \begin{itemize}
        \item Try to still use
          \code{meta/recipes-bsp/u-boot/u-boot.inc}
      \end{itemize}
  \end{itemize}
\end{frame}

\subsection{Kernel}

\begin{frame}
  \frametitle{Linux kernel recipes in Yocto}
  \begin{itemize}
    \item There are mainly two ways of compiling a kernel:
      \begin{itemize}
        \item By creating a custom kernel recipe, inheriting
          \code{kernel.bbclass}
        \item By using the \code{linux-yocto} packages, provided in
          Poky, for very complex needs
      \end{itemize}
    \item The kernel used is selected in the machine file thanks to:
      \code{PREFERRED_PROVIDER_virtual/kernel}
    \item Its version is defined with:
      \code{PREFERRED_VERSION_<kernel_provider>}
  \end{itemize}
\end{frame}

\begin{frame}[fragile]
  \frametitle{Linux Yocto 1/3}
  \begin{itemize}
    \item \code{linux-yocto} is a set of recipes with advanced features to
      build a mainline kernel
    \item \code{PREFERRED_PROVIDER_virtual/kernel = "linux-yocto"}
    \item \usebeamercolor[fg]{code} \path{PREFERRED_VERSION_linux-yocto = "5.14%"}
  \end{itemize}
\end{frame}

\begin{frame}[fragile]
  \frametitle{Linux Yocto 2/3}
  \begin{itemize}
    \item Automatically applies configuration fragments listed in \code{SRC_URI} with a \code{.cfg} extension
      \begin{block}{}
        \begin{minted}{sh}
SRC_URI += "file://defconfig        \
            file://nand-support.cfg \
            file://ethernet-support.cfg"
        \end{minted}
      \end{block}
  \end{itemize}
\end{frame}

\begin{frame}
  \frametitle{Linux Yocto 3/3}
  \begin{itemize}
    \item Another way of configuring \code{linux-yocto} is by using
      \emph{Advanced Metadata}.
    \item It is a powerful way of splitting the configuration and the
      patches into several pieces.
    \item It is designed to provide a very configurable kernel, at the cost
      of higher complexity.
    \item The full documentation can be found at
      \url{https://docs.yoctoproject.org/kernel-dev/advanced.html\#working-with-advanced-metadata-yocto-kernel-cache}
  \end{itemize}
\end{frame}

\begin{frame}
  \frametitle{Linux Yocto: Kernel Metadata 1/2}
  \begin{itemize}
    \item Kernel Metadata is a way to organize and to split the
      kernel configuration and patches in little pieces each providing
      support for one feature.
    \item Two main configuration variables help taking advantage of
      this:
      \begin{description}
        \item[LINUX\_KERNEL\_TYPE] \code{standard} (default),
          \code{tiny} or \code{preempt-rt}
          \begin{itemize}
            \item \code{standard}: generic Linux kernel policy.
            \item \code{tiny}: bare minimum configuration, for small
              kernels.
            \item \code{preempt-rt}: applies the \code{PREEMPT_RT}
              patch.
          \end{itemize}
        \item[KERNEL\_FEATURES] List of features to enable. Features
          are sets of patches and configuration fragments.
      \end{description}
  \end{itemize}
\end{frame}

\begin{frame}[fragile]
  \frametitle{Linux Yocto: Kernel Metadata 2/2}
  \begin{itemize}
    \item Kernel Metadata description files have their own syntax to describe an optional kernel feature
    \item A basic feature is defined as a patch to apply and a
      configuration fragment to add
    \item Simple example, \code{features/nunchuk.scc}
      \begin{block}{}
        \begin{minted}{sh}
define KFEATURE_DESCRIPTION "Enable Nunchuk driver"

kconf hardware enable-nunchuk-driver.cfg
patch Add-nunchuk-driver.patch
        \end{minted}
      \end{block}
    \item To integrate the feature into the kernel image:
      \code{KERNEL_FEATURES += "features/nunchuk.scc"}
  \end{itemize}
\end{frame}
