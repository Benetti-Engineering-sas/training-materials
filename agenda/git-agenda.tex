\documentclass[a4paper,12pt,obeyspaces,spaces,hyphens]{article}

\usepackage{agenda}

\hypersetup{pdftitle={Git training},
  pdfauthor={Bootlin}}

\renewcommand{\arraystretch}{2.0}

\begin{document}

\setlength{\arrayrulewidth}{0.8pt}

\begin{center}
\LARGE
Source control using Git training\\
\large
1-day session
\end{center}
\vspace{1cm}

\small
\newcolumntype{g}{>{\columncolor{fedarkblue}}m{4cm}}
\newcolumntype{h}{>{\columncolor{felightblue}}X}

\arrayrulecolor{lightgray} {
  \setlist[1]{itemsep=-5pt}
  \begin{tabularx}{\textwidth}{|g|h|}
    {\bf Title} & {\bf Source control using Git training} \\
    \hline

    {\bf Overview} &
    Introduction to source control and Git \par
    Git glossary\par
    Git basics\par
    Branches\par
    Remotes\par
    Collaborating\par
    Workflows\par
    \\
    \hline

    {\bf Duration} & {\bf One} day - 8 hours.
    \newline 40\% of lectures, 60\% of practical labs. \\
    \hline

    {\bf Language} & Oral lectures: English, French.
    \newline Materials: English.\\
    \hline

    {\bf Audience} & Companies already using or interested in using
    Git to managed their source code.\\
    \hline

    {\bf Prerequisites} & {\bf None}\\
    \hline
  \end{tabularx}

  \begin{tabularx}{\textwidth}{|g|h|}
    {\bf Required equipment} &
    {\bf For on-site sessions only.}
    \newline Everything is supplied by Bootlin in public
    sessions.
    \begin{itemize}
    \item Video projector
    \item PC computers with at least 8 GB of RAM, and Ubuntu Linux
    installed in a {\bf free partition of at least 20 GB}.
    \item We need Ubuntu Desktop 18.04 (Xubuntu and
    other variants are fine). We don't support other
    distributions, because we can't test all possible package versions.
    \item {\bf Connection to the Internet} (direct or through the
    company proxy).
    \item {\bf PC computers with valuable data must be backed up}
    before being used in our sessions.  Some people have already made
    mistakes during our sessions and damaged work data.
    \end{itemize}\\
    \hline

    {\bf Materials} & Electronic copies of presentations and
    labs.
    \newline Electronic copy of lab files.\\
    \hline

\end{tabularx}}
\normalsize

\section{Morning}

\feagendatwocolumn
{Lecture - Source control}
{
  \begin{itemize}
  \item Version control systems: principles
  \item Centralized vs. distributed version control systems
  \item Git glossary and internal representation of data
  \item Git basics: installation and configuration, getting a repository, making commits, etc.
  \item Basic Git workflow: understanding {\em HEAD}, {\em index} and {\em working directory}
  \end{itemize}
}
{Lab - Basic Git usage}
{
  \begin{itemize}
  \item Installing and setting up Git
  \item Getting a repository
  \item Listing changes
  \item First commits
  \item Undoing changes
  \end{itemize}
}
\\
\feagendatwocolumn
{Lecture - Organizing sources}
{
  \begin{itemize}
  \item Navigating in sources and history
  \item Branches: creation, deletion, merging
  \item Rewriting history with {\em rebase}
  \item Tags
  \item Temporary storage of changes with {\em stash}
  \end{itemize}
}
{Lab - Everyday Git usage}
{
  \begin{itemize}
  \item Initializing a Git repository
  \item Creating and moving between branches
  \item Merging
  \item Rebasing
  \item Cleaning up a branch
  \item Tagging commits
  \end{itemize}
}

\section{Afternoon}

\feagendatwocolumn
{Lecture - Collaborating}
{
  \begin{itemize}
  \item Git {\em remotes}: interacting with other developers
  \item Importing changes: fetching changes from other developers, applying patches
  \item Sending changes: pushing commits and sending patches
  \item Getting more efficient
  \end{itemize}
}
{Lab - Collaborating}
{
  \begin{itemize}
  \item Working with multiple remotes
  \item Creating patches
  \item Applying patches
  \item Exporting a repository
  \item First Pull request
  \end{itemize}
}
\\
\feagendatwocolumn
{Lecture - Workflows}
{
  \begin{itemize}
  \item Centralized workflow, often used in internal company projects
  \item Linux Kernel workflow, often used in open-source communities
  \end{itemize}
}
{Lecture - Advanced source organization}
{
  \begin{itemize}
  \item Referencing Git repositories from another repository with {\em git submodules}
  \item Managing easily a large number of Git repositories with Google's {\em repo} tool
  \item Rewriting branches in an automated way with {\em filter-branch}
  \end{itemize}
}

\end{document}
