\documentclass[a4paper,12pt,obeyspaces,spaces,hyphens]{article}

\def \trainingtype{online}
\def \agendalanguage{english}

\input{agenda/audio.inc}

\usepackage{agenda}

\begin{document}

\feshowtitle

\feshowinfo

\section{Half day 1}

\feagendatwocolumn
{Lecture - Digital Audio Representation}
{
  \begin{itemize}
  \item What is sound?
  \item Sampling theory
  \item Sample size, sample rate
  \item Audio formats: I2S, LJ, RJ, DSPA, DSPB
  \item AC97
  \item IEC 61937 (S/PDIF and HDMI)
  \item PDM
  \end{itemize}
  \vspace{0.5em}
  {\em Introducing the basic notions used for representing audio waveforms.}
}
{Lecture - Hardware}
{
  \begin{itemize}
  \item Signals
  \item CPU Digital Audio Interfaces
  \item CODEC Digital Audio Interfaces
  \item Amplifiers
  \item Clocks and clock providers
  \end{itemize}
  \vspace{0.5em}
  {\em Presenting the hardware involved in the audio playback or capture.}
}

\feagendatwocolumn
{Lecture - Linux kernel ASoC subsystem}
{
  \begin{itemize}
  \item ASoC: the ALSA System-on-Chip subsystem in the Linux kernel
  \item Describing audio cards with Device Tree: {\em
      audio-graph-card}, {\em simple-audio-card}
  \item Linux kernel drivers for audio cards
  \item Linux kernel drivers for audio codecs
  \item Controls
  \item Linux kernel drivers for CPU audio interfaces
  \end{itemize}
  \vspace{0.5em}
  {\em Introducing the Linux kernel subsystem for audio on embedded systems.}
}
{Demo - Audio card examples}
{
  \begin{itemize}
  \item Walk-through of DT-only audio cards
  \item Walk-through of card drivers
  \item Walk-through of audio codec drivers
  \item Walk-through of CPU audio interface drivers
  \end{itemize}
  \vspace{0.5em}
  {\em Presenting existing sound card support.}
}

\section{Half day 2}

\feagendatwocolumn
{Lecture - Linux kernel helpers}
{
  \begin{itemize}
  \item {\em regmap}, {\em regcache}
  \item DMA handling
  \end{itemize}
  \vspace{0.5em}
  {\em Presenting the common helper APIs.}
}
{Lecture - Routing}
{
  \begin{itemize}
  \item Routing audio
  \item DAPM: Dynamic Audio Power Management
  \end{itemize}
  \vspace{0.5em}
  {\em Presenting the audio routes and power managment.}
}

\feagendatwocolumn
{Lecture - More audio components}
{
  \begin{itemize}
  \item Auxiliary devices, amplifiers, muxing
  \item Jack detection
  \item Asynchronous Sample Rate Converter
  \end{itemize}
  \vspace{0.5em}
  {\em Presenting more components of the sound cards.}
}
{Demo - Complex audio card examples}
{
  \begin{itemize}
  \item Examples of DT-only complex audio cards
  \item Examples of complex card drivers
  \end{itemize}
  \vspace{0.5em}
  {\em Presenting existing advanced sound card support.}
}

\section{Half day 3}

\feagendatwocolumn
{Lecture - Troubleshooting}
{
  \begin{itemize}
  \item Debugging
  \item {\em vizdapm}
  \end{itemize}
  \vspace{0.5em}
  {\em Presenting how to debug common issues.}
}
{Demo - Debugging}
{
  \begin{itemize}
  \item Examples of common issues and their resolutions
  \end{itemize}
}

\feagendatwocolumn
{Lecture - Userspace, hardware configuration}
{
  \begin{itemize}
  \item ALSA plugins
  \item \code{asound.conf}
  \item Sound card configuration
  \end{itemize}
  \vspace{0.5em}
  {\em Configuring the userspace audio paths and effects.}
}
{Demo - Card configuration examples}
{
  \begin{itemize}
  \item Reordering channels
  \item Splitting channels
  \item Resampling
  \item Mixing
  \item LADSPA
  \end{itemize}
  \vspace{0.5em}
  {\em Exercising the ALSA plugins.}
}

\feagendatwocolumn
{Lecture - Userspace, controls configuration}
{
  \begin{itemize}
  \item {\em amixer}
  \item {\em alsamixer}
  \item Userspace API
  \item Saving state: {\em alsactl}, \code{asound.state}
  \end{itemize}
  \vspace{0.5em}
  {\em Configuring the userspace audio paths and effects.}
}
{Demo - Configuring controls}
{
  \begin{itemize}
  \item {\em alsamixer} demonstration
  \item \code{asound.state} examples
  \item Custom application
  \end{itemize}
  \vspace{0.5em}
  {\em Configuring the sound card controls.}
}

\section{Half day 3}

\feagendaonecolumn
{Lecture - Userspace, playing and capturing audio}
{
  \begin{itemize}
  \item Userspace ALSA API
  \end{itemize}
  \vspace{0.5em}
  {\em Playing and capturing audio samples.}
}

\feagendatwocolumn
{Lecture - Pipewire}
{
  \begin{itemize}
  \item Pipewire introduction
  \item Pipewire configuration
  \item Pipewire tools (pwdump, pw-cli, ...)
  \item Pipewire session and policy management
  \item Pipewire modules and filtering
  \item {\em WirePlumber}
  \end{itemize}
  \vspace{0.5em}
  {\em Using Pipewire as the sound server.}
}
{Demo - Pipewire}
{
  \begin{itemize}
  \item Running pipewire on the target
  \item Inspecting the configuration and setting properties
  \item Dynamic routing and patchbay
  \item Using modules and Filter-Chain
  \end{itemize}
  \vspace{0.5em}
  {\em Running Pipewire and exercising advanced configuration.}
}

\feagendatwocolumn
{Lecture - The GStreamer multimedia framework}
{
  \begin{itemize}
  \item {\em GStreamer}
  \item GStreamer pipelines
  \item GStreamer plugins
  \end{itemize}
  \vspace{0.5em}
  {\em Using Gstreamer for audio capture and playback.}
}
{Demo - GStreamer}
{
  \begin{itemize}
  \item Running gstreamer on the target
  \item Creating multiple pipelines
  \end{itemize}
  \vspace{0.5em}
  {\em Running Gstreamer using different audio pipelines.}
}

\end{document}
