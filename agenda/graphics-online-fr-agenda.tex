\documentclass[a4paper,12pt,obeyspaces,spaces,hyphens]{article}

\def \trainingtype{online}
\def \agendalanguage{french}

\input{agenda/graphics.inc}

\usepackage{agenda}

\begin{document}

\feshowtitle

\feagendasummaryitem{Titre}{
  {\bf \trainingtitle{}}
}
\feagendasummaryitem{Objectifs\newline opérationnels}{
  \traininggoals{}
}
\feagendasummaryitem{Durée}{
  \feshowduration{}
}
\feagendasummaryitem{Méthodes\newline pédagogiques}{
  \begin{itemize}
    \vspace{-0.5cm}
  \item Présentations animées par le formateur, par
    visioconférence. Les participants peuvent poser des questions
    à tout instant.
  \item Démonstrations pratiques réalisées par le formateur, par
    vidéo-conférence. Les participants peuvent poser des questions
    à tout instant.
  \item Messagerie instantanée pour questions entre les sessions
    (réponse sous 24h, hors week-end et jours fériés)
  \item Version électronique des supports de présentation, des
    instructions et des données de travaux pratiques. Les supports
    sont librement disponibles sur
    \href{https://bootlin.com/doc/training/graphics}{bootlin.com/doc/training/graphics}.
    \vspace{-0.5cm}
  \end{itemize}
}
\feagendasummaryitem{Formateur}{
  Un des ingénieurs mentionnés sur :
  \newline \url{https://bootlin.com/training/trainers/}
}
\feagendasummaryitem{Langue}{
  \traininglanguages{}
}
\feagendasummaryitem{Public visé}{
  Développeurs de systèmes multimédia utilisant le
  noyau Linux
}
\feagendasummaryitem{Pré-requis}{
  \begin{itemize}
    \prerequisiteclanguage
    \prerequisitekernel
    \prerequisiteenglish
  \end{itemize}
}
\feagendasummaryitem{Équipement nécessaire}{
  \begin{itemize}
  \item Ordinateur avec le système d'exploitation de votre choix, équipé du
    navigateur Google Chrome ou Chromium pour la conférence vidéo.
  \item Une webcam et un micro (de préférence un casque avec micro)
  \item Une connexion à Internet à haut débit
  \end{itemize}
}
\certificate{}
\disabilities{}

\section{1\textsuperscript{ère} demi-journée}

\feagendatwocolumn
{Présentation - Représentation des images et des couleurs}
{
  \begin{itemize}
  \item Lumière, pixels et images
  \item Échantillonage, domaine de fréquence, aliasing
  \item Quantification et représentation des couleurs
  \item Espaces colorimétriques et canaux, canal alpha
  \item Sous-échantillonnage YUV et chroma
  \item Plans de données de pixels, ordre d'analyse
  \item Formats de pixels, codes FourCC codes, modificateurs
  \end{itemize}
  \vspace{0.5em}
  {\em Introduction aux notions de base utilisées pour représenter les images en couleur.}
}
{Présentation - Dessin des pixels}
{
  \begin{itemize}
  \item Accès aux données de pixels et itération
  \item Concepts autour de la pixellisation
  \item Dessin de rectangles
  \item Dessin de gradients linéaires
  \item Dessin de disques
  \item Dessin de gradients circulaires
  \item Dessin de lignes
  \item Aliasing de lignes et de formes, dessin sub-pixel
  \item Cercles et coordonnées polaires
  \item Courbes paramétriques
  \end{itemize}
  \vspace{0.5em}
  {\em Comment accéder aux données de pixels en mémoire et dessiner des formes simples.}
}
\\
\feagendatwocolumn
{Présentation - Opérations sur les pixels}
{
  \begin{itemize}
  \item Copie de région
  \item Alpha blending
  \item Keying de couleur
  \item Mise à l'échelle et interpolation
  \item Filtrage linéaire et convolution
  \item Filtres de floutage
  \item Dithering
  \end{itemize}
  \vspace{0.5em}
  {\em Notions de base autour du filtrage, avec des exemples d'utilisation très courants.}
}
{Démo - Dessin et opérations}
{
  \begin{itemize}
  \item Exemples de dessin de divers types de formes et de régions
  \item Exemples d'opérations de base sur les pixels
  \end{itemize}
  \vspace{0.5em}
  {\em Illustration des concepts présentés au fur et à mesure.}
}

\section{2\textsuperscript{ème} demi-journée}

\feagendatwocolumn
{Présentation - Vue d'ensembe des composants du pipeline et généralités}
{
  \begin{itemize}
  \item Types d'implémentations de matériel graphique
  \item Mémoire graphique et buffers
  \item Pipelines graphiques
  \item Vue d'ensemble du matériel d'affichage, de rendu et de vidéo
  \end{itemize}
  \vspace{0.5em}
  {\em Présentation du matériel impliqué dans les pipelines graphiques.}
}
{Présentation - Matériel d'affichage}
{
  \begin{itemize}
  \item Technologies d'affichage visuel : CRT, plasma, LCD, OLED, EPD
  \item Timings d'affichage, modes et EDID
  \item Interfaces d'affichage : VGA, DVI, HDMI, DP, LVDS, DSI, DP
  \item Bridges et transcodeurs
  \end{itemize}
  \vspace{0.5em}
  {\em Présentation du fonctionnement interne du matériel d'affichage.}
}
\\

\feagendatwocolumn
{Présentation - Spécificités du matériel de rendu}
{
  \begin{itemize}
  \item Digital Signal Processors (DSPs)
  \item Accélérateurs matériels dédiés
  \item Graphics Processing Unit (GPUs)
  \end{itemize}
  \vspace{0.5em}
  {\em Description de l'architecture du matériel de traitement et de rendu.}
}
{Présentation - Intégration système, mémoire et performance}
{
  \begin{itemize}
  \item Intégration graphique et mémoire
  \item Mémoire partagée pour les graphiques
  \item Contraintes et performance de la mémoire graphique
  \item Soulager le processeur en utilisant du matériel dédié au graphisme
  \item Conseils pour les performances graphiques
  \end{itemize}
  \vspace{0.5em}
  {\em Sujets autour de l'intégration système, la gestion de la mémoire et les performances.}
}

\section{3\textsuperscript{ème} demi-journée}

\feagendatwocolumn
{Présentation - Pile d'affichage}
{
  \begin{itemize}
  \item Vue d'ensemble indépendante du système : noyau, affichage et
        rendu en espace utilisateur
  \item Vue d'ensemble de la partie dans le noyau Linux
  \item Vue d'ensemble de la partie bas niveau en espace utilisateur
  \item X Window et Wayland
  \item Bibliothèques graphiques haut niveau et environnements de bureau
  \end{itemize}
  \vspace{0.5em}
  {\em Présentation des composants nécessaires à un traitement graphique
       moderne, et comment ceux-ci sont répartis entre les espace noyau et
       utilisateur}
}
{Présentation - Aspects noyau TTY et device framebuffer}
{
  \begin{itemize}
  \item Introduction au sous-système TTY de Linux
  \item Terminaux virtuels et graphiques
  \item Basculer entre terminaux virtuels et graphiques
  \end{itemize}
  \vspace{0.5em}
  \begin{itemize}
  \item Vue d'ensemble de fbdev
  \item Opérations de base de fbdev
  \item Limitations de fbdev
  \end{itemize}
  \vspace{0.5em}
  {\em Comment les TTYs interagissent avec les graphiques sous Linux et
       brève présentation de fbdev et pourquoi ce composant n'est plus
       recommandé}
}
\\

\feagendatwocolumn
{Présentation - DRM dans le noyau}
{
  \begin{itemize}
  \item Devices DRM
  \item Identification et fonctionnalités des pilotes DRM
  \item Maître DRM, "magic authentification"
  \item Gestion de la mémoire des DRM
  \item API "dumb buffer" de DRM KMS
  \item Modificateurs et FourCCs dans DRM
  \item Détection des ressources dans DRM KMS
  \item Modes DRM KMS
  \item Gestion de framebuffer dans DRM KMS
  \item Ancien système de configuration de DRM KMS et échange de pages
  \item Notification d´évènements dans DRM
  \item Propriétés d'objets dans DRM KMS
  \item DRM KMS atomic
  \item Rendu DRM
  \item Partage mémoire sans copie (dma-buf) avec DRM Prime
  \item Barrières d'objets DRM sync
  \item Débug et documentation dans DRM
  \end{itemize}
  \vspace{0.5em}
  {\em Une présentation complète de l'interface DRM.}
}
{Démo - Aspects noyau}
{
  \begin{itemize}
  \item Terminaux virtuels et TTYs dans Linux
  \item Configuration des modes DRM KMS
  \item Visite guidée d'un pilote DRM KMS
  \item Visite guidée d'un pilote de rendu DRM
  \end{itemize}
  \vspace{0.5em}
  {\em Illustration du fonctionnement en espace noyau.}
}

\section{4\textsuperscript{ème} demi-journée}

\feagendatwocolumn
{Présentation - Aspects X Window en espace utilisateur}
{
  \begin{itemize}
  \item Protocole X11 et son architecture
  \item Extensions au protocole X11
  \item Architecture de Xorg et accélération
  \item Présentation des pilotes Xorg
  \item Accélération X11 et OpenGL : GLX et DRI2
  \item Utilisation de Xorg, intégration et configuration
  \item Principaux problèmes avec X11
  \item Débug et documentation de Xorg
  \end{itemize}
  \vspace{0.5em}
  {\em Présentation des tous les aspects de X11 et Xorg.}
}
{Présentation - Aspects Wayland en espace utilisateur}
{
  \begin{itemize}
  \item Vue d'ensemble et paradigmes de Wayland
  \item Protocole Wayland et son architecture
  \item Détails sur le coeur du protocole Wayland
  \item Protocoles supplémentaires de Wayland
  \item Interface asynchrone de Wayland
  \item Intégration OpenGL de Wayland
  \item Statut et adoption de Wayland
  \item Débug et documentation de Wayland
  \end{itemize}
  \vspace{0.5em}
  {\em Une présentation approfondie de Wayland.}
}\\

\feagendatwocolumn
{Présentation - Aspects Mesa 3D en espace utilisateur}
{
  \begin{itemize}
  \item APIs de rendu 3D standardisées : OpenGL, OpenGL ES, EGL and Vulkan
  \item Vue d'ensemble de Mesa 3D
  \item Principaux détails d'implémentation de Mesa 3D
  \item Détails internes de Mesa 3D : Gallium 3D
  \item Détails internes de Mesa 3D : représentations intermédiaires
  \item Generic Buffer Management (GBM) dans Mesa 3D
  \item Point sur la prise en charge du matériel par Mesa 3D
  \item Mesa 3D comparée aux implémentations propriétaires
  \item Prise en charge du matériel par Mesa 3D : débug et documentation
  \end{itemize}
  \vspace{0.5em}
  {\em Présentation des APIs 3D et implémentation de Mesa 3D.}
}
{Démo - Aspects en espace utilisateur}
{
  \begin{itemize}
  \item Visite guidée du code de Xorg
  \item Visite guidée du coeur du compositeur Wayland
  \item Exemples de clients Wayland
  \item Visite guidée du code de Mesa
  \item Exemples OpenGL et EGL
  \end{itemize}
  \vspace{0.5em}
  {\em Illustration des aspects en espace utilisateur et
       d'implémentations de clients et de serveurs.}
}

\feagendaonecolumn
{Questions / réponses}
{
  \begin{itemize}
  \item Questions et réponses avec les participants à propos des sujets abordés.
  \item Présentations supplémentaires s'il reste du temps, en fonction des demandes
        de la majorité des participants.
  \end{itemize}
}

\end{document}
