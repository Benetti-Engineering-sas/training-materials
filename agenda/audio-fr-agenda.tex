\documentclass[a4paper,12pt,obeyspaces,spaces,hyphens]{article}

\def \trainingtype{onsite}
\def \agendalanguage{french}

\input{agenda/audio.inc}

\usepackage{agenda}

\begin{document}

\feshowtitle

\feshowinfo

\section{1\textsuperscript{er} jour - Matin}

\feagendatwocolumn
{Cours - Représentation audio numérique}
{
  \begin{itemize}
  \item Qu'est-ce que le son?
  \item Théorie de l'échantillonage
  \item Taille des échantillons, fréquence d'échantillonage
  \item Formats audio : I2S, LJ, RJ, DSPA, DSPB
  \item AC97
  \item IEC 61937 (S/PDIF and HDMI)
  \item PDM
  \end{itemize}
  \vspace{0.5em}
  {\em Introduction des notions de base utilisées pour représenter des ondes audio.}
}
{Cours - Matériel}
{
  \begin{itemize}
  \item Signaux
  \item Interfaces audio numériques sur les System-on-chip
  \item Interfaces audio numériques sur les codecs audio
  \item Amplificateurs
  \item Horloges et fournisseurs d'horloges
  \end{itemize}
  \vspace{0.5em}
  {\em Présentation du matériel impliqué dans la lecture ou l'enregistrement audio.}
}

\feagendaonecolumn
{Cours - Le sous-système ASoC du noyau Linux}
{
  \begin{itemize}
  \item ASoC : le sous-système ALSA pour les System-on-chip dans le noyau Linux
  \item Description des cartes audio dans le Device Tree : {\em
      audio-graph-card}, {\em simple-audio-card}
  \item Drivers dans le noyau Linux pour les cartes audio
  \item Drivers dans le noyau Linux pour les codecs audio
  \item Controles audio dans le kernel
  \item Drivers dans le noyau Linux pour les interfaces audio des
    System-on-chip
  \end{itemize}
  \vspace{0.5em}
  {\em Introduction au sous-système du noyau Linux pour la gestion audio dans les systèmes embarqués.}
}

\section{1\textsuperscript{er} jour - Après-midi}

\feagendatwocolumn
{Cours - Mécanismes supplémentaires du noyau Linux relatifs à l'audio}
{
  \begin{itemize}
  \item {\em regmap}, {\em regcache}
  \item Support DMA
  \end{itemize}
  \vspace{0.5em}
  {\em Présentation des mécanismes du noyau Linux couramment utilisés en lien avec le support audio.}
}
{Cours - Autres composants audio}
{
  \begin{itemize}
  \item Périphériques auxilliaires, amplificateurs, muxing
  \item Detection Jack
  \item Convertisseur de fréquence d'échantillonage asynchrone
  \end{itemize}
  \vspace{0.5em}
  {\em Présentation de composants supplémentaires utilisés pour certaines cartes audio.}
}

\feagendaonecolumn
{Cours - Routage}
{
  \begin{itemize}
  \item Routage de flux audio
  \item DAPM : {\em Dynamic Audio Power Management}
  \end{itemize}
  \vspace{0.5em}
  {\em Présentation du routage audio et de la gestion d'énergie pour l'audio.}
}

\section{2\textsuperscript{ème} jour - Matin}

\feagendatwocolumn
{Cours - Espace utilisateur et configuation matérielle}
{
  \begin{itemize}
  \item Plug-ins ALSA
  \item \code{asound.conf}
  \item Configuration de cartes son en espace utilisateur
  \end{itemize}
  \vspace{0.5em}
  {\em Configuration des chemins et effets audio depuis l'espace utilisateur.}
}
{Démo - Exemples de configuration de cartes audio}
{
  \begin{itemize}
  \item Changement d'ordre des canaux audio
  \item Séparation de canaux audio
  \item Ré-échantillonage
  \item Mixing
  \item LADSPA
  \end{itemize}
  \vspace{0.5em}
  {\em Mise en oeuvre des plug-ins ALSA}
}

\feagendatwocolumn
{Cours - Espace utilisateur et configuration des contrôles audio}
{
  \begin{itemize}
  \item {\em amixer}
  \item {\em alsamixer}
  \item API en espace utilisateur
  \item Sauvegarde de l'état des contrôles : {\em alsactl}, \code{asound.state}
  \end{itemize}
  \vspace{0.5em}
  {\em Configuration des chemins et effets audio depuis l'espace utilisateur.}
}
{Démo - Configuration des contrôles}
{
  \begin{itemize}
  \item Démonstration de {\em alsamixer}
  \item Exemples de fichiers \code{asound.state}
  \item Application custom
  \end{itemize}
  \vspace{0.5em}
  {\em Configuration des contrôles de cartes audio.}
}

\feagendaonecolumn
{Cours - Espace utilisateur : lecture et enregistrement audio}
{
  \begin{itemize}
  \item API de ALSA en espace utilisateur
  \end{itemize}
  \vspace{0.5em}
  {\em Lecture et enregistrement de samples audio}
}

\section{2\textsuperscript{ème} jour - Après-midi}

\feagendatwocolumn
{Cours - Résolution de problèmes}
{
  \begin{itemize}
  \item Résolution de problèmes courants
  \item {\em vizdapm}
  \end{itemize}
  \vspace{0.5em}
  {\em Présentation de la résolution de problèmes courants.}
}
{Démo - Résolution de problèmes}
{
  \begin{itemize}
  \item Exemples de problèmes courants et leur résolution.
  \end{itemize}
}

\feagendatwocolumn
{Cours - Pipewire}
{
  \begin{itemize}
  \item Introduction à Pipewire
  \item Configuration de Pipewire
  \item Outils autour de Pipewire (pwdump, pw-cli, ...)
  \item Gestion de session et de {\em policy} avec Pipewire
  \item Modules de Pipewire et filtrage
  \item {\em WirePlumber}
  \end{itemize}
  \vspace{0.5em}
  {\em Utiliser Pipewire comme serveur audio.}
}
{Démo - Pipewire}
{
  \begin{itemize}
  \item Lancer pipewire sur une cible embarquée
  \item Inspecter la configuration et les propriétés
  \item Routage dynamique et {\em patchbay}
  \item Utilisation des modules et de {\em Filter-Chain}
  \end{itemize}
  \vspace{0.5em}
  {\em Utiliser Pipewire et des configurations avancées.}
}

\feagendatwocolumn
{Cours - Le framework multimedia GStreamer}
{
  \begin{itemize}
  \item {\em GStreamer}
  \item Pipelines GStreamer
  \item Plugins GStreamer
  \end{itemize}
  \vspace{0.5em}
  {\em Utiliser GStreamer pour la lecture et l'enregistrement audio.}
}
{Démo - GStreamer}
{
  \begin{itemize}
  \item Lancer gstreamer sur une cible embarquée
  \item Créer différents pipelines
  \end{itemize}
  \vspace{0.5em}
  {\em Exécuter Gstreamer et utiliser différents pipelines audio.}
}

\end{document}
