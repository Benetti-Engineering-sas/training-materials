\documentclass[a4paper,12pt,obeyspaces,spaces,hyphens]{article}

\def \trainingtype{online}
\def \agendalanguage{english}

\input{agenda/autotools.inc}

\usepackage{agenda}

\begin{document}

\feshowtitle

\feagendasummaryitem{Title}{
  {\bf \trainingtitle{}}
}
\feagendasummaryitem{Training objectives}{
  \traininggoals{}
}
\feagendasummaryitem{Duration}{
  \feshowduration{}
}
\onsitepedagogics{40}{60}{autotools}
\feagendasummaryitem{Trainer}{
  {\bf Thomas Petazzoni}. Thomas is a major
  Buildroot developer since 2009, an activity through which he has
  gained a good knowledge of {\em autoconf}, {\em automake} and {\em
    libtool}.
}
\feagendasummaryitem{Language}{
  Oral lectures: English, French.
  \newline Materials: English.
}
\feagendasummaryitem{Audience}{
  Companies already using or interested in using
  {\em autotools} to build their software components.
}
\feagendasummaryitem{Prerequisites}{
  \begin{itemize}
    \prerequisitecommandline
  \end{itemize}
}
\feagendasummaryitem{Required equipment}{
  \begin{itemize}
  \item Computer with the operating system of your choice, with the
    Google Chrome or Chromium browser for videoconferencing
  \item Webcam and microphone (preferably from an audio headset)
  \item High speed access to the Internet
  \item For people interested in our optional practical labs,
    an installation of VirtualBox and about 30 GB of free
    disk space.
  \end{itemize}
}
\certificate{}
\disabilities{}

\section{Half day 1}

\feagendatwocolumn
{Lecture - Overview and usage of {\em autotools}}
{
  \begin{itemize}
  \item What the {\em autotools} are, what the alternatives are, and
    what they are useful for.
  \item Usage of an existing software component using the {\em
      autotools}: configuring and building the software component.
  \item Standard Makefile targets, filesystem hierarchy, configuration variables
  \item System types: build, host, target
  \item Cross-compilation
  \item Out of tree build
  \item Diverted installation
  \item Cache variables
  \item Using {\em autoreconf}
  \end{itemize}
}
{Lab - Usage of an existing software component using the {\em autotools}}
{
  \begin{itemize}
  \item First build of an {\em autotools} package
  \item Out-of-tree build and cross-compilation
  \item Overriding cache variables
  \item Using {\em autoreconf}
  \end{itemize}
}

\feagendaonecolumn
{Lecture - autoconf/automake: the basics}
{
  \begin{itemize}
  \item \code{configure.ac} language and basic macros
  \item \code{AC_CONFIG_FILES} and {\em output variables}
  \item Minimal \code{Makefile.am}
  \end{itemize}
}

\feagendaonecolumn
{Lab - autoconf/automake: the basics}
{
  \begin{itemize}
  \item Your first \code{configure.ac}
  \item Adding and building a program
  \item Going further: \code{autoscan} and \code{make dist}
  \end{itemize}
}

\section{Half day 2}

\feagendatwocolumn
{Lecture - Autoconf advanced}
{
  \begin{itemize}
  \item Configuration header
  \item Checking for functions, headers, libraries
  \item Custom tests
  \item Handling external software and optional features
  \item \code{pkg-config}
  \end{itemize}
}
{Lecture - Automake advanced}
{
  \begin{itemize}
  \item Subdirectories
  \item Conditionals
  \item Shared libraries
  \item Misc: variables, macro and auxiliarly directories, silent
    rules, etc.
  \end{itemize}
}

\feagendaonecolumn
{Lab - Implement more advanced options}
{
  \begin{itemize}
  \item Use \code{AC_ARG_ENABLE} and \code{config.h}
  \item Implement a shared library
  \item Switch to multiple directories
  \item Make the compilation of programs conditional
  \item Use \code{pkg-config}
  \end{itemize}
}

\end{document}
