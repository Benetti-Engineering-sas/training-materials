\section{Writing the kernel and DTB on the SD card}

In order to let the kernel boot on the board autonomously, we can
copy the kernel image and DTB in the boot partition we created
previously.

Insert the SD card in your PC, it will get auto-mounted. Copy the
kernel and device tree:

\begin{bashinput}
$ sudo cp arch/arm/boot/dts/stm32mp157a-dk1.dtb arch/arm/boot/zImage /media/$USER/boot/
$ sudo umount /media/$USER/boot
\end{bashinput}
\normalsize

Insert the SD card back in the board and reset it. You should now be
able to load the DTB and kernel image from the SD card and boot with:

\begin{ubootinput}
=> ext2load mmc 0:4 0xc0000000 zImage
=> ext2load mmc 0:4 0xc4000000 stm32mp157a-dk1.dtb
=> bootz 0xc0000000 - 0xc4000000
\end{ubootinput}

You are now ready to modify \code{bootcmd} to boot the board
from SD card. But first, save the settings for booting from
\code{tftp}:

\begin{ubootinput}
=> setenv bootcmdtftp %\$%{bootcmd}
\end{ubootinput}

This will be useful to switch back to \code{tftp} booting mode
later in the labs.

Finally, using \code{editenv bootcmd}, adjust \code{bootcmd} so that
the Discovery board starts using the kernel from the SD card.

Now, reset the board to check that it boots in the same way from the
SD card. Check that this is really your own version of the kernel
that's running\footnote{Look at the kernel log. You will find the
kernel version number as well as the date when it was compiled.
That's very useful to check that you're not loading an older version
of the kernel instead of the one that you've just compiled.}.

