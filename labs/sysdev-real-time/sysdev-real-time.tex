\subchapter{Real-time - Timers and scheduling latency}{Objective:
  Learn how to handle real-time processes and practice with the
  different real-time modes. Measure scheduling latency.}

After this lab, you will:
\begin{itemize}
\item Be able to check clock accuracy.
\item Be able to start processes with real-time priority.
\item Be able to build a real-time application against the standard
POSIX real-time API, and against Xenomai's POSIX skin.
\item Have compared scheduling latency on your system, between a
  standard kernel, a kernel with \code{PREEMPT_RT} and a kernel
  with Xenomai.
\end{itemize}

\section{Setup}

Go to the \code{$HOME/__SESSION_NAME__-labs/realtime/} directory.

Install the \code{netcat} package.

\section{Testing Linux, with and without preemption}

In this section, we will compare the real-time behavior of the Linux
kernel in four different configurations:

\begin{itemize}
\item Without preemption, without high-resolution timers
\item Without preemption, with high-resolution timers
\item With preemption enabled from the mainline kernel
\item With full preemption enabled from the {\em PREEMPT\_RT} project
\end{itemize}

First of all, let's prepare a root filesystem for our experiments.

\subsection{Root filesystem}

Create an \code{nfsroot} directory.

To compare real-time latency between standard Linux and Xenomai, we
are going to need a root filesystem and a build environment that
supports Xenomai. Let's build this with Buildroot.

Download and extract the latest Buildroot 2021.02.X sources.

Configure Buildroot with the following settings,
using the \code{/} command in \code{make
menuconfig} to find parameters by their name:

\begin{itemize}
\item In \code{Target}:
   \begin{itemize}
   \item \code{Target architecture}: \code{ARM (little endian)}
   \item \code{Target Architecture Variant}: \code{cortex-A5}
   \item Enable the \code{Enable VFP extension support} option
   \end{itemize}
\item In \code{Toolchain}:
   \begin{itemize}
   \item \code{Toolchain type}: \code{External toolchain}
   \item \code{Toolchain}: \code{Linaro ARM 2018.05}
   \end{itemize}
\item In \code{System configuration}:
   \begin{itemize}
   \item in \code{Run a getty (login prompt) after boot},  \code{TTY port}: \code{ttyS0}
   \end{itemize}
\item In \code{Target packages}:
   \begin{itemize}
   \item Enable \code{Show packages that are also provided by busybox}.
         We need this to build the standard \code{netcat} command, not
         provided in the default BusyBox configuration.
   \item In \code{Networking applications}, enable \code{netcat}
   \end{itemize}
\end{itemize}

Now, build your root filesystem.

At the end of the build job, extract the
\code{output/images/rootfs.tar} archive in the \code{nfsroot}
directory.

The last thing to do is to add a few files that we will need in our
tests:

\begin{terminalinput}
$ cp data/* nfsroot/root
\end{terminalinput}

\subsection{Downloading Linux kernel sources and patches}

We will use a recent kernel version that is supported by the {\em
  PREEMPT\_RT} patchset.

So, go to \url{https://kernel.org/pub/linux/kernel/projects/rt/} and
download the patch \code{patch-5.11.4-rt11.patch.xz}: it is the {\em
  PREEMPT\_RT} patch for Linux 5.11.4.

Then go to \url{https://kernel.org} and download the exact version
(including at the update number level) corresponding to the patch you
downloaded.

\subsection{Compile a standard Linux kernel}

Extract the sources of your 5.11.4 kernel but don't apply the {\em
  PREEMPT\_RT} patches yet.

Configure your kernel for your Xplained board, and then make sure
that the below settings are disabled:
\kconfig{CONFIG_PROVE_LOCKING}, \kconfig{CONFIG_DEBUG_LOCK_ALLOC},
\kconfig{CONFIG_DEBUG_MUTEXES} and \kconfig{CONFIG_DEBUG_SPINLOCK}.

Also, for the moment, disable the \kconfig{CONFIG_HIGH_RES_TIMERS}
option which impact we want to measure.

Compile and boot the Xplained board by mounting the root filesystem that you
built. As usual, login as \code{root}, there is no password.

\subsection{Measure the effect of high-resolution timers}

The root filesystem was built with the {\em glibc} C library, because
it has better support for the POSIX RT API. We need to compile our
test application with the toolchain that Buildroot used.

Let's configure our \code{PATH} to use this toolchain:

%\simall
\begin{terminalinput}
$ export PATH=$HOME/__SESSION_NAME__-labs/realtime/buildroot-2021.02.X/output/host/usr/bin:$PATH
\end{terminalinput}
\normalsize

Have a look at the \code{rttest.c} source file available in
\code{root/} in the \code{nfsroot/} directory. See how it shows the
resolution of the \code{CLOCK_MONOTONIC} clock.

Now compile this program:
\begin{terminalinput}
$ arm-linux-gnueabihf-gcc -o rttest rttest.c
\end{terminalinput}

Execute the program on the board. Is the clock resolution good or bad?
Compare it to the timer tick of your system, as defined by
\kconfig{CONFIG_HZ}.

Copy the results in a file, in order to be able to compare them
with further results.

Obviously, this resolution will not provide accurate sleep times, and
this is because our kernel doesn't use high-resolution timers. So
let's add back the \kconfig{CONFIG_HIGH_RES_TIMERS} option in the kernel
configuration.

Recompile your kernel, boot your Xplained with the new version, and
check the new resolution. Better, isn't it?

\subsection{Testing the non-preemptible kernel}

Now, do the following tests:
\begin{itemize}
\item Run the \code{rttest} program with nothing special and write
  down the results.
\item Run \code{rttest} again and at the same time, add some workload
  to the board, by running \code{/root/doload 300 > /dev/null 2>&1 &}
  on the board, and using \code{netcat 192.168.0.100 5566} on your
  workstation in order to flood the network interface of the Xplained
  board (where \code{192.168.0.100} is the IP address of the Xplained
  board).
\item Run \code{rttest} once again with the workload, but by running
  the program in the \code{SCHED_FIFO} scheduling class at priority
  \code{99}, using the \code{chrt} command.
\end{itemize}

\subsection{Testing the preemptible kernel}

Recompile your kernel with \kconfig{CONFIG_PREEMPT} enabled, which
enables kernel preemption (except for critical sections protected by
spinlocks).

Run the \code{rttest} program again, with workload, under this new
preemptible kernel and compare the results.

\subsection{Compiling and testing the PREEMPT\_RT kernel}

Apply the {\em PREEMPT\_RT} patch that you have downloaded previously
to your Linux kernel sources.

Configure your kernel with \kconfig{CONFIG_PREEMPT_RT} and boot it.

Repeat the tests and compare the results again. You should see a massive
improvement in the maximum latency.

\section{Testing Linux with the Xenomai real-time core}

\subsection{Enable Xenomai in Buildroot}

Go back to the Buildroot 2021.02.X folder used earlier in this lab. Run
\code{make menuconfig}, and enable {\em Xenomai} in \code{Target
  packages} $\rightarrow$ \code{Real-Time} $\rightarrow$ \code{Xenomai
  Userspace}. Then for the sub-options:

\begin{itemize}
\item Set \code{Custom version} to \code{3.1}
\item Set \code{Xenomai core} to \code{Cobalt}
\item Disable SMP support
\end{itemize}

Rebuild Buildroot by running \code{make}. Deploy the new root
filesystem tarball in your \code{nfsroot} directory.

\subsection{Build a Xenomai-enabled kernel}

As explained in the lectures, Xenomai needs a patched kernel, with the
{\em I-Pipe} patch. Such patches can be found at
\url{https://xenomai.org/downloads/ipipe/}. We will be using
\code{ipipe-core-4.19.144-arm-10.patch} located at
\url{https://xenomai.org/downloads/ipipe/v4.x/arm/ipipe-core-4.19.144-arm-10.patch}. Download
this patch.

Then, go to {\em kernel.org} and download the 4.19.144 Linux kernel
source code, and extract it.

Now, to apply the {\em I-Pipe} patch and add the Xenomai kernel
modules to our kernel sources, we will use a helper script provided by
Xenomai itself. Go to Buildroot \code{output/build/xenomai-3.1/}
folder, and run:

\begin{terminalinput}
$ ./scripts/prepare-kernel.sh \
    --linux=$HOME/embedded-linux-labs/realtime/linux-4.19.144/ \
    --ipipe=$HOME/embedded-linux-labs/realtime/ipipe-core-4.19.144-arm-10.patch \
    --arch=arm
\end{terminalinput}

Now that your kernel is patched, you can configure it:

\begin{terminalinput}
$ make ARCH=arm sama5_defconfig
\end{terminalinput}

Make sure \code{CONFIG_XENOMAI} is enabled.

We are going to use the Linaro toochain used by Buildroot to compile the
kernel, so we will need to update our \code{CROSS_COMPILE} setting:

\begin{terminalinput}
$ export CROSS_COMPILE=arm-linux-gnueabihf-
\end{terminalinput}

Build your kernel. Copy the kernel image and the Device Tree to your TFTP
directory, and boot this new kernel.

\subsection{Testing Xenomai latencies}

To run our \code{rttest} application, we need to compile it against
the Xenomai user-space libraries. This can be done by using the
\code{xeno-config} script provided by Xenomai, as follows:

%\small
\begin{terminalinput}
$ cd $HOME/__SESSION_NAME__-labs/realtime/nfsroot/root
$ export PATH=$HOME/__SESSION_NAME__-labs/realtime/buildroot-2021.02.X/output/host/usr/bin:$PATH
$ arm-linux-gnueabihf-gcc -o rttest rttest.c \
    $(DESTDIR=../../buildroot-2021.02.X/output/staging/ \
    ../../buildroot-2021.02.X/output/staging/usr/bin/xeno-config \
    --skin=posix --cflags --ldflags)
\end{terminalinput}
\normalsize

Run the following commands on the board:

\begin{terminalinput}
$ echo 0 > /proc/xenomai/latency
\end{terminalinput}

This will disable the timer compensation feature of Xenomai. This
feature allows Xenomai to adjust the timer programming to take into
account the time the system needs to schedule a task after being woken
up by a timer. However, this feature needs to be calibrated
specifically for each system. By disabling this feature, we will have
raw Xenomai results, that could be further improved by doing proper
calibration of this compensation mechanism.

Run the tests again, compare the results.
