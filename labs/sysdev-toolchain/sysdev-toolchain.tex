\subchapter{Building a cross-compiling toolchain}{Objective: Learn how
  to compile your own cross-compiling toolchain for the uClibc C
  library}

After this lab, you will be able to:

\begin{itemize}
\item Configure the {\em crosstool-ng} tool
\item Execute {\em crosstool-ng} and build up your own cross-compiling toolchain
\end{itemize}

\section{Setup}

Go to the \code{$HOME/__SESSION_NAME__-labs/toolchain} directory.

For this lab, you need a system or VM with a least 4 GB of RAM.

\section{Install needed packages}

Install the packages needed for this lab:

\begin{bashinput}
$ sudo apt install build-essential git autoconf bison flex texinfo help2man gawk libtool-bin libncurses5-dev unzip
\end{bashinput}

\section{Getting Crosstool-ng}

Let's download the sources of Crosstool-ng, through its git
source repository, and switch to the 1.25 release:

\begin{bashinput}
$ git clone https://github.com/crosstool-ng/crosstool-ng
$ cd crosstool-ng/
$ git checkout crosstool-ng-1.25.0
\end{bashinput}

\section{Building and installing Crosstool-ng}

As we are not building Crosstool-ng from a release archive but from
a git repository, we first need to generate a \code{configure} script and
more generally all the generated files that are shipped in the source
archive for a release:

\begin{bashinput}
$ ./bootstrap
\end{bashinput}

We can then either install Crosstool-ng globally on the system, or keep it
locally in its download directory. We'll choose the latter
solution. As documented at
\url{https://crosstool-ng.github.io/docs/install/#hackers-way}, do:

\begin{bashinput}
$ ./configure --enable-local
$ make
\end{bashinput}

Then you can get Crosstool-ng help by running

\begin{bashinput}
$ ./ct-ng help
\end{bashinput}

\section{Configure the toolchain to produce}

A single installation of Crosstool-ng allows to produce as many
toolchains as you want, for different architectures, with different C
libraries and different versions of the various components.

Crosstool-ng comes with a set of ready-made configuration files for
various typical setups: Crosstool-ng calls them {\em samples}. They can be
listed by using \code{./ct-ng list-samples}.

We will load the
\ifdefstring{\labboard}{qemu}{Cortex A9}{Cortex A5}
sample\ifdefstring{\labboard}{discovery}{, as Crosstool-ng doesn't have
any sample for Cortex A7 yet}. Load it with the \code{./ct-ng} command.

Then, to refine the configuration, let's run the \code{menuconfig} interface:

\begin{bashinput}
$ ./ct-ng menuconfig
\end{bashinput}

In \code{Path and misc options}:
\begin{itemize}
\item Change \code{Maximum log level to see} to \code{DEBUG} (look for
\code{LOG_DEBUG} in the interface, using the \code{/} key) so that
  we can have more details on what happened during the build in case
  something went wrong.
\end{itemize}

\ifdefstring{\labboard}{discovery}{
In \code{Target options}:
\begin{itemize}
\item Set \code{Emit assembly for CPU} (\code{ARCH_CPU}) to \code{cortex-a7}.
\item Set \code{Use specific FPU} (\code{ARCH_FPU}) to \code{vfpv4}.
\end{itemize}
}{}

In \code{Toolchain options}:
\begin{itemize}
\item Set \code{Tuple's vendor string} (\code{TARGET_VENDOR}) to \code{training}.
\item Set \code{Tuple's alias} (\code{TARGET_ALIAS}) to \code{arm-linux}.
      This way, we will be able to use the compiler as \code{arm-linux-gcc}
      instead of \code{arm-training-linux-uclibcgnueabihf-gcc}, which is
      much longer to type.
\end{itemize}

In \code{C-library}:
\begin{itemize}
  \item If not set yet, set \code{C library} to \code{uClibc}
        (\code{LIBC_UCLIBC})
  \item Keep the default version that is proposed
  \item If needed, enable \code{Add support for IPv6}
        (\code{LIBC_UCLIBC_IPV6})\footnote{
        That's needed to use the toolchain in Buildroot, which only
        accepts toolchains with IPv6 support},
        \code{Add support for WCHAR} (\code{LIBC_UCLIBC_WCHAR}) and
	\code{Support stack smashing protection (SSP)}
        (\code{LIBC_UCLIBC_HAS_SSP})
\end{itemize}

In \code{C compiler}:
\begin{itemize}
  \item Keep \code{Version of gcc} set to \code{11.2.0}, as it is by
        default. We need to stick to gcc 11.x, as Buildroot 2022.02
        which we are going to use later doesn't support gcc 12.x
        toolchains yet: gcc 12.x was released after Buildroot 2022.02.
  \item Make sure that \code{C++} (\code{CC_LANG_CXX}) is enabled
\end{itemize}

In \code{Debug facilities}:
\begin{itemize}
\item Disable {\em duma}, {\em gdb}, {\em ltrace}. They will not be
  needed for the labs that will use this toolchain.
\item Keep only \code{strace} (\code{DEBUG_STRACE}) enabled
\end{itemize}

Explore the different other available options by traveling through the
menus and looking at the help for some of the options. Don't hesitate
to ask your trainer for details on the available options. However,
remember that we tested the labs with the configuration described
above. You might waste time with unexpected issues if you customize the
toolchain configuration.

\section{Produce the toolchain}

Nothing is simpler:

\begin{bashinput}
$ ./ct-ng build
\end{bashinput}

The toolchain will be installed by default in \code{$HOME/x-tools/}.
That's something you could have changed in Crosstool-ng's configuration.

And wait!

\section{Testing the toolchain}

You can now test your toolchain by adding
\code{$HOME/x-tools/arm-training-linux-uclibcgnueabihf/bin/} to your
\code{PATH} environment variable and compiling the simple
\code{hello.c} program in your main lab directory with
\code{arm-linux-gcc}:

\begin{bashinput}
$ arm-linux-gcc -o hello hello.c
\end{bashinput}

You can use the \code{file} command on your binary to make sure it has
correctly been compiled for the ARM architecture.

Did you know that you can still execute this binary from your x86 host?
To do this, install the QEMU user emulator, which just emulates target
instruction sets, not an entire system with devices:

\begin{bashinput}
$ sudo apt install qemu-user
\end{bashinput}

Now, try to run QEMU ARM user emulator:

\begin{bashinput}
$ qemu-arm hello
/lib/ld-uClibc.so.0: No such file or directory
\end{bashinput}

What's happening is that \code{qemu-arm} is missing the shared C library
(compiled for ARM) that this binary uses. Let's find it in our newly
compiled toolchain:

\begin{bashinput}
$ find ~/x-tools -name ld-uClibc.so.0
\end{bashinput}
\begin{terminaloutput}
/home/tux/x-tools/arm-training-linux-uclibcgnueabihf/arm-training-linux-uclibcgnueabihf/sysroot/lib/ld-uClibc.so.0
\end{terminaloutput}

We can now use the \code{-L} option of \code{qemu-arm} to let it know
where shared libraries are:

\begin{bashinput}
$ qemu-arm -L ~/x-tools/arm-training-linux-uclibcgnueabihf/arm-training-linux-uclibcgnueabihf/sysroot hello
\end{bashinput}

\begin{terminaloutput}
Hello world!
\end{terminaloutput}

\section{Cleaning up}

{\em Do this only if you have limited storage space. In case you made a
mistake in the toolchain configuration, you may need to run Crosstool-ng
again, keeping generated files would save a significant amount of time.}

To save about 11 GB of storage space, do a \code{./ct-ng clean} in the
Crosstool-NG source directory. This will remove the source code of the
different toolchain components, as well as all the generated files
that are now useless since the toolchain has been installed in
\code{$HOME/x-tools}.
