\subchapter
{Debugging memory issues}
{Objectives:
  \begin{itemize}
    \item Memory leak and misbehavior detection with {\em valgrind} and
            {\em vgdb}.
  \end{itemize}
}

\section{valgrind \& vgdb}

Go into the \code{valgrind folder} and compile \code{valgrind.c} with debugging
information using:

\begin{bashinput}
$ cd /home/$USER/debugging-labs/nfsroot/root/valgrind
$ make
\end{bashinput}

Then run it on the target. Do you notice any problem? Does it run correctly?
Even though there is no segfault, an application might have some memory leaks
or even out-of-bounds accesses, uninitialized memory, etc.

Now, run the command again with valgrind using the following command:

\begin{bashinput}
$ valgrind --leak-check=full ./faulty_mem_app
\end{bashinput}

You'll see various errors found by valgrind
\begin{itemize}
  \item Invalid memory write
  \item Uninitialized memory
  \item Memory leaks
\end{itemize}

In order to pinpoint exactly each error and be able to debug with gdb, vgdb
can be used. We will do that remotely on the host using gdb-multiarch. First, we
need to run valgrind with vgdb enabled on the target:

\begin{bashinput}
$ cd /root/valgrind
$ valgrind --vgdb-error=0 --leak-check=full ./faulty_mem_app
\end{bashinput}

Then, in order to do remote debugging, we also need to run vgdb in listen mode.
Start another terminal in SSH on the target and run the following command:

\begin{bashinput}
$ vgdb --port=1234
\end{bashinput}

On the computer side, start gdb-multiarch and give it the \code{faulty_mem_app}
binary which will allow to detect the architecture and read symbols:
\begin{bashinput}
$ gdb-multiarch ./faulty_mem_app
\end{bashinput}

Finally, we'll need to connect to vgdb using the following gdb command:
\begin{bashinput}
(gdb) target remote 192.168.0.100:1234
\end{bashinput}

You will then be able to debug each error using gdb and valgrind will interrupt
the program each time it detects an error. Try to solve all the problems that
were encountered by valgrind.

NOTE: The backtrace for leaks is not shown on the target because all libraries
are stripped and thus do not have any debugging symbols anymore. This leads to
the impossibility to use the dwarf information for backtracing.
