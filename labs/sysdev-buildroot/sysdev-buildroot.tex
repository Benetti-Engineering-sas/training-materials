\subchapter{Using a build system, example with Buildroot}{Objectives:
  discover how a build system is used and how it works, with the
  example of the Buildroot build system. Build a Linux system with
  libraries and make it work on the board.}

\section{Setup}

Create the \code{$HOME/__SESSION_NAME__-labs/buildroot} directory and
go into it.

\section{Get Buildroot and explore the source code}

The official Buildroot website is available at
\url{https://buildroot.org/}. Download the latest 2022.02.<n> (Long Term Support)
version which we have tested for this lab. Uncompress the tarball
and go inside the Buildroot source directory.

Several subdirectories or files are visible, the most important ones
are:

\begin{itemize}
\item \code{boot} contains the Makefiles and configuration items
  related to the compilation of common bootloaders (Grub, U-Boot,
  Barebox, etc.)
\item \code{configs} contains a set of predefined configurations,
  similar to the concept of defconfig in the kernel.
\item \code{docs} contains the documentation for Buildroot. You can
  start reading \code{manual/manual.html} which is the main Buildroot
  documentation;
\item \code{fs} contains the code used to generate the various root
  filesystem image formats
\item \code{linux} contains the Makefile and configuration items
  related to the compilation of the Linux kernel
\item \code{Makefile} is the main Makefile that we will use to use
  Buildroot: everything works through Makefiles in Buildroot;
\item \code{package} is a directory that contains all the Makefiles,
  patches and configuration items to compile the user space
  applications and libraries of your embedded Linux system. Have a
  look at various subdirectories and see what they contain;
\item \code{system} contains the root filesystem skeleton and the {\em
    device tables} used when a static \code{/dev} is used;
\item \code{toolchain} contains the Makefiles, patches and
  configuration items to generate the cross-compiling toolchain.
\end{itemize}

\section{Configure Buildroot}

In our case, we would like to:

\begin{itemize}
\item Generate an embedded Linux system for ARM;
\item Use an already existing external toolchain instead of having
  Buildroot generating one for us;
\item Integrate {\em BusyBox}, {\em alsa-utils} and {\em vorbis-tools}
  in our embedded Linux system;
\item Integrate the target filesystem into a tarball
\end{itemize}

To run the configuration utility of Buildroot, simply run:

\bashcmd{$ make menuconfig}

Set the following options. Don't hesitate to press the \code{Help}
button whenever you need more details about a given option:

\begin{itemize}
\item \code{Target options}
  \begin{itemize}
  \item \code{Target Architecture}: \code{ARM (little endian)}
  \ifdefstring{\labboard}{discovery}{
  \item \code{Target Architecture Variant}: \code{cortex-A7}
  }{
   \ifdefstring{\labboard}{qemu}{
   \item \code{Target Architecture Variant}: \code{cortex-A9}
   \item \code{Enable NEON SIMD extension support}: Enabled
   }{
   \item \code{Target Architecture Variant}: \code{cortex-A5}
   }
   \item \code{Enable VFP extension support}: Enabled
  }
  \item \code{Target ABI}: \code{EABIhf}
  \ifdefstring{\labboard}{discovery}{
  \item \code{Floating point strategy}: \code{VFPv4}
  }{
   \ifdefstring{\labboard}{qemu}{
   \item \code{Floating point strategy}: \code{VFPv3-D16}
   }{
   \item \code{Floating point strategy}: \code{VFPv4-D16}
   }
  }
  \end{itemize}
\item \code{Toolchain}
  \begin{itemize}
  \item \code{Toolchain type}: \code{External toolchain}
  \item \code{Toolchain}: \code{Custom toolchain}
  \item \code{Toolchain path}: use the toolchain you built:
    \code{/home/<user>/x-tools/arm-training-linux-uclibcgnueabihf}
    (replace \code{<user>} by your actual user name)
  \item \code{External toolchain gcc version}: \code{11.x}
  \item \code{External toolchain kernel headers series}: \code{5.13.x or later}
  \item \code{External toolchain C library}: \code{uClibc/uClibc-ng}
  \item We must tell Buildroot about our toolchain configuration, so
    select \code{Toolchain has WCHAR support?},
    \code{Toolchain has SSP support?} and
    \code{Toolchain has C++ support?}.
    Buildroot will check these parameters anyway.
  \end{itemize}
\item \code{Target packages}
  \begin{itemize}
  \item Keep \code{BusyBox} (default version) and keep the BusyBox
    configuration proposed by Buildroot;
  \item \code{Audio and video applications}
    \begin{itemize}
    \item Select \code{alsa-utils}
    \item \code{ALSA utils selection}
    \begin{itemize}
         \item Disable \code{alsactl} (we don't use it)
	 \item Keep \code{alsamixer} (good to see, and needed to pull
	       the \code{ncurses} dependency used in the next lab).
         \item Select \code{speaker-test}
         \ifdefstring{\labboard}{qemu}{
            \item Select \code{aplay/arecord} (to play a WAV file as long as the
                  OGG file doesn't play properly).
         }{}
     \end{itemize}
    \item Select \code{vorbis-tools}
    \end{itemize}
  \end{itemize}
\item \code{Filesystem images}
  \begin{itemize}
  \item Select \code{tar the root filesystem}
  \end{itemize}
\end{itemize}

Exit the menuconfig interface. Your configuration has now been saved
to the \code{.config} file.

\section{Generate the embedded Linux system}

Just run:

\bashcmd{$ make}

Buildroot will first create a small environment with the external
toolchain, then download, extract, configure, compile and install each
component of the embedded system.

All the compilation has taken place in the \code{output/} subdirectory. Let's
explore its contents:

\begin{itemize}

\item \code{build}, is the directory in which each component built by
  Buildroot is extracted, and where the build actually takes place

\item \code{host}, is the directory where Buildroot installs some
  components for the host. As Buildroot doesn't want to depend on too
  many things installed in the developer machines, it installs some
  tools needed to compile the packages for the target. In our case it
  installed {\em pkg-config} (since the version of the host may be ancient)
  and tools to generate the root filesystem image ({\em genext2fs},
  {\em makedevs}, {\em fakeroot}).

\item \code{images}, which contains the final images produced by
  Buildroot. In our case it's just a tarball of the filesystem, called
  \code{rootfs.tar}, but depending on the Buildroot configuration,
  there could also be a kernel image or a bootloader image.

\item \code{staging}, which contains the “build” space of the target
  system. All the target libraries, with headers and documentation. It
  also contains the system headers and the C library, which in our
  case have been copied from the cross-compiling toolchain.

\item \code{target}, is the target root filesystem. All applications
  and libraries, usually stripped, are installed in this
  directory. However, it cannot be used directly as the root
  filesystem, as all the device files are missing: it is not possible
  to create them without being root, and Buildroot has a policy of not
  running anything as root.

\end{itemize}

\section{Run the generated system}

Go back to the \code{$HOME/__SESSION_NAME__-labs/buildroot/} directory. Create
a new \code{nfsroot} directory that is going to hold our system,
exported over NFS. Go into this directory, and untar the rootfs using:

\bashcmd{$ tar xvf ../buildroot-2022.02.<n>/output/images/rootfs.tar}

Add our \code{nfsroot} directory to the list of directories exported
by NFS in \code{/etc/exports}, and make sure the board uses it too.

Boot the board, and log in (\code{root} account, no password).

You should now have a shell, where you will be able to run
\code{ogg123} like you used to in the previous lab.

\ifdefstring{\labboard}{qemu}
{{\bf Known issue}: as in the previous lab, the \code{sample.ogg} sample
may play in a weird way in QEMU. To have a correct result instead, use \code{aplay}
to play the \code{sample.wav} file provided together with \code{sample.ogg}.}
{}

\section{Going further}

\begin{itemize}

\item Add dropbear (SSH server and client) to the list of packages
  built by Buildroot and log to your target system using an ssh client
  on your development workstation. Hint: you will have to set a
  non-empty password for the root account on your target for this to
  work.

\item Add a new package in Buildroot for the GNU Gtypist game. Read
  the Buildroot documentation to see how to add a new
  package. Finally, add this package to your target system, compile it
  and run it. The newest versions require a library that is not fully
  supported by Buildroot, so you'd better stick with the latest
  version in the 2.8 series.

  \ifdefstring{\labboard}{xplained}{
\item Flash the new system on the flash of the board
  \begin{itemize}
  \item First, in buildroot, select the UBIFS filesystem image type.
  \item You'll also need to provide buildroot some information on the
    underlying device that will store the filesystem. In our case, the
    logical eraseblock size is 124KiB, the minimum I/O unit size is
    2048 and the Maximum logical eraseblock (LEB) count is 133.
  \item Then, once the image has been generated, update your rootfs
    volume.
  \end{itemize}
  }{}

\end{itemize}
