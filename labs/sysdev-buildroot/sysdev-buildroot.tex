\subchapter{Using a build system, example with Buildroot}{Objectives:
  discover how a build system is used and how it works, with the
  example of the Buildroot build system. Build a Linux system with
  libraries and make it work on the board.}

\section{Goals}

Compared to the previous lab, we are going to build a more elaborate
system, still containing {\em alsa-utils} (and of course its {\em
alsa-lib} dependency) and {\em libgpiod}, but this time using Buildroot,
an automated build system.

The automated build system will also allow us to add more packages
and play real audio on our system, thanks to the {\em Music Player
Daemon (mpd)} (\url{https://www.musicpd.org/} and its {\em mpc} client.

\section{Setup}

Go to the \code{$HOME/__SESSION_NAME__-labs/buildroot} directory.

\section{Get Buildroot and explore the source code}

The official Buildroot website is available at
\url{https://buildroot.org/}. Download the latest 2022.02.<n> (Long Term Support)
version which we have tested for this lab. Uncompress the tarball
and go inside the Buildroot source directory.

Several subdirectories or files are visible, the most important ones
are:

\begin{itemize}
\item \code{boot} contains the Makefiles and configuration items
  related to the compilation of common bootloaders (GRUB, U-Boot,
  Barebox, etc.)
\item \code{configs} contains a set of predefined configurations,
  similar to the concept of defconfig in the kernel.
\item \code{docs} contains the documentation for Buildroot. You can
  start reading \code{manual/manual.html} which is the main Buildroot
  documentation;
\item \code{fs} contains the code used to generate the various root
  filesystem image formats
\item \code{linux} contains the Makefile and configuration items
  related to the compilation of the Linux kernel
\item \code{Makefile} is the main Makefile that we will use to use
  Buildroot: everything works through Makefiles in Buildroot;
\item \code{package} is a directory that contains all the Makefiles,
  patches and configuration items to compile the user space
  applications and libraries of your embedded Linux system. Have a
  look at various subdirectories and see what they contain;
\item \code{system} contains the root filesystem skeleton and the {\em
    device tables} used when a static \code{/dev} is used;
\item \code{toolchain} contains the Makefiles, patches and
  configuration items to generate the cross-compiling toolchain.
\end{itemize}

\section{Prepare an overlay directory}

To play audio in this lab, we will need to add the kernel modules
to the root filesystem generated by Buildroot, and make sure that
the \code{snd-usb-audio} module is automatically loaded at boot time.

So, let's prepare an overlay directory (in the main lab directory)
that Buildroot will use after building the root filesystem:

\begin{bashinput}
mkdir -p rootfs-overlay/lib
cp -a ../tinysystem/nfsroot/lib/modules rootfs-overlay/lib/
\end{bashinput}

Also add a \code{etc/init.d/S03modprobe} executable file
to the overlay directory, with the below contents:

\begin{verbatim}
#!/bin/sh
modprobe snd-usb-audio
modprobe nunchuk
\end{verbatim}

The \code{S03} prefix of this file makes sure it is executed
before {\em Music Player Daemon}, which will have a greater
prefix (\code{S95}). That's necessary because this daemon will
fail to start is audio is not available.

We're also loading the \code{nunchuk.ko} module as we are
going to test the Nunchuk later in this lab.

Let's also add music files \footnote{For the most part, these are public domain
music files, except a small sample file... See the \code{README.txt}
file in the directory containing the files.}
for MPD to play:

\begin{bashinput}
mkdir -p rootfs-overlay/var/lib/mpd/music
cp data/music/* rootfs-overlay/var/lib/mpd/music/
\end{bashinput}

\section{Configure Buildroot}

In our case, we would like to:

\begin{itemize}
\item Generate an embedded Linux system for ARM;
\item Use an already existing external toolchain instead of having
  Buildroot generating one for us;
\item Integrate {\em BusyBox}, {\em alsa-utils}, {\em libgpiod},
  {\em mpd}, {\em mpc} and {\em evtest} in our embedded Linux system;
\item Integrate the target filesystem into a tarball
\end{itemize}

To run the configuration utility of Buildroot, simply run:

\bashcmd{$ make menuconfig}

Set the following options. Don't hesitate to press the \code{Help}
button whenever you need more details about a given option:

\begin{itemize}
\item \code{Target options}
  \begin{itemize}
  \item \code{Target Architecture}: \code{ARM (little endian)}
  \ifdefstring{\labboard}{discovery}{
  \item \code{Target Architecture Variant}: \code{cortex-A7}
  }{
   \ifdefstring{\labboard}{qemu}{
   \item \code{Target Architecture Variant}: \code{cortex-A9}
   \item \code{Enable NEON SIMD extension support}: Enabled
   }{
   \item \code{Target Architecture Variant}: \code{cortex-A5}
   }
   \item \code{Enable VFP extension support}: Enabled
  }
  \item \code{Target ABI}: \code{EABIhf}
  \ifdefstring{\labboard}{discovery}{
  \item \code{Floating point strategy}: \code{VFPv4}
  }{
   \ifdefstring{\labboard}{qemu}{
   \item \code{Floating point strategy}: \code{VFPv3-D16}
   }{
   \item \code{Floating point strategy}: \code{VFPv4-D16}
   }
  }
  \end{itemize}
\item \code{Toolchain}
  \begin{itemize}
  \item \code{Toolchain type}: \code{External toolchain}
  \item \code{Toolchain}: \code{Custom toolchain}
  \item \code{Toolchain path}: use the toolchain you built:
    \code{/home/<user>/x-tools/arm-training-linux-uclibcgnueabihf}
    (replace \code{<user>} by your actual user name)
  \item \code{External toolchain gcc version}: \code{11.x}
  \item \code{External toolchain kernel headers series}: \code{5.16.x or later}
  \item \code{External toolchain C library}: \code{uClibc/uClibc-ng}
  \item We must tell Buildroot about our toolchain configuration, so
    select \code{Toolchain has WCHAR support?},
    \code{Toolchain has SSP support?} and
    \code{Toolchain has C++ support?}.
    Buildroot will check these parameters anyway.
  \end{itemize}
\item \code{System configuration}
  \begin{itemize}
  \item \code{Root filesystem overlay directories}: \code{../rootfs-overlay}
  \end{itemize}
\item \code{Target packages}
  \begin{itemize}
  \item Keep \code{BusyBox} (default version) and keep the BusyBox
    configuration proposed by Buildroot;
  \item \code{Audio and video applications}
    \begin{itemize}
    \item Select \code{alsa-utils}, and in the submenu:
    \begin{itemize}
         \item Only keep \code{speaker-test}
    \end{itemize}
    \item Select \code{mpd}, and in the submenu:
    \begin{itemize}
         \item Keep only \code{alsa}, \code{vorbis} and \code{tcp sockets}
    \end{itemize}
    \item Select \code{mpd-mpc}.
    \end{itemize}
  \item \code{Hardware handling}
    \begin{itemize}
	 \item Select \code{evtest}\\
	       This userspace application allows to test events from
	       input devices. This way, we will be able to test the
	       Nunchuk by getting details about which buttons were
	       pressed.
    \end{itemize}
  \end{itemize}
\item \code{Filesystem images}
  \begin{itemize}
  \item Select \code{tar the root filesystem}
  \end{itemize}
\end{itemize}

Exit the menuconfig interface. Your configuration has now been saved
to the \code{.config} file.

\section{Generate the embedded Linux system}

Just run:

\bashcmd{$ make}

Buildroot will first create a small environment with the external
toolchain, then download, extract, configure, compile and install each
component of the embedded system.

All the compilation has taken place in the \code{output/} subdirectory. Let's
explore its contents:

\begin{itemize}

\item \code{build}, is the directory in which each component built by
  Buildroot is extracted, and where the build actually takes place

\item \code{host}, is the directory where Buildroot installs some
  components for the host. As Buildroot doesn't want to depend on too
  many things installed in the developer machines, it installs some
  tools needed to compile the packages for the target. In our case it
  installed {\em pkg-config} (since the version of the host may be ancient)
  and tools to generate the root filesystem image ({\em genext2fs},
  {\em makedevs}, {\em fakeroot}).

\item \code{images}, which contains the final images produced by
  Buildroot. In our case it's just a tarball of the filesystem, called
  \code{rootfs.tar}, but depending on the Buildroot configuration,
  there could also be a kernel image or a bootloader image.

\item \code{staging}, which contains the “build” space of the target
  system. All the target libraries, with headers and documentation. It
  also contains the system headers and the C library, which in our
  case have been copied from the cross-compiling toolchain.

\item \code{target}, is the target root filesystem. All applications
  and libraries, usually stripped, are installed in this
  directory. However, it cannot be used directly as the root
  filesystem, as all the device files are missing: it is not possible
  to create them without being root, and Buildroot has a policy of not
  running anything as root.

\end{itemize}

\section{Run the generated system}

Go back to the \code{$HOME/__SESSION_NAME__-labs/buildroot/} directory. Create
a new \code{nfsroot} directory that is going to hold our system,
exported over NFS. Go into this directory, and untar the rootfs using:

\bashcmd{$ tar xvf ../buildroot-2022.02.<n>/output/images/rootfs.tar}

Add our \code{nfsroot} directory to the list of directories exported
by NFS in \code{/etc/exports}, and make sure the board uses it too.

Boot the board, and log in (\code{root} account, no password).

You should now reach a shell.

Check that the \code{snd_usb_audio} module is loaded as expected.

\section{Testing music playback with mpd and mpc}

Using the \code{ps} command, check that the \code{mpd} server
was started by the system, as implemented by the
\code{/etc/init.d/S95mpd} script.

If that's the case, you are now ready to run \code{mpc} client commands
to control music playback. First, let's make \code{mpd} process the
newly added music files. Run this command on the target:

\bashcmd{# mpc update}

You should see the files getting indexed, by displaying the contents
of the \code{/var/log/mpd.log} file:

\begin{terminaloutput}
Jan 01 00:04 : exception: Failed to open '/var/lib/mpd/state': No such file or directory
Jan 01 00:15 : update: added /2-arpent.ogg
Jan 01 00:15 : update: added /6-le-baguette.ogg
Jan 01 00:15 : update: added /4-land-of-pirates.ogg
Jan 01 00:15 : update: added /3-chronos.ogg
Jan 01 00:15 : update: added /1-sample.ogg
Jan 01 00:15 : update: added /7-fireworks.ogg
Jan 01 00:15 : update: added /5-ukulele-song.ogg
\end{terminaloutput}

You can also check the list of available files:
\begin{terminaloutput}
# mpc listall
1-sample.ogg
2-arpent.ogg
5-ukulele-song.ogg
3-chronos.ogg
7-fireworks.ogg
6-le-baguette.ogg
4-land-of-pirates.ogg
\end{terminaloutput}

To play files, you first need to create a playlist. Let's create a
currently playlist by adding all music files to it:

\bashcmd{# mpc add /}

You should now be able to start playing the songs in the playlist:

\bashcmd{# mpc play}

Here are a few further commands for controlling playback:
\begin{itemize}
\item \code{mpc volume +5}: increase the volume by 5\%
\item \code{mpc volume -5}: reduce the volume by 5\%
\item \code{mpc prev}: switch to the previous song in the playlist.
\item \code{mpc next}: switch to the next song in the playlist.
\item \code{mpc toggle}: toggle between pause and playback modes.
\end{itemize}

Later, we will compile and debug a custom MPD client application.

\section{Analyzing dependencies}

It's always useful to understand the dependencies drawn by the
packages we build.

First we need to install a few host packages:

\bashcmd{$ sudo apt install graphviz xdot}

Now, let's use Buildroot's target to generate a
dependency graph:

\bashcmd{$ make graph-depends}

We can now study the dependency graph:

\bashcmd{$ xdot output/graphs/graph-depends.dot}

In particular, you can see that adding MPD and its client
required to compile {\em Meson} for the host, and in turn,
{\em Python 3} for the host too. This substantially contributed to the
build time.

\section{Testing the Nunchuk}

Now that we have compiled \code{evtest} for the target, thanks to
Buildroot, we can now test the input events coming from the
Nunchuk.

\begin{bashinput}
# evtest
No device specified, trying to scan all of /dev/input/event*
Available devices:
/dev/input/event0:	pmic_onkey
/dev/input/event1:	Logitech Inc. Logitech USB Headset H340 Consumer Control
/dev/input/event2:	Logitech Inc. Logitech USB Headset H340
/dev/input/event3:	Wii Nunchuk
Select the device event number [0-3]:
\end{bashinput}

Enter the number corresponding to the Nunchuk device.

You can now press the Nunchuk buttons, use the joypad, and see which
input events are emitted.

By the way, you can also test which input events are exposed by the
driver for your audio headset, which doesn't mean that they physically
exist.

\section{Going further}

{\em If you finish your lab before the others}

\begin{itemize}
\item For more music playing fun, you can install the \code{ario} MPD
  client on your host machine (\code{sudo apt install ario}), configure
  it to connect to the IP address of your target system with the default
  port, and you will also be able to control playback from your host
  machine.
\end{itemize}

