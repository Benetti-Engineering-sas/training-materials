\subchapter
{Solving an application crash}
{Objectives:
  \begin{itemize}
    \item Analysis of compiled C code with compiler-explorer to understand
          optimizations.
    \item Managing {\em gdb} from the command line.
    \item Debugging a crashed application using a coredump with {\em gdb}.
    \item Using {\em gdb} Python scripting capabilities.
  \end{itemize}
}

\section{Compiler explorer}

Go to \url{https://godbolt.org/} and paste the content of \code{swap_bytes.c}.
Select the correct compiler for armv7 and observe the generated assembly. Try
to modify the compiler options to optimize the generation (-O3). Observe the
result.

\section{Using GDB}

Take our \code{linked_list.c} program. It uses the \code{<sys/queue.h>} header
which provides multiple linked-list implementations. This program creates and
fill a linked list with the names read from a file. Compile it using the
following command:

\begin{bashinput}
$ cd /home/$USER/debugging-labs/nfsroot/root/gdb/
$ make
\end{bashinput}

By default, it will look for a \code{word_list} file located in the current
directory. This program should display the list of words that were read from
the file.

\begin{bashinput}
$ ./linked_list
\end{bashinput}

From what you can see, it actually crashes ! So we will use GDB to debug that
program. We will do that remotely since our target does not embed a full gdb,
only a gdbserver, a lightweight gdb server that allows connecting with a remote
full feature GDB. Start our program using gdbserver in multi mode:

\begin{bashinput}
$ gdbserver --multi :2000 ./linked_list
\end{bashinput}

On the host side install gdb-multiarch if not already done and attach to this
process using gdb-multiarch:

\begin{bashinput}
$ sudo apt install gdb-multiarch
$ gdb-multiarch ./linked_list
(gdb) target extended-remote 192.168.0.100:2000
(gdb) set sysroot /home/<user>/debugging-labs/buildroot/output/staging/
\end{bashinput}

Then continue the execution and try to find the error using GDB. There are
multiple ways to debug such program. We will track down up to the error in order
to understand

{\em TIP: you can execute command automatically at GDB startup by putting them
in a \code{~/.gdbinit} file. For instance, history can be enabled with
\code{set history save on} and pretty printing of structure with
\code{set print pretty on}.}

{\em TIP: GDB features a TUI which can be spawned using Ctrl + X + A. You can
switch from the command line to the TUI view using Ctrl X + O.}

{\em TIP: in gdb, not only values can be displayed using \code{p} command but
functions can also be called directly from gdb ! Try to call
\code{display_linked_list()}.}

{\em NOTE: you can exit gdbserver from the connected gdb process using the
\code{monitor exit} command.}

\section{Using a coredump with GDB}

Sometimes, the problems only arise in production and you can only gather data
once the application crashed. This is also something that can be used if the
crash is not reproducible but crashes only once in a while.  If so, we can use
the kernel coredump support to generate a core dump of the faulty application
and do a post-mortem analysis.

First of all, we need to enable kernel coredumping support of programs:

\begin{bashinput}
$ ulimit -c unlimited
\end{bashinput}

Then, run the program normally:

\begin{bashinput}
$ ./linked_list
Segmentation fault (core dumped)
\end{bashinput}

When crashing, a \code{core} file will be generated. On your desktop computer,
fix the permissions using:
\begin{bashinput}
$ sudo chown $USER:$USER core
\end{bashinput}

Then, open this file using \code{gdb-multiarch}:

\begin{bashinput}
$ gdb-multiarch ./linked_list ./core
\end{bashinput}

You can then inspect the program state (memory, registers, etc) at the time it
crashed. While less dynamic, it allows to pinpoint the place that triggered the 
crash.

\section{GDB Python support}

When developing and debugging applications, sometimes we often uses the same
set of commands over and over under GDB. Rather than doing so, we can create 
python scripts that are integrated with GDB.

In order to display our program list from GDB, we provide a python GDB script
named \code{linked_list.py} that displays this list. You will need to fill two
parts of this script to display a complete list correctly. This python script
takes the list head name and the next field name as parameters.

The part to be filled in are the pretty printer struct formatting and the
iteration on the list. We would like to display each \code{struct name} as
\code{index: name}. In order to access a struct field in gdb python, you can use
\code{self.val['field_name']}.

Once done, you can use the following commands to test your script:

\begin{bashinput}
(gdb) source linked_list.py
(gdb) printslist name_list next
\end{bashinput}
