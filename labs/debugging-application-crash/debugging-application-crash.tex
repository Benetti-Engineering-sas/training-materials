\subchapter
{Solving an application crash}
{Objectives:
  \begin{itemize}
    \item Analysis of compiled C code with compiler-explorer to understand
          optimizations.
    \item Managing {\em gdb} from the command line.
    \item Debugging a crashed application using a coredump with {\em gdb}.
    \item Using {\em gdb} Python scripting capabilities.
    \item (Bonus) configuring an IDE for debugging
  \end{itemize}
}

\section{Compiler explorer}

Go to \url{https://godbolt.org/} and paste the content of
\code{/home/$USER/debugging-labs/nfsroot/root/compiler_explorer/swap_bytes.c}.
Select the correct compiler for armv7 and observe the generated assembly. Try
to modify the compiler options to optimize the generation (-O3). Observe the
result.

\section{Using GDB}

Take our \code{linked_list.c} program. It uses the \code{<sys/queue.h>} header
which provides multiple linked-list implementations. This program creates and
fill a linked list with the names read from a file. Compile it from your
development host using the following command:

\begin{bashinput}
$ cd /home/$USER/debugging-labs/nfsroot/root/gdb/
$ make
\end{bashinput}

By default, it will look for a \code{word_list} file located in the current
directory. This program, when run on the target, should display the list of
words that were read from the file.

\begin{bashinput}
$ ./linked_list
\end{bashinput}

From what you can see, it actually crashes! So we will use GDB to debug that
program. We will do that remotely since our target does not embed a full gdb,
only a gdbserver, a lightweight gdb server that allows connecting with a remote
full feature GDB. Start our program on the target using gdbserver in multi mode:

\begin{bashinput}
$ gdbserver --multi :2000 ./linked_list
\end{bashinput}

On the host side install gdb-multiarch if not already done and attach to this
process using gdb-multiarch:

\begin{bashinput}
$ sudo apt install gdb-multiarch
$ gdb-multiarch ./linked_list
(gdb) target extended-remote 192.168.0.100:2000
(gdb) set sysroot /home/<user>/debugging-labs/buildroot/output/staging/
\end{bashinput}

Then continue the execution and try to find the error using GDB. There are
multiple ways to debug such program. Here are a few hints to help you find the bug root
cause:
\begin{itemize}
  \item Instead of waiting for the segfault to happen, you can use breakpoints
  to stop at a specific place BEFORE the segfault happens
  \item The bug is pretty reproductible here, so do not hesitate to restart the
  program multiple times during your analysis to properly understand the path
  leading to the bug, and to run it step by step
  \item If your breakpoint is in a loop, it may be tedious to type "continue"
  multiple times to reach to execution site you are interested in: remember
  that you can use conditional breakpoints and watchpoints to make a breakpoint
  hit on a specific loop occurrence !
  \item It is necessary to understand the data structures your code is using to
  understand why you eventually get a segfault: take some time to get familiar
  with the \code{SLIST_HEAD} and \code{SLIST_ENTRY} macros, and to print them
  during debugging to see how data is organized.
  \begin{itemize}
    \item You can also refer to the \code{macro} command documentation to learn
    how to interpret macros in gdb during the program execution
  \end{itemize}
  \item You mave even use and call existing code during your debug session: you
  can for example try to print the list by executing
  \code{display_linked_list()} with the \code{p} gdb command !
\end{itemize}

{\em TIP: you can execute command automatically at GDB startup by putting them
in a \code{~/.gdbinit} file. For instance, history can be enabled with
\code{set history save on} and pretty printing of structure with
\code{set print pretty on}.}

{\em TIP: GDB features a TUI which can be spawned using Ctrl + X + A. You can
switch from the command line to the TUI view using Ctrl X + O.}

{\em NOTE: you can exit gdbserver from the connected gdb process using the
\code{monitor exit} command.}

\section{Using a coredump with GDB}

Sometimes, the problems only arise in production and you can only gather data
once the application crashed. This is also something that can be used if the
crash is not reproducible but crashes only once in a while.  If so, we can use
the kernel coredump support to generate a core dump of the faulty application
and do a post-mortem analysis.

First of all, we need to enable kernel coredumping support of programs. On the
target, run:

\begin{bashinput}
$ ulimit -c unlimited
\end{bashinput}

Then, run the program normally:

\begin{bashinput}
$ ./linked_list
Segmentation fault (core dumped)
\end{bashinput}

When crashing, a \code{core} file will be generated. On your desktop computer,
fix the permissions using:
\begin{bashinput}
$ sudo chown $USER:$USER core
\end{bashinput}

Then, open this file using \code{gdb-multiarch}:

\begin{bashinput}
$ gdb-multiarch ./linked_list ./core
\end{bashinput}

You can then inspect the program state (memory, registers, etc) at the time it
crashed. While less dynamic, it allows to pinpoint the place that triggered the
crash. Here are a few hints to help you manipulate the coredump:
\begin{itemize}
  \item Since the coredump is only a memory snapshot (the program is not
  running anymore), you can not run any process-controlling command in gdb
  (\code{run}, \code{continue}, \code{step}...)
  \item However, you can print any variable which is in scope. You can even
  define your own commands to ease printing our custom list:
  \begin{itemize}
    \item Use the \code{define} command to start defining a command, use
    \code{end} to mark its end
    \item In between, knowing how the data is organized, you can print some
    data fields with \code{p}
    \item You can use gdb custom variables to navigate into the list : refresh
    your custom variable to make it point to the next element of the list each
    time you call your custom command
  \end{itemize}
\end{itemize}

\section{GDB Python support}

When developing and debugging applications, sometimes we often uses the same
set of commands over and over under GDB. Rather than doing so, we can create
python scripts that are integrated with GDB.

In order to display our program list from GDB, we provide a python GDB script
named \code{linked_list.py} that displays this list. You will need to fill two
parts of this script to display a complete list correctly. This python script
takes the list head name and the next field name as parameters.

The part to be filled in are the pretty printer struct formatting and the
iteration on the list. We would like to display each \code{struct name} as
\code{index: name}. In order to access a struct field in gdb python, you can use
\code{self.val['field_name']}.

Once done, you can use the following commands in your development host gdb
client session to test your script:

\begin{bashinput}
(gdb) source linked_list.py
(gdb) printslist name_list next
\end{bashinput}

\section{(Bonus) Configuring an IDE for debugging}

While using bare gdb commands in console is enough to debug your application issues, you
may want to use some frontend to do so, for example by using GDB features directly
in your IDE. Multiple frontends are able to interact with GDB thanks to its
\href{https://sourceware.org/gdb/current/onlinedocs/gdb.html/GDB_002fMI.html}{MI
interface}
For this example, we will configure
\href{https://code.visualstudio.com/}{Visual Studio Code}. Start by installing
the IDE.

The firmware used for this training labs is in fact already running a gdbserver
right at boot, listening on port 9876 : thanks to the \code{--multi} option,
this server can be configured by the client to debug any target of interest.

We must then provide instructions to VSCode about how to start a debug session.
You may have noticed that the labs data also provide a \code{.vscode} directory
along the program sources. If you open vscode in the sources directory, it will
recognize this \code{.vscode} directory and load corresponding configuration

The \code{.vscode} directory contains a minimalistic configuration to allow
remote debugging, thanks to two files:
\begin{itemize}
  \item \code{launch.json} contains a debug configuration which indicates what
  is the program to debug, how to connect to gdbserver and to configure it, where to
  find the missing symbols (our staging directory), etc.
  \item \code{tasks.json} contains a default build task: we have defined in our
  debug configuration that we want to run a build before each debug session (so
  we are sure that the program under debug is in sync with our sources), so
  we have to define this build task. Note that since we are using a NFS root,
  this build task also takes in charge the "deploy" step !
\end{itemize}

You can then open VSCode in the sources directory:
\begin{bashinput}
  cd ~/debugging-labs/root/gdb
  code .
\end{bashinput}
Before being able to debug your program onto the target, you will need to install
\href{https://marketplace.visualstudio.com/items?itemName=ms-vscode.cpptools}{the
official C/C++} extension (you can do it directly from the "Extensions" tab in
the left column).

You can now start the debugging session (hit the "Run and Debug"
tab on the left column, then "Debug linked list" in the list on top left, or
directly hit "F5" to start the debug session).
VSCode will start our cross-gdb and connect it to the remote gdbserver, and you
will be able to do everything we have done in the console so far:
\begin{itemize}
  \item Add breakpoints by clicking on the line number you want to break in
  \item Set conditions on breakpoint by issuing right-click on the breakpoint
  (red circle on the line number), "Edit breakpoint"
  \item In the debugging tab (left column), you can
  \begin{itemize}
    \item Check variables which are in scope
    \item take a look at your call stack
    \item add expression to watch for specific values
  \end{itemize}
  \item Control the program execution with the "Continue", "Step Over", "Step
  Into", "Step Out", "Restart", "Stop" buttons (in the upper right section)
  \item Use directly bare gdb commands in the debugging console: open the
  "Debug console" tab on the lower part of the screen, then type any command
  learnt so far, but remember to prefix it with \code {-exec}
\end{itemize}

You can also refer to
\href{https://code.visualstudio.com/docs/editor/debugging}{VSCode debug
configuration documentation} to learn more about available configuration options
