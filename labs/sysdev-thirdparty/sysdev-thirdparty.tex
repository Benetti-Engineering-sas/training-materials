\subchapter{Third party libraries and applications}{Objective: Learn
  how to leverage existing libraries and applications: how to
  configure, compile and install them}

To illustrate how to use existing libraries and applications, we will
extend the small root filesystem built in the {\em A tiny embedded
  system} lab to add the {\em ALSA} libraries and tools and an audio
playback application using the {\em ALSA} libraries. {\em ALSA} stands for
({\em Advanced Linux Sound Architecture}, and is the Linux audio subsystem.

We'll see that manually re-using existing libraries is quite tedious,
so that more automated procedures are necessary to make it
easier. However, learning how to perform these operations manually
will significantly help you when you face issues with more
automated tools.

\section{Audio support in the Kernel}

Recompile your kernel with audio support. The options we want are:
\kconfig{CONFIG_SOUND}, \kconfig{CONFIG_SND}, \kconfig{CONFIG_SND_USB} and
\kconfig{CONFIG_SND_USB_AUDIO}.

At this stage, the easiest solution to update your kernel is probably
to get back to copying it to RAM from \code{tftp}. Anyway, we will have
to modify U-Boot environment variables, as we are going to switch back to
NFS booting anyway.

Make sure that your board still boots with this new kernel.

\section{Figuring out library dependencies}

We're going to integrate the \code{alsa-utils} and \code{ogg123}
executables. As most software components, they in turn depend
on other libraries, and these dependencies are different depending
on the configuration chosen for them. In our case, the dependency
chain for \code{alsa-utils} is quite simple, it only depends on
the \code{alsa-lib} library.

The dependencies are a bit more complex for \code{ogg123}. It is part
of \code{vorbis-tools}, that depend on \code{libao} and
\code{libvorbis}. \code{libao} in turn depends on \code{alsa-lib}, and
\code{libvorbis} on \code{libogg}.

\code{libao}, \code{alsa-utils} and \code{alsa-lib} are here
to abstract the use of {\em ALSA}. \code{vorbis-tools}, \code{libvorbis} and
\code{libogg} are used to handle the audio files encoded using the Ogg
container and Vorbis codec, which are quite common.

So, we end up with the following dependency tree:

\begin{center}
\includegraphics[width=\textwidth]{labs/sysdev-thirdparty/alsa-dependencies.pdf}
\end{center}

Of course, all these libraries rely on the C library, which is not
mentioned here, because it is already part of the root filesystem
built in the {\em A tiny embedded system} lab. You might wonder how to
figure out this dependency tree by yourself. Basically, there are
several ways, that can be combined:

\begin{itemize}
\item Read the library documentation, which often mentions the
  dependencies;
\item Read the help message of the \code{configure script} (by running
  \code{./configure --help}).
\item By running the \code{configure} script, compiling and looking
  at the errors.
\end{itemize}

To configure, compile and install all the components of our system,
we're going to start from the bottom of the tree with {\em alsa-lib},
then continue with {\em alsa-utils}, \emph{libao}, {\em libogg},
and \emph{libvorbis}, to finally compile \emph{vorbis-tools}.

\section{Preparation}

For our cross-compilation work, we will need two separate spaces:
\begin{itemize}
\item A \emph{staging} space in which we will directly install all the
  packages: non-stripped versions of the libraries, headers,
  documentation and other files needed for the compilation. This
  \emph{staging} space can be quite big, but will not be used on our
  target, only for compiling libraries or applications;
\item A \emph{target} space, in which we will only copy the required
  files from the \emph{staging} space: binaries and libraries, after
  stripping, configuration files needed at runtime, etc. This target
  space will take a lot less space than the \emph{staging} space, and
  it will contain only the files that are really needed to make the
  system work on the target.
\end{itemize}

To sum up, the {\em staging} space will contain everything that's
needed for compilation, while the {\em target} space will contain only
what's needed for execution.

So, in \code{$HOME/__SESSION_NAME__-labs/thirdparty}, create two
directories: \code{staging} and \code{target}.

For the target, we need a basic system with BusyBox and
initialization scripts. We will re-use the system built in the {\em A
  tiny embedded system} lab, so copy this system in the target
directory:

\bashcmd{$ cp -a $HOME/__SESSION_NAME__-labs/tinysystem/nfsroot/* target/}

Note that for this lab, a lot of typing will be required. To save time
typing, we advise you to copy and paste commands from the electronic
version of these instructions.

\section{Testing}

Make sure the \code{target/} directory is exported by your NFS server
to your board by modifying \code{/etc/exports} and restarting your NFS
server.

Make your board boot from this new directory through NFS.

\section{alsa-lib}

\code{alsa-lib} is a library supposed to handle the interaction with
the ALSA subsystem. It is available at
\url{https://alsa-project.org}. Download version 1.2.3.2 (there's an
issue in version 1.2.4 for the moment), and extract it
in \code{$HOME/__SESSION_NAME__-labs/thirdparty/}.

{\bf Tip}: if the website for any of the source packages that we
need to download in the next sections is down, a great mirror
that you can use is \url{http://sources.buildroot.net/}.

Back to \code{alsa-lib} sources, look at the \code{configure} script
and see see that it has been generated by \code{autoconf} (the header
contains a sentence like {\em Generated by GNU Autoconf 2.69}). Most of
the time, \code{autoconf} comes with \code{automake}, that generates
Makefiles from \code{Makefile.am} files. So alsa-lib uses a rather
common build system. Let's try to configure and build it:

\begin{bashinput}
$ ./configure
$ make
\end{bashinput}

You can see that the files are getting compiled with gcc, which
generates code for x86 and not for the target platform. This is
obviously not what we want, so we clean-up the object and tell the
configure script to use the ARM cross-compiler:

\begin{bashinput}
$ make clean
$ CC=arm-linux-gcc ./configure
\end{bashinput}

Of course, the \code{arm-linux-gcc} cross-compiler must be in your
\code{PATH} prior to running the configure script. The \code{CC} environment
variable is the classical name for specifying the compiler to
use.

Quickly, you should get an error saying:

%\footnotesize
\begin{terminaloutput}
checking whether we are cross compiling... configure: error: in `.../thirdparty/alsa-lib-1.1.6':
configure: error: cannot run C compiled programs.
If you meant to cross compile, use `--host'.
See `config.log' for more details
\end{terminaloutput}
\normalsize

If you look at the \code{config.log} file, you can see that the
\code{configure} script compiles a binary with the cross-compiler
and then tries to run it on the development workstation. This is a
rather usual thing to do for a \code{configure} script, and that's
why it tests so early that it's actually doable, and bails out if not.

Obviously, it cannot work in our case, and the scripts exits. The job
of the \code{configure} script is to test the configuration of the system. To
do so, it tries to compile and run a few sample applications to test
if this library is available, if this compiler option is supported,
etc. But in our case, running the test examples is definitely not
possible.

We need to tell the \code{configure} script that we are cross-compiling, and
this can be done using the \code{--build} and \code{--host} options,
as described in the help of the \code{configure} script:

\begin{verbatim}
System types:
  --build=BUILD	configure for building on BUILD [guessed]
  --host=HOST	cross-compile to build programs to run on HOST [BUILD]
\end{verbatim}

The \code{--build} option allows to specify on which system the
package is built, while the \code{--host} option allows to specify on
which system the package will run. By default, the value of the
\code{--build} option is guessed and the value of \code{--host} is the
same as the value of the \code{--build} option. The value is guessed
using the \code{./config.guess} script, which on your system should
return \code{i686-pc-linux-gnu}. See
\url{https://www.gnu.org/software/autoconf/manual/html_node/Specifying-Names.html}
for more details on these options.

So, let's override the value of the \code{--host} option:

\bashcmd{$ CC=arm-linux-gcc ./configure --host=arm-linux}

The \code{configure} script should end properly now, and create a
Makefile. Run the \code{make} command, which should run just fine.

Look at the result of compiling in \code{src/.libs}: a set of object files
and a set of \code{libasound.so*} files.

The \code{libasound.so*} files are a dynamic version of the
library. The shared library itself is \code{libasound.so.2.0.0}, it has
been generated by the following command line:

\begin{bashinput}
$ arm-linux-gcc -shared conf.o confmisc.o input.o output.o async.o error.o dlmisc.o socket.o shmarea.o userfile.o names.o -lm -ldl -lpthread -lrt -Wl,-soname -Wl,libasound.so.2 -o libasound.so.2.0.0
\end{bashinput}

And creates the symbolic links \code{libasound.so} and
\code{libasound.so.2}.

\begin{bashinput}
$ ln -s libasound.so.2.0.0 libasound.so.2
$ ln -s libasound.so.2.0.0 libasound.so
\end{bashinput}

These symlinks are needed for two different reasons:

\begin{itemize}
\item \code{libasound.so} is used at compile time when you want to
  compile an application that is dynamically linked against the
  library. To do so, you pass the \code{-lLIBNAME} option to the
  compiler, which will look for a file named
  \code{lib<LIBNAME>.so}. In our case, the compilation option is
  \code{-lasound} and the name of the library file is
  \code{libasound.so}. So, the \code{libasound.so} symlink is needed
  at compile time;
\item \code{libasound.so.2} is needed because it is the {\em SONAME}
  of the library. {\em SONAME} stands for {\em Shared Object Name}. It
  is the name of the library as it will be stored in applications
  linked against this library. It means that at runtime, the dynamic
  loader will look for exactly this name when looking for the shared
  library. So this symbolic link is needed at runtime.
\end{itemize}

To know what's the {\em SONAME} of a library, you can use:
\bashcmd{$ arm-linux-readelf -d libasound.so.2.0.0}

and look at the \code{(SONAME)} line. You'll also see that this
library needs the C library, because of the \code{(NEEDED)} line on
\code{libc.so.0}.

The mechanism of \code{SONAME} allows to change the library without
recompiling the applications linked with this library. Let's say that
a security problem is found in the alsa-lib release that provides
libasound 2.0.0, and fixed in the next alsa-lib release, which will
now provide libasound 2.0.1.

You can just recompile the library, install it on your target system,
change the \code{libasound.so.2} link so that it points to
\code{libasound.so.2.0.1} and restart your applications. And it will
work, because your applications don't look specifically for
\code{libasound.so.2.0.0} but for the {\em SONAME}
\code{libasound.so.2}.

However, it also means that as a library developer, if you break the
ABI of the library, you must change the {\em SONAME}: change from
\code{libasound.so.2} to \code{libasound.so.3}.

Finally, the last step is to tell the \code{configure} script where the
library is going to be installed. Most \code{configure} scripts consider that
the installation prefix is \code{/usr/local/} (so that the library is
installed in \code{/usr/local/lib}, the headers in
\code{/usr/local/include}, etc.). But in our system, we simply want
the libraries to be installed in the \code{/usr} prefix, so let's tell
the \code{configure} script about this:

\begin{bashinput}
$ CC=arm-linux-gcc ./configure --host=arm-linux --disable-python --prefix=/usr
$ make
\end{bashinput}

For this library, this option may not change anything to the resulting
binaries, but for safety, it is always recommended to make sure that
the prefix matches where your library will be running on the target
system.

Do not confuse the {\em prefix} (where the application or library will
be running on the target system) from the location where the
application or library will be installed on your host while building
the root filesystem.

For example, libasound will be installed in
\code{$HOME/__SESSION_NAME__-labs/thirdparty/target/usr/lib/} because this is
the directory where we are building the root filesystem, but once our
target system will be running, it will see libasound in
\code{/usr/lib}.

The prefix corresponds to the path in the target system and {\bf
  never} on the host. So, one should {\bf never} pass a prefix like
\code{$HOME/__SESSION_NAME__-labs/thirdparty/target/usr}, otherwise at
runtime, the application or library may look for files inside this
directory on the target system, which obviously doesn't exist! By
default, most build systems will install the application or library in
the given prefix (\code{/usr} or \code{/usr/local}), but with most
build systems (including {\em autotools}), the installation prefix can
be overridden, and be different from the configuration prefix.

We now only have the installation process left to do.

First, let's make the installation in the {\em staging} space:
\bashcmd{$ make DESTDIR=$HOME/__SESSION_NAME__-labs/thirdparty/staging install}

Now look at what has been installed by alsa-lib:
\begin{itemize}
\item Some configuration files in \code{/usr/share/alsa}
\item The headers in \code{/usr/include}
\item The shared library and its libtool (\code{.la}) file in \code{/usr/lib}
\item A pkgconfig file in \code{/usr/lib/pkgconfig}. We'll come back
  to these later
\end{itemize}

Finally, let's install the library in the {\em target} space:

\begin{enumerate}
\item Create the \code{target/usr/lib} directory, it will contain the
  stripped version of the library
\item Copy the dynamic version of the library. Only
  \code{libasound.so.2} and \code{libasound.so.2.0.0} are needed,
  since \code{libasound.so.2} is the {\em SONAME} of the library and
  \code{libasound.so.2.0.0} is the real binary:
  \begin{itemize}
  \item \bashcmd{$ cp -a staging/usr/lib/libasound.so.2* target/usr/lib}
  \end{itemize}
\item Measure the size of the \code{target/usr/lib/libasound.so.2.0.0}
  library before stripping.
\item Strip the library:
  \begin{itemize}
  \item \bashcmd{$ arm-linux-strip target/usr/lib/libasound.so.2.0.0}
  \end{itemize}
\item Measure the size of the \code{target/usr/lib/libasound.so.2.0.0}
  library library again after stripping. How many unnecessary bytes
  were saved?
\end{enumerate}

And we're done with alsa-lib!

\section{Alsa-utils}

Download alsa-utils from the ALSA offical webpage. We tested the lab
with version 1.2.4.

Once uncompressed, we quickly discover that the alsa-utils build
system is based on the {\em autotools}, so we will work once again
with a regular \code{configure} script.

As we've seen previously, we will have to provide the prefix and host
options and the CC variable:

\bashcmd{$ CC=arm-linux-gcc ./configure --host=arm-linux --prefix=/usr}

Now, we should quiclky get an error in the execution of the
\code{configure} script:

\begin{verbatim}
checking for libasound headers version >= 1.0.27... not present.
configure: error: Sufficiently new version of libasound not found.
\end{verbatim}

Again, we can check in \code{config.log} what the \code{configure}
script is trying to do:

%\footnotesize
\begin{terminaloutput}
configure:7146: checking for libasound headers version >= 1.0.27
configure:7208: arm-linux-gcc -c -g -O2  conftest.c >&5
conftest.c:12:10: fatal error: alsa/asoundlib.h: No such file or directory
\end{terminaloutput}
\normalsize

Of course, since {\em alsa-utils} uses {\em alsa-lib}, it includes
its header file! So we need to tell the C compiler where the headers
can be found: there are not in the default directory
\code{/usr/include/}, but in the \code{/usr/include} directory of our
{\em staging} space. The help text of the \code{configure} script says:

\begin{verbatim}
  CPPFLAGS              C/C++/Objective C preprocessor flags,
                        e.g. -I<include�dir> if you have headers
                        in a nonstandard directory <include�dir>
\end{verbatim}

Let's use it:

\begin{bashinput}
$ CPPFLAGS=-I$HOME/__SESSION_NAME__-labs/thirdparty/staging/usr/include \%\linebreak
CC=arm-linux-gcc %\% \linebreak
./configure --host=arm-linux --prefix=/usr
\end{bashinput}

Now, it should stop a bit later, this time with the error:
\begin{verbatim}
checking for libasound headers version >= 1.0.27... found.
checking for snd_ctl_open in -lasound... no
configure: error: No linkable libasound was found.
\end{verbatim}

The \code{configure} script tries to compile an application against {\em
  libasound} (as can be seen from the \code{-lasound} option): {\em
  alsa-utils} uses {\em alsa-lib}, so the \code{configure} script
wants to make sure this library is already installed. Unfortunately,
the \code{ld} linker doesn't find it. So, let's tell the
linker where to look for libraries using the \code{-L} option followed
by the directory where our libraries are (in
\code{staging/usr/lib}). This \code{-L} option can be passed to the
linker by using the \code{LDFLAGS} at configure time, as told by the
help text of the \code{configure} script:

\begin{verbatim}
  LDFLAGS       linker flags, e.g. -L<lib dir> if you have
                libraries in a nonstandard directory <lib dir>
\end{verbatim}

Let's use this \code{LDFLAGS} variable:

\begin{bashinput}
$ LDFLAGS=-L$HOME/__SESSION_NAME__-labs/thirdparty/staging/usr/lib \ 
     CPPFLAGS=-I$HOME/__SESSION_NAME__-labs/thirdparty/staging/usr/include \ 
     CC=arm-linux-gcc \
     ./configure --host=arm-linux --prefix=/usr
\end{bashinput}

Once again, it should fail a bit further down the tests, this time
complaining about a missing {\em curses helper header}. {\em curses}
or {\em ncurses} is a graphical framework to design UIs in the
terminal. This is only used by {\em alsamixer}, one of the tools
provided by {\em alsa-utils}, that we are not going to use.
Hence, we can just disable the build of {\em alsamixer}.

Of course, if we wanted it, we would have had to build {\em ncurses} first,
just like we built {\em alsa-lib}. We will also need to disable support
for {\em xmlto} that generates the documentation.

\begin{bashinput}
$ LDFLAGS=-L$HOME/__SESSION_NAME__-labs/thirdparty/staging/usr/lib \
     CPPFLAGS=-I$HOME/__SESSION_NAME__-labs/thirdparty/staging/usr/include \
     CC=arm-linux-gcc \
     ./configure --host=arm-linux --prefix=/usr \
     --disable-alsamixer --disable-xmlto
\end{bashinput}

Then, run the compilation with \code{make}. Hopefully, it works!

Let's now begin the installation process.  Before really installing in
the staging directory, let's install in a dummy directory, to see
what's going to be installed (this dummy directory will not be used
afterwards, it is only to verify what will be installed before
polluting the staging space):

\bashcmd{$ make DESTDIR=/tmp/alsa-utils/ install}

The \code{DESTDIR} variable can be used with all Makefiles based on
\code{automake}. It allows to override the installation directory:
instead of being installed in the configuration prefix directory, the
files will be installed in \code{DESTDIR/configuration-prefix}.

Now, let's see what has been installed in \code{/tmp/alsa-utils/} (run
\code{tree /tmp/alsa-utils}):

\begin{verbatim}
/tmp/alsa-utils/
|-- lib
|   |-- systemd
|   |   `-- system
|   |       |-- alsa-restore.service
|   |       |-- alsa-state.service
|   |       `-- sound.target.wants
|   |           |-- alsa-restore.service -> ../alsa-restore.service
|   |           `-- alsa-state.service -> ../alsa-state.service
|   `-- udev
|       `-- rules.d
|           |-- 89-alsa-ucm.rules
|           `-- 90-alsa-restore.rules
|-- usr
|   |-- bin
|   |   |-- aconnect
|   |   |-- alsabat
|   |   |-- alsaloop
|   |   |-- alsatplg
|   |   |-- alsaucm
|   |   |-- amidi
|   |   |-- amixer
|   |   |-- aplay
|   |   |-- aplaymidi
|   |   |-- arecord -> aplay
|   |   |-- arecordmidi
|   |   |-- aseqdump
|   |   |-- aseqnet
|   |   |-- axfer
|   |   |-- iecset
|   |   `-- speaker-test
|   |-- sbin
|   |   |-- alsabat-test.sh
|   |   |-- alsaconf
|   |   |-- alsactl
|   |   `-- alsa-info.sh
|   `-- share
|       |-- alsa
|       |   |-- init
|       |   |   |-- 00main
|       |   |   |-- ca0106
|       |   |   |-- default
|       |   |   |-- hda
|       |   |   |-- help
|       |   |   |-- info
|       |   |   `-- test
|       |   `-- speaker-test
|       |       `-- sample_map.csv
|       |-- man
|       |   |-- fr
|       |   |   `-- man8
|       |   |       `-- alsaconf.8
|       |   |-- man1
|       |   |   |-- aconnect.1
|       |   |   |-- alsabat.1
|       |   |   |-- alsactl.1
|       |   |   |-- alsa-info.sh.1
|       |   |   |-- alsaloop.1
|       |   |   |-- amidi.1
|       |   |   |-- amixer.1
|       |   |   |-- aplay.1
|       |   |   |-- aplaymidi.1
|       |   |   |-- arecord.1 -> aplay.1
|       |   |   |-- arecordmidi.1
|       |   |   |-- aseqdump.1
|       |   |   |-- aseqnet.1
|       |   |   |-- axfer.1
|       |   |   |-- axfer-list.1
|       |   |   |-- axfer-transfer.1
|       |   |   |-- iecset.1
|       |   |   `-- speaker-test.1
|       |   |-- man7
|       |   `-- man8
|       |       `-- alsaconf.8
|       `-- sounds
|           `-- alsa
|               |-- Front_Center.wav
|               |-- Front_Left.wav
|               |-- Front_Right.wav
|               |-- Noise.wav
|               |-- Rear_Center.wav
|               |-- Rear_Left.wav
|               |-- Rear_Right.wav
|               |-- Side_Left.wav
|               `-- Side_Right.wav
`-- var
    `-- lib
        `-- alsa

24 directories, 63 files
\end{verbatim}

So, we have:
\begin{itemize}
\item The systemd service definitions in \code{lib/systemd}
\item The udev rules in \code{lib/udev}
\item The alsa-utils binaries in \code{/usr/bin} and \code{/usr/sbin}
\item Some sound samples in \code{/usr/share/sounds}
\item The various translations in \code{/usr/share/locale}
\item The manual pages in \code{/usr/share/man/}, explaining how to
  use the various tools
\item Some configuration samples in \code{/usr/share/alsa}.
\end{itemize}

Now, let's make the installation in the {\em staging} space:

\bashcmd{$ make DESTDIR=$HOME/__SESSION_NAME__-labs/thirdparty/staging/ install}

Then, let's install only the necessary files in the {\em target}
space, manually:

\begin{bashinput}
$ cd ..
$ cp -a staging/usr/bin/a* staging/usr/bin/speaker-test target/usr/bin/
$ cp -a staging/usr/sbin/alsa* target/usr/sbin
$ arm-linux-strip target/usr/bin/a*
$ arm-linux-strip target/usr/bin/speaker-test
$ arm-linux-strip target/usr/sbin/alsactl
$ mkdir -p target/usr/share/alsa/pcm
$ cp -a staging/usr/share/alsa/alsa.conf* target/usr/share/alsa/
$ cp -a staging/usr/share/alsa/cards target/usr/share/alsa/
$ cp -a staging/usr/share/alsa/pcm/default.conf target/usr/share/alsa/pcm/
\end{bashinput}

And we're finally done with {\em alsa-utils}!

Now test that all is working fine by running the \code{speaker-test} util on
your board, with the headset provided by your instructor plugged
in. You may need to add the missing libraries from the toolchain
install directory.

{\bf Caution}: don't copy the \code{dmix.conf} file. \code{speaker-test}
will tell you that it cannot find this file, but it won't work if you
copy this file from the staging area.

The sound you get will be mainly noise (as what you would get by
running \code{speaker-test} on your PCs). This is a way to test
all possible frequencies, but is not really meant for a human
to listen to. At least, sound output is
showing some signs of life! It will get much better when we play
samples with \code{ogg123}.

\section{libogg}

Now, let's work on {\em libogg}. Download the 1.3.4 version from
\url{https://xiph.org} and extract it.

Configuring {\em libogg} is very similar to the configuration of the
previous libraries:

\bashcmd{$ CC=arm-linux-gcc ./configure --host=arm-linux --prefix=/usr}

Of course, compile the library:

\bashcmd{$ make}

Installation to the {\em staging} space can be done using the
classical \code{DESTDIR} mechanism:

\bashcmd{$ make DESTDIR=$HOME/__SESSION_NAME__-labs/thirdparty/staging/ install}

And finally, only install manually in the {\em target} space the files
needed at runtime:

\begin{bashinput}
$ cd ..
$ cp -a staging/usr/lib/libogg.so.0* target/usr/lib/
$ arm-linux-strip target/usr/lib/libogg.so.0.8.4
\end{bashinput}

Done with {\em libogg}!

\section{libvorbis}

{\em Libvorbis} is the next step. Grab the 1.3.7 version from
\url{https://xiph.org} and uncompress it.

Once again, the {\em libvorbis} build system is a nice example of what can
be done with a good usage of the autotools. Cross-compiling {\em
libvorbis} is very easy, and almost identical to what we've seen
with {\em alsa-utils}. First, the \code{configure} step:

\bashcmd{$ CC=arm-linux-gcc ./configure --host=arm-linux --prefix=/usr}

It will fail with:

\begin{verbatim}
configure: error: Ogg >= 1.0 required !
\end{verbatim}

By running \code{./configure --help}, you will find the
\code{--with-ogg-libraries} and \code{--with-ogg-includes} options.
Use these:

\begin{bashinput}
$ CC=arm-linux-gcc ./configure --host=arm-linux --prefix=/usr \
    --with-ogg-includes=$HOME/__SESSION_NAME__-labs/thirdparty/staging/usr/include \
    --with-ogg-libraries=$HOME/__SESSION_NAME__-labs/thirdparty/staging/usr/lib
\end{bashinput}

Then, compile the library:

\bashcmd{$ make}

Install it in the {\em staging} space:

\bashcmd{$ make DESTDIR=$HOME/__SESSION_NAME__-labs/thirdparty/staging/ install}

And install only the required files in the {\em target} space:

\begin{bashinput}
$ cd ..
$ cp -a staging/usr/lib/libvorbis.so.0* target/usr/lib/
$ arm-linux-strip target/usr/lib/libvorbis.so.0.4.9
$ cp -a staging/usr/lib/libvorbisfile.so.3* target/usr/lib/
$ arm-linux-strip target/usr/lib/libvorbisfile.so.3.3.8
\end{bashinput}

And we're done with {\em libvorbis}!

\section{libao}

Now, let's work on {\em libao}. Download the 1.2.0 version from
\url{https://xiph.org} and extract it.

Configuring {\em libao} is once again fairly easy, and similar to
every sane autotools based build system:

\begin{bashinput}
$ LDFLAGS=-L$HOME/__SESSION_NAME__-labs/thirdparty/staging/usr/lib \
    CPPFLAGS=-I$HOME/__SESSION_NAME__-labs/thirdparty/staging/usr/include \
    CC=arm-linux-gcc ./configure --host=arm-linux --prefix=/usr
\end{bashinput}

Of course, compile the library:

\bashcmd{$ make}

Installation to the {\em staging} space can be done using the
classical \code{DESTDIR} mechanism:

\bashcmd{$ make DESTDIR=$HOME/__SESSION_NAME__-labs/thirdparty/staging/ install}

And finally, install manually the only needed files at runtime in the
{\em target} space:

\begin{bashinput}
$ cd ..
$ cp -a staging/usr/lib/libao.so.4* target/usr/lib/
$ arm-linux-strip target/usr/lib/libao.so.4.1.0
\end{bashinput}

We will also need the alsa plugin that is loaded dynamically by
{\em libao} at startup:
\begin{bashinput}
$ mkdir -p target/usr/lib/ao/plugins-4/
$ cp -a staging/usr/lib/ao/plugins-4/libalsa.so target/usr/lib/ao/plugins-4/
\end{bashinput}

Done with {\em libao}!

\section{vorbis-tools}

Finally, thanks to all the libraries we compiled previously, all the
dependencies are ready. We can now build the vorbis tools themselves.
Download the 1.4.2 version from the official website, at
\url{https://xiph.org/}. As usual, extract the tarball.

Before starting the configuration, let's have a look at the available
options by running \code{./configure --help}. Many options are
available. We see that we can, in addition to the usual autotools
configuration options:

\begin{itemize}
\item Enable/Disable the various tools that are going to be built:
  \code{ogg123}, \code{oggdec}, \code{oggenc}, etc.
\item Enable or disable support for various other codecs: FLAC, Speex,
  etc.
\item Enable or disable the use of various libraries that can
  optionally be used by the vorbis tools
\end{itemize}

So, let's begin with our usual \code{configure} line:

\begin{bashinput}
$ LDFLAGS=-L$HOME/__SESSION_NAME__-labs/thirdparty/staging/usr/lib \
    CPPFLAGS=-I$HOME/__SESSION_NAME__-labs/thirdparty/staging/usr/include \
    CC=arm-linux-gcc \
    ./configure --host=arm-linux --prefix=/usr
\end{bashinput}

At the end, you should see the following warning:

\begin{verbatim}
configure: WARNING: Prerequisites for ogg123 not met, ogg123 will be skipped.
Please ensure that you have POSIX threads, libao, and (optionally) libcurl
libraries and headers present if you would like to build ogg123.
\end{verbatim}

Which is unfortunate, since we precisely want \code{ogg123}.

If you look back at the script output, you should see that at some point
that it tests for {\em libao} and fails to find it:

\begin{verbatim}
checking for AO... no
configure: WARNING: libao too old; >= 1.0.0 required
\end{verbatim}

If you look into the \code{config.log} file now, you should find
something like:

\begin{verbatim}
configure:22343: checking for AO
configure:22351: $PKG_CONFIG --exists --print-errors "ao >= 1.0.0"
Package ao was not found in the pkg-config search path.
Perhaps you should add the directory containing `ao.pc'
to the PKG_CONFIG_PATH environment variable
No package 'ao' found
\end{verbatim}

In this case, the \code{configure} script uses the {\em pkg-config} system to
get the configuration parameters to link the library against {\em libao}. By
default, {\em pkg-config} looks in \code{/usr/lib/pkgconfig/} for
\code{.pc} files, and because the \code{libao-dev} package is probably
not installed in your system the \code{configure} script will not find
{\em libao} library that we just compiled.

It would have been worse if we had the package installed, because it
would have detected and used our host package to compile {\em libao}, which,
since we're cross-compiling, is a pretty bad thing to do.

This is one of the biggest issue with cross-compilation: mixing host
and target libraries, because build systems have a tendency to look
for libraries in the default paths.

So, now, we must tell {\em pkg-config} to look inside the
\code{/usr/lib/pkgconfig/} directory of our {\em staging} space. This
is done through the \code{PKG_CONFIG_LIBDIR} environment variable, as
explained in the manual page of \code{pkg-config}.

Moreover, the \code{.pc} files contain references to paths. For
example, in
\code{$HOME/__SESSION_NAME__-labs/thirdparty/staging/usr/lib/pkgconfig/ao.pc},
we can see:

\begin{verbatim}
prefix=/usr
exec_prefix=${prefix}
libdir=${exec_prefix}/lib
includedir=${prefix}/include
[...]
Libs: -L${libdir} -lao
Cflags: -I${includedir}
\end{verbatim}

So we must tell \code{pkg-config} that these paths are not absolute,
but relative to our {\em staging} space. This can be done using the
\code{PKG_CONFIG_SYSROOT_DIR} environment variable.

Then, let's run the configuration of the vorbis-tools again, passing
the \code{PKG_CONFIG_LIBDIR} and \code{PKG_CONFIG_SYSROOT_DIR}
environment variables:

\begin{bashinput}
$ LDFLAGS=-L$HOME/__SESSION_NAME__-labs/thirdparty/staging/usr/lib \
    CPPFLAGS=-I$HOME/__SESSION_NAME__-labs/thirdparty/staging/usr/include \
    PKG_CONFIG_LIBDIR=$HOME/__SESSION_NAME__-labs/thirdparty/staging/usr/lib/pkgconfig \
    PKG_CONFIG_SYSROOT_DIR=$HOME/__SESSION_NAME__-labs/thirdparty/staging \
    CC=arm-linux-gcc \
    ./configure --host=arm-linux --prefix=/usr
\end{bashinput}
\normalsize

Now, the \code{configure} script should end properly, we can now start the
compilation:
\bashcmd{$ make}

It should fail with the following cryptic message:
\begin{terminaloutput}
make[2]: Entering directory '/home/tux/__SESSION_NAME__-labs/thirdparty/vorbis-tools-1.4.0/ogg123'
if arm-linux-gcc -DSYSCONFDIR=\"/usr/etc\" -DLOCALEDIR=\"/usr/share/locale\" -DHAVE_CONFIG_H -I. -I. -I.. -I/usr/include -I../include -I../intl  -I/home/tux/__SESSION_NAME__-labs/thirdparty/staging/usr/include  -O2 -Wall -ffast-math -fsigned-char -g -O2 -MT audio.o -MD -MP -MF ".deps/audio.Tpo" -c -o audio.o audio.c; \
then mv -f ".deps/audio.Tpo" ".deps/audio.Po"; else rm -f ".deps/audio.Tpo"; exit 1; fi
In file included from audio.c:22:
/usr/include/stdio.h:27:10: fatal error: bits/libc-header-start.h: No such file or directory
\end{terminaloutput}
\normalsize

You can notice that \code{/usr/include} is added to the include paths.
Again, this is not what we want because it contains includes for the
host, not the target. It is coming from the autodetected value for
\code{CURL_CFLAGS}.

Add the \code{--without-curl} option to the \code{configure} invocation,
restart the compilation.

Finally, it builds!

Now, install the vorbis-tools to the {\em staging} space using:

\bashcmd{$ make DESTDIR=$HOME/__SESSION_NAME__-labs/thirdparty/staging/ install}

And then install them in the {\em target} space:

\begin{bashinput}
$ cd ..
$ cp -a staging/usr/bin/ogg* target/usr/bin
$ arm-linux-strip target/usr/bin/ogg*
\end{bashinput}

You can now test that everything works! Run \code{ogg123} on the
sample file found in \code{thirdparty/data}.

There should still be one missing C library object. Copy it, and you
should get:
+
\begin{verbatim}
ERROR: Failed to load plugin /usr/lib/ao/plugins-4/libalsa.so => dlopen() failed
=== Could not load default driver and no driver specified in config file. Exiting.
\end{verbatim}

This error message is unfortunately not sufficient to figure out what's going wrong.
It's a good opportunity to use the \code{strace} utility (covered in
upcoming lectures) to get more details about what's going on. To do so,
you can use the one built by Crosstool-ng inside the
toolchain \code{target/usr/bin} directory.

You can now run \code{ogg123} through \code{strace}:
\bashcmd{$ strace ogg123 /sample.ogg}

You can see that the command fails to open the \code{ld-uClibc.so.1} file:
\begin{terminaloutput}
open("/lib/ld-uClibc.so.1", O_RDONLY)   = -1 ENOENT (No such file or directory)
open("/lib/ld-uClibc.so.1", O_RDONLY)   = -1 ENOENT (No such file or directory)
open("/usr/lib/ld-uClibc.so.1", O_RDONLY) = -1 ENOENT (No such file or directory)
open("/usr/X11R6/lib/ld-uClibc.so.1", O_RDONLY) = -1 ENOENT (No such file or directory)
open("/home/tux/__SESSION_NAME__-labs/thirdparty/staging/usr/lib/ld-uClibc.so.1", O_RDONLY) = -1 ENOENT (No such file or directory)
write(2, "ERROR: Failed to load plugin ", 29ERROR: Failed to load plugin ) = 29
write(2, "/usr/lib/ao/plugins-4/libalsa.so", 32/usr/lib/ao/plugins-4/libalsa.so) = 32
write(2, " => dlopen() failed\n", 20 => dlopen() failed
\end{terminaloutput}

Now, look for \code{ld-uClibc.so.1} in the toolchain. You can see that both \code{ld-uClibc.so.1}
and \code{ld-uClibc.so.0} are symbolic links to the same file. So, create the missing
link under \code{target/lib} and run \code{ogg123} again.

Everything should work fine now. Enjoy the sound sample!

\ifdefstring{\labboard}{qemu}
{{\bf Known issue}: the \code{sample.ogg} sample will play in a weird way
in QEMU (too slow, apparently, we haven't found any solution yet). To have a
correct result instead, use the \code{aplay} command from alsa-utils, and play the
\code{sample.wav} file provided together with \code{sample.ogg}.}
{}

To finish this lab completely, and to be consistent with what we've done before,
let's strip the libraries in \code{target/lib}:
\bashcmd{$ arm-linux-strip target/lib/*}
