\subchapter
{Debugging application issues}
{Objectives:
  \begin{itemize}
    \item Analyze dynamic library calls from an application using
            {\em ltrace}.
    \item Using {\em strace} to analyze program syscalls.
  \end{itemize}
}

\section{ltrace}

Go into the \code{ltrace} folder

\begin{bashinput}
$ cd /root/ltrace/
\end{bashinput}

From there, run the \code{traceme} application:

\begin{bashinput}
$ export LD_LIBRARY_PATH=$PWD
$ ./authent
Error: failed to authenticate the user !
\end{bashinput}

As you can see, it seems our application is failing to correctly authenticate
the system. Using {\em ltrace}, trace the application in order to understand
what is going on. From what we can see, some function is failing and seems to
returns 1 instead of 0.

In order to overload this check, we can use a \code{LD_PRELOAD} a library.
We'll override the \code{al_authent_user()} based on the
\code{authent_library.h} definitions. Create a file \code{overload.c} which
override the \code{al_authent_user()}, prints the user, password and returns 0. 
Compile it using the following command line:

\begin{bashinput}
$ gcc -fPIC -shared overload.c -o overload.so
\end{bashinput}

Finally, run your application and preload the new library using the following
command:
\begin{bashinput}
$ LD_PRELOAD=./overload.so ./authent
\end{bashinput}

\section{strace}

\code{strace} is useful to debug an application when you don't have the source.
For that example, use the \code{strace_me} binary that is present in on the
target in \code{/root/strace} and run it with strace:

\begin{bashinput}
$ cd /root/strace
$ strace ./strace_me
\end{bashinput}

Based on the output and running strace with other options, try to answer the
following questions:
\begin{itemize}
  \item What are the files that are opened by this binary ?
  \item How many time is \code{read()} called ?
  \item Which \code{openat} system call failed ?
  \item How many system calls are issued by the program ?
\end{itemize}