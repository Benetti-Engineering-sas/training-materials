\subchapter
{System wide profiling and tracing}
{Objectives:
  \begin{itemize}
    \item System profiling with {\em perf}.
    \item IRQ latencies using {\em ftrace}.
    \item Tracing and visualizing system activity using {\em kernelshark} or
          {\em LTTng}
  \end{itemize}
}

\section{System profiling with {\em perf}}



\section{IRQ latencies using {\em ftrace}}

{\em ftrace} features a specific tracers for irq latency which is named irqsoff.
Using this tracer, analyze the system irqs latency. Run the following command
for a few seconds and hit \code{Ctrl + [C]} to stop it.

\begin{bashinput}
$ trace-cmd record -p irqsoff
^C
\end{bashinput}

Once done, you can visualize which section of code generated too much latency by
using:

\begin{bashinput}
$ trace-cmd report
\end{bashinput}

This will display a trace of the longest chain of calls while itnerrupts were
disabled. Based on this report, can you find the code that generates this
latency ?

\section{LTTng}

Go into the primes directory and open the \code{primes.c} file. This file
computes primes number and output them in a file provided as an argument.
Compile it using the provided toolchain

\begin{bashinput}
  $(CROSS_COMPILE)-gcc -Wall -Werror primes.c -o primes
\end{bashinput}

In order to observe our program performances, we want to instrument it with
tracepoints. We would like to know two things:

\begin{itemize}
  \item How much time it takes to generate a prime number.
  \item How much times it takes to write data to the output file.
\end{itemize}

In order to do so, add tracepoints to your program which will allows to measure
this. We'll add 4 tracepoints:

\begin{itemize}
  \item One for the start of prime number computation
  \item Another for the end of computation
  \item One for the start of writing to disk
  \item Another for the end of writing to disk
\end{itemize}

For that, create a tracepoint provider header file template named
\code{primes-tp.h} and another one for the tracepoint provider named
\code{primes-tp.c}. You can then use the new tracepoints in your program to
trace specific points of execution.

Once added, you can compile your application using the following command.

\begin{bashinput}
$ $(CROSS_COMPILE)-gcc -I. primes-tp.c primes.c  -llttng-ust -o primes
\end{bashinput}

Enable the program tracepoints, run it and collect tracepoints.

Download \code{tracecompass} latest version from the eclipse website:
(https://www.eclipse.org/tracecompass/) and extract it using:

\begin{bashinput}
  tar -xvf trace-compass-*.tar.gz
\end{bashinput}

Run it
\begin{bashinput}
$ cd trace-compass*
$ ./tracecompass
\end{bashinput}
