\subchapter{Using a build system, example with Buildroot}{Objectives:
  discover how a build system is used and how it works, with the
  example of the Buildroot build system.}

\section{Goals}

Compared to the previous lab, we are going to build a more elaborate
system, still containing {\em alsa-utils} (and of course its {\em
alsa-lib} dependency), but this time using Buildroot,
an automated build system.

The automated build system will also allow us to add more packages
and play real audio on our system, thanks to the {\em Music Player
Daemon (mpd)} (\url{https://www.musicpd.org/} and its {\em mpc} client.

{\em Important note: because of the current sound playing issues
mentioned before, this lab will be less exhaustive compared to our
instructions on real hardware. You should be able to run the commands
in the QEMU emulated machine though, proving that the tools were built
correctly. So, we will build tools like mpd and mpc, but won't test
them because of the absence of sound.}

\section{Setup}

Go to the \code{$HOME/__SESSION_NAME__-labs/buildroot} directory.

\section{Get Buildroot and explore the source code}

The official Buildroot website is available at
\url{https://buildroot.org/}. Clone the {\em Git} repository:

\begin{bashinput}
git clone https://git.buildroot.net/buildroot
cd buildroot
\end{bashinput}

Now checkout the tag corresponding to the latest 2023.02.<n> release (Long
Term Support), which we have tested for this lab.

Several subdirectories or files are visible, the most important ones
are:

\begin{itemize}
\item \code{boot} contains the Makefiles and configuration items
  related to the compilation of common bootloaders (GRUB, U-Boot,
  Barebox, etc.)
\item \code{board} contains board specific configurations and
  root filesystem overlays.
\item \code{configs} contains a set of predefined configurations,
  similar to the concept of defconfig in the kernel.
\item \code{docs} contains the documentation for Buildroot.
\item \code{fs} contains the code used to generate the various root
  filesystem image formats
\item \code{linux} contains the Makefile and configuration items
  related to the compilation of the Linux kernel
\item \code{Makefile} is the main Makefile that we will use to use
  Buildroot: everything works through Makefiles in Buildroot;
\item \code{package} is a directory that contains all the Makefiles,
  patches and configuration items to compile the user space
  applications and libraries of your embedded Linux system. Have a
  look at various subdirectories and see what they contain;
\item \code{system} contains the root filesystem skeleton and the {\em
    device tables} used when a static \code{/dev} is used;
\item \code{toolchain} contains the Makefiles, patches and
  configuration items to generate the cross-compiling toolchain.
\end{itemize}


\section{Configure Buildroot}

In our case, we would like to:

\begin{itemize}
\item Generate an embedded Linux system for ARM;
\item Use an already existing external toolchain instead of having
  Buildroot generating one for us;
\item Compile the Linux kernel and deploy its modules in the root
  filesystem;
\item Integrate {\em BusyBox}, {\em alsa-utils},
  {\em mpd}, {\em mpc} and {\em evtest} in our embedded Linux system;
\item Integrate the target filesystem into a tarball
\end{itemize}

To run the configuration utility of Buildroot, simply run:

\bashcmd{$ make menuconfig}

Set the following options. Don't hesitate to press the \code{Help}
button whenever you need more details about a given option:

\begin{itemize}
\item \code{Target options}
  \begin{itemize}
  \item \code{Target Architecture}: \code{ARM (little endian)}
   \ifdefstring{\labboard}{stm32mp1}{
   \item \code{Target Architecture Variant}: \code{cortex-A7}
  }{
   \ifdefstring{\labboard}{qemu}{
   \item \code{Target Architecture Variant}: \code{cortex-A9}
   \item \code{Enable NEON SIMD extension support}: Enabled
   \item \code{Enable VFP extension support}: Enabled
   }{%beaglebone
   \item \code{Target Architecture Variant}: \code{cortex-A8}
   }
  }
  \item \code{Target ABI}: \code{EABIhf}
  \ifdefstring{\labboard}{stm32mp1}{
  \item \code{Floating point strategy}: \code{VFPv4}
  }{
   \ifdefstring{\labboard}{qemu}{
   \item \code{Floating point strategy}: \code{VFPv3-D16}
   }{%beaglebone
   \item \code{Floating point strategy}: \code{VFPv3-D16}
   }
  }
  \end{itemize}
\item \code{Toolchain}
  \begin{itemize}
  \item \code{Toolchain type}: \code{External toolchain}
  \item \code{Toolchain}: \code{Custom toolchain}
  \item \code{Toolchain path}: use the toolchain you built:
    \code{/home/<user>/x-tools/arm-training-linux-musleabihf}
    (replace \code{<user>} by your actual user name)
  \item \code{External toolchain gcc version}: \code{12.x}
  \item \code{External toolchain kernel headers series}: \texttt{\workingkernel.x}
  \item \code{External toolchain C library}: \code{musl (experimental)}
  \item We must tell Buildroot about our toolchain configuration, so
    select \code{Toolchain has SSP support?} and
    \code{Toolchain has C++ support?}.
    Buildroot will check these parameters anyway.
  \end{itemize}
\item \code{Target packages}
  \begin{itemize}
  \item Keep \code{BusyBox} (default version) and keep the BusyBox
    configuration proposed by Buildroot;
  \item \code{Audio and video applications}
    \begin{itemize}
    \item Select \code{alsa-utils}, and in the submenu:
    \begin{itemize}
         \item Select \code{alsamixer}. You will be able to
	       test this application too, and that will also pull
	       the \code{ncurses} library, which we will also use
	       in the next lab.
         \item Select \code{speaker-test}
    \end{itemize}
    \item Select \code{mpd}, and in the submenu:
    \begin{itemize}
         \item Keep only \code{alsa}, \code{vorbis} and \code{tcp sockets}
    \end{itemize}
    \item Select \code{mpd-mpc}.
    \end{itemize}
  \end{itemize}
\item \code{Filesystem images}
  \begin{itemize}
  \item Select \code{tar the root filesystem}
  \end{itemize}
\end{itemize}

Exit the menuconfig interface. Your configuration has now been saved
to the \code{.config} file.

\section{Generate the embedded Linux system}

Just run:

\bashcmd{$ make}

Buildroot will first create a small environment with the external
toolchain, then download, extract, configure, compile and install each
component of the embedded system.

All the compilation has taken place in the \code{output/} subdirectory. Let's
explore its contents:

\begin{itemize}

\item \code{build}, is the directory in which each component built by
  Buildroot is extracted, and where the build actually takes place

\item \code{host}, is the directory where Buildroot installs some
  components for the host. As Buildroot doesn't want to depend on too
  many things installed in the developer machines, it installs some
  tools needed to compile the packages for the target. In our case it
  installed {\em pkg-config} (since the version of the host may be ancient)
  and tools to generate the root filesystem image ({\em genext2fs},
  {\em makedevs}, {\em fakeroot}).

\item \code{images}, which contains the final images produced by
  Buildroot. In our case it contains a tarball of the filesystem, called
  \code{rootfs.tar}, plus the compressed kernel and Device Tree binary.
  Depending on the configuration, there could also a bootloader binary
  or a full SD card image.

\item \code{staging}, which contains the “build” space of the target
  system. All the target libraries, with headers and documentation. It
  also contains the system headers and the C library, which in our
  case have been copied from the cross-compiling toolchain.

\item \code{target}, is the target root filesystem. All applications
  and libraries, usually stripped, are installed in this
  directory. However, it cannot be used directly as the root
  filesystem, as all the device files are missing: it is not possible
  to create them without being root, and Buildroot has a policy of not
  running anything as root.

\end{itemize}

\section{Run the generated system}

Go back to the \code{$HOME/__SESSION_NAME__-labs/buildroot/} directory. Create
a new \code{nfsroot} directory that is going to hold our system,
exported over NFS. Go into this directory, and untar the rootfs using:

\bashcmd{$ tar xvf ../buildroot/output/images/rootfs.tar}

Add our \code{nfsroot} directory to the list of directories exported
by NFS in \code{/etc/exports}.

Also update the kernel and Device Tree binaries used by your board,
from the ones compiled by Buildroot in \code{output/images/}.

Boot the board, and log in (\code{root} account, no password).

You should now reach a shell.

Even though we have no sound at the moment, you can run \code{speaker-test}
to check that this application works. You can also test the \code{alsamixer}
command too.

By running the \code{ps} command, you may also check whether the \code{mpd}
server was started on your system. However, as said earlier, we won't try
to test it as we currently have no sound in QEMU on Ubuntu 22.04.

\section{Analyzing dependencies}

It's always useful to understand the dependencies drawn by the
packages we build.

First we need to install a {\em Graphviz}:

\bashcmd{$ sudo apt install graphviz}

Now, let's use Buildroot's target to generate a
dependency graph:

\bashcmd{$ make graph-depends}

We can now study the dependency graph:

\bashcmd{$ evince output/graphs/graph-depends.pdf}

In particular, you can see that adding MPD and its client
required to compile {\em Meson} for the host, and in turn,
{\em Python 3} for the host too. This substantially contributed to the
build time.


