\subchapter{Bootloader - U-Boot}{Objectives: Set up serial
  communication, compile and install the U-Boot bootloader, use basic
  U-Boot commands, set up TFTP communication with the development
  workstation.}

As the bootloader is the first piece of software executed by a
hardware platform, the installation procedure of the bootloader is
very specific to the hardware platform. There are usually two cases:

\begin{itemize}

\item The processor offers nothing to ease the installation of the
  bootloader, in which case the JTAG has to be used to initialize
  flash storage and write the bootloader code to flash. Detailed
  knowledge of the hardware is of course required to perform these
  operations.

\item The processor offers a monitor, implemented in ROM, and through
  which access to the memories is made easier.

\end{itemize}

The STM32MP1 SoC, falls into the second category. The monitor
integrated in the ROM reads the SD card to search for a valid
bootloader (the boot mode is actually configurable via a few input
pins). In case no bootloader is found, it will operate in a fallback
mode, that will allow to use an external tool to reflash some
executable through USB. Therefore, either by using an MMC/SD card or
that fallback mode, we can start up an STM32MP1-based board without
having anything installed on it.

\section{Setting up serial communication with the board}

Plug the USB-A to micro USB-B cable on the Discovery board. There is
only one micro USB port on the board, it is CN11, also named ST-LINK.
This is a debug interface and exposes multiple debugging interfaces,
including a serial interface. When plugged in your computer, a serial
port should appear, \code{/dev/ttyACM0}.

You can also see this device appear by looking at the output of
\code{dmesg}.

To communicate with the board through the serial port, install a
serial communication program, such as \code{picocom}:

\bashcmd{$ sudo apt install picocom}

You also need to make your user belong to the \code{dialout} group to be
allowed to write to the serial console:

\bashcmd{$ sudo adduser $USER dialout}

{\bf Important}: for the group change to be effective, you have to
{\em completely log out} from your session and log in again (no need to
reboot). A workaround is to run \code{newgrp dialout}, but it is not global.
You have to run it in each terminal.

Run \inlinebash{$ picocom -b 115200 /dev/ttyACM0}, to start serial
communication on \code{/dev/ttyACM0}, with a baudrate of 115200.

If you wish to exit \code{picocom}, press \code{[Ctrl][a]} followed by
\code{[Ctrl][x]}.

\section{U-Boot setup}

The boot process is done in two steps with the ROM monitor trying to
execute a first piece of software, called {\em fsbl}, from its
internal SRAM, that will initialize the DRAM, a second program, {\em
ssbl} that will in turn load Linux and execute it.

In our case, both programs are provided by U-Boot.

Download U-Boot:

\begin{bashinput}
$ git clone https://gitlab.denx.de/u-boot/u-boot.git
$ cd u-boot
$ git checkout v2021.01
\end{bashinput}

Get an understanding of U-Boot's configuration and compilation steps
by reading the \code{README} file, and specifically the {\em Building
the Software} section.

Basically, you need to:

\begin{enumerate}

\item Specify the cross-compiler prefix
(the part before \code{gcc} in the cross-compiler executable name):
\bashcmd{$ export CROSS_COMPILE=arm-linux-}

\item Run \inlinebash{$ make <NAME>_defconfig}, where the list of available
  configurations can be found in the \code{configs/} directory. There
  are three flavors of the stm32mp15 configuration: two to boot using
  secure boot and a basic one (\code{stm32mp15_basic}). This is the
  one we will use.

\item Now that you have a valid initial configuration, you can now
  run \inlinebash{$ make menuconfig} to further edit your bootloader features.
  \begin{itemize}

  \item In the \code{SPL / TPL} submenu, disable \code{Support an environment}, as
     SPL will jump to the second stage bootloader U-Boot, and not directly to Linux.

  \item In the \code{Environment} submenu, we will configure U-Boot so
    that it stores its environment inside a file called {\em
      uboot.env} in an ext4 filesystem:
    \begin{itemize}
    \item Disable \code{Environment is not stored}. We want changes to variables to
        be persistent across reboots
    \item Enable \code{Environment is in a EXT4 filesystem}. Disable all other
        options for environment storage (e.g. MMC, SPI, UBI)
% Environment in MMC also works, but we need to be careful with the size available
% and the offset defined. tests showed that if MMC is selected it will take
% precedence over ext4
    \item \code{Name of the block device for the environment}: \code{mmc}
    \item \code{Device and partition for where to store the environment in
        EXT4}: \code{0:4}
    \item \code{Name of the EXT4 file to use for the environment}: \code{/uboot.env}
    \end{itemize}

  \item In the \code{Device Drivers} $\rightarrow$ \code{Watchdog
      Timer Support} submenu, disable \code{IWDG watchdog
      driver for STM32 MP's family}, so that U-Boot doesn't start the
    watchdog.
  \end{itemize}

\item In recent versions of U-Boot and for some boards, you will
  need to have the Device Tree compiler:

\bashcmd{$ sudo apt install device-tree-compiler}

\item Finally, run \inlinebash{$ make
    DEVICE_TREE=stm32mp157a-dk1}\footnote{You can speed up the
    compiling by using the \code{-jX} option with \code{make}, where X
    is the number of parallel jobs used for compiling. Twice the
    number of CPU cores is a good value.}, which will build
  U-Boot. The \code{DEVICE_TREE} variable specifies the specific
  Device Tree that describes our hardware board, in our case the
  STM32MP1 Discover Kit 1. Alternatively, if you wish to run just \code{make},
  specify our board's device tree name on
  \code{Device Tree Control} $\rightarrow$ \code{Default Device Tree for DT Control}
  option.
\end{enumerate}

\section{Flashing U-Boot}

The ROM monitor will look for the first stage bootloader in a
partition named \code{fsbl1}. If it cannot find a valid bootloader in
this partition, it will then try to load it from a partition named
\code{fslb2}. This first stage bootloader (in our case the U-Boot SPL)
will load the second bootloader (U-Boot itself) from the partition
named \code{ssbl}. Finally, U-Boot will store its environment in the
fourth partition, which we'll name \code{bootfs}.

So, as far as bootloaders are concerned, the SD card partitioning will
look like:

\begin{verbatim}
Number  Start   End      Size     File system  Name    Flags
 1      2048s   4095s    2048s                 fsbl1
 2      4096s   6143s    2048s                 fsbl2
 3      6144s   10239s   4096s                 ssbl
 4      10240s  131071s  120832s               bootfs
\end{verbatim}

Plug the SD card your instructor gave you on your workstation. Type
the \code{dmesg} command to see which device is used by your
workstation. In case the device is \code{/dev/mmcblk0}, you will see
something like

\begin{verbatim}
[46939.425299] mmc0: new high speed SDHC card at address 0007
[46939.427947] mmcblk0: mmc0:0007 SD16G 14.5 GiB
\end{verbatim}

The device file name may be different (such as \code{/dev/sdb}
if the card reader is connected to a USB bus (either internally
or using a USB card reader).

In the following instructions, we will assume that your SD card is
seen as \code{/dev/mmcblk0} by your PC workstation.

Type the \code{mount} command to check your currently mounted
partitions. If SD partitions are mounted, unmount them:

\bashcmd{$ sudo umount /dev/mmcblk0p*}

Then, clear possible SD card contents remaining from previous
training sessions (only the first megabytes matter):

\bashcmd{$ sudo dd if=/dev/zero of=/dev/mmcblk0 bs=4k count=200}

Now, let's use the \code{parted} command to create the partitions that
we are going to use:

\bashcmd{$ sudo parted /dev/mmcblk0}

The ROM monitor handles {\em GPT} partition tables, let's create one:

\begin{verbatim}
(parted) mklabel gpt
\end{verbatim}

Then, the 4 partitions are created with:
\begin{verbatim}
(parted) mkpart fsbl1 0% 4095s
(parted) mkpart fsbl2 4096s 6143s
(parted) mkpart ssbl 6144s 10239s
(parted) mkpart bootfs 10240s 131071s
\end{verbatim}

You can verify everything looks right with:

\begin{verbatim}
(parted) unit s
(parted) print
\end{verbatim}

Once done, quit:
\begin{verbatim}
(parted) quit
\end{verbatim}


Now, format the boot partition as an ext2 filesystem. This is where
U-Boot saves its environment:
\bashcmd{$ sudo mkfs.ext2 -L boot /dev/mmcblk0p4}

The U-Boot SPL is {\em fsbl}. Write it in both fsbl partitions:

\begin{bashinput}
$ sudo dd if=spl/u-boot-spl.stm32 of=/dev/mmcblk0p1 bs=1M conv=fdatasync
$ sudo dd if=spl/u-boot-spl.stm32 of=/dev/mmcblk0p2 bs=1M conv=fdatasync
\end{bashinput}

Then flash {\em ssbl}, this is the main U-Boot binary:

\bashcmd{$ sudo dd if=u-boot.img of=/dev/mmcblk0p3 bs=1M conv=fdatasync}

\section{Testing U-Boot}

Insert the SD card in the board slot. You can now power-up the board
by connecting the USB-C cable to the board, in CN6, \code{PWR_IN} and
to your PC at the other end. Check that it boots your new bootloaders.
You can verify this by checking the build dates:

\begin{verbatim}
U-Boot SPL 2021.01-00307-gd71be19902 (Jan 12 2021 - 10:02:03 +0000)
Model: STMicroelectronics STM32MP157A-DK1 Discovery Board
RAM: DDR3-DDR3L 16bits 533000Khz
WDT:   Not found!

U-Boot 2021.01-00307-gd71be19902 (Jan 12 2021 - 10:02:03 +0000)

CPU: STM32MP157AAC Rev.B
Model: STMicroelectronics STM32MP157A-DK1 Discovery Board
Board: stm32mp1 in basic mode (st,stm32mp157a-dk1)
Board: MB1272 Var1.0 Rev.C-01
DRAM:  512 MiB
Clocks:
- MPU : 650 MHz
- MCU : 208.878 MHz
- AXI : 266.500 MHz
- PER : 24 MHz
- DDR : 533 MHz
WDT:   Not found!
NAND:  0 MiB
MMC:   STM32 SD/MMC: 0
Loading Environment from EXT4... ** File not found /uboot.env **

** Unable to read "/uboot.env" from mmc0:4 **
In:    serial
Out:   serial
Err:   serial
****************************************************
*        WARNING 500mA power supply detected       *
*     Current too low, use a 3A power supply!      *
****************************************************

Net:   eth0: ethernet@5800a000
Hit any key to stop autoboot:  0 
Boot over mmc0!
Saving Environment to EXT4... File System is consistent
done
OK
switch to partitions #0, OK
mmc0 is current device
** Unrecognized filesystem type **
STM32MP> 
\end{verbatim}

In U-Boot, type the \code{help} command, and explore the few commands
available.

\section{Setting up Ethernet communication}

Later on, we will transfer files from the development workstation to
the board using the TFTP protocol, which works on top of an Ethernet
connection.

To start with, install and configure a TFTP server on your development
workstation, as detailed in the bootloader slides.

With a network cable, connect the Ethernet port labelled ETH0/GETH of
your board to the one of your computer. If your computer already has a
wired connection to the network, your instructor will provide you with
a USB Ethernet adapter. A new network interface should appear on your
Linux system.

Find the name of this interface by typing:
\ubootcmd{=> ip a}

The network interface name is likely to be
\code{enxxx}\footnote{Following the {\em Predictable Network Interface
Names} convention:
\url{https://www.freedesktop.org/wiki/Software/systemd/PredictableNetworkInterfaceNames/}}.
If you have a pluggable Ethernet device, it's easy to identify as it's
the one that shows up after pluging in the device.

Then, instead of configuring the host IP address from NetWork Manager’s graphical interface,
let’s do it through its command line interface, which is so much easier to use:

\bashcmd{$ nmcli con add type ethernet ifname en... ip4 192.168.0.1/24}

Now, configure the network on the board in U-Boot by setting the \code{ipaddr}
and \code{serverip} environment variables:

\begin{ubootinput}
=> setenv ipaddr 192.168.0.100
=> setenv serverip 192.168.0.1
\end{ubootinput}

To make these settings permanent, save the environment:

\begin{ubootinput}
=> saveenv
\end{ubootinput}

You can then test the TFTP connection. First, put a small text file in
the directory exported through TFTP on your development
workstation. Then, from U-Boot, do:

\begin{ubootinput}
=> tftp 0xc4000000 textfile.txt
\end{ubootinput}

The \code{tftp} command should have downloaded the
\code{textfile.txt} file from your development workstation into
the board's memory at location \code{0xc4000000}\footnote{
This location is part of the board DRAM. If you want
to check where this value comes from, you can check the stm32mp157
datasheet at
\url{https://www.st.com/resource/en/reference_manual/dm00327659.pdf}.
It's a big document (more than 4,000 pages). In this document, look
for \code{Memory organization} and you will find the SoC memory map.
You will see that the address range for the memory controller ({\em
DDRC}) starts at \code{0xc0000000} and ends at \code{0xdfffffff}. This
shows that the \code{0xc4000000} address is within the address range
for RAM. You can also try with other values in the same address range.}.

You can verify that the download was successful by dumping the
contents of the memory:

\begin{ubootinput}
=> md 0xc4000000
\end{ubootinput}

We will see in the next labs how to use U-Boot to download, flash and
boot a kernel.

\section{Rescue binaries}

If you have trouble generating binaries that work properly, or later
make a mistake that causes you to lose your bootloader binaries, you
will find working versions under \code{data/} in the current lab
directory.
