\subchapter{Lab7: Create a custom image}{The highest level of customization in Poky}

During this lab, you will:
\begin{itemize}
  \item Write a full customized image recipe
  \item Choose the exact packages you want on your board
\end{itemize}

\section{Add a basic image recipe}

A build is configured by two top level recipes: the machine recipe and the image
one. The image recipe is the top configuration file for the generated rootfs and
the packages it includes. Our aim in this lab is to define a custom image from
scratch to allow a precise selection of packages on the target. To
show how to deal with real world configuration and how the Yocto Project can be
used in the industry we will, in addition to the production image recipe you
will use in the final product, create a development one including debug tools
and show how to link the two of them to avoid configuration duplication.

First add a custom image recipe in the \code{meta-bootlinlabs} layer. We will name it
\code{bootlinlabs-image-minimal}. You can find information on how to create a custom
image on the dedicated Yocto Project development manual at
\url{https://www.yoctoproject.org/docs/2.5/dev-manual/dev-manual.html}. There
are different ways to customize an image, we here want to create a full recipe,
using a custom \code{.bb} file.

Do not forget to inherit from the \code{core-image} class.

\section{Select the images capabilities and packages}

You can control the packages built and included into the final image with the
\code{IMAGE_INSTALL} configuration variable. It is a list of packages to be
built. You can also use package groups to include a bunch of programs, generally
enabling a functionality, such as \code{packagegroup-core-boot} which adds the
minimal set of packages required to boot an image (i.e. a shell or a kernel).

You can find the package groups under the \code{packagegroups} directories. To
have a list of the available one:
\begin{verbatim}
find -name packagegroups
\end{verbatim}

Open some of them to read their description and have an idea about the
capabilities they provide. Then update the installed packages of the image
recipe and don't forget to add the nInvaders one!

\section{Add a custom package group}

We just saw it is possible to use package groups to organize and select the
packages instead of having a big blob of configuration in the image recipe
itself. We will here create a custom package for game related recipes.

With the above documentation, create a \code{packagegroup-bootlinlabs-games} group
which inherits from the \code{packagegroup} class. Add the nInvaders program
into its runtime dependencies.

Now update the image recipe to include the package group instead of the
nInvaders program directly.

\section{Differentiate the production recipe from the debug one}

You can enable the debugging capabilities of your image just by changing the
BitBake target when building the whole system. We want here to have a common
base for both the production and the debug images, but also take into account the
possible differences. In our example only the built package list will change.

Create a debug version of the previous image recipe, and name it
\code{bootlinlabs-image-minimal-dbg}. Try to avoid duplicating code! Then add the
\code{dbg-pkgs} to the image features list. It is also recommended to
update the recipe's description, and to add extra debugging tools.

Build the new debug image with BitBake and check the previously
included packages are present in the newly generated rootfs.
