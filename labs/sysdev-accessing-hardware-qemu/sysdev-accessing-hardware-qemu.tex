\subchapter{Accessing Hardware Devices}
{Objective: learn how to access hardware devices.}

\section{Goals}

Now that we have access to a command line shell thanks to a working root
filesystem, we can now explore existing devices.

{\bf Important:} {\em The manipulations in this lab are limited because
we're working with an emulated platform, not with real hardware.
It would be best to get your hands on the hardware platforms
we support. Our instructions for such platforms really cover
practising with internal and external hardware devices.}

\section{Setup}

Go to the \code{$HOME/__SESSION_NAME__-labs/hardware} directory,
which provides useful files for this lab.

However, we will go on booting the system through NFS, using the
root filesystem built by the previous lab.

\section{Exploring /dev}

Start by exploring \code{/dev} on your target system. Here are a few
noteworthy device files that you will see:

\begin{itemize}
 \item {\em Terminal devices}: devices starting with \code{tty}.
       Terminals are user interfaces taking text as
       input and producing text as output, and are typically used by
       interactive shells. In particular, you will find
       \code{console} which matches the device specified through
       \code{console=} in the kernel command line. You will also find
       the {\tt \ttyname} device file.
 \item {\em Pseudo-terminal devices}: devices starting with \code{pty},
       used when you connect through SSH for example. Those are virtual
       devices, but there are so many in \code{/dev} that we wanted
       to give a description here.
 \item {MMC device(s) and partitions}: devices starting with
       \code{mmcblk}. You should here recognize the MMC device(s)
       on your system and the associated partitions.
\end{itemize}

Don't hesitate to explore \code{/dev} on your workstation too
and ask any questions to your instructor.

\section{Exploring /sys}

The next thing you can explore is the {\em Sysfs} filesystem.

A good place to start is \code{/sys/class}, which exposes devices
classified by the kernel frameworks which manage them.

For example, go to \code{/sys/class/net}, and you will see all the
networking interfaces on your system, whether they are internal,
external or virtual ones.

Find which subdirectory corresponds to the network connection
to your host system, and then check device properties such as:
\begin{itemize}
   \item \code{speed}: will show you whether this is a gigabit
         or hundred megabit interface.
   \item \code{address}: will show the device MAC address. No
	 need to get it from a complex command!
   \item \code{statistics/rx_bytes} will show you how many bytes
	 were received on this interface.
\end{itemize}

Don't hesitate to look for further interesting properties by yourself!

You can also check whether \code{/sys/class/thermal} exists and is not
empty on your system. That's the thermal framework, and it allows
to access temperature measures from the thermal sensors on your system.

Next, you can now explore all the buses (virtual or physical) available
on your system, by checking the contents of \code{/sys/bus}.

In particular, go to \code{/sys/bus/mmc/devices} to see all the
MMC devices on your system. Go inside the directory for the first device
and check several files (for example):

\begin{itemize}
\item \code{serial}: the serial number for your device.
\item \code{preferred_erase_size}: the preferred erase block for your
      device. It's recommended that partitions start at multiples of this
      size.
\item \code{name}: the product name for your device. You could display
      it in a user interface or log file, for example.
\item \code{date}: apparently the manufacturing date for the device.
\end{itemize}

Don't hesitate to spend more time exploring \code{/sys} on your system
and asking questions to your instructor.

That's all for now!
