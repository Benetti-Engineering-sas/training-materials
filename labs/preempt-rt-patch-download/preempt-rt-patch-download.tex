\subchapter{Lab1: Building and Booting a Preempt-RT Kernel}{Download, Configure, Build and Boot}

During this lab, you will:
\begin{itemize}
	\item Configure the Buildroot Build-system to generate an image based on the upstream linux-rt repository
	\item Configure the kernel to enable full preemption
	\item Boot the system and check that it runs preempt-rt
\end{itemize}

\section{Initial Setup}
As specified in the Buildroot
manual\footnote{\url{https://buildroot.org/downloads/manual/manual.html\#requirement-mandatory}},
Buildroot requires a few packages to be installed on your
machine. Let's install them using Ubuntu's package manager:

\begin{bashinput}
sudo apt install sed make binutils gcc g++ bash patch \
  gzip bzip2 perl tar cpio python unzip rsync wget libncurses-dev
\end{bashinput}

\section{Download Buildroot}

Since we're going to do Buildroot development, let's clone the
Buildroot source code from its Git repository:

\begin{bashinput}
git clone https://git.busybox.net/buildroot
\end{bashinput}

In case this is blocked on your network, you can download the Buildroot
tarball \code{buildroot-2023.11.1.tar.bz2} from
\code{https://buildroot.org/downloads/} and extract it. However in this
case, you won't be able to use {\em Git} to visualize your changes and
keep track of them.

Go into the newly created \code{buildroot} directory.

We're going to start a branch from the {\em 2023.11.1} Buildroot
release, with which this training has been tested.

\begin{bashinput}
git checkout -b bootlinlabs 2023.11.1
\end{bashinput}

\section{Configuring Buildroot}

The buildroot configuration is provided in the lab materials. Copy the configuration
in the "configs" folder of your buildroot installation :

\begin{bashinput}
	cp ~/preempt-rt/preempt-rt-lab-data/stm32mp157a_dk1_rt_defconfig configs
\end{bashinput}

We'll use that configuration as a basis for our setup:

\begin{bashinput}
make stm32mp157a_dk1_rt_defconfig
\end{bashinput}

\section{Kernel configuration}
The standard way to configure the kernel is through the \code{make menuconfig} interface. Here, we're building everything using
Buildroot, because it's an easy way to build a fully integrated image, with our custom kernel but also our custom applications.

We'll use Buildroot's \code{make linux-menuconfig} to modify our kernel configuration,
it's strictly equivalent to the \code{make menuconfig} command from the kernel's source tree.

\begin{bashinput}
	make linux-menuconfig
\end{bashinput}

The default kernel configuration file for this platform isn't made for a -RT kernel. Building an kernel with the Preempt-RT patch is not enough to
benefit from the full kernel preemption, we also need to enable it. In the menuconfig interface, you'll find the \code{Preemption Model} setting under the \code{General Setup} category.

We want to enable the \code{Fully Preemptible Kernel} mode, but it's not proposed as an available choice by default. To enable it, we first need to enable the \code{expert} mode, by selecting the \code{CONFIG_EXPERT} option. Once enabled, we can select the Preempt-RT mode!

\begin{bashinput}
make
\end{bashinput}

While the build is ongoing, don't hesitate to take a look at the latest
version of the patchset:

\begin{bashinput}
wget https://cdn.kernel.org/pub/linux/kernel/projects/rt/6.2/patches-6.2-rc3-rt1.tar.gz
\end{bashinput}

Look at the \code{series} file for more information about each individual patch.

\section{Setting up serial communication with the board - STM32MP157}

The STM32MP1 devkit's serial port can be accessed through the micro-USB connector.

Once the USB cable is plugged in, a new serial port
should appear: \code{/dev/ttyACM0}.  You can also see this device
appear by looking at the output of \code{dmesg}.

To communicate with the board through the serial port, install a
serial communication program, such as \code{picocom}:

\begin{bashinput}
sudo apt install picocom
\end{bashinput}

If you run \code{ls -l /dev/ttyACM0}, you can also see that only
\code{root} and users belonging to the \code{dialout} group have
read and write access to this file. Therefore, you need to add your user
to the \code{dialout} group:

\begin{bashinput}
sudo adduser $USER dialout
\end{bashinput}

{\bf Important}: for the group change to be effective, in Ubuntu 18.04, you have to
{\em completely reboot} the system \footnote{As explained on
\url{https://askubuntu.com/questions/1045993/after-adding-a-group-logoutlogin-is-not-enough-in-18-04/}.}.
A workaround is to run \code{newgrp dialout}, but it is not global.
You have to run it in each terminal.

Now, you can run \code{picocom -b 115200 /dev/ttyACM0}, to start serial
communication on \code{/dev/ttyACM0}, with a baudrate of \code{115200}. If
you wish to exit \code{picocom}, press \code{[Ctrl][a]} followed by
\code{[Ctrl][x]}.

There should be nothing on the serial line so far, as the board is not
powered up yet.

\section{Flash the image to the SDCard}

\section{Checking that we run a Patched kernel}

To check that we are indeed running a kernel with the preempt-RT patch applied and
full kernel preemption enabled, we have 2 main ways of checing.

First, use the \code{uname -a} command. Running an RT kernel should show the \code{PREEMPT_RT} version item.

The other way to check is to look at the file \code{/sys/kernel/realtime}. It's content is always '1', but the file only exists for RT kernels.

Take a look at the boot logs, can you see something that's not going right? Why?

\section{Fixing our first kernel bug}

You should see quite a lot of \code{BUG: scheduling while atomic} messages. The
actual issue is non-trivial to fix, but it's a nice example to analyse.

