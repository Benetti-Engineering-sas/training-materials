\subchapter{Bootloader - U-Boot}{Objectives: Set up serial
  communication, compile and install the U-Boot bootloader, use basic
  U-Boot commands, set up TFTP communication with the development
  workstation.}

As the bootloader is the first piece of software executed by a
hardware platform, the installation procedure of the bootloader is
very specific to the hardware platform. There are usually two cases:

\begin{itemize}

\item The processor offers nothing to ease the installation of the
  bootloader, in which case a JTAG interface has to be used to initialize
  flash storage and write the bootloader code to flash. Detailed
  knowledge of the hardware is of course required to perform these
  operations.

\item The processor offers a monitor, implemented in ROM, and through
  which access to the memories is made easier.

\end{itemize}

The Xplained board, which uses the SAMA5D3 SoCs, falls into the second
category. The monitor integrated in the ROM reads the MMC/SD card to
search for a valid bootloader before looking at the internal NAND
flash for a bootloader. In case nothing is available, it will operate
in a fallback mode, that will allow to use an external tool
to reflash some bootloader through USB. Therefore, either by using an MMC/SD
card or that fallback mode, we can start up a SAMA5D3-based board
without having anything installed on it.

\section{Downloading Microchip's flashing tool}

Go to the \code{$HOME/__SESSION_NAME__-labs/bootloader} directory.\\
We're going to use that fallback mode, and its associated tool,
\code{sam-ba} ({\em SAM Boot Assistant}).\\
We first need to download this tool, from Microchip's website\footnote{
In case this website is down, you can also find this
tool on \url{https://bootlin.com/labs/tools/}.}.

\begin{bashinput}
$ wget https://ww1.microchip.com/downloads/en/DeviceDoc/\
    sam-ba_3.3.1-linux_x86_64.tar.gz
$ tar xf sam-ba_3.3.1-linux_x86_64.tar.gz
\end{bashinput}

\section{Setting up serial communication with the board}

Plug the USB-to-serial cable on the Xplained board. The blue end of
the cable is going to \code{GND} on \code{J23}, red on \code{RXD} and
green on \code{TXD}. When plugged in your computer, a serial port
should appear, \code{/dev/ttyUSB0}.

\begin{center}
\includegraphics[width=8cm]{labs/sysdev-u-boot/xplained-serial-connector.jpg}
\end{center}

You can also see this device appear by looking at the output of
\code{dmesg}.

To communicate with the board through the serial port, install a
serial communication program, such as \code{picocom}:

\begin{bashinput}
$ sudo apt install picocom
\end{bashinput}

You also need to make your user belong to the \code{dialout} group to be
allowed to write to the serial console:

\begin{bashinput}
$ sudo adduser $USER dialout
\end{bashinput}

{\bf Important}: for the group change to be effective, you have to
{\em completely log out} from your session and log in again (no need to
reboot). A workaround is to run \code{newgrp dialout}, but it is not global.
You have to run it in each terminal.

Run \inlinebash{$ picocom -b 115200 /dev/ttyUSB0}, to start serial
communication on \code{/dev/ttyUSB0}, with a baudrate of 115200.

You can now power-up the board by connecting the micro-USB cable to
the board, and to your PC at the other end. If a system was previously
installed on the board, you should be able to interact with it
through the serial line.

If you wish to exit \code{picocom}, press \code{[Ctrl][a]} followed by
\code{[Ctrl][x]}.

\section{Bootloader setup}

The boot process is done in two steps with the ROM monitor trying to
execute a first piece of software, called \em{U-Boot Single Program
Loader (U-Boot SPL)}, from its internal SRAM. It will initialize the
DRAM, load \em{U-Boot} that will in turn load Linux and execute it.

As far as bootloaders are concerned, the layout of the NAND flash will
look like:

\begin{center}
  \includegraphics[width=\textwidth]{labs/sysdev-u-boot/flash-map.pdf}
\end{center}

\begin{itemize}
\item Offset \code{0x0} for the first stage bootloader is dictated by
  the hardware: the ROM code of the SAMA5D3 looks for a first stage
  bootloader at offset \code{0x0} in the NAND flash.
\item Offset \code{0x40000} for the second stage bootloader is decided
  by the first stage bootloader. This can be changed modifying the
  value of \projconfig{u-boot}{CONFIG_SYS_NAND_U_BOOT_OFFS} in
  \projfile{u-boot}{include/configs/sama5d3_xplained.h} in U-Boot
  sources.
\item Offset \code{0x140000} of the U-Boot environment is decided by
  U-Boot's configuration (\projconfig{u-boot}{CONFIG_ENV_OFFSET}).
\end{itemize}

\section{Configuring and compiling U-Boot}

Get U-Boot sources:

\begin{bashinput}
$ git clone https://gitlab.denx.de/u-boot/u-boot
$ cd u-boot
$ git checkout v2020.07
\end{bashinput}

Note that the latest versions of U-Boot (v2021.01 at the time of this
writing) are broken for the SAMA5D3 Xplained board. We're trying
to solve this!

Get an understanding of U-Boot's
configuration and compilation steps by reading the \code{README} file,
and specifically the {\em Building the Software} section.

Basically, you need to:

\begin{itemize}

\item Set the \code{CROSS_COMPILE} environment variable;

\item Run \inlinebash{$ make <NAME>_defconfig}, where the list of available
  configurations can be found in the \code{configs/} directory. There
  are two flavors of the Xplained configuration: one to run from the
  SD card (\code{sama5d3_xplained_mmc}) and one to run from the NAND
  flash (\code{sama5d3_xplained_nandflash}). Since we're going to boot
  on the NAND, use the latter.

\item Now that you have a valid initial configuration, you can now
  run \inlinebash{$ make menuconfig} to further edit your bootloader features.

\item In recent versions of U-Boot and for some boards, you will
  need to have the Device Tree compiler:

\begin{bashinput}
$ sudo apt install device-tree-compiler
\end{bashinput}

\item Finally, run \code{make}, which should build the two stages of U-Boot:
\begin{itemize}
  \item First stage bootloader (SPL): \code{spl/u-boot-spl.bin}
  \item Second stage bootloader: \code{u-boot.bin}
\end{itemize}

\end{itemize}

Look at the size of the binaries. \code{spl/u-boot-spl.bin} should fit
in the SoC SRAM (64 KB) and according to our flash layout, \code{u-boot.bin}
is supposed to fit between flash offset \code{0x40000} and offset \code{0x100000},
corresponding to a maximum size of \code{786432} bytes.  Make sure that
both binaries fit.

\section{Flashing SPL and U-Boot}

To put the board in {\em Boot monitor mode} and use \code{sam-ba} for
flashing, you first need to remove the \code{NAND CS} jumper on the board.
It's next to the pin header closest to the Micro-USB plug.
Now, press the \code{RESET} button.  On the serial port, you should see
\code{RomBoot}. You should also see that a \code{/dev/ttyACM0} device
file has appeared.

Put the jumper back.

Getting back to the \code{bootloader} directory, use \code{sam-ba} to erase
NAND flash before writing images to it:

\begin{bashinput}
$ ./sam-ba_3.3.1/sam-ba -p serial -b sama5d3-xplained -a nandflash -c erase
\end{bashinput}

Then flash the U-Boot SPL:

\begin{bashinput}
$ ./sam-ba_3.3.1/sam-ba -p serial -b sama5d3-xplained -a nandflash \%\linebreak% -c writeboot:u-boot/spl/u-boot-spl.bin
\end{bashinput}

Then flash the U-Boot binary:

\begin{bashinput}
$ ./sam-ba_3.3.1/sam-ba -p serial -b sama5d3-xplained -a nandflash \%\linebreak% -c write:u-boot/u-boot.bin:0x40000
\end{bashinput}

\section{Testing U-Boot}

Reset the board and check that it boots your new bootloaders. You can
verify this by checking the build dates:

\input{../labs/sysdev-u-boot/boot.log.tex}

Interrupt the countdown to enter the U-Boot shell:
\begin{verbatim}
=>
\end{verbatim}

In U-Boot, type the \code{help} command, and explore the few commands
available.

\section{Setting up Ethernet communication}

Later on, we will transfer files from the development workstation to
the board using the TFTP protocol, which works on top of an Ethernet
connection.

To start with, install and configure a TFTP server on your development
workstation, as detailed in the bootloader slides.

With a network cable, connect the Ethernet port labelled ETH0/GETH of
your board to the one of your computer. If your computer already has a
wired connection to the network, your instructor will provide you with
a USB Ethernet adapter. A new network interface should appear on your
Linux system.

Find the name of this interface by typing:
\begin{bashinput}
$ ip a
\end{bashinput}

The network interface name is likely to be
\code{enxxx}\footnote{Following the {\em Predictable Network Interface
Names} convention:
\url{https://www.freedesktop.org/wiki/Software/systemd/PredictableNetworkInterfaceNames/}}.
If you have a pluggable Ethernet device, it's easy to identify as it's
the one that shows up after pluging in the device.

Then, instead of configuring the host IP address from NetWork Manager’s graphical interface,
let’s do it through its command line interface, which is so much easier to use:

\begin{bashinput}
$ nmcli con add type ethernet ifname en... ip4 192.168.0.1/24
\end{bashinput}

Now, configure the network on the board in U-Boot by setting the \code{ipaddr}
and \code{serverip} environment variables:

\begin{ubootinput}
=> setenv ipaddr 192.168.0.100
=> setenv serverip 192.168.0.1
\end{ubootinput}

The first time you use your board, you also need to set the MAC
address in U-boot:

\begin{ubootinput}
=> setenv ethaddr 12:34:56:ab:cd:ef
\end{ubootinput}

To make these settings permanent, save the environment:

\begin{ubootinput}
=> saveenv
\end{ubootinput}

You can then test the TFTP connection. First, put a small text file in
the directory exported through TFTP on your development
workstation. Then, from U-Boot, do:

\begin{ubootinput}
=> tftp 0x22000000 textfile.txt
\end{ubootinput}

The \code{tftp} command should have downloaded the
\code{textfile.txt} file from your development workstation into
the board's memory at location \code{0x22000000}\footnote{
This location is part of the board DRAM. If you want
to check where this value comes from, you can check the Atmel SAMA5D3
datasheet at
\url{https://ww1.microchip.com/downloads/en/DeviceDoc/Atmel-11121-32-bit-Cortex-A5-Microcontroller-SAMA5D3_Datasheet_B.pdf}.
It's a big document (more than 1,900 pages). In this document, look
for \code{Memory Mapping} and you will find the SoC memory map. You
will see that the address range for the memory controller ({\em DDRC
S}) starts at \code{0x20000000} and ends at \code{0x3fffffff}. This
shows that the \code{0x22000000} address is within the address range
for RAM. You can also try with other values in the same address range,
knowing that our board only has 256 MB of RAM (that's
\code{0x10000000}, so the physical RAM probably ends at
\code{0x30000000}).}.

You can verify that the download was successful by dumping the
contents of the memory:

\begin{ubootinput}
=> md 0x22000000
\end{ubootinput}

We will see in the next labs how to use U-Boot to download, flash and
boot a kernel.

\section{Rescue binaries}

If you have trouble generating binaries that work properly, or later
make a mistake that causes you to lose your bootloader binaries, you
will find working versions under \code{data/} in the current lab
directory.
