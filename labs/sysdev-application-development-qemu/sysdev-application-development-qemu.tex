\subchapter{Application development}{Objective: Compile and run your
  own ncurses application on the target.}

\section{Setup}

Go to the \code{$HOME/__SESSION_NAME__-labs/appdev} directory.

\section{Compile your own application}

We will re-use the system built during the {\em Buildroot lab} and add
to it our own application.

In the lab directory the file \code{app.c} contains a very simple
{\em ncurses} application. It is a simple game where you need to reach
a target using the arrow keys of your keyboard.  We will compile and
integrate this simple application to our Linux system.

Buildroot has generated toolchain wrappers in
\code{output/host/bin}, which make it easier to use the toolchain,
since these wrappers pass some mandatory flags (especially the
\code{--sysroot} {\em gcc} flag, which tells {\em gcc} where to look
for the headers and libraries).

Let's add this directory to our \code{PATH}:

\begin{bashinput}
$ export PATH=$HOME/__SESSION_NAME__-labs/buildroot/buildroot/output/host/bin:$PATH
\end{bashinput}
\normalsize

Let's try to compile the application:

\bashcmd{$ arm-linux-gcc -o app app.c}

It complains about undefined references to some symbols. This is
normal, since we didn't tell the compiler to link with the necessary
libraries. So let's use \code{pkg-config} to query the {\em
pkg-config} database about the location of the header files and the
list of libraries needed to build an application against
{\em ncurses}\footnote{Again, \code{output/host/bin} has a special
\code{pkg-config} that automatically knows where to look, so it
already knows the right paths to find \code{.pc} files and their
sysroot.}:

\bashcmd{$ arm-linux-gcc -o app app.c $(pkg-config --libs --cflags ncurses)}

Our application is now compiled! Copy the generated binary to the NFS
root filesystem (in the \code{root/} directory for example), start
your system, and run your application!

