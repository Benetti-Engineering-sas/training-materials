\subchapter{Kernel - Cross-compiling}{Objective: Learn how to
cross-compile a kernel for an ARM target platform.}

After this lab, you will be able to:
\begin{itemize}
\item Checkout a stable version of the Linux kernel
\item Set up a cross-compiling environment
\item Cross compile the kernel for the \labboarddescription
\item Use U-Boot to download the kernel
\item Check that the kernel you compiled starts the system
\end{itemize}

\section{Setup}

Stay in the \code{$HOME/__SESSION_NAME__-labs/kernel} directory.

\section{Choose a particular stable version of Linux}

We will use \texttt{linux-\workingkernel.x}, which corresponds to an
LTS release, and which this lab was tested with.

First, let's get the list of branches on our \code{stable} remote tree:

\begin{verbatim}
cd linux
git branch -a
\end{verbatim}

As we will do our labs with the Linux \workingkernel,
the remote branch we are interested in is
\texttt{remotes/stable/linux-\workingkernel.y}.

First, execute the following command to check which version you
currently have:

\begin{verbatim}
make kernelversion
\end{verbatim}

You can also open the \code{Makefile} and look at the beginning of it
to check this information.

Now, let's create a local branch starting from that remote branch:
\begin{bashinput}
git checkout stable/linux-%\workingkernel%.y
\end{bashinput}

Check the version again using the \code{make kernelversion} command
to make sure you now have a \workingkernel.x version.

\section{Cross-compiling environment setup}

To cross-compile Linux, you need to have a cross-compiling
toolchain. We will use the cross-compiling toolchain that we
previously produced, so we just need to make it available in the PATH:

\bashcmd{$ export PATH=$HOME/x-tools/arm-training-linux-uclibcgnueabihf/bin:$PATH}

Also, don't forget to either:

\begin{itemize}
\item Define the value of the \code{ARCH} and \code{CROSS_COMPILE}
  variables in your environment (using \code{export})
\item {\bf Or} specify them on the command line at every invocation of
  \code{make}, i.e.: \code{make ARCH=... CROSS_COMPILE=... <target>}
\end{itemize}

\section{Linux kernel configuration}

\ifdefstring{\labboard}{stm32mp1}
{
The standard configuration for this kernel is \code{multi_v7_defconfig},
but this will generate a pretty big kernel with support for many other
SoCs. However, we can reduce it to compile faster and get a small
kernel.

So, apply this configuration, and than run \code{make menuconfig},
and in the \code{System Type} menu, remove support for all the
SoCs except the STM32MP157 ones. Don't forget to disable the TI ones
too which are in a submenu.
}
{
By running \code{make help}, find the proper Makefile target to
configure the kernel
\ifdefstring{\labboard}{qemu}
{for the ARM Vexpress boards (\code{vexpress_defconfig}).}
}

\ifdefstring{\labboard}{qemu}
{Also start \code{make menuconfig} to
add \kconfig{CONFIG_DEVTMPFS_MOUNT} to your configuration.}
{}

\section{Cross compiling}

You're now ready to cross-compile your kernel. Simply run:

\bashcmd{$ make}

and wait a while for the kernel to compile. Don't forget to use
\code{make -j<n>} if you have multiple cores on your machine!

Look at the end of the kernel build output to see which file contains
the kernel image. You can also see the Device Tree \code{.dtb} files
which got compiled. Find which \code{.dtb} file corresponds to your
board.

Copy the linux kernel image and DTB files to the TFTP server
home directory.

\section{Load and boot the kernel using U-Boot}

As we are going to boot the Linux kernel from U-Boot,
we need to set the \code{bootargs} environment corresponding
to the Linux kernel command line:

\begin{ubootinput}
=> setenv bootargs console=%\console%
=> saveenv
\end{ubootinput}
We will use TFTP to load the kernel image on the board:

\begin{itemize}

\item On your workstation, copy the \code{zImage} and DTB
(\texttt\dtbname) to the directory exposed by the TFTP server.

\item On the target (in the U-Boot prompt), load \code{zImage} from
TFTP into RAM:
\begin{ubootinput}
=> tftp %\zimageboardaddr% zImage
\end{ubootinput}

\item Now, also load the DTB file into RAM:
\begin{ubootinput}
=> tftp %\dtbboardaddr% %\dtbname%
\end{ubootinput}

\item Boot the kernel with its device tree:
\begin{ubootinput}
=> bootz %\zimageboardaddr% - %\dtbboardaddr%
\end{ubootinput}

\end{itemize}

You should see Linux boot and finally panicking. This is expected: we
haven't provided a working root filesystem for our device yet.

You can now automate all this every time the board is booted or
reset. Reset the board, and customize \code{bootcmd}:

\begin{ubootinput}
=> setenv bootcmd 'tftp %\zimageboardaddr\ zImage; tftp \dtbboardaddr\ \dtbname; bootz \zimageboardaddr\ - \dtbboardaddr'%
=> saveenv
\end{ubootinput}

Restart the board to make sure that booting the kernel is now automated.

\ifdefstring{\labboard}{stm32mp1}
{
\section{Writing the kernel and DTB on the SD card}

In order to let the kernel boot on the board autonomously, we can
copy the kernel image and DTB in the boot partition we created
previously.

Insert the SD card in your PC, it will get auto-mounted. Copy the
kernel and device tree:

\begin{bashinput}
$ sudo cp arch/arm/boot/dts/stm32mp157a-dk1.dtb arch/arm/boot/zImage /media/$USER/boot/
$ sudo umount /media/$USER/boot
\end{bashinput}
\normalsize

Insert the SD card back in the board and reset it. You should now be
able to load the DTB and kernel image from the SD card and boot with:

\begin{ubootinput}
=> ext2load mmc 0:4 0xc0000000 zImage
=> ext2load mmc 0:4 0xc4000000 stm32mp157a-dk1.dtb
=> bootz 0xc0000000 - 0xc4000000
\end{ubootinput}

You are now ready to modify \code{bootcmd} to boot the board
from SD card. But first, save the settings for booting from
\code{tftp}:

\begin{ubootinput}
=> setenv bootcmdtftp %\$%{bootcmd}
\end{ubootinput}

This will be useful to switch back to \code{tftp} booting mode
later in the labs.

Finally, using \code{editenv bootcmd}, adjust \code{bootcmd} so that
the Discovery board starts using the kernel from the SD card.

Now, reset the board to check that it boots in the same way from the
SD card. Check that this is really your own version of the kernel
that's running\footnote{Look at the kernel log. You will find the
kernel version number as well as the date when it was compiled.
That's very useful to check that you're not loading an older version
of the kernel instead of the one that you've just compiled.}.
}
{}
