\subchapter{Kernel - Cross-compiling}{Objective: Learn how to
cross-compile a kernel for an ARM target platform.}

After this lab, you will be able to:
\begin{itemize}
\item Checkout a stable version of the Linux kernel
\item Set up a cross-compiling environment
\item Cross compile the kernel for the \labboarddescription
\item Use U-Boot to download the kernel
\item Check that the kernel you compiled starts the system
\end{itemize}

\section{Setup}

Stay in the \code{$HOME/__SESSION_NAME__-labs/kernel} directory.

\section{Choose a particular stable version of Linux}

We will use \texttt{linux-\workingkernel.x},
\ifdefstring{\labboard}{qemu}{}{
  \ifdefstring{\labboard}{beagleplay}{}{which corresponds to an LTS release, and}
}
which this lab was tested with.

First, let's get the list of branches we have available:

\begin{verbatim}
cd linux
git branch -a
\end{verbatim}

As we will do our labs with the Linux \workingkernel,
the remote branch we are interested in is\\%without this, you get a line overflow, weird
\texttt{remotes/stable/linux-\workingkernel.y}.

First, execute the following command to check which version you
currently have:

\begin{verbatim}
make kernelversion
\end{verbatim}

You can also open the \code{Makefile} and look at the beginning of it
to check this information.

Now, let's create a local branch starting from that remote branch:
\begin{bashinput}
git checkout stable/linux-%\workingkernel%.y
\end{bashinput}

Check the version again using the \code{make kernelversion} command
to make sure you now have a \workingkernel.x version.

\section{Cross-compiling environment setup}

To cross-compile Linux, you need to have a cross-compiling
toolchain. We will use the cross-compiling toolchain that we
previously produced, so we just need to make it available in the PATH:

\begin{bashinput}
$ export PATH=$HOME/x-tools/%\ifdefstring{\labboard}{beagleplay}{aarch64-training-linux-musl}{arm-training-linux-musleabihf}%/bin:$PATH
\end{bashinput}

Also, don't forget to either:

\begin{itemize}
\item Define the value of the \code{ARCH} and \code{CROSS_COMPILE}
  variables in your environment (using \code{export})
\item {\bf Or} specify them on the command line at every invocation of
  \code{make}, i.e.: \code{make ARCH=... CROSS_COMPILE=... <target>}
\end{itemize}
\ifdefstring{\labboard}{beagleplay}{
\textbf{Important:} The majority of tools use \code{aarch64} to define 64 bits
ARM processors. However in specific cases, the tools may use \code{arm64} to 
define such architecture. In the case of the kernel Makefile we will use the
\code{arm64} one.
}{}
\section{Linux kernel configuration}

By running \code{make help}, look for the proper Makefile target to
configure the kernel for your processor.

\ifdefstring{\labboard}{stm32mp1}
{The standard configuration for this kernel is actually \code{multi_v7_defconfig},
but this will generate a pretty big kernel with support for many other
SoCs. However, we can reduce it to compile faster and get a small
kernel.}{}

\ifdefstring{\labboard}{beaglebone}
{In our case look for a configuration for boards in the OMAP2 and
later family which the AM335x found in the BeagleBone belongs to.}{}
\ifdefstring{\labboard}{beagleplay}
{If you search for a configuration file that corresponds to the \code{arm64}
architecture that the kernel will use, you might see that only one
\code{defconfig} file is available. We will therefore load this basic
configuration file and modify it later.}{}
\ifdefstring{\labboard}{qemu}
{In course case, use the configuration for the ARM Vexpress boards
(\code{vexpress_defconfig}).}{}


So, apply this configuration, and then run \code{make menuconfig}.

\begin{itemize}
\item Disable \kconfig{CONFIG_GCC_PLUGINS} if it is set. This will skip
  building special {\em gcc} plugins, which would require extra dependencies
  for the build.
\ifdefstring{\labboard}{stm32mp1}{
  \item In the \code{System Type} menu, remove support for all the SoCs except
  the STM32MP157 ones. Don't forget to disable the TI ones too which are
  in a submenu.
  \item Disable \kconfig{CONFIG_DRM}, which will skip support for many display
  controller and GPU drivers.}{}
\ifdefstring{\labboard}{qemu}{
  Also start \code{make menuconfig} to
  add \kconfig{CONFIG_DEVTMPFS_MOUNT} to your configuration.}{}
\ifdefstring{\labboard}{beaglebone}{
\item Add options to support USB host and networking over USB device:
\begin{itemize}
  \item \kconfigval{CONFIG_USB}{y}
  \item \kconfigval{CONFIG_USB_GADGET}{y}
  \item \kconfigval{CONFIG_USB_MUSB_HDRC}{y} {\em Driver for the USB OTG
        controller}
  \item \kconfigval{CONFIG_USB_MUSB_DSPS}{y}
  \item \kconfigval{CONFIG_USB_MUSB_DUAL_ROLE}{y} {\em Use the USB OTG
        both in host and device (gadget) modes. This will be needed
        later to use the board's USB host port.}
  \item Check the dependencies of \kconfig{CONFIG_AM335X_PHY_USB}
        and find the way to set \kconfigval{CONFIG_AM335X_PHY_USB}{y}
  \item Find the "USB Gadget precomposed configurations" menu
        and set it to {\em static} instead of {\em module}
        so that \kconfigval{CONFIG_USB_ETH}{y}
\end{itemize}
\item Also compile your kernel with \kconfigval{CONFIG_INPUT_EVDEV}{y},
  to have the same default setting as in our labs with the STM32MP1 boards.
}{}
\ifdefstring{\labboard}{beagleplay}{
\item In the \code{Platform Selection} menu, remove support for all the SoCs except
the for Texas Instruments Inc. K3 multicore SoC architecture.
\item Disable \kconfig{CONFIG_DRM}, which will skip support for many display
controller and GPU drivers.
}{}
\end{itemize}

\ifdefstring{\labboard}{stm32mp1}{
Please note that this will definitely not build the smallest
and most optimized kernel for STM32MP1: \code{multi_v7_defconfig}
enables plenty of features and drivers that will not be useful on our
particular board.}{}

\section{Cross compiling}

You're now ready to cross-compile your kernel. Simply run:

\bashcmd{$ make}

and wait a while for the kernel to compile. Don't forget to use
\code{make -j<n>} if you have multiple cores on your machine!

Look at the end of the kernel build output to see which file contains
the kernel image.

Also look in the Device Tree Source directory to see which \code{.dtb}
files got compiled. Find which \code{.dtb} file corresponds to your board.

\ifdefstring{\labboard}{beaglebone}
{For the BeagleBone Black, its \code{am335x-boneblack.dtb},
and for the BeagleBone Black Wireless, its
\code{am335x-boneblack-wireless.dtb}.}
{}

\section{Load and boot the kernel using U-Boot}

As we are going to boot the Linux kernel from U-Boot,
we need to set the \code{bootargs} environment corresponding
to the Linux kernel command line:

\begin{ubootinput}
=> setenv bootargs console=%\console%
=> saveenv
\end{ubootinput}
We will use TFTP to load the kernel image on the board:

\begin{itemize}

\item On your workstation, copy the {\tt \ifdefstring{\labboard}{beagleplay}{Image.gz}{zImage}} and DTB
(\texttt{\dtname}\texttt{.dtb}) to the directory exposed by the TFTP server.

\item On the target (in the U-Boot prompt), load {\tt \ifdefstring{\labboard}{beagleplay}{Image.gz}{zImage}} from
TFTP into RAM:
\begin{ubootinput}
=> tftp %\zimageboardaddr% %\ifdefstring{\labboard}{beagleplay}{Image.gz}{zImage}%
\end{ubootinput}

\item Now, also load the DTB file into RAM:
\begin{ubootinput}
=> tftp %\dtbboardaddr% %\dtname%.dtb
\end{ubootinput}

\item Boot the kernel with its device tree:
\begin{ubootinput}
=> %\ifdefstring{\labboard}{beagleplay}{booti}{bootz} \zimageboardaddr% - %\dtbboardaddr%
\end{ubootinput}

\end{itemize}

\if\defstring{\labboard}{beagleplay}
This last command should show you an error message of this type:
\bashcmd{kernel_comp_addr_r or kernel_comp_size is not provided!}

This is because the boot image that we use, \code{Image.gz}, is compressed, and
therfore, needs to be uncompressed by U-Boot before continue booting. To do so
U-Boot needs to know the maximum size of the uncompressed image and where to
store it.

If you look at the size of the uncompressed kernel (\code{Image} file),
you can estimate that 32 MB (0x2000000) is a reasonable upper bound
for the size of the uncompressed kernel, even with a more exhaustive
configuration.

This gives us,

\begin{ubootinput}
=> setenv kernel_comp_addr_r 0x85000000
=> setenv kernel_comp_size 0x2000000
=> saveenv
\end{ubootinput}

Now you can retry the \code{booti} command and see the kernel be uncompressed
and then loaded.
\fi

You should see Linux boot and finally panicking. This is expected: we
haven't provided a working root filesystem for our device yet.

You can now automate all this every time the board is booted or
reset. Reset the board, and customize \code{bootcmd}:

\begin{ubootinput}
=> setenv bootcmd 'tftp %\zimageboardaddr\ \ifdefstring{\labboard}{beagleplay}{Image.gz}{zImage}; tftp \dtbboardaddr\ {\dtname}.dtb; \ifdefstring{\labboard}{beagleplay}{booti}{bootz} \zimageboardaddr\ - \dtbboardaddr'%
=> saveenv
\end{ubootinput}

Restart the board to make sure that booting the kernel is now automated.
