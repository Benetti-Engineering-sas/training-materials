\subchapter{Lab4: Create a Yocto layer}{Add a custom layer to the Yocto project for
	your project needs}

During this lab, you will:
\begin{itemize}
  \item Create a new Yocto layer
  \item Interface this custom layer to the existing Yocto project
  \item Use applications from custom layers
\end{itemize}

This lab extends the previous one, in order to fully understand how to interface
a custom project to the basic Yocto project.

\section{Tools}

You can access the configuration and state of layers with the
\code{bitbake-layers} command. This command can also be used to retrieve useful
information about available recipes. Try the following commands:
\begin{verbatim}
bitbake-layers show-layers
bitbake-layers show-recipes linux-yocto
bitbake-layers show-overlayed
bitbake-layers create-layer
\end{verbatim}

\section{Create a new layer}

With the above commands, create a new Yocto layer named
\code{meta-bootlinlabs} with a priority of \code{7}. Before doing that,
return to your project root directory, where by convention all layers are
stored (\code{cd $HOME/__SESSION_NAME__-labs/}).

Before using the new layer, we need to configure its generated configuration
files. You can start with the \code{README} file which is not used in the build
process but contains information related to layer maintenance. You can then
check, and adapt if needed, the global layer configuration file located in the
\code{conf} directory of your custom layer.

\section{Integrate a layer to the build}

To be fair, we already used and integrated a layer in our build configuration
during the first lab, with \ifdefstring{\labboard}{discovery}
{\code{meta-st-stm32}} {\code{meta-ti}}. This layer was responsible for \ifdefstring{\labboard}{discovery} {\code{stm32mp1 DK1}} {\code{BeagleBone Black}} support in Yocto. We have to do the same for our
\code{meta-bootlinlabs} now.

There is a file which contains all the paths of the layers we use. Try to find it
without looking back to the first lab. Then add the full path to our newly
created layer to the list of layers.

Validate the integration of the \code{meta-bootlinlabs} layer with:
\begin{verbatim}
bitbake-layers show-layers
\end{verbatim}

and make sure you don't have any warning from bitbake.

\section{Add a recipe to the layer}

In the previous lab we introduced a recipe for the nInvaders game. We included
it to the existing \code{meta} layer. While this approach give a working result,
the Yocto logic is not respected. You should instead always \textbf{use a custom layer}
to add recipes or to customize the existing ones. To illustrate this we will
move our previously created nInvaders recipe into the \code{meta-bootlinlabs} layer.

You can check the nInvaders recipe is part of the \code{meta} layer first:
\begin{verbatim}
bitbake-layers show-recipes ninvaders
\end{verbatim}

Then move the nInvaders recipe to the \code{meta-bootlinlabs} layer. You can check that
the nInvaders recipe is now part of the layer with the \code{bitbake-layers} command.
