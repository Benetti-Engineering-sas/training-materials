\subchapter{Lab1: First Yocto Project build}{Your first dive into Yocto
Project and its build mechanism}

During this lab, you will:
\begin{itemize}
  \item Set up an OpenEmbedded environment
  \item Configure the project and choose a target
  \item Build your first Poky image
\end{itemize}

\section{Setup}

Before starting this lab, make sure your home directory is not
encrypted using eCryptFS. OpenEmbedded cannot be used on top of an eCryptFS file
system due to limitations in file name lengths.

Go to the \code{$HOME/__SESSION_NAME__-labs/} directory.

Install the required packages:
\begin{bashinput}
sudo apt install bc build-essential chrpath cpio diffstat gawk git python3 texinfo wget gdisk libssl-dev lz4
\end{bashinput}

\section{Download Yocto}

Download the \code{kirkstone} version of Poky:
\begin{bashinput}
git clone https://git.yoctoproject.org/git/poky
cd $HOME/__SESSION_NAME__-labs/poky
git checkout -b kirkstone-4.0.5 kirkstone-4.0.5
\end{bashinput}

Return to your project root directory (\code{cd $HOME/__SESSION_NAME__-labs/})
and download the OpenEmbedded and STM32MP layers:
\begin{bashinput}
git clone -b kirkstone https://git.openembedded.org/meta-openembedded
git clone https://github.com/STMicroelectronics/meta-st-stm32mp
cd meta-st-stm32mp
git checkout openstlinux-5.15-yocto-kirkstone-mp1-v22.11.23
\end{bashinput}

\section{Set up the build environment}

Check you're using Bash. This is the default shell when using Ubuntu.

Export all needed variables and set up the build directory:
\begin{bashinput}
cd $HOME/__SESSION_NAME__-labs
source poky/oe-init-build-env
\end{bashinput}

You must specify which machine is your target. By default it
is \code{qemu}. We need to build an image for an \code{stm32mp1}.
Update the \code{MACHINE} configuration variable accordingly.

Also, if you need to save disk space on your computer you can add \code{INHERIT
+= "rm_work"} in the previous configuration file. This will remove the
package work directory once a package is built.

Don't forget to make the configuration aware of the OpenEmbedded and
STM32MP layers. Edit the layer configuration file
(\code{$BUILDDIR/conf/bblayers.conf}) and append the full path to the
\code{meta-openembedded/meta-oe}, \code{meta-openembedded/meta-python} and \code{meta-st-stm32mp} directory to the
\code{BBLAYERS} variable.

\section{Build your first image}

Now that you're ready to start the compilation, simply run:
\begin{bashinput}
bitbake core-image-minimal
\end{bashinput}

Once the build finished, you will find the output images under
\code{$BUILDDIR/tmp/deploy/images/stm32mp1}.

\section{Set up the SD card}

In this first lab we will use an SD card to store the bootloader, kernel and
root filesystem files. To generate the final image, you will find a script in
\code{$BUILDDIR/tmp/deploy/images/stm32mp1/scripts}.

Execute it (replace \code{157d} with \code{157a} depending on your board variant):
\begin{bashinput}
./create_sdcard_from_flashlayout.sh \
../flashlayout_core-image-minimal/trusted/FlashLayout_sdcard_stm32mp157d-dk1-trusted.tsv
\end{bashinput}

Flash the SD card with that image:
\begin{bashinput}
umount /dev/mmcblk0p*
sudo dd if=../FlashLayout_sdcard_stm32mp157d-dk1-trusted.raw of=/dev/mmcblk0 bs=8M conv=fdatasync
\end{bashinput}

\section{Setting up serial communication with the board}

Plug the USB-A to micro USB-B cable on the Discovery board. There is
only one micro USB port on the board, it is CN11, also named ST-LINK.
This is a debug interface and exposes multiple debugging interfaces,
including a serial interface. When plugged in your computer, a serial
port should appear, \code{/dev/ttyACM0}.

You can also see this device appear by looking at the output of
\code{dmesg}.

To communicate with the board through the serial port, install a
serial communication program, such as \code{picocom}:

\begin{bashinput}
sudo apt install picocom
\end{bashinput}

You also need to make your user belong to the \code{dialout} group to be
allowed to write to the serial console:

\begin{bashinput}
sudo adduser $USER dialout
\end{bashinput}

{\bf Important}: for the group change to be effective, you have to
{\em completely log out} from your session and log in again (no need to
reboot). A workaround is to run \code{newgrp dialout}, but it is not global.
You have to run it in each terminal.

Run \code{picocom -b 115200 /dev/ttyACM0}, to start serial
communication on \code{/dev/ttyACM0}, with a baudrate of 115200.

If you wish to exit \code{picocom}, press \code{[Ctrl][a]} followed by
\code{[Ctrl][x]}.

There should be nothing on the serial line so far, as the board is not
powered up yet.

\section{Boot}

Insert the SD card in the dedicated slot on the Discovery.  You can
now power-up the board by connecting the USB-C cable to the board, in
CN6, \code{PWR_IN} and to your PC at the other end.  You should see
boot messages on the console. Wait until the login prompt, then enter
\code{root} as user.  Congratulations! The board has booted and you
now have a shell.

\section{Known issues on stmp32mp157d-dk1}

If your board doesn't boot any more, even after making no changes to the
micro SD card, even after unplugging the USB-C power cable, you may need
to {\bf unplug both} the USB micro-B (used for serial) and USB-C power cables
to get the board to boot again. It's probably because the USB micro-B is
also a power source.


